\section{Context as Material: A Theoretical and Computational Framework
for Large Language Model
Systems}\label{context-as-material-a-theoretical-and-computational-framework-for-large-language-model-systems}

\subsection{Abstract}\label{abstract}

The paradigm shift in software engineering precipitated by Large
Language Models (LLMs)---often characterized as "Software
3.0"---necessitates a fundamental ontological re-evaluation of the
medium in which computation occurs. No longer defined by deterministic
logic gates or explicit instruction sets, the primary unit of
computation has shifted to the "token," and the primary workspace to the
"context window." This report proposes the "Context as Material"
framework, which posits that the context window must be treated not as
an abstract, infinite canvas for text, but as a finite physical resource
governed by distinct material properties, thermodynamic-like laws of
entropy, and rigorous economic constraints. By synthesizing foundational
anthropological theories---specifically Clifford Geertz's "thick
description," Lucy Suchman's "situated actions," and Bruno Latour's
"Actor-Network Theory"---with contemporary empirical research on
"context rot," "lost-in-the-middle" phenomena, and "agentic context
engineering," we establish a hierarchical taxonomy of context
management. This hierarchy, conceptualized as the \textbf{Material
Pyramid}, spans five distinct strata: \textbf{Foundations},
\textbf{Material}, \textbf{Physics}, \textbf{Architecture}, and
\textbf{Economics}. Through this lens, we argue that the efficacy of
autonomous agents is bounded by the physical dynamics of their context,
requiring a transition from the linguistic art of "prompt engineering"
to the material science of "context engineering."

\subsection{I. Introduction}\label{i.-introduction}

The history of computer science is largely a history of abstraction.
From vacuum tubes to transistors, from assembly language to
object-oriented programming, the trajectory has been one of distancing
the developer from the physical constraints of the hardware. However,
the advent of Large Language Models (LLMs) and the subsequent rise of
Generative AI have reversed this trend, reintroducing a form of stubborn
materiality to the forefront of software development. In this new era,
often termed "Software 3.0" by Andrej Karpathy, the neural network acts
as the central processing unit, but the "code" is no longer a static
binary; it is a fluid, probabilistic stream of natural language tokens
fed into a context window.\textsuperscript{1}

This report argues that the "context window" is not merely a data
buffer. It is a material substance that underpins the cognitive
capabilities of the artificial agent. Like any material, it occupies
space (VRAM), possesses mass (latency), generates friction (inference
cost), and is subject to degradation (context rot).\textsuperscript{3}
The prevailing view of context as an ephemeral, abstract "prompt" is
insufficient for building robust, autonomous systems. As agents are
tasked with increasingly complex, long-horizon workflows---debugging
codebases, managing financial portfolios, or navigating legal
precedents---the management of this material becomes the defining
engineering challenge.

We introduce the "Context as Material" framework to formalize this
challenge. Structured as a pyramid, this framework grounds the ephemeral
nature of language in the concrete realities of computation. At the
base, we find the \textbf{Foundations}, borrowing from anthropology and
sociology to define what constitutes "meaningful" context in a
socio-technical system. Above this lies the \textbf{Material}, the raw
technical substrate of tokens and Key-Value (KV) caches that physically
embody memory.\textsuperscript{5} This material is governed by the
\textbf{Physics} of the transformer architecture, which dictates how
attention flows, accumulates, or dissipates across the sequence
length.\textsuperscript{6} The \textbf{Architecture} layer describes the
systems---such as Agentic Context Engineering (ACE) and Dynamic
Cheatsheets---designed to structure and refine this
material.\textsuperscript{7} Finally, the \textbf{Economics} layer
analyzes the cost-value optimization of context as a scarce, tradable
asset.\textsuperscript{8}

This report provides an exhaustive analysis of each layer, weaving
together theoretical insights with empirical data to offer a
comprehensive guide for the design of next-generation AI systems.

\subsection{II. Foundations: The Anthropological
Substrate}\label{ii.-foundations-the-anthropological-substrate}

To engineer context effectively, one must first understand what context
\emph{is}. In computer science, context is often reduced to
"state"---the values of variables at a given time. However, for an LLM
effectively simulating human reasoning, "state" is insufficient. We must
turn to the "Foundations" layer of the Material Pyramid, which draws
upon the rich traditions of anthropology, ethnography, and Science and
Technology Studies (STS) to define the semiotic density required for
intelligence.

\subsubsection{A. Thick Description and Semiotic
Density}\label{a.-thick-description-and-semiotic-density}

The challenge of prompting an LLM to perform a complex task is
structurally identical to the challenge of an ethnographer attempting to
understand a foreign culture. In his 1973 seminal work, \emph{The
Interpretation of Cultures}, Clifford Geertz introduced the distinction
between "thin description" and "thick description".\textsuperscript{9}
Geertz used the example of a wink: physically, it is merely a
contraction of the eyelid (thin description). However, as a social act,
it could be a conspiratorial signal, a mockery, a twitch, or a practice
run. To understand the wink, one must describe the "web of
significance"---the complex, layered hierarchies of meaning---that
surround the physical action.\textsuperscript{11}

In the domain of LLMs, a "thin description" corresponds to a naive,
zero-shot prompt: "Translate this text" or "Fix this bug." Such prompts
provide the model with the raw signal but lack the semiotic scaffolding
necessary to disambiguate intent. The model, trained on the vast,
contradictory "culture" of the internet, requires "thick description" to
narrow its probabilistic output space to a desired
behavior.\textsuperscript{12} Thick description in context engineering
involves layering the prompt not just with instructions, but with
provenance, constraints, emotional tone, relational history, and "emic"
(insider) perspectives.\textsuperscript{9}

The "Context as Material" framework posits that this "thickness" is a
quantifiable property of the context material. Just as a physical
material must have sufficient density to support a load, the context
must have sufficient "semiotic density" to support complex
reasoning.\textsuperscript{13} Geertz's assertion that culture is
"semiotic" implies that the context window is a space where signs
(tokens) interact to produce meaning.\textsuperscript{9} If the context
is too "thin," the semiotic reactions fail to ignite, leading to generic
or hallucinated outputs---the AI equivalent of mistaking a wink for a
twitch.

\subsubsection{B. Situated Actions and the Failure of
Planning}\label{b.-situated-actions-and-the-failure-of-planning}

The necessity of dynamic context is further illuminated by Lucy
Suchman's theory of "situated actions," presented in \emph{Plans and
Situated Actions} (1987).\textsuperscript{14} Suchman critiqued the
classical AI view that intelligent behavior results from the execution
of a pre-formulated plan. Instead, she argued that plans are merely
resources for action, but the action itself is "situated"---it emerges
from the immediate, improvised interaction between the actor and their
environment.\textsuperscript{14}

This distinction is critical for the design of autonomous agents. A
rigid "System Prompt" acts as a plan---a static prescription of
behavior. However, as the agent interacts with a user or a tool, the
"situation" evolves. If the agent relies solely on the static plan (the
system prompt), it fails to adapt to the contingencies of the
moment.\textsuperscript{16} Suchman's work suggests that context is not
a static background but an active, fluid constituent of intelligence.
The "material" of context must be plastic; it must be capable of being
reshaped in real-time by the "situatedness" of the
interaction.\textsuperscript{17}

This theoretical stance underpins modern architectures like "Dynamic
Cheatsheets" and "Agentic Context Engineering" (ACE), which reject
static documentation in favor of context that updates based on the
program state.\textsuperscript{18} These systems acknowledge that
intelligence is not "inside the skull" (or the weights) but emerges from
the "intra-action" of the agent and its context.\textsuperscript{17}

\subsubsection{C. Actor-Network Theory and Material
Semiotics}\label{c.-actor-network-theory-and-material-semiotics}

The agency of the context window itself is best understood through the
lens of Bruno Latour's Actor-Network Theory (ANT).\textsuperscript{20}
ANT posits that agency is not a unique property of humans but is
distributed across a heterogeneous network of human and non-human
"actants." In an LLM system, the non-human actants---the specific tokens
in the prompt, the retrieval algorithm, the KV cache, the GPU
memory---are not passive tools. They actively mediate the interaction,
enabling or constraining specific outcomes.\textsuperscript{22}

Latour argues that we must treat these non-human actors with "analytical
equality".\textsuperscript{22} In the context of "Context Rot"
(discussed in the Physics layer), a "distractor" token is a hostile
actant; it actively works to disrupt the network of
meaning.\textsuperscript{3} Conversely, a well-engineered "few-shot
example" is an ally, stabilizing the network. The act of context
engineering is thus a political act of "parliamentary" assembly:
deciding which actants (information chunks) are allowed representation
in the "parliament of things" (the context window).\textsuperscript{21}

Furthermore, the concept of "material-semiotics," developed by Donna
Haraway and Karen Barad, bridges the gap between the physical and the
conceptual.\textsuperscript{23} Barad's notion of "intra-action" posits
that distinct entities do not precede their interaction; rather, they
emerge \emph{through} their entanglement.\textsuperscript{25} Applied to
LLMs, this implies that the "knowledge" of the model is not a static
retrieval from a database. It is intra-actively produced at the moment
of inference by the entanglement of the model's weights (past) and the
context window (present). The engineer, by defining the "agential cut"
(what is included or excluded from context), literally constructs the
reality in which the agent exists.\textsuperscript{27}

\subsection{III. Material: The Substance of
Computation}\label{iii.-material-the-substance-of-computation}

Having established the theoretical necessity of context, we descend the
pyramid to the "Material" layer. Here, we strip away the metaphors of
"mind" and "language" to examine the raw, physical substance that
enables LLM cognition. In the "Context as Material" framework, context
is a tangible resource stored in High-Bandwidth Memory (HBM), occupying
physical space and consuming physical energy.

\subsubsection{A. The Materiality of Tokens and the KV
Cache}\label{a.-the-materiality-of-tokens-and-the-kv-cache}

The atomic unit of this material is the "token"---a discrete chunk of
information that serves as the fundamental currency of the LLM economy.
However, during the inference process, tokens are transmuted into a much
heavier, more cumbersome substance: the Key-Value (KV)
cache.\textsuperscript{5}

In the Transformer architecture, the attention mechanism calculates the
relationship between the current token and every previous token in the
sequence. To generate the \$N+1\$ token, the model needs the Key (K) and
Value (V) matrices for tokens \$1\$ through \$N\$. Without caching, the
model would be forced to recompute these matrices from scratch at every
single step, leading to a quadratic increase in computational cost
(\$O(N\^{}2)\$) and making real-time generation
impossible.\textsuperscript{29}

The KV cache serves as the "physical memory" of the LLM. It is a
dedicated allocation of VRAM (Video RAM) on the GPU where these
intermediate representations are stored.\textsuperscript{5} This cache
is the material embodiment of the "context window." It has a definite
mass, measured in gigabytes. For example, a 175-billion parameter model
with a long context window (e.g., 128k tokens) can generate a KV cache
so massive that it exceeds the memory capacity of a single A100
GPU.\textsuperscript{28}

This materiality introduces the "Memory Wall"---a bottleneck where the
speed of inference is limited not by the speed of computation (FLOPS)
but by the bandwidth available to move this massive KV material between
memory and the compute cores.\textsuperscript{28} We observe here a
direct physical manifestation of Haraway's material-semiotics: the
"meaning" (semiotics) the model can generate is strictly bounded by the
physical "matter" (VRAM) available to store its
history.\textsuperscript{23} If the material storage runs out, the
context is truncated, and the "web of significance" is severed.

\subsubsection{B. The Context Window as RAM: The LLM
OS}\label{b.-the-context-window-as-ram-the-llm-os}

Andrej Karpathy crystallizes this material view by analogizing the LLM
to a Central Processing Unit (CPU) and the context window to Random
Access Memory (RAM).\textsuperscript{1} In this "LLM OS" paradigm, the
Large Language Model is the kernel---the core processing engine. The
context window is the working memory where the active state of the
application is loaded.\textsuperscript{1}

Just as a traditional Operating System (OS) manages RAM to allow
multiple applications to run without crashing, an "LLM OS" must manage
the context window. It must decide what data to "page in" (retrieve from
long-term storage/RAG), what to "page out" (summarize or discard), and
how to organize the memory space to prevent "segmentation faults"
(hallucinations or context overflow).\textsuperscript{1}

This analogy elevates "Context Engineering" from a linguistic art to a
systems engineering discipline. It is akin to memory management in
classical computing. The context engineer is responsible for "malloc"
(memory allocation) and "garbage collection" (pruning irrelevant
tokens). The finite size of the context window (e.g., 8k, 32k, 128k
tokens) acts as the physical constraint of the RAM stick. If the
engineer fills the RAM with "bloatware" (verbose, irrelevant text), the
system thrashes; it spends all its compute cycles processing noise
rather than generating signal.\textsuperscript{31}

\subsubsection{C. Software 3.0: Vibe Coding and Natural Language
Programming}\label{c.-software-3.0-vibe-coding-and-natural-language-programming}

This material shift marks the transition to "Software 3.0." Software 1.0
was explicit code (C++, Python) written by humans. Software 2.0 was
neural networks (weights) learned from data, opaque to human inspection.
Software 3.0, as defined by Karpathy, is the programming of these neural
networks using natural language prompts.\textsuperscript{1}

In Software 3.0, the prompt \emph{is} the code. However, unlike the
rigid syntax of C++, this code is "soft" and probabilistic. Karpathy
refers to this as "vibe coding"---programming through intent and persona
rather than explicit logic.\textsuperscript{1} But this softness does
not imply a lack of structure. The "Context as Material" framework
argues that Software 3.0 requires even \emph{more} rigorous structural
engineering because the material (language) is inherently unstable. The
engineer must build "containment structures" within the
prompt---schemas, delimiters, and few-shot examples---to constrain the
probabilistic "fluid" of the model into reliable execution
paths.\textsuperscript{31}

\subsection{IV. Physics: The Dynamics of
Context}\label{iv.-physics-the-dynamics-of-context}

If context is a material, it follows that it must be subject to physical
laws. It does not behave uniformly; it exhibits varying density, decay,
interference patterns, and gravitational pulls. Recent empirical
research has uncovered specific "physics" governing how LLMs interact
with information distributed across this material substrate.

\subsubsection{A. The "Lost in the Middle"
Phenomenon}\label{a.-the-lost-in-the-middle-phenomenon}

One of the most significant discoveries in the physics of context is the
"U-shaped" performance curve, widely known as the "Lost in the Middle"
phenomenon.\textsuperscript{6} Research by Liu et al. (2023)
demonstrates that LLMs are not isotropic in their attention
capabilities. When presented with a long sequence of information (e.g.,
a "haystack" of documents), the model\textquotesingle s ability to
retrieve a specific fact (the "needle") depends heavily on its
position.\textsuperscript{6}

\begin{itemize}
\item
  \textbf{Primacy Bias:} Information placed at the very beginning of the
  context window is retrieved with high accuracy. The model's attention
  mechanism seems to "anchor" on the initial tokens, treating them as
  foundational instructions.
\item
  \textbf{Recency Bias:} Information at the very end of the
  context---closest to the generation point---is also highly salient.
  These tokens are "fresh" in the recurrence of the autoregressive
  process.
\item
  \textbf{The Trough of Ignorance:} However, information buried in the
  middle of a long context sequence suffers from significant retrieval
  degradation. The performance curve dips dramatically in the center,
  forming a "U" shape.
\end{itemize}

This phenomenon suggests that the "material" of context has variable
density. The edges are dense and solid, providing reliable support for
reasoning, while the center is porous and unstable.\textsuperscript{32}
This finding challenges the naive assumption that "more context is
better." Simply extending the context window (e.g., to 1 million tokens)
does not guarantee that the model can effectively utilize that
space.\textsuperscript{33} The physics of the attention
mechanism---specifically the accumulation of attention scores and the
limitations of Rotary Positional Embeddings (RoPE)---creates a "fog of
war" in the middle of the sequence, where the signal-to-noise ratio
drops precipitously.\textsuperscript{34}

\subsubsection{B. Context Rot and
Entropy}\label{b.-context-rot-and-entropy}

Complementing the positional bias is the concept of "Context Rot," a
term coined to describe the degradation of model performance as the
volume of irrelevant information increases.\textsuperscript{3} Just as
physical materials degrade due to entropy, the semantic integrity of the
context window degrades as "noise" accumulates.

Experiments by Chroma Research (2025) reveal that this rot is not
linear. They evaluated 18 state-of-the-art LLMs using a "Needle in a
Haystack" (NIAH) setup with varying degrees of
complexity.\textsuperscript{3} The findings were stark:

\begin{itemize}
\item
  \textbf{Distractor Interference:} Adding "distractors"---related but
  irrelevant information---disproportionately impairs the
  model\textquotesingle s ability to retrieve the target information.
  The presence of distractors doesn\textquotesingle t just dilute the
  signal; it actively confuses the attention heads, leading to
  "hallucinations" (confident errors) or "refusals" (conservative
  abstentions).\textsuperscript{3}
\item
  \textbf{Semantic Rot:} The rot is most severe when the distractors are
  semantically similar to the needle. This suggests an interference
  pattern where the vector representations of similar concepts overlap,
  making it difficult for the model to resolve the correct
  entity.\textsuperscript{4}
\item
  \textbf{Shuffled vs. Ordered:} Surprisingly, in some cases,
  \emph{shuffled} (incoherent) contexts performed better than logically
  ordered ones.\textsuperscript{35} This counter-intuitive finding
  suggests that models may over-attend to the "narrative flow" of
  ordered text, getting lost in the story, whereas shuffled text forces
  a more mechanical, keyword-based retrieval.
\end{itemize}

The physics of context rot implies a \textbf{Second Law of Context
Dynamics}: in a closed context window (like a long conversation), the
ratio of useful information (signal) to useless information (noise)
tends to decrease over time. Without active external work (energy input
in the form of curation and pruning), the system will inevitably devolve
into a state of maximum entropy, characterized by hallucinations and
incoherence.\textsuperscript{3}

\subsubsection{C. The Non-Uniformity of
Attention}\label{c.-the-non-uniformity-of-attention}

The underlying mechanism driving these phenomena is the non-uniformity
of the attention mechanism. While theoretically, the Transformer
architecture allows every token to attend to every other token
(\$O(N\^{}2)\$ complexity), in practice, the learned attention weights
are not distributed evenly.\textsuperscript{29} The model learns
heuristics during pre-training that prioritize certain positions and
token types.

This creates "gravity wells" in the context window. The system prompt
(start) and the user query (end) act as massive bodies that warp the
attention field, pulling the model\textquotesingle s focus. The "middle"
tokens, lacking this gravitational pull, drift in a low-attention void.
Understanding these physics is crucial for the "Architecture" layer.
Engineers cannot simply dump data into the context window; they must
structure it to exploit the gravitational pull of the start and end
tokens, placing critical instructions and schemas where the physics of
the model will naturally highlight them.\textsuperscript{32}

\subsection{V. Architecture: Structuring the
Material}\label{v.-architecture-structuring-the-material}

Recognizing the material properties and physical constraints of context,
we move to the \textbf{Architecture} layer. This layer deals with the
design of systems that can engineer context to maximize utility and
minimize rot. The transition here is from the static art of "prompt
engineering" to the dynamic systems engineering of "Agentic Context
Engineering."

\subsubsection{A. From Prompting to Agentic Context Engineering
(ACE)}\label{a.-from-prompting-to-agentic-context-engineering-ace}

Traditional prompt engineering treats the prompt as a static
artifact---a "spell" cast once to invoke a result. However, for
autonomous agents operating over long time horizons, static prompts fail
due to the "situated" nature of the task and the inevitable accumulation
of entropy.\textsuperscript{36} The \textbf{Agentic Context Engineering
(ACE)} framework, introduced by Zhang et al. (2025), represents the
architectural response to these challenges.\textsuperscript{7}

ACE treats context not as a static text but as an evolving "playbook."
It identifies two primary failure modes in traditional context
management:

\begin{enumerate}
\def\labelenumi{\arabic{enumi}.}
\item
  \textbf{Brevity Bias:} When context is summarized to save space, the
  model tends to discard nuanced, domain-specific details in favor of
  generic abstractions. This "thinning" of the description leads to a
  loss of competence.\textsuperscript{19}
\item
  \textbf{Context Collapse:} Repeatedly rewriting or summarizing the
  context history leads to a degradation of information fidelity,
  similar to the generation loss in repeatedly copying a JPEG image.
  Over time, the context "collapses" into a noisy, low-fidelity
  state.\textsuperscript{19}
\end{enumerate}

\subsubsection{B. The Generator-Reflector-Curator
Triad}\label{b.-the-generator-reflector-curator-triad}

To solve these problems, ACE introduces a modular architecture that
separates the cognitive functions of the agent into three distinct
roles, creating a "self-improving loop".\textsuperscript{7}

\begin{enumerate}
\def\labelenumi{\arabic{enumi}.}
\item
  \textbf{The Generator:} This module is responsible for producing
  candidate strategies and reasoning paths based on the current context.
  It generates the "forward pass" of action, proposing how to handle a
  new query or task.\textsuperscript{38}
\item
  \textbf{The Reflector:} This module acts as the "critic." It analyzes
  the outcomes of the Generator\textquotesingle s actions,
  distinguishing between success and failure. The Reflector applies a
  critical gaze to the agent\textquotesingle s own history, extracting
  "lessons learned." This is a computational implementation of Donald
  Schön's concept of "reflection-in-action"---the ability of a
  practitioner to think about what they are doing while doing
  it.\textsuperscript{37}
\item
  \textbf{The Curator:} This is the memory manager. It integrates the
  lessons extracted by the Reflector into the persistent context store
  (the playbook). Crucially, the Curator uses "delta
  updates"---incremental additions of specific insights---rather than
  wholesale rewrites of the context. This "append-only" or "patch-based"
  strategy prevents Context Collapse and ensures that the context grows
  denser and more refined over time.\textsuperscript{37}
\end{enumerate}

\subsubsection{C. Dynamic Cheatsheets and Focused
Retrieval}\label{c.-dynamic-cheatsheets-and-focused-retrieval}

The ACE framework aligns with the concept of \textbf{Dynamic
Cheatsheets}.\textsuperscript{18} Instead of providing the agent with a
massive, static manual (which would suffer from Lost-in-the-Middle
effects), the system maintains a "cheatsheet" of relevant strategies,
code snippets, and heuristics that updates in real-time.

To combat "Context Rot," architectures are also moving away from
brute-force RAG (Retrieval-Augmented Generation) toward
\textbf{Graph-Based Focused Retrieval}.\textsuperscript{4} In standard
RAG, retrieving a chunk of text often brings along
"distractors"---irrelevant surrounding sentences that rot the context.
In Graph-Based retrieval, the system models the knowledge base as a
graph of nodes (concepts) and edges (relations). When the agent needs
information, it traverses the graph to retrieve only the specific nodes
and edges required, assembling a "pure signal" context with minimal
noise.\textsuperscript{4}

This approach allows for "Context-Bench" style evaluations, where agents
are tested not just on their ability to answer questions, but on their
ability to "engineer" their own context---deciding what to retrieve,
what to keep, and what to discard.\textsuperscript{36} The agent becomes
an active curator of its own material reality.

\subsection{VI. Economics: The Value of
Context}\label{vi.-economics-the-value-of-context}

The final layer of the pyramid is \textbf{Economics}. The manipulation
of context material is not free; it consumes computational energy and
financial capital. The "Context as Material" framework allows us to
quantify the value of context and optimize its usage in a market-driven
environment.

\subsubsection{A. Token Economics: Cache Write vs. Cache
Read}\label{a.-token-economics-cache-write-vs.-cache-read}

The primary cost driver in LLM systems is the processing of tokens.
However, the pricing models of major providers like Anthropic reveal a
crucial distinction that reflects the underlying materiality of the KV
cache.\textsuperscript{8}

\begin{itemize}
\item
  \textbf{Cache Write Cost:} Processing \emph{new} tokens is expensive
  because the model must perform the full matrix calculations to
  generate the KV states. For Claude 3.5 Sonnet, this costs
  \textbf{\$3.75 per million tokens}.\textsuperscript{8}
\item
  \textbf{Cache Read Cost:} Reusing \emph{existing} context (cached KV
  states) is significantly cheaper because the heavy computation is
  already done. Accessing cached tokens costs only \textbf{\$0.30 per
  million tokens}---a \textbf{90\% reduction} in
  cost.\textsuperscript{8}
\end{itemize}

This pricing structure fundamentally alters the economics of context. It
transforms context from a "disposable good" (computed once and
discarded) into a "durable asset" (computed once and reused). The
initial processing of a prompt is a \textbf{Capital Expenditure
(CapEx)}---an investment in building a material state. Subsequent uses
of that prompt are \textbf{Operational Expenditures (OpEx)}.

\subsubsection{B. The Attention Budget and
Latency}\label{b.-the-attention-budget-and-latency}

Beyond monetary cost, there is the cost of time (latency). The "mass" of
the context window creates inertia. Processing a 100k token context for
every turn induces unacceptable latency, often measured in tens of
seconds.\textsuperscript{8}

We can conceptualize this as an \textbf{"Attention Budget."} An agent
has a finite amount of "attention seconds" it can spend per interaction.
If the context is too "heavy" (too many tokens), the agent spends its
entire budget just "loading" the context, leaving no time for "thinking"
(generation).

The use of "Prompt Caching" significantly impacts this budget. By
pre-loading heavy context materials (e.g., a codebase, a legal library)
into the cache, the "Time to First Token" (TTFT) can be reduced by up to
\textbf{85\%}.\textsuperscript{8} This allows for "thicker" descriptions
(more context) without the latency penalty.

\subsubsection{C. Return on Context
(RoC)}\label{c.-return-on-context-roc}

The ultimate economic metric for the context engineer is the
\textbf{Return on Context (RoC)}. This metric asks: "Does the addition
of these 1,000 tokens of history improve the outcome sufficiently to
justify the incremental cost and latency?"

Empirical results from the ACE framework demonstrate a high RoC. By
using dynamic, curated context playbooks, agents achieved:

\begin{itemize}
\item
  \textbf{+10.6\% performance gain} on general agent
  benchmarks.\textsuperscript{7}
\item
  \textbf{+8.6\% performance gain} on finance-specific reasoning
  tasks.\textsuperscript{7}
\item
  \textbf{-86.9\% reduction in adaptation latency}.\textsuperscript{41}
\item
  \textbf{-75\% reduction in rollout costs}.\textsuperscript{38}
\end{itemize}

These figures prove that treating context as a scarce resource that must
be engineered---rather than an infinite bucket to be filled---maximizes
the efficient conversion of tokens into intelligence.

\subsection{VII. Conclusion}\label{vii.-conclusion}

The "Context as Material" framework offers a unified, multi-layered
ontology for understanding and engineering Large Language Model systems.
It moves the discourse beyond the superficial "tips and tricks" of
prompt engineering to a rigorous science of material management.

By traversing the \textbf{Material Pyramid}, we see that effective AI
systems are built on:

\begin{enumerate}
\def\labelenumi{\arabic{enumi}.}
\item
  \textbf{Foundations:} A deep understanding of semiotics and situated
  action (Geertz, Suchman, Latour) that defines \emph{meaning}.
\item
  \textbf{Material:} A mastery of the physical substrate (KV Cache,
  VRAM) that defines \emph{capacity}.
\item
  \textbf{Physics:} A respect for the thermodynamic laws (Entropy,
  Gravity) that define \emph{dynamics}.
\item
  \textbf{Architecture:} The implementation of modular systems (ACE,
  Generator-Reflector-Curator) that define \emph{structure}.
\item
  \textbf{Economics:} An optimization of the cost-value function (Cache
  Pricing, RoC) that defines \emph{viability}.
\end{enumerate}

As we enter the era of Software 3.0, the primary skill of the AI
engineer evolves. It is no longer about writing the perfect line of
code; it is about curating the perfect window of context. It is a shift
from the digital to the material, from the abstract to the situated, and
from the static to the evolving. The context window is the new silicon;
how we refine, structure, and utilize this material will determine the
intelligence of the systems we build.

\subsubsection{Table 1: The Material Pyramid of
Context}\label{table-1-the-material-pyramid-of-context}

\begin{longtable}[]{@{}
  >{\raggedright\arraybackslash}p{(\linewidth - 6\tabcolsep) * \real{0.2500}}
  >{\raggedright\arraybackslash}p{(\linewidth - 6\tabcolsep) * \real{0.2500}}
  >{\raggedright\arraybackslash}p{(\linewidth - 6\tabcolsep) * \real{0.2500}}
  >{\raggedright\arraybackslash}p{(\linewidth - 6\tabcolsep) * \real{0.2500}}@{}}
\toprule\noalign{}
\begin{minipage}[b]{\linewidth}\raggedright
\textbf{Level}
\end{minipage} & \begin{minipage}[b]{\linewidth}\raggedright
\textbf{Core Concept}
\end{minipage} & \begin{minipage}[b]{\linewidth}\raggedright
\textbf{Key Theoretical/Technical Pillars}
\end{minipage} & \begin{minipage}[b]{\linewidth}\raggedright
\textbf{Primary Challenge}
\end{minipage} \\
\begin{minipage}[b]{\linewidth}\raggedright
\textbf{I. Foundations}
\end{minipage} & \begin{minipage}[b]{\linewidth}\raggedright
Semiotics \& Agency
\end{minipage} & \begin{minipage}[b]{\linewidth}\raggedright
Thick Description \textsuperscript{9}, Situated Action
\textsuperscript{14}, ANT \textsuperscript{21}, Intra-action
\textsuperscript{23}
\end{minipage} & \begin{minipage}[b]{\linewidth}\raggedright
Defining "meaning" and "agency" in non-human systems.
\end{minipage} \\
\begin{minipage}[b]{\linewidth}\raggedright
\textbf{II. Material}
\end{minipage} & \begin{minipage}[b]{\linewidth}\raggedright
The Token \& Cache
\end{minipage} & \begin{minipage}[b]{\linewidth}\raggedright
KV Cache \textsuperscript{29}, Context Window as RAM
\textsuperscript{1}, Memory Wall \textsuperscript{28}
\end{minipage} & \begin{minipage}[b]{\linewidth}\raggedright
VRAM limitations, HBM bandwidth bottlenecks.
\end{minipage} \\
\begin{minipage}[b]{\linewidth}\raggedright
\textbf{III. Physics}
\end{minipage} & \begin{minipage}[b]{\linewidth}\raggedright
Dynamics of Attention
\end{minipage} & \begin{minipage}[b]{\linewidth}\raggedright
Lost in the Middle \textsuperscript{6}, Context Rot \textsuperscript{3},
U-Shaped Curve \textsuperscript{32}
\end{minipage} & \begin{minipage}[b]{\linewidth}\raggedright
Non-uniform attention, entropy of information, distractor interference.
\end{minipage} \\
\begin{minipage}[b]{\linewidth}\raggedright
\textbf{IV. Architecture}
\end{minipage} & \begin{minipage}[b]{\linewidth}\raggedright
Structured Management
\end{minipage} & \begin{minipage}[b]{\linewidth}\raggedright
ACE \textsuperscript{7}, Dynamic Cheatsheets \textsuperscript{39},
Generator-Reflector-Curator \textsuperscript{38}
\end{minipage} & \begin{minipage}[b]{\linewidth}\raggedright
Context Collapse, Brevity Bias, Signal-to-Noise maintenance.
\end{minipage} \\
\begin{minipage}[b]{\linewidth}\raggedright
\textbf{V. Economics}
\end{minipage} & \begin{minipage}[b]{\linewidth}\raggedright
Cost-Value Optimization
\end{minipage} & \begin{minipage}[b]{\linewidth}\raggedright
Prompt Caching \textsuperscript{8}, Return on Context (RoC), Latency
Budget
\end{minipage} & \begin{minipage}[b]{\linewidth}\raggedright
Balancing "Thick Description" (quality) with inference costs (price).
\end{minipage} \\
\midrule\noalign{}
\endhead
\bottomrule\noalign{}
\endlastfoot
\end{longtable}

\subsubsection{Table 2: Comparative Analysis of Context
Paradigms}\label{table-2-comparative-analysis-of-context-paradigms}

\begin{longtable}[]{@{}
  >{\raggedright\arraybackslash}p{(\linewidth - 6\tabcolsep) * \real{0.2500}}
  >{\raggedright\arraybackslash}p{(\linewidth - 6\tabcolsep) * \real{0.2500}}
  >{\raggedright\arraybackslash}p{(\linewidth - 6\tabcolsep) * \real{0.2500}}
  >{\raggedright\arraybackslash}p{(\linewidth - 6\tabcolsep) * \real{0.2500}}@{}}
\toprule\noalign{}
\begin{minipage}[b]{\linewidth}\raggedright
\textbf{Feature}
\end{minipage} & \begin{minipage}[b]{\linewidth}\raggedright
\textbf{Prompt Engineering (Software 2.0)}
\end{minipage} & \begin{minipage}[b]{\linewidth}\raggedright
\textbf{Context Engineering (Software 3.0)}
\end{minipage} & \begin{minipage}[b]{\linewidth}\raggedright
\textbf{Agentic Context Engineering (ACE)}
\end{minipage} \\
\begin{minipage}[b]{\linewidth}\raggedright
\textbf{Metaphor}
\end{minipage} & \begin{minipage}[b]{\linewidth}\raggedright
Casting a Spell (Incantation).
\end{minipage} & \begin{minipage}[b]{\linewidth}\raggedright
Managing RAM (OS Kernel).
\end{minipage} & \begin{minipage}[b]{\linewidth}\raggedright
Evolving a Playbook (Scientific Method).
\end{minipage} \\
\begin{minipage}[b]{\linewidth}\raggedright
\textbf{Temporal State}
\end{minipage} & \begin{minipage}[b]{\linewidth}\raggedright
Static / Stateless.
\end{minipage} & \begin{minipage}[b]{\linewidth}\raggedright
Persistent / Stateful.
\end{minipage} & \begin{minipage}[b]{\linewidth}\raggedright
Dynamic / Self-Improving.\textsuperscript{7}
\end{minipage} \\
\begin{minipage}[b]{\linewidth}\raggedright
\textbf{Structure}
\end{minipage} & \begin{minipage}[b]{\linewidth}\raggedright
Text block.
\end{minipage} & \begin{minipage}[b]{\linewidth}\raggedright
Hierarchical / Graph.
\end{minipage} & \begin{minipage}[b]{\linewidth}\raggedright
Modular (Generator-Reflector-Curator).\textsuperscript{38}
\end{minipage} \\
\begin{minipage}[b]{\linewidth}\raggedright
\textbf{Handling Noise}
\end{minipage} & \begin{minipage}[b]{\linewidth}\raggedright
Ignore / Hope.
\end{minipage} & \begin{minipage}[b]{\linewidth}\raggedright
Pruning / Garbage Collection.
\end{minipage} & \begin{minipage}[b]{\linewidth}\raggedright
Active Curation / Delta Updates.\textsuperscript{37}
\end{minipage} \\
\begin{minipage}[b]{\linewidth}\raggedright
\textbf{Failure Mode}
\end{minipage} & \begin{minipage}[b]{\linewidth}\raggedright
Hallucination.
\end{minipage} & \begin{minipage}[b]{\linewidth}\raggedright
Context Rot / Lost in Middle.
\end{minipage} & \begin{minipage}[b]{\linewidth}\raggedright
Context Collapse.\textsuperscript{19}
\end{minipage} \\
\begin{minipage}[b]{\linewidth}\raggedright
\textbf{Theoretical Basis}
\end{minipage} & \begin{minipage}[b]{\linewidth}\raggedright
Linguistics / NLP.
\end{minipage} & \begin{minipage}[b]{\linewidth}\raggedright
Systems Engineering.
\end{minipage} & \begin{minipage}[b]{\linewidth}\raggedright
Situated Action / Material Semiotics.\textsuperscript{23}
\end{minipage} \\
\midrule\noalign{}
\endhead
\bottomrule\noalign{}
\endlastfoot
\end{longtable}

\subsubsection{Table 3: Economic Impact of Context Caching (Claude 3.5
Sonnet)}\label{table-3-economic-impact-of-context-caching-claude-3.5-sonnet}

\textsuperscript{8}

\begin{longtable}[]{@{}
  >{\raggedright\arraybackslash}p{(\linewidth - 6\tabcolsep) * \real{0.2500}}
  >{\raggedright\arraybackslash}p{(\linewidth - 6\tabcolsep) * \real{0.2500}}
  >{\raggedright\arraybackslash}p{(\linewidth - 6\tabcolsep) * \real{0.2500}}
  >{\raggedright\arraybackslash}p{(\linewidth - 6\tabcolsep) * \real{0.2500}}@{}}
\toprule\noalign{}
\begin{minipage}[b]{\linewidth}\raggedright
\textbf{Operation}
\end{minipage} & \begin{minipage}[b]{\linewidth}\raggedright
\textbf{Cost (per Million Tokens)}
\end{minipage} & \begin{minipage}[b]{\linewidth}\raggedright
\textbf{Latency Impact}
\end{minipage} & \begin{minipage}[b]{\linewidth}\raggedright
\textbf{Economic Classification}
\end{minipage} \\
\begin{minipage}[b]{\linewidth}\raggedright
\textbf{Standard Input}
\end{minipage} & \begin{minipage}[b]{\linewidth}\raggedright
\$3.00
\end{minipage} & \begin{minipage}[b]{\linewidth}\raggedright
High (Linear growth)
\end{minipage} & \begin{minipage}[b]{\linewidth}\raggedright
OpEx (Recurring)
\end{minipage} \\
\begin{minipage}[b]{\linewidth}\raggedright
\textbf{Cache Write}
\end{minipage} & \begin{minipage}[b]{\linewidth}\raggedright
\$3.75 (+25\%)
\end{minipage} & \begin{minipage}[b]{\linewidth}\raggedright
High (Initial Load)
\end{minipage} & \begin{minipage}[b]{\linewidth}\raggedright
CapEx (Investment)
\end{minipage} \\
\begin{minipage}[b]{\linewidth}\raggedright
\textbf{Cache Read}
\end{minipage} & \begin{minipage}[b]{\linewidth}\raggedright
\$0.30 (-90\%)
\end{minipage} & \begin{minipage}[b]{\linewidth}\raggedright
Low (-85\% TTFT)
\end{minipage} & \begin{minipage}[b]{\linewidth}\raggedright
OpEx (Optimized)
\end{minipage} \\
\midrule\noalign{}
\endhead
\bottomrule\noalign{}
\endlastfoot
\end{longtable}

Note: Data derived from Anthropic pricing models.\textsuperscript{8}
TTFT = Time To First Token.

\paragraph{Works cited}\label{works-cited}

\begin{enumerate}
\def\labelenumi{\arabic{enumi}.}
\item
  Software 3.0: Software is Changing Again and Again - Superagentic AI,
  accessed December 10, 2025,
  \href{https://super-agentic.ai/resources/super-posts/software-3-0-software-is-changing-again-and-again}{\ul{https://super-agentic.ai/resources/super-posts/software-3-0-software-is-changing-again-and-again}}
\item
  Software 3.0: Karpathy\textquotesingle s AI Vision Reshaping
  Development - WillDom, accessed December 10, 2025,
  \href{https://willdom.com/blog/software-3-0-ai-reshaping-development-future/}{\ul{https://willdom.com/blog/software-3-0-ai-reshaping-development-future/}}
\item
  Papers Explained 445: Context Rot \textbar{} by Ritvik Rastogi -
  Medium, accessed December 10, 2025,
  \href{https://ritvik19.medium.com/papers-explained-443-context-rot-4bbd72d77631}{\ul{https://ritvik19.medium.com/papers-explained-443-context-rot-4bbd72d77631}}
\item
  Context Rot is a Big Problem, and Graphs Are the Promising Fix for
  Coding Agents - Medium, accessed December 10, 2025,
  \href{https://medium.com/@animesh1997/context-rot-is-a-big-problem-and-graphs-are-the-promising-fix-for-coding-agents-30be152c49c6}{\ul{https://medium.com/@animesh1997/context-rot-is-a-big-problem-and-graphs-are-the-promising-fix-for-coding-agents-30be152c49c6}}
\item
  From Theory to Practice: Demystifying the Key-Value Cache in Modern
  LLMs, accessed December 10, 2025,
  \href{https://alain-airom.medium.com/from-theory-to-practice-demystifying-the-key-value-cache-in-modern-llms-9674e9f904a5}{\ul{https://alain-airom.medium.com/from-theory-to-practice-demystifying-the-key-value-cache-in-modern-llms-9674e9f904a5}}
\item
  Lost in the Middle: How Language Models Use Long Contexts -
  ResearchGate, accessed December 10, 2025,
  \href{https://www.researchgate.net/publication/378284067_Lost_in_the_Middle_How_Language_Models_Use_Long_Contexts}{\ul{https://www.researchgate.net/publication/378284067\_Lost\_in\_the\_Middle\_How\_Language\_Models\_Use\_Long\_Contexts}}
\item
  Agentic Context Engineering: Evolving Contexts for Self-Improving
  Language Models - arXiv, accessed December 10, 2025,
  \href{https://arxiv.org/html/2510.04618v1}{\ul{https://arxiv.org/html/2510.04618v1}}
\item
  Prompt caching with Claude \textbar{} Claude, accessed December 10,
  2025,
  \href{https://www.anthropic.com/news/prompt-caching}{\ul{https://www.anthropic.com/news/prompt-caching}}
\item
  A Summary of Thick Description The Interpretation of Cultures -
  Scribd, accessed December 10, 2025,
  \href{https://www.scribd.com/document/168611383/A-Summary-of-Thick-Description}{\ul{https://www.scribd.com/document/168611383/A-Summary-of-Thick-Description}}
\item
  accessed December 10, 2025,
  \href{https://www.scirp.org/reference/ReferencesPapers?ReferenceID=2132836\#:~:text=Article\%20citationsMore\%3E\%3E-,Geertz\%2C\%20C.,York\%2C\%20NY\%3A\%20Basic\%20Books.}{\ul{https://www.scirp.org/reference/ReferencesPapers?ReferenceID=2132836\#:\textasciitilde:text=Article\%20citationsMore\%3E\%3E-,Geertz\%2C\%20C.,York\%2C\%20NY\%3A\%20Basic\%20Books.}}
\item
  Thick description \textbar{} Literary Theory and Criticism Class Notes
  - Fiveable, accessed December 10, 2025,
  \href{https://fiveable.me/literary-theory-criticism/unit-8/thick-description/study-guide/9FRywJGWCH1SlgzL}{\ul{https://fiveable.me/literary-theory-criticism/unit-8/thick-description/study-guide/9FRywJGWCH1SlgzL}}
\item
  Thick Description - Google Docs, accessed December 10, 2025,
  \href{https://docs.google.com/document/d/1bMbIEc8MjErUYW8S4socA8O7QgNZ-sBd3fRXHetQnMw/preview?hgd=1}{\ul{https://docs.google.com/document/d/1bMbIEc8MjErUYW8S4socA8O7QgNZ-sBd3fRXHetQnMw/preview?hgd=1}}
\item
  Thick description - Wikipedia, accessed December 10, 2025,
  \href{https://en.wikipedia.org/wiki/Thick_description}{\ul{https://en.wikipedia.org/wiki/Thick\_description}}
\item
  Human-Machine Reconfigurations: Plans and Situated Actions: 2nd
  Edition - ResearchGate, accessed December 10, 2025,
  \href{https://www.researchgate.net/publication/265092509_Human-Machine_Reconfigurations_Plans_and_Situated_Actions_2nd_Edition}{\ul{https://www.researchgate.net/publication/265092509\_Human-Machine\_Reconfigurations\_Plans\_and\_Situated\_Actions\_2nd\_Edition}}
\item
  accessed December 10, 2025,
  \href{https://www.biblio.com/9780521337397\#:~:text=Plans\%20and\%20Situated\%20Actions\%3A\%20The,Press\%202nd\%20Edition\%20\%7C\%209780521337397\%20\%7C\%20Biblio}{\ul{https://www.biblio.com/9780521337397\#:\textasciitilde:text=Plans\%20and\%20Situated\%20Actions\%3A\%20The,Press\%202nd\%20Edition\%20\%7C\%209780521337397\%20\%7C\%20Biblio}}
\item
  {[}PDF{]} Plans and Situated Actions: The Problem of Human-Machine
  Communication (Learning in Doing: Social, \textbar{} Semantic Scholar,
  accessed December 10, 2025,
  \href{https://www.semanticscholar.org/paper/Plans-and-Situated-Actions\%3A-The-Problem-of-in-Suchman/5416463537f8c6be1199951b4fd6f8d5dae14920}{\ul{https://www.semanticscholar.org/paper/Plans-and-Situated-Actions\%3A-The-Problem-of-in-Suchman/5416463537f8c6be1199951b4fd6f8d5dae14920}}
\item
  Social Situatedness of Natural and Artificial Intelligence - DiVA
  portal, accessed December 10, 2025,
  \href{http://www.diva-portal.org/smash/get/diva2:3017/FULLTEXT02}{\ul{http://www.diva-portal.org/smash/get/diva2:3017/FULLTEXT02}}
\item
  Dynamic Cheatsheet: Adaptive Reference - Emergent Mind, accessed
  December 10, 2025,
  \href{https://www.emergentmind.com/topics/dynamic-cheatsheet}{\ul{https://www.emergentmind.com/topics/dynamic-cheatsheet}}
\item
  Agentic Context Engineering: Learning Comprehensive Contexts for
  Self-Improving Language Models \textbar{} OpenReview, accessed
  December 10, 2025,
  \href{https://openreview.net/forum?id=eC4ygDs02R}{\ul{https://openreview.net/forum?id=eC4ygDs02R}}
\item
  Video: Actor Network Theory \textbar{} Diagram, Critiques \& Examples
  - Study.com, accessed December 10, 2025,
  \href{https://study.com/academy/lesson/video/latours-actor-network-theory.html}{\ul{https://study.com/academy/lesson/video/latours-actor-network-theory.html}}
\item
  Actor--network theory - Wikipedia, accessed December 10, 2025,
  \href{https://en.wikipedia.org/wiki/Actor\%E2\%80\%93network_theory}{\ul{https://en.wikipedia.org/wiki/Actor\%E2\%80\%93network\_theory}}
\item
  Latour\textquotesingle s Actor Network Theory - Simply Psychology,
  accessed December 10, 2025,
  \href{https://www.simplypsychology.org/actor-network-theory.html}{\ul{https://www.simplypsychology.org/actor-network-theory.html}}
\item
  towards-posthumanist-design - Wild Pear CIC, accessed December 10,
  2025,
  \href{https://wildpearcic.co.uk/towards-posthumanist-design}{\ul{https://wildpearcic.co.uk/towards-posthumanist-design}}
\item
  Situated Knowledges: The Science Question in Feminism and the
  Privilege of Partial Perspective, accessed December 10, 2025,
  \href{https://commons.princeton.edu/hum583-f21/wp-content/uploads/sites/283/2021/08/Haraway-Situated-Knowledges.pdf}{\ul{https://commons.princeton.edu/hum583-f21/wp-content/uploads/sites/283/2021/08/Haraway-Situated-Knowledges.pdf}}
\item
  Meeting the Universe Halfway: Quantum Physics and the Entanglement of
  Matter and Meaning. By Karen Barad. Durham, N.C., accessed December
  10, 2025,
  \href{https://www.cambridge.org/core/journals/hypatia/article/meeting-the-universe-halfway-quantum-physics-and-the-entanglement-of-matter-and-meaning-by-karen-barad-durham-nc-duke-university-press-2007/5717D38D111DF2ED914359CDCBE6E1E0}{\ul{https://www.cambridge.org/core/journals/hypatia/article/meeting-the-universe-halfway-quantum-physics-and-the-entanglement-of-matter-and-meaning-by-karen-barad-durham-nc-duke-university-press-2007/5717D38D111DF2ED914359CDCBE6E1E0}}
\item
  Methodologies in New Materialism and Discourse Analysis, accessed
  December 10, 2025,
  \href{https://discourseanalyzer.com/methodologies-in-new-materialism-and-discourse-analysis/}{\ul{https://discourseanalyzer.com/methodologies-in-new-materialism-and-discourse-analysis/}}
\item
  Abstract What constitutes \textquotesingle knowledge\textquotesingle{}
  that we acquire, and how do we acquire it? Through Donna
  Haraway\textquotesingle s theoretical recon - IDA, accessed December
  10, 2025,
  \href{https://ida.mtholyoke.edu/server/api/core/bitstreams/473399db-eaaa-42ee-bf98-5e7b50d9aa73/content}{\ul{https://ida.mtholyoke.edu/server/api/core/bitstreams/473399db-eaaa-42ee-bf98-5e7b50d9aa73/content}}
\item
  Scalable Inference with RDMA and Tiered KV Caching \textbar{} by
  Nadeem Khan(NK) \textbar{} LearnWithNK \textbar{} Nov, 2025, accessed
  December 10, 2025,
  \href{https://medium.com/learnwithnk/scalable-inference-with-rdma-and-tiered-kv-caching-9d7e494a863b}{\ul{https://medium.com/learnwithnk/scalable-inference-with-rdma-and-tiered-kv-caching-9d7e494a863b}}
\item
  The Secret Behind Fast LLM Inference: Unlocking the KV Cache - Towards
  AI, accessed December 10, 2025,
  \href{https://pub.towardsai.net/the-secret-behind-fast-llm-inference-unlocking-the-kv-cache-9c13140b632d}{\ul{https://pub.towardsai.net/the-secret-behind-fast-llm-inference-unlocking-the-kv-cache-9c13140b632d}}
\item
  Andrej Karpathy, accessed December 10, 2025,
  \href{https://karpathy.ai/}{\ul{https://karpathy.ai/}}
\item
  Andrej Karpathy on Software 3.0: Software in the Age of AI \textbar{}
  by Ben Pouladian \textbar{} Medium, accessed December 10, 2025,
  \href{https://medium.com/@ben_pouladian/andrej-karpathy-on-software-3-0-software-in-the-age-of-ai-b25533da93b6}{\ul{https://medium.com/@ben\_pouladian/andrej-karpathy-on-software-3-0-software-in-the-age-of-ai-b25533da93b6}}
\item
  Why Language Models Are ``Lost in the Middle'' - Towards AI, accessed
  December 10, 2025,
  \href{https://pub.towardsai.net/why-language-models-are-lost-in-the-middle-629b20d86152}{\ul{https://pub.towardsai.net/why-language-models-are-lost-in-the-middle-629b20d86152}}
\item
  Found in the Middle: How Language Models Use Long Contexts Better via
  Plug-and-Play Positional Encoding - arXiv, accessed December 10, 2025,
  \href{https://arxiv.org/html/2403.04797v1}{\ul{https://arxiv.org/html/2403.04797v1}}
\item
  Never Lost in the Middle: Mastering Long-Context Question Answering
  with Position-Agnostic Decompositional Training - arXiv, accessed
  December 10, 2025,
  \href{https://arxiv.org/html/2311.09198v2}{\ul{https://arxiv.org/html/2311.09198v2}}
\item
  P6: Context Rot -- Hamel\textquotesingle s Blog, accessed December 10,
  2025,
  \href{https://hamel.dev/notes/llm/rag/p6-context_rot.html}{\ul{https://hamel.dev/notes/llm/rag/p6-context\_rot.html}}
\item
  Context-Bench: Benchmarking LLMs on Agentic Context Engineering
  \textbar{} Letta, accessed December 10, 2025,
  \href{https://www.letta.com/blog/context-bench}{\ul{https://www.letta.com/blog/context-bench}}
\item
  Agentic Context Engineering - Sundeep Teki, accessed December 10,
  2025,
  \href{https://www.sundeepteki.org/blog/agentic-context-engineering}{\ul{https://www.sundeepteki.org/blog/agentic-context-engineering}}
\item
  Agentic Context Engineering (ACE) - Emergent Mind, accessed December
  10, 2025,
  \href{https://www.emergentmind.com/topics/agentic-context-engineering-ace}{\ul{https://www.emergentmind.com/topics/agentic-context-engineering-ace}}
\item
  Paper page - Agentic Context Engineering: Evolving Contexts for
  Self-Improving Language Models - Hugging Face, accessed December 10,
  2025,
  \href{https://huggingface.co/papers/2510.04618}{\ul{https://huggingface.co/papers/2510.04618}}
\item
  The Complexity Trap: Simple Observation Masking Is as Efficient as LLM
  Summarization for Agent Context Management - arXiv, accessed December
  10, 2025,
  \href{https://arxiv.org/html/2508.21433v1}{\ul{https://arxiv.org/html/2508.21433v1}}
\item
  {[}2510.04618{]} Agentic Context Engineering: Evolving Contexts for
  Self-Improving Language Models - arXiv, accessed December 10, 2025,
  \href{https://arxiv.org/abs/2510.04618}{\ul{https://arxiv.org/abs/2510.04618}}
\end{enumerate}
