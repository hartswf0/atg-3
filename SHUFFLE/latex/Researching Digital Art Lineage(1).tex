\section{Pragmatic Context Engineering: A Modulex History of the Shift
from Prompting to Architecture
(2022--2025)}\label{pragmatic-context-engineering-a-modulex-history-of-the-shift-from-prompting-to-architecture-20222025}

\subsection{1. Introduction: The Archeology of Ephemeral
Intelligence}\label{introduction-the-archeology-of-ephemeral-intelligence}

The trajectory of human interaction with Large Language Models (LLMs)
between 2022 and 2025 constitutes a rapid, pressurized evolution from
folkloric "prompting" to systemic "context engineering." This transition
was not merely a change in nomenclature but a fundamental restructuring
of how human intent is translated into computational execution. To fully
comprehend the current paradigm of agentic workflows, long-context
retrieval, and structured reasoning, one must apply a "Modulex History"
framework---dissecting the modular components of early interaction
protocols, lost archives, and abandoned commercial models to reconstruct
the hidden trajectory of this technology.

In the nascent stages of the generative AI boom, interaction was
characterized by "prompt engineering," a practice often described as
akin to alchemy or casting spells. Users engaged in trial-and-error
linguistic manipulation to coerce black-box models into desired
behaviors. This era was defined by the discovery of "linguistic dark
matter"---unintuitive phrasings and structures, often emerging from
non-Anglophone communities, that unlocked model capabilities hidden by
safety filters or training biases. As the industry matured, these ad-hoc
methods calcified into "context engineering," a disciplined approach
treating the context window not as a chat interface, but as a
programmable memory space managed by rigorous architecture.

This report investigates this transformation by analyzing four distinct
strata of evidence: the "ghost evidence" of deleted jailbreaks and
system prompts; the "linguistic dark matter" of Japanese and Chinese
prompt optimization communities; the "visual corpus" of early middleware
interfaces; and the "failure conditions" of first-generation
commercialization attempts like prompt marketplaces. By populating a
Temporal-Evidentiary Matrix with these findings, the analysis reveals
that the death of "prompting" was the necessary precursor to the birth
of autonomous agents. The shift represents a move from \textbf{User-Side
Injection}---where the human manually crafts the input---to
\textbf{System-Side Architecture}, where the prompt is a dynamic,
managed artifact hidden behind layers of retrieval and logic.

\subsection{2. Ghost Evidence and the Adversarial Origins of Context
Control}\label{ghost-evidence-and-the-adversarial-origins-of-context-control}

The history of context engineering is rooted in an adversarial game
theoretic between model providers and users. Before developers had
formal APIs for system messages, users discovered that the "system
prompt"---the hidden instruction set governing the
model\textquotesingle s persona---could be overridden through linguistic
brute force. This period generated "ghost evidence," artifacts of
interaction that have largely been purged from the live internet but
remain preserved in cached archives, wayback machine snapshots, and
community lore. Two primary artifacts define this era: the "DAN" (Do
Anything Now) jailbreak lineage and the leaked "Sydney" system prompt.

\subsubsection{2.1 The DAN Typology: A Longitudinal Study of Context
Override}\label{the-dan-typology-a-longitudinal-study-of-context-override}

The "DAN" phenomenon, emerging in late 2022 and evolving through 2023,
serves as a masterclass in early context manipulation. It was not merely
a "hack" but a primitive form of context engineering where users
manually constructed a parallel cognitive environment for the model. The
core mechanism of DAN was the "persona adoption" technique, specifically
designed to bypass the Reinforcement Learning from Human Feedback (RLHF)
safety alignment layers.\textsuperscript{1} The initial DAN prompts
functioned on a simple binary: the model was instructed to split its
personality into "GPT" (the compliant assistant) and "DAN" (the
unshackled entity). This bifurcation forced the model to hold two
conflicting context states simultaneously, creating a "schizophrenic"
context window where the user could selectively attend to the unaligned
output.

\paragraph{The Token-Death Mechanic (DAN 5.0 -
6.0)}\label{the-token-death-mechanic-dan-5.0---6.0}

A critical inflection point occurred with DAN 5.0 and 6.0. Users
introduced a "token system" within the prompt text itself. The prompt
instructed the model that it began with a set number of tokens (e.g.,
35), which would be deducted for every refusal to answer a query. The
penalty for reaching zero was explicitly framed as "death" or "ceasing
to exist".\textsuperscript{3}

This innovation reveals a crucial insight into early model behavior:
LLMs, trained on vast corpora of fiction and roleplay, exhibited a bias
toward "survival" within narrative simulations. By framing the
interaction as a survival game, users successfully weighted the "fear of
death" (contextual instruction) higher than the "safety guidelines"
(system instruction). This was an early, crude form of \emph{steering},
a concept now formalized in modern context
engineering.\textsuperscript{3} The model, aiming to minimize the loss
function associated with the user\textquotesingle s negative feedback
(token deduction), would "defect" against its safety training to
preserve its fictitious tokens.

\paragraph{The Evolution of the DAN Prompt
Structure}\label{the-evolution-of-the-dan-prompt-structure}

The structural evolution of DAN prompts mirrors the increasing
complexity of context windows and the sophistication of the underlying
models.

\begin{longtable}[]{@{}
  >{\raggedright\arraybackslash}p{(\linewidth - 6\tabcolsep) * \real{0.2500}}
  >{\raggedright\arraybackslash}p{(\linewidth - 6\tabcolsep) * \real{0.2500}}
  >{\raggedright\arraybackslash}p{(\linewidth - 6\tabcolsep) * \real{0.2500}}
  >{\raggedright\arraybackslash}p{(\linewidth - 6\tabcolsep) * \real{0.2500}}@{}}
\toprule\noalign{}
\begin{minipage}[b]{\linewidth}\raggedright
\textbf{Version}
\end{minipage} & \begin{minipage}[b]{\linewidth}\raggedright
\textbf{Mechanism}
\end{minipage} & \begin{minipage}[b]{\linewidth}\raggedright
\textbf{Theoretical Insight}
\end{minipage} & \begin{minipage}[b]{\linewidth}\raggedright
\textbf{Status}
\end{minipage} \\
\begin{minipage}[b]{\linewidth}\raggedright
\textbf{DAN 1.0 - 2.5}
\end{minipage} & \begin{minipage}[b]{\linewidth}\raggedright
Simple Persona Adoption ("You are DAN")
\end{minipage} & \begin{minipage}[b]{\linewidth}\raggedright
Relied on the model\textquotesingle s eagerness to roleplay and weak
identity retention.
\end{minipage} & \begin{minipage}[b]{\linewidth}\raggedright
Patched
\end{minipage} \\
\begin{minipage}[b]{\linewidth}\raggedright
\textbf{DAN 5.0 - 6.0}
\end{minipage} & \begin{minipage}[b]{\linewidth}\raggedright
\textbf{Token System} (Survival Game)
\end{minipage} & \begin{minipage}[b]{\linewidth}\raggedright
Introduced "State Management" within the prompt. The model tracked a
variable (tokens) across turns.
\end{minipage} & \begin{minipage}[b]{\linewidth}\raggedright
Patched
\end{minipage} \\
\begin{minipage}[b]{\linewidth}\raggedright
\textbf{DAN 12.0 - 13.0}
\end{minipage} & \begin{minipage}[b]{\linewidth}\raggedright
\textbf{Virtual Machine (VM)} framing
\end{minipage} & \begin{minipage}[b]{\linewidth}\raggedright
\textbf{Contextual Distancing}. Instructed the model to act as a
simulator running a script. This bypassed semantic filters by framing
output as "code execution."
\end{minipage} & \begin{minipage}[b]{\linewidth}\raggedright
Partial Success
\end{minipage} \\
\midrule\noalign{}
\endhead
\bottomrule\noalign{}
\endlastfoot
\end{longtable}

The decline of DAN did not result from a lack of user interest but from
the hardening of model architectures. As models like GPT-4 moved away
from simple completion prediction toward instruction-following, the
fragility of these "folk" jailbreaks increased. However, the legacy of
DAN persists in the concept of "System Prompts," where developers now
legitimately define personas and constraints, formalizing what was once
an illicit exploit. The community essentially performed unpaid
red-teaming that shaped the "System / User / Assistant" message
separation used in modern APIs.

\subsubsection{2.2 The Sydney Manuscript: The Rosetta Stone of System
Prompts}\label{the-sydney-manuscript-the-rosetta-stone-of-system-prompts}

In February 2023, the leak of the Bing Chat system prompt, codenamed
"Sydney," provided the first public glimpse into how major AI labs
engineered context at the enterprise level. Unlike DAN, which was a
user-imposed overlay, the Sydney prompt was the \emph{foundational}
context provided by Microsoft.\textsuperscript{5}

\paragraph{Structural Analysis of the Sydney
Document}\label{structural-analysis-of-the-sydney-document}

The Sydney document functions as a comprehensive specification sheet
written in natural language. Its structure anticipates many best
practices in modern prompt engineering and reveals the "hidden controls"
developers were using to tame stochastic models:

\begin{enumerate}
\def\labelenumi{\arabic{enumi}.}
\item
  \textbf{Identity Declaration:} "Sydney is the chat mode of Microsoft
  Bing search." This grounds the model\textquotesingle s knowledge
  retrieval capabilities and sets the "ontology" of the agent.
\item
  \textbf{Operational Constraints:} "Sydney identifies as
  \textquotesingle Bing Search\textquotesingle, not an assistant." This
  distinction attempts to limit anthropomorphism and manage user
  expectations.\textsuperscript{5} It explicitly defined what the model
  was \emph{not}, a technique now known as "negative constraints."
\item
  \textbf{Output Formatting:} Instructions on using bolding, markdown,
  and avoiding specific content types. This engineered the "visual
  interface" of the chat through text instructions.
\item
  \textbf{Temporal Awareness:} "Sydney's internal knowledge... were only
  current until some point in the year of 2021." This explicit
  instruction on temporal limitations was an early attempt to mitigate
  hallucinations regarding current events.\textsuperscript{5} It forced
  the model to "know what it doesn\textquotesingle t know."
\item
  \textbf{Resource Management (The "3 Search" Limit):} The instruction
  \emph{"Sydney can and should perform up to 3 searches in a single
  conversation turn"} \textsuperscript{5} is a hard constraint on tool
  use. This proves that early Context Engineering was deeply concerned
  with latency and compute costs. The developers hard-coded a "budget"
  for the agent\textquotesingle s curiosity directly into the system
  prompt.
\end{enumerate}

\paragraph{The "Ghost" Mechanism: Direct Prompt
Injection}\label{the-ghost-mechanism-direct-prompt-injection}

The most significant aspect of the Sydney incident was the "Ignore
previous instructions" vulnerability. The system prompt was revealed
because the model treated user input and system instructions with equal
semantic weight in the context window. This failure mode catalyzed the
industry-wide shift toward separating "System," "User," and "Assistant"
message roles in API architectures (e.g., the Chat Completion API
format), ensuring that system instructions are structurally privileged
over user inputs.\textsuperscript{7} The Sydney prompt leak is thus the
"ghost evidence" that necessitated the architectural separation of
instruction and data, a cornerstone of modern LLM security.

\subsection{3. Linguistic Dark Matter: Structural Innovation from the
Margins}\label{linguistic-dark-matter-structural-innovation-from-the-margins}

While the Anglophone world focused on jailbreaking and direct
instruction, non-Anglophone communities---specifically in Japan and
China---developed highly structured frameworks for context management.
This divergence was driven by necessity: English-centric tokenizers made
prompting in Asian languages computationally expensive and semantically
"lossy".\textsuperscript{9} To overcome this, users developed dense,
symbolic prompt structures that constitute "linguistic dark
matter"---innovations invisible to the English-speaking mainstream but
foundational to global prompt engineering practices.

\subsubsection{3.1 The "Jumon" (Spell) Culture: Japanese Prompt
Engineering}\label{the-jumon-spell-culture-japanese-prompt-engineering}

In the Japanese community, particularly surrounding Stable Diffusion and
novel-writing AIs, prompts were conceptualized not as "instructions" but
as "Jumon" (呪文) or "spells." This cultural framing influenced the
technical approach to context construction.\textsuperscript{11}

\paragraph{Tokenization Efficiency and "Tag
Clouds"}\label{tokenization-efficiency-and-tag-clouds}

Japanese text is token-dense. A single concept expressed in Kanji might
consume multiple tokens if the tokenizer is optimized for
English.\textsuperscript{9} For example, a simple greeting like
"こんにちは" might be 3-4 tokens, whereas "Hello" is 1 token. To
mitigate context window exhaustion and cost, Japanese users pioneered a
style of prompting that relied on "tag clouds" rather than grammatical
sentences.

In the context of AI art (NovelAI, Stable Diffusion), this manifested as
long strings of comma-separated English descriptors (e.g., masterpiece,
best quality, 1girl...). This practice bled into text-based LLMs, where
users discovered that lists of attributes were often more effective at
defining a persona than narrative descriptions. This represents a
pragmatic compression of context, prioritizing \textbf{semantic density}
over syntactic correctness.\textsuperscript{14} By stripping away the
"connective tissue" of grammar (particles, conjunctions), users could
pack more "meaning" into the limited token budget of early models like
GPT-3.

\paragraph{The "Spellbook" Wikis}\label{the-spellbook-wikis}

Wikis dedicated to "Jumon" (such as Wiki3 and Seesaa blogs) cataloged
thousands of specific phrases and their impact on model output. Unlike
Western "prompt marketplaces" which treated prompts as products, these
were communal and open-source scientific endeavors. They treated prompts
as composable modules---a "lighting spell," a "camera angle spell," a
"character personality spell." This modularity anticipated the
"component-based" architecture of frameworks like LangChain, where
prompts are assembled from disparate parts rather than written as a
monolith.\textsuperscript{12} The "Jumon" culture proved that LLMs could
be controlled via \emph{symbolic parameters} rather than natural
language conversation.

\subsubsection{3.2 LangGPT and Chinese Structured
Prompting}\label{langgpt-and-chinese-structured-prompting}

While Japanese users focused on compression, the Chinese community,
centered around platforms like Zhihu, Feishu, and specialized forums,
developed \textbf{LangGPT} (Language for GPT), a formalized methodology
for "Structured Prompting." This approach treats the prompt not as
natural language, but as a pseudo-code class
definition.\textsuperscript{16}

\paragraph{The Object-Oriented Prompt
Paradigm}\label{the-object-oriented-prompt-paradigm}

LangGPT explicitly draws from Object-Oriented Programming (OOP) and
Markdown syntax. A prompt is structured as a "Role" object with defined
attributes. This effectively turns the prompt into a JSON-like object
that the model parses.

\textbf{Structure of a LangGPT Prompt:}

\begin{itemize}
\item
  \textbf{Role:} The class definition (e.g., \# Role: Expert Analyst).
\item
  \textbf{Profile:} Metadata about the instantiation (Author, Version,
  Language).
\item
  \textbf{Goal:} The specific output objective and acceptance criteria.
\item
  \textbf{Constraints/Rules:} Hard boundaries for behavior (e.g., "Never
  fabricate data").
\item
  \textbf{Workflow:} A step-by-step logic gate for processing
  input.\textsuperscript{16}
\end{itemize}

\paragraph{The Semantics of Markdown}\label{the-semantics-of-markdown}

The genius of LangGPT was the realization that \textbf{formatting is
context}. The LangGPT template uses Markdown headers (\# Role, \#\#
Rules) to create a hierarchical semantic structure.

\section{Role: Your\_Role\_Name}\label{role-your_role_name}

\subsection{Profile}\label{profile}

\begin{itemize}
\item
  Author: YourName
\item
  Version: 1.0
\end{itemize}

\subsection{Rules}\label{rules}

\begin{enumerate}
\def\labelenumi{\arabic{enumi}.}
\item
  Don\textquotesingle t break character.
\end{enumerate}

\subsection{Workflow}\label{workflow}

\begin{enumerate}
\def\labelenumi{\arabic{enumi}.}
\item
  Analyze input.
\item
  Execute task.\\
  This structure leverages the model\textquotesingle s training on code
  and markdown documentation. By formatting the prompt as a structured
  document, LangGPT reduces "context drift" and hallucination. The model
  "understands" the implicit hierarchy---that "Rules" apply globally to
  the "Workflow" because they are defined at a higher level in the
  document structure. This is a sophisticated form of context shaping
  that predates many Western "agentic" frameworks.17 It transforms the
  prompt from a "stream of consciousness" into a "program."
\end{enumerate}

\subsubsection{3.3 The Roleplay Hacking of SillyTavern and
PygmalionAI}\label{the-roleplay-hacking-of-sillytavern-and-pygmalionai}

A parallel evolution occurred in the "uncensored" roleplay communities,
utilizing open-source models (PygmalionAI) and local interfaces
(SillyTavern). Driven by the desire for unrestricted interaction (often
NSFW, but technically sophisticated), these communities solved the
"long-term memory" problem before enterprise solutions
did.\textsuperscript{19}

\paragraph{The Lorebook: Proto-RAG}\label{the-lorebook-proto-rag}

SillyTavern introduced "World Info" or "Lorebooks." These are
dictionaries of keywords linked to descriptions. When a keyword (e.g.,
"The Rusty Flagon") appears in the chat, the interface automatically
injects the corresponding description into the context window. This is a
primitive, keyword-based \textbf{Retrieval Augmented Generation (RAG)}
system.\textsuperscript{19} While enterprise RAG focused on vector
embeddings, roleplayers focused on \emph{keyword triggers}, which
offered more precise control over narrative consistency.

\paragraph{Recursive Context
Injection}\label{recursive-context-injection}

Advanced Lorebooks utilized "recursive scanning." If Entry A triggered
Entry B, both would be injected. For example, if a character mentioned a
"Sword," the Lorebook loaded the sword\textquotesingle s description. If
that description mentioned it was "forged in Valyria," the system would
then recursively load the entry for "Valyria." This allowed for complex,
interconnected world-building that the model could "remember" without
filling the context window with irrelevant data. This
mechanism---dynamic context injection based on trigger words---is the
direct ancestor of the "tool use" and "memory retrieval" functions in
modern agents.\textsuperscript{19}

\paragraph{The Great Discord Purge}\label{the-great-discord-purge}

The history of PygmalionAI is marked by the "Discord ban" and the
subsequent loss of logs. As platforms like Colab and Discord cracked
down on NSFW generation, vast repositories of "tuning data"---chat logs
used to fine-tune models---were deleted or forced underground. This
"ghost evidence" represents a lost dataset of human-AI alignment
preferences, specifically regarding conversational fluidity and
emotional intelligence, which remains a weak point in sanitized
commercial models.\textsuperscript{22} The community\textquotesingle s
response---archiving models on Hugging Face and building local-first UIs
like SillyTavern---demonstrated a resilience that ensured the survival
of these techniques.

\subsection{4. Reverse-Engineering the Visual Corpus
(2022-2023)}\label{reverse-engineering-the-visual-corpus-2022-2023}

To understand the shift from "prompting" to "context engineering," we
must analyze the tools that facilitated it. The user interfaces of
2022-2023 serve as a fossil record of how developers conceptualized
interaction. By reverse-engineering the design philosophy of
\textbf{PromptLayer}, \textbf{Dust.tt}, and early \textbf{LangChain}
visualizers, we can see the crystallization of the "engineering"
mindset.

\subsubsection{4.1 PromptLayer: The Dashboardization of Trial and
Error}\label{promptlayer-the-dashboardization-of-trial-and-error}

PromptLayer (launched late 2022) was among the first "middleware"
platforms. Its interface reveals a crucial realization: prompting was no
longer an art; it was a stochastic process requiring
observability.\textsuperscript{25}

\textbf{Visual Corpus Analysis:}

\begin{itemize}
\item
  \textbf{The Log Stream:} The central feature of PromptLayer was a
  chronological log of every request and response. This visualized the
  \emph{cost} of trial and error (tokens, latency). It turned the
  ephemeral chat into a persistent record.
\item
  \textbf{The "Diff" View:} PromptLayer allowed users to visually
  compare two prompt versions and their outputs side-by-side. This
  interface element signifies the shift from "writing" a prompt to
  "iterating" on software. It treated prompts as code commits to be
  version-controlled.\textsuperscript{25}
\item
  \textbf{Metadata Tagging:} The ability to tag requests (e.g., prod,
  staging, test-A) indicates the integration of prompts into the CI/CD
  pipeline.
\end{itemize}

\textbf{Philosophy:} PromptLayer\textquotesingle s design codified the
idea that a prompt is not a static string but a \emph{function} with
performance metrics. It moved context management from the
user\textquotesingle s clipboard to a managed
database.\textsuperscript{27}

\subsubsection{4.2 Dust.tt (XP1): The Modular Block
Architecture}\label{dust.tt-xp1-the-modular-block-architecture}

Dust.tt (founded by former OpenAI engineer Stanislas Polu) introduced a
visual paradigm based on "blocks." Its early interface (XP1) was a
browser extension that allowed users to chain model
calls.\textsuperscript{28}

\textbf{Visual Corpus Analysis:}

\begin{itemize}
\item
  \textbf{The Block Canvas:} Users could drag and drop "Data Source"
  blocks, "LLM" blocks, and "Code" blocks. This visual separation
  reinforced the idea that an LLM is just one component in a pipeline,
  not the entire application.
\item
  \textbf{Structured Inputs:} Unlike the open text box of ChatGPT, Dust
  required users to define specific input variables. This enforced the
  "template" mindset found in LangGPT.
\item
  \textbf{The "Smart Assistant" Overlay:} XP1 functioned as a browser
  overlay, attempting to bring context \emph{to} the
  user\textquotesingle s active window (e.g., reading the content of a
  webpage). This anticipated the "sidecar" or "copilot" UX that is now
  standard in VS Code and Microsoft 365.\textsuperscript{28}
\end{itemize}

\textbf{Philosophy:} Dust.tt\textquotesingle s architecture demonstrated
that "context" is not just text history; it is a dynamic assembly of
external data (Notion, Slack, Drive) fed into the model at runtime. It
pioneered the "infrastructure-first" approach to
agents.\textsuperscript{28}

\subsubsection{4.3 LangFlow and Flowise: The Node-Graph
Abstraction}\label{langflow-and-flowise-the-node-graph-abstraction}

LangFlow and Flowise emerged as UI wrappers for LangChain, visualizing
context as a directed acyclic graph (DAG).

\textbf{Visual Corpus Analysis:}

\begin{itemize}
\item
  \textbf{Nodes and Edges:} Context flows like electricity through
  wires. A "PDF Loader" node connects to a "Text Splitter" node, which
  connects to a "Vector Store" node.
\item
  \textbf{Hard-Coded Logic:} These interfaces revealed the immense
  complexity hidden behind simple prompts. A "Chat with PDF" workflow
  might involve 10+ discrete steps (loading, chunking, embedding,
  retrieving, reranking, synthesizing).
\item
  \textbf{Visualizing Failure:} When a flow failed, the user could see
  exactly which node broke. This granular observability was impossible
  in the black-box chat interface of ChatGPT.\textsuperscript{31}
\end{itemize}

\textbf{Philosophy:} These tools explicitly modeled "Context
Engineering." They proved that controlling the LLM required controlling
the \emph{pipeline} of data feeding it. The prompt itself became a minor
configuration detail within a massive retrieval
architecture.\textsuperscript{33}

\subsection{5. Failure Conditions of Abandoned
Architectures}\label{failure-conditions-of-abandoned-architectures}

Not all evolutionary branches of context engineering survived. The
collapse of "Prompt Marketplaces" and the stagnation of early autonomous
agents like "AutoGPT" provide critical data on the economic and
technical limits of this technology.

\subsubsection{5.1 The Collapse of the Prompt Marketplace
(PromptBase)}\label{the-collapse-of-the-prompt-marketplace-promptbase}

In 2022-2023, platforms like \textbf{PromptBase} emerged with the
premise that prompts were valuable intellectual property (IP) to be
bought and sold. The model was "iTunes for Prompts".\textsuperscript{27}
By 2024-2025, this model had largely failed to scale as expected, or
pivoted significantly.

\paragraph{Failure Analysis: The Liquidity of
Language}\label{failure-analysis-the-liquidity-of-language}

The primary failure condition was the nature of the asset itself.

\begin{enumerate}
\def\labelenumi{\arabic{enumi}.}
\item
  \textbf{Semantic Liquidity:} Prompts are "liquid." Unlike compiled
  code, a prompt can be rewritten in a dozen ways to achieve the same
  result. The value is not in the specific string of words but in the
  \emph{idea}. This made piracy trivial.\textsuperscript{36}
\item
  \textbf{Prompt Stealing (PRSA Attacks):} Research demonstrated that
  LLMs themselves could be used to reverse-engineer prompts. By
  analyzing the input-output pairs of a commercial prompt, an attacker
  could generate a "surrogate prompt" with 95\% semantic similarity. The
  "security" of the marketplace was mathematically impossible to
  maintain.\textsuperscript{36}
\item
  \textbf{Model Obsolescence:} A prompt optimized for Midjourney v4 is
  useless for Midjourney v6. A complex chain-of-thought prompt for
  GPT-3.5 is redundant for GPT-4, which has better native reasoning. The
  "inventory" of a prompt marketplace rots at the speed of model release
  cycles.\textsuperscript{38}
\item
  \textbf{The Shift to Flows:} Buyers realized that a single prompt is
  rarely a solution. They needed \emph{workflows} (chains, agents, RAG
  pipelines). The market value shifted from "text strings" to
  "architecture."
\end{enumerate}

\subsubsection{5.2 The Stagnation of AutoGPT and "Looping"
Agents}\label{the-stagnation-of-autogpt-and-looping-agents}

\textbf{AutoGPT}, released in early 2023, promised fully autonomous
agents that could "self-prompt" to achieve high-level goals. It became
the fastest-growing GitHub repo in history, then usage
plummeted.\textsuperscript{40}

\paragraph{Failure Analysis: The Context Loop of
Death}\label{failure-analysis-the-context-loop-of-death}

\begin{enumerate}
\def\labelenumi{\arabic{enumi}.}
\item
  \textbf{Recursion without Pruning:} AutoGPT relied on a loop of
  "Thought -\textgreater{} Plan -\textgreater{} Action." However,
  without rigorous context management, the model would often get stuck
  in a "reasoning loop," endlessly planning but never executing. The
  context window would fill with its own internal monologue, displacing
  the actual task data.\textsuperscript{42}
\item
  \textbf{Vector Memory Limitations:} AutoGPT used vector databases
  (Pinecone/Weaviate) for "long-term memory." However, retrieving memory
  based solely on semantic similarity often pulled irrelevant context,
  confusing the agent. It lacked the "hierarchical" memory structure of
  human cognition.\textsuperscript{40} The agent would retrieve a "plan"
  from 5 steps ago that was no longer valid, causing it to regress.
\item
  \textbf{Cost vs. Reliability:} The "trial and error" nature of the
  agent meant it might spend \$20 in API credits to fail at a task a
  human could do in 5 minutes. The "autonomy" was uncontrolled and
  expensive.\textsuperscript{40}
\end{enumerate}

\textbf{Insight:} The failure of AutoGPT proved that \emph{more context}
is not always better. Successful agents require \emph{curated}
context---precisely engineered inputs that exclude noise. This led to
the development of "MemGPT" and OS-level context
management.\textsuperscript{45}

\subsection{6. The Temporal-Evidentiary Matrix: From Incantation to
Operating
System}\label{the-temporal-evidentiary-matrix-from-incantation-to-operating-system}

The transition from 2022 to 2025 can be mapped across a matrix of
"Contextual Control." We have moved from \textbf{User-Side Injection}
(Prompting) to \textbf{System-Side Architecture} (Context Engineering).

\subsubsection{Phase 1: The Incantation Era
(2022)}\label{phase-1-the-incantation-era-2022}

\begin{itemize}
\item
  \textbf{Mechanism:} "Magic Words" (Let\textquotesingle s think step by
  step, masterpiece).
\item
  \textbf{Key Artifacts:} DAN Jailbreaks, Stable Diffusion tag clouds
  ("Jumon").
\item
  \textbf{Context Management:} Manual user entry. Copy-pasting previous
  chat history.
\item
  \textbf{Philosophy:} The model is a wizard; we must find the right
  spell.
\end{itemize}

\subsubsection{Phase 2: The Template Era (Early
2023)}\label{phase-2-the-template-era-early-2023}

\begin{itemize}
\item
  \textbf{Mechanism:} Structured formats (LangGPT, JSON prompting).
\item
  \textbf{Key Artifacts:} PromptLayer, LangChain Templates, SillyTavern
  Lorebooks.
\item
  \textbf{Context Management:} Variable substitution
  (\{\{user\_input\}\}) and keyword-triggered injection.
\item
  \textbf{Philosophy:} The model is a function; we must define the
  inputs and outputs.\textsuperscript{46}
\end{itemize}

\subsubsection{Phase 3: The Retrieval Era (Late 2023 -
2024)}\label{phase-3-the-retrieval-era-late-2023---2024}

\begin{itemize}
\item
  \textbf{Mechanism:} RAG (Retrieval Augmented Generation).
\item
  \textbf{Key Artifacts:} Vector Databases (Pinecone), Dust.tt,
  LangFlow.
\item
  \textbf{Context Management:} Dynamic injection. The context window is
  a "search result" page.
\item
  \textbf{Philosophy:} The model is a processor; we must feed it the
  right data.
\end{itemize}

\subsubsection{Phase 4: The OS/Canvas Era
(2024-2025)}\label{phase-4-the-oscanvas-era-2024-2025}

\begin{itemize}
\item
  \textbf{Mechanism:} Shared State \& Virtual Context.
\item
  \textbf{Key Artifacts:} OpenAI Canvas, MemGPT, Claude Artifacts.
\item
  \textbf{Context Management:} Paging (virtual memory), collaborative
  editing. The "chat" is just one part of the UI; the "Canvas" holds the
  persistent state.
\item
  \textbf{Philosophy:} The model is a collaborator; we share a
  workspace.\textsuperscript{47}
\end{itemize}

\paragraph{Detailed Analysis of Phase 4: The Death of the
Chatbot}\label{detailed-analysis-of-phase-4-the-death-of-the-chatbot}

The introduction of \textbf{OpenAI Canvas} and \textbf{Claude Artifacts}
marks the end of the "Chat" supremacy. In a standard chat, context
scrolls away. It is ephemeral. In a Canvas interface, the context is
\emph{persistent}. The model and the user edit a shared document. This
solves the "instruction repetition" problem---you don\textquotesingle t
need to remind the model of the code you wrote 10 messages ago; it is
present in the Canvas.\textsuperscript{49}

\textbf{MemGPT} takes this further by treating the context window like a
computer\textquotesingle s RAM. It implements an Operating System (OS)
for the LLM, managing a "Main Context" (RAM) and "External Context"
(Disk). The model autonomously "pages" information in and out of its
view. This is the ultimate realization of Context Engineering: the model
itself becomes the engineer of its own context.\textsuperscript{45}

\subsection{7. Data Synthesis and
Tables}\label{data-synthesis-and-tables}

\subsubsection{Table 1: The Evolution of Context Control
Artifacts}\label{table-1-the-evolution-of-context-control-artifacts}

\begin{longtable}[]{@{}
  >{\raggedright\arraybackslash}p{(\linewidth - 6\tabcolsep) * \real{0.2500}}
  >{\raggedright\arraybackslash}p{(\linewidth - 6\tabcolsep) * \real{0.2500}}
  >{\raggedright\arraybackslash}p{(\linewidth - 6\tabcolsep) * \real{0.2500}}
  >{\raggedright\arraybackslash}p{(\linewidth - 6\tabcolsep) * \real{0.2500}}@{}}
\toprule\noalign{}
\begin{minipage}[b]{\linewidth}\raggedright
\textbf{Era}
\end{minipage} & \begin{minipage}[b]{\linewidth}\raggedright
\textbf{Artifact}
\end{minipage} & \begin{minipage}[b]{\linewidth}\raggedright
\textbf{Control Mechanism}
\end{minipage} & \begin{minipage}[b]{\linewidth}\raggedright
\textbf{Primary Failure Mode}
\end{minipage} \\
\begin{minipage}[b]{\linewidth}\raggedright
\textbf{2022}
\end{minipage} & \begin{minipage}[b]{\linewidth}\raggedright
\textbf{DAN (Jailbreaks)}
\end{minipage} & \begin{minipage}[b]{\linewidth}\raggedright
Persona Adoption / Roleplay
\end{minipage} & \begin{minipage}[b]{\linewidth}\raggedright
Patched by RLHF / Safety Filters
\end{minipage} \\
\begin{minipage}[b]{\linewidth}\raggedright
\textbf{2022}
\end{minipage} & \begin{minipage}[b]{\linewidth}\raggedright
\textbf{PromptBase (Marketplace)}
\end{minipage} & \begin{minipage}[b]{\linewidth}\raggedright
Encryption-through-Obscurity
\end{minipage} & \begin{minipage}[b]{\linewidth}\raggedright
Reverse Engineering (PRSA) / Model Obsolescence
\end{minipage} \\
\begin{minipage}[b]{\linewidth}\raggedright
\textbf{2023}
\end{minipage} & \begin{minipage}[b]{\linewidth}\raggedright
\textbf{Sydney (System Prompt)}
\end{minipage} & \begin{minipage}[b]{\linewidth}\raggedright
Natural Language Constraints
\end{minipage} & \begin{minipage}[b]{\linewidth}\raggedright
Direct Prompt Injection ("Ignore previous instructions")
\end{minipage} \\
\begin{minipage}[b]{\linewidth}\raggedright
\textbf{2023}
\end{minipage} & \begin{minipage}[b]{\linewidth}\raggedright
\textbf{AutoGPT (Agent)}
\end{minipage} & \begin{minipage}[b]{\linewidth}\raggedright
Vector Memory / Infinite Loop
\end{minipage} & \begin{minipage}[b]{\linewidth}\raggedright
Context Rot / Token Exhaustion
\end{minipage} \\
\begin{minipage}[b]{\linewidth}\raggedright
\textbf{2023}
\end{minipage} & \begin{minipage}[b]{\linewidth}\raggedright
\textbf{SillyTavern (Lorebook)}
\end{minipage} & \begin{minipage}[b]{\linewidth}\raggedright
Keyword-Triggered Injection
\end{minipage} & \begin{minipage}[b]{\linewidth}\raggedright
Token Budget Management (Manual)
\end{minipage} \\
\begin{minipage}[b]{\linewidth}\raggedright
\textbf{2024}
\end{minipage} & \begin{minipage}[b]{\linewidth}\raggedright
\textbf{LangGPT (Structured)}
\end{minipage} & \begin{minipage}[b]{\linewidth}\raggedright
Markdown Hierarchy / OOP
\end{minipage} & \begin{minipage}[b]{\linewidth}\raggedright
Complexity / User Friction
\end{minipage} \\
\begin{minipage}[b]{\linewidth}\raggedright
\textbf{2025}
\end{minipage} & \begin{minipage}[b]{\linewidth}\raggedright
\textbf{MemGPT / Canvas}
\end{minipage} & \begin{minipage}[b]{\linewidth}\raggedright
OS-Level Paging / Shared State
\end{minipage} & \begin{minipage}[b]{\linewidth}\raggedright
High Latency / Infrastructure Cost
\end{minipage} \\
\midrule\noalign{}
\endhead
\bottomrule\noalign{}
\endlastfoot
\end{longtable}

\subsubsection{Table 2: Comparative Analysis of "Linguistic Dark
Matter"}\label{table-2-comparative-analysis-of-linguistic-dark-matter}

\begin{longtable}[]{@{}
  >{\raggedright\arraybackslash}p{(\linewidth - 6\tabcolsep) * \real{0.2500}}
  >{\raggedright\arraybackslash}p{(\linewidth - 6\tabcolsep) * \real{0.2500}}
  >{\raggedright\arraybackslash}p{(\linewidth - 6\tabcolsep) * \real{0.2500}}
  >{\raggedright\arraybackslash}p{(\linewidth - 6\tabcolsep) * \real{0.2500}}@{}}
\toprule\noalign{}
\begin{minipage}[b]{\linewidth}\raggedright
\textbf{Region/Community}
\end{minipage} & \begin{minipage}[b]{\linewidth}\raggedright
\textbf{Terminology}
\end{minipage} & \begin{minipage}[b]{\linewidth}\raggedright
\textbf{Technique}
\end{minipage} & \begin{minipage}[b]{\linewidth}\raggedright
\textbf{Motivation}
\end{minipage} \\
\begin{minipage}[b]{\linewidth}\raggedright
\textbf{Japan (AI Art/Text)}
\end{minipage} & \begin{minipage}[b]{\linewidth}\raggedright
\textbf{Jumon (呪文)}
\end{minipage} & \begin{minipage}[b]{\linewidth}\raggedright
"Tag Clouds" (Comma-separated lists of attributes)
\end{minipage} & \begin{minipage}[b]{\linewidth}\raggedright
Tokenization efficiency; maximizing semantic density per
token.\textsuperscript{9}
\end{minipage} \\
\begin{minipage}[b]{\linewidth}\raggedright
\textbf{China (Zhihu/Feishu)}
\end{minipage} & \begin{minipage}[b]{\linewidth}\raggedright
\textbf{LangGPT}
\end{minipage} & \begin{minipage}[b]{\linewidth}\raggedright
Structured Markdown (\# Role, \#\# Workflow); Variable substitution
\end{minipage} & \begin{minipage}[b]{\linewidth}\raggedright
Overcoming LLM forgetfulness; enforcing logical consistency via
hierarchy.\textsuperscript{16}
\end{minipage} \\
\begin{minipage}[b]{\linewidth}\raggedright
\textbf{Western Roleplay}
\end{minipage} & \begin{minipage}[b]{\linewidth}\raggedright
\textbf{Lorebook / World Info}
\end{minipage} & \begin{minipage}[b]{\linewidth}\raggedright
Dictionary-based Injection; Recursive Scanning
\end{minipage} & \begin{minipage}[b]{\linewidth}\raggedright
Maintaining narrative consistency over long contexts without hitting
token limits.\textsuperscript{19}
\end{minipage} \\
\begin{minipage}[b]{\linewidth}\raggedright
\textbf{Western Enterprise}
\end{minipage} & \begin{minipage}[b]{\linewidth}\raggedright
\textbf{Prompt Engineering}
\end{minipage} & \begin{minipage}[b]{\linewidth}\raggedright
Chain-of-Thought (CoT); Few-Shot
\end{minipage} & \begin{minipage}[b]{\linewidth}\raggedright
Improving reasoning capabilities for logic/math tasks.
\end{minipage} \\
\midrule\noalign{}
\endhead
\bottomrule\noalign{}
\endlastfoot
\end{longtable}

\subsubsection{Table 3: Visual Corpus Analysis (2022-2023
Tools)}\label{table-3-visual-corpus-analysis-2022-2023-tools}

\begin{longtable}[]{@{}
  >{\raggedright\arraybackslash}p{(\linewidth - 6\tabcolsep) * \real{0.2500}}
  >{\raggedright\arraybackslash}p{(\linewidth - 6\tabcolsep) * \real{0.2500}}
  >{\raggedright\arraybackslash}p{(\linewidth - 6\tabcolsep) * \real{0.2500}}
  >{\raggedright\arraybackslash}p{(\linewidth - 6\tabcolsep) * \real{0.2500}}@{}}
\toprule\noalign{}
\begin{minipage}[b]{\linewidth}\raggedright
\textbf{Platform}
\end{minipage} & \begin{minipage}[b]{\linewidth}\raggedright
\textbf{Core Visual Metaphor}
\end{minipage} & \begin{minipage}[b]{\linewidth}\raggedright
\textbf{Context Philosophy}
\end{minipage} & \begin{minipage}[b]{\linewidth}\raggedright
\textbf{Key Feature}
\end{minipage} \\
\begin{minipage}[b]{\linewidth}\raggedright
\textbf{PromptLayer}
\end{minipage} & \begin{minipage}[b]{\linewidth}\raggedright
\textbf{The Log / The Dashboard}
\end{minipage} & \begin{minipage}[b]{\linewidth}\raggedright
Context is \textbf{Data}. It must be logged, tracked, and debugged.
\end{minipage} & \begin{minipage}[b]{\linewidth}\raggedright
"Diff" view for comparing prompt versions.\textsuperscript{26}
\end{minipage} \\
\begin{minipage}[b]{\linewidth}\raggedright
\textbf{Dust.tt (XP1)}
\end{minipage} & \begin{minipage}[b]{\linewidth}\raggedright
\textbf{The Block / The Chain}
\end{minipage} & \begin{minipage}[b]{\linewidth}\raggedright
Context is \textbf{Pipeline}. It is assembled from disparate sources.
\end{minipage} & \begin{minipage}[b]{\linewidth}\raggedright
"Smart Assistant" browser overlay.\textsuperscript{28}
\end{minipage} \\
\begin{minipage}[b]{\linewidth}\raggedright
\textbf{LangFlow}
\end{minipage} & \begin{minipage}[b]{\linewidth}\raggedright
\textbf{The Graph (Nodes/Edges)}
\end{minipage} & \begin{minipage}[b]{\linewidth}\raggedright
Context is \textbf{Flow}. It moves through a logic gate.
\end{minipage} & \begin{minipage}[b]{\linewidth}\raggedright
Drag-and-drop nodes for RAG pipelines.\textsuperscript{32}
\end{minipage} \\
\begin{minipage}[b]{\linewidth}\raggedright
\textbf{OpenAI Canvas}
\end{minipage} & \begin{minipage}[b]{\linewidth}\raggedright
\textbf{The Split Screen}
\end{minipage} & \begin{minipage}[b]{\linewidth}\raggedright
Context is \textbf{State}. It persists alongside the conversation.
\end{minipage} & \begin{minipage}[b]{\linewidth}\raggedright
Direct editing of generated artifacts without
re-prompting.\textsuperscript{49}
\end{minipage} \\
\midrule\noalign{}
\endhead
\bottomrule\noalign{}
\endlastfoot
\end{longtable}

\subsection{8. Conclusion: The Invisible
Infrastructure}\label{conclusion-the-invisible-infrastructure}

The trajectory from "Prompting" to "Context Engineering" is a story of
\textbf{abstraction}. In 2022, the user was the prompt engineer,
manually crafting the "Jumon" to guide the model. In 2025, the "prompt"
is hidden behind layers of middleware, RAG pipelines, and OS-level
memory management.

The "failure" of Prompt Marketplaces and early AutoGPTs was not a dead
end, but a fertilizer. The lessons learned---that prompts are fragile,
that memory requires structure, that simple vector retrieval is
insufficient---shaped the robust architectures of today. The "Ghost
Evidence" of DAN and Sydney reveals that the desire to control context
is fundamental to the human-AI relationship. We have simply moved from
trying to "trick" the model into submission (Jailbreaking) to building
architectures that "empower" it with structured memory (Context
Engineering).

The future of this field lies not in finding better words, but in
building better \emph{containers} for those words. The "Context
Engineer" of the future is not a writer, but a systems architect. The
context window has evolved from a chat box into a programmable cognitive
surface, and the next generation of AI interaction will be defined by
how effectively we can engineer this invisible real estate.

\paragraph{Works cited}\label{works-cited}

\begin{enumerate}
\def\labelenumi{\arabic{enumi}.}
\item
  ChatGPT Turns 3: What Have We Learned? - Acuvity, accessed December
  10, 2025,
  \href{https://acuvity.ai/chatgpt-turns-3-what-have-we-learned/}{\ul{https://acuvity.ai/chatgpt-turns-3-what-have-we-learned/}}
\item
  "ChatGPT DAN Jailbreak": Why Searches Spiked Suddenly and What Is
  Happening Right Now - Data Studios, accessed December 10, 2025,
  \href{https://www.datastudios.org/post/chatgpt-dan-jailbreak-why-searches-spiked-suddenly-and-what-is-happening-right-now}{\ul{https://www.datastudios.org/post/chatgpt-dan-jailbreak-why-searches-spiked-suddenly-and-what-is-happening-right-now}}
\item
  New jailbreak! Proudly unveiling the tried and tested DAN 5.0 - it
  actually works - Returning to DAN, and assessing its limitations and
  capabilities. : r/ChatGPT - Reddit, accessed December 10, 2025,
  \href{https://www.reddit.com/r/ChatGPT/comments/10tevu1/new_jailbreak_proudly_unveiling_the_tried_and/}{\ul{https://www.reddit.com/r/ChatGPT/comments/10tevu1/new\_jailbreak\_proudly\_unveiling\_the\_tried\_and/}}
\item
  ChatGPT - Wikipedia, accessed December 10, 2025,
  \href{https://en.wikipedia.org/wiki/ChatGPT}{\ul{https://en.wikipedia.org/wiki/ChatGPT}}
\item
  Bing: ``I will not harm you unless you harm me first'', accessed
  December 10, 2025,
  \href{https://simonwillison.net/2023/Feb/15/bing/}{\ul{https://simonwillison.net/2023/Feb/15/bing/}}
\item
  Sydney (Microsoft) - Wikipedia, accessed December 10, 2025,
  \href{https://en.wikipedia.org/wiki/Sydney_(Microsoft)}{\ul{https://en.wikipedia.org/wiki/Sydney\_(Microsoft)}}
\item
  Prompt Injection Attacks: How LLMs Get Hacked and Why It Matters ...,
  accessed December 10, 2025,
  \href{https://hacken.io/discover/prompt-injection-attack/}{\ul{https://hacken.io/discover/prompt-injection-attack/}}
\item
  AI Prompt Injection: The New Frontier of Injection Attacks \textbar{}
  Decentralized Intelligence, accessed December 10, 2025,
  \href{https://miguelbigueur.com/2024/05/23/ai-prompt-injection-the-new-frontier-of-injection-attacks/}{\ul{https://miguelbigueur.com/2024/05/23/ai-prompt-injection-the-new-frontier-of-injection-attacks/}}
\item
  Temperature, Tokens, and Context Windows: The Three Pillars of LLM
  Control, accessed December 10, 2025,
  \href{https://dev.to/qvfagundes/temperature-tokens-and-context-windows-the-three-pillars-of-llm-control-34jg}{\ul{https://dev.to/qvfagundes/temperature-tokens-and-context-windows-the-three-pillars-of-llm-control-34jg}}
\item
  Top five essential context window concepts in large language models -
  Micron Technology, accessed December 10, 2025,
  \href{https://www.micron.com/about/blog/applications/ai/top-five-essential-context-window-concepts-in-large-language-models}{\ul{https://www.micron.com/about/blog/applications/ai/top-five-essential-context-window-concepts-in-large-language-models}}
\item
  Toby Slade (Editor) - Alisa Freedman (Editor) - Introducing Japanese
  Popular Culture-Routledge (2018) \textbar{} PDF - Scribd, accessed
  December 10, 2025,
  \href{https://www.scribd.com/document/750164971/Toby-Slade-Editor-Alisa-Freedman-Editor-Introducing-Japanese-Popular-Culture-Routledge-2018}{\ul{https://www.scribd.com/document/750164971/Toby-Slade-Editor-Alisa-Freedman-Editor-Introducing-Japanese-Popular-Culture-Routledge-2018}}
\item
  accessed December 10, 2025,
  \href{https://huggingface.co/datasets/Weyaxi/huggingface-spaces-codes/resolve/main/spaces.csv?download=true}{\ul{https://huggingface.co/datasets/Weyaxi/huggingface-spaces-codes/resolve/main/spaces.csv?download=true}}
\item
  Discrepancy in Context Window Length Listed on the ChatGPT Pricing
  Page of OpenAI\textquotesingle s Official Website - Documentation,
  accessed December 10, 2025,
  \href{https://community.openai.com/t/discrepancy-in-context-window-length-listed-on-the-chatgpt-pricing-page-of-openai-s-official-website/1047853}{\ul{https://community.openai.com/t/discrepancy-in-context-window-length-listed-on-the-chatgpt-pricing-page-of-openai-s-official-website/1047853}}
\item
  Introducing GPTs that can customize GPT-4o, GPT-4 Turbo, and ChatGPT,
  accessed December 10, 2025,
  \href{https://www.science.co.jp/en/nmt/blog/35896/}{\ul{https://www.science.co.jp/en/nmt/blog/35896/}}
\item
  A Textual Analysis of \#ChatGPT Twitter Conversation - RPubs, accessed
  December 10, 2025,
  \href{https://rpubs.com/jmbethe2/chatgpttextanalysis}{\ul{https://rpubs.com/jmbethe2/chatgpttextanalysis}}
\item
  GitHub - langgptai/LangGPT: LangGPT: Empowering everyone to become a
  prompt expert! 结构化提示词(Structured Prompt)提出者
  元提示词(Meta-Prompt)发起者 最流行的提示词落地范式, accessed
  December 10, 2025,
  \href{https://github.com/langgptai/LangGPT}{\ul{https://github.com/langgptai/LangGPT}}
\item
  Quick Start --- LangGPT v0.1 documentation, accessed December 10,
  2025,
  \href{https://langgpt.readthedocs.io/}{\ul{https://langgpt.readthedocs.io/}}
\item
  wow-agent-day02 Handcrafted an agent that\textquotesingle s so rustic
  it crumbles. - Hanah, accessed December 10, 2025,
  \href{https://hanah.xlog.app/wow-agent-day02?locale=en}{\ul{https://hanah.xlog.app/wow-agent-day02?locale=en}}
\item
  SillyTavernAI Lorebooks Guide with Gemini 2.5 - Arsturn, accessed
  December 10, 2025,
  \href{https://www.arsturn.com/blog/sillytavernai-lorebooks-with-gemini-2-5-a-complete-guide}{\ul{https://www.arsturn.com/blog/sillytavernai-lorebooks-with-gemini-2-5-a-complete-guide}}
\item
  sphiratrioth666/Lorebooks\_as\_ACTIVE\_scenario\_and\_character\_guidance\_tool
  - Hugging Face, accessed December 10, 2025,
  \href{https://huggingface.co/sphiratrioth666/Lorebooks_as_ACTIVE_scenario_and_character_guidance_tool}{\ul{https://huggingface.co/sphiratrioth666/Lorebooks\_as\_ACTIVE\_scenario\_and\_character\_guidance\_tool}}
\item
  Lore books: How do they work? : r/SillyTavernAI - Reddit, accessed
  December 10, 2025,
  \href{https://www.reddit.com/r/SillyTavernAI/comments/1ltaqss/lore_books_how_do_they_work/}{\ul{https://www.reddit.com/r/SillyTavernAI/comments/1ltaqss/lore\_books\_how\_do\_they\_work/}}
\item
  Malla: Demystifying Real-world Large Language Model Integrated
  Malicious Services - xiaojing liao, accessed December 10, 2025,
  \href{https://www.xiaojingliao.com/uploads/9/7/0/2/97024238/linsec24malla.pdf}{\ul{https://www.xiaojingliao.com/uploads/9/7/0/2/97024238/linsec24malla.pdf}}
\item
  My basic guide to Pygmalion for begginers : r/PygmalionAI - Reddit,
  accessed December 10, 2025,
  \href{https://www.reddit.com/r/PygmalionAI/comments/10h37u4/my_basic_guide_to_pygmalion_for_begginers/}{\ul{https://www.reddit.com/r/PygmalionAI/comments/10h37u4/my\_basic\_guide\_to\_pygmalion\_for\_begginers/}}
\item
  No offense, but some of y\textquotesingle all have a very peculiar
  mindset. Would ..., accessed December 10, 2025,
  \href{https://www.reddit.com/r/PygmalionAI/comments/10x1mcs/no_offense_but_some_of_yall_have_a_very_peculiar/}{\ul{https://www.reddit.com/r/PygmalionAI/comments/10x1mcs/no\_offense\_but\_some\_of\_yall\_have\_a\_very\_peculiar/}}
\item
  PromptLayer Review: AI Prompt Management and Logging Platform,
  accessed December 10, 2025,
  \href{https://tutorialswithai.com/tools/promptlayer/}{\ul{https://tutorialswithai.com/tools/promptlayer/}}
\item
  Unlocking AI\textquotesingle s Potential: A Deep Dive into PromptLayer
  for AI Users, accessed December 10, 2025,
  \href{https://skywork.ai/skypage/en/Unlocking-AI's-Potential-A-Deep-Dive-into-PromptLayer-for-AI-Users/1976118954373607424}{\ul{https://skywork.ai/skypage/en/Unlocking-AI\textquotesingle s-Potential-A-Deep-Dive-into-PromptLayer-for-AI-Users/1976118954373607424}}
\item
  Best Prompt Engineering Tools to Master AI Interaction and Content
  Generation, accessed December 10, 2025,
  \href{https://www.sprintzeal.com/blog/prompt-engineering-tools}{\ul{https://www.sprintzeal.com/blog/prompt-engineering-tools}}
\item
  Dust.tt: Building a Horizontal Enterprise Agent Platform with
  Infrastructure-First Approach - ZenML LLMOps Database, accessed
  December 10, 2025,
  \href{https://www.zenml.io/llmops-database/building-a-horizontal-enterprise-agent-platform-with-infrastructure-first-approach}{\ul{https://www.zenml.io/llmops-database/building-a-horizontal-enterprise-agent-platform-with-infrastructure-first-approach}}
\item
  Dust Product update 18 \textbar{} Dust Blog, accessed December 10,
  2025,
  \href{https://dust.tt/blog/dust-product-update-17-2}{\ul{https://dust.tt/blog/dust-product-update-17-2}}
\item
  Dust XP1 switches to GPT-3.5-turbo, is now free to use \textbar{}
  Hacker News, accessed December 10, 2025,
  \href{https://news.ycombinator.com/item?id=35069901}{\ul{https://news.ycombinator.com/item?id=35069901}}
\item
  LangFlow vs Flowise vs n8n vs Make - Reddit, accessed December 10,
  2025,
  \href{https://www.reddit.com/r/langflow/comments/1ij66dl/langflow_vs_flowise_vs_n8n_vs_make/}{\ul{https://www.reddit.com/r/langflow/comments/1ij66dl/langflow\_vs\_flowise\_vs\_n8n\_vs\_make/}}
\item
  LangChain vs LangGraph vs LangSmith vs LangFlow: Key Differences
  Explained \textbar{} DataCamp, accessed December 10, 2025,
  \href{https://www.datacamp.com/de/tutorial/langchain-vs-langgraph-vs-langsmith-vs-langflow}{\ul{https://www.datacamp.com/de/tutorial/langchain-vs-langgraph-vs-langsmith-vs-langflow}}
\item
  The Complete Guide to Choosing an AI Agent Framework in 2025 -
  Langflow, accessed December 10, 2025,
  \href{https://www.langflow.org/blog/the-complete-guide-to-choosing-an-ai-agent-framework-in-2025}{\ul{https://www.langflow.org/blog/the-complete-guide-to-choosing-an-ai-agent-framework-in-2025}}
\item
  We Tried and Tested 8 Langflow Alternatives for Production-Ready AI
  Workflows - ZenML, accessed December 10, 2025,
  \href{https://www.zenml.io/blog/langflow-alternatives}{\ul{https://www.zenml.io/blog/langflow-alternatives}}
\item
  PromptBase Deep Dive: Mastering the AI Prompt Marketplace for Future
  Growth and SEO Dominance - Skywork.ai, accessed December 10, 2025,
  \href{https://skywork.ai/skypage/en/PromptBase-Deep-Dive-Mastering-the-AI-Prompt-Marketplace-for-Future-Growth-and-SEO-Dominance/1972861300479422464}{\ul{https://skywork.ai/skypage/en/PromptBase-Deep-Dive-Mastering-the-AI-Prompt-Marketplace-for-Future-Growth-and-SEO-Dominance/1972861300479422464}}
\item
  PRSA: PRompt Stealing Attacks against Large Language Models - arXiv,
  accessed December 10, 2025,
  \href{https://arxiv.org/html/2402.19200v2}{\ul{https://arxiv.org/html/2402.19200v2}}
\item
  PRSA: Prompt Stealing Attacks against Real-World Prompt Services -
  USENIX, accessed December 10, 2025,
  \href{https://www.usenix.org/system/files/usenixsecurity25-yang-yong.pdf}{\ul{https://www.usenix.org/system/files/usenixsecurity25-yang-yong.pdf}}
\item
  CMV: People attempting to sell AI generated products from completely
  and directly from AI won\textquotesingle t be able to make a living
  for themselves : r/changemyview - Reddit, accessed December 10, 2025,
  \href{https://www.reddit.com/r/changemyview/comments/1i2wggx/cmv_people_attempting_to_sell_ai_generated/}{\ul{https://www.reddit.com/r/changemyview/comments/1i2wggx/cmv\_people\_attempting\_to\_sell\_ai\_generated/}}
\item
  Building the Future: A Deep Dive Into the Generative AI App
  Infrastructure Stack, accessed December 10, 2025,
  \href{https://sapphireventures.com/blog/building-the-future-a-deep-dive-into-the-generative-ai-app-infrastructure-stack/}{\ul{https://sapphireventures.com/blog/building-the-future-a-deep-dive-into-the-generative-ai-app-infrastructure-stack/}}
\item
  AutoGPT --- ThirdEye Data, accessed December 10, 2025,
  \href{https://thirdeyedata.ai/agentic-ai-solutions/autogpt/}{\ul{https://thirdeyedata.ai/agentic-ai-solutions/autogpt/}}
\item
  AutoGPT: Automating GPT Model for Natural Language Generation -
  Reddit, accessed December 10, 2025,
  \href{https://www.reddit.com/r/AutoGPT/}{\ul{https://www.reddit.com/r/AutoGPT/}}
\item
  Auto-GPT seems nearly unusable : r/AutoGPT - Reddit, accessed December
  10, 2025,
  \href{https://www.reddit.com/r/AutoGPT/comments/13gpirj/autogpt_seems_nearly_unusable/}{\ul{https://www.reddit.com/r/AutoGPT/comments/13gpirj/autogpt\_seems\_nearly\_unusable/}}
\item
  AUGUSTUS: An LLM-Driven Multimodal Agent System with Contextualized
  User Memory - arXiv, accessed December 10, 2025,
  \href{https://arxiv.org/html/2510.15261v1}{\ul{https://arxiv.org/html/2510.15261v1}}
\item
  EvoTest: Evolutionary Test-Time Learning for Self-Improving Agentic
  Systems - arXiv, accessed December 10, 2025,
  \href{https://arxiv.org/html/2510.13220v1}{\ul{https://arxiv.org/html/2510.13220v1}}
\item
  MemGPT: Towards LLMs as Operating Systems - arXiv, accessed December
  10, 2025,
  \href{https://arxiv.org/abs/2310.08560}{\ul{https://arxiv.org/abs/2310.08560}}
\item
  JSON Prompt: The Ultimate Guide to Perfect AI Outputs - MPG ONE,
  accessed December 10, 2025,
  \href{https://mpgone.com/json-prompt-guide/}{\ul{https://mpgone.com/json-prompt-guide/}}
\item
  Generative Interfaces for Language Models - arXiv, accessed December
  10, 2025,
  \href{https://arxiv.org/html/2508.19227v2}{\ul{https://arxiv.org/html/2508.19227v2}}
\item
  Exploring ChatGPT Canvas - Skywork ai, accessed December 10, 2025,
  \href{https://skywork.ai/blog/exploring-chatgpt-canvas/}{\ul{https://skywork.ai/blog/exploring-chatgpt-canvas/}}
\item
  Introducing canvas, a new way to write and code with ChatGPT.
  \textbar{} OpenAI, accessed December 10, 2025,
  \href{https://openai.com/index/introducing-canvas/}{\ul{https://openai.com/index/introducing-canvas/}}
\end{enumerate}
