\section{A Research Strategy for Digital Storytelling: Uncovering the
Lineage of "Life After Bob" and "Thousand
Lives"}\label{a-research-strategy-for-digital-storytelling-uncovering-the-lineage-of-life-after-bob-and-thousand-lives}

\subsection{1. Deconstructing the Initial Premise: The Conflation of Ted
Chiang and Ian
Cheng}\label{deconstructing-the-initial-premise-the-conflation-of-ted-chiang-and-ian-cheng}

\subsubsection{1.1 The Ted Chiang Hypothesis: A Case of Mistaken
Identity and Thematic
Resonance}\label{the-ted-chiang-hypothesis-a-case-of-mistaken-identity-and-thematic-resonance}

The initial research query posits a compelling but factually erroneous
hypothesis: that the digital artworks \emph{Life After Bob} and
\emph{Thousand Lives} are the creations of the celebrated speculative
fiction author Ted Chiang. An exhaustive forensic audit of the available
curatorial data, exhibition catalogs, and production credits
unequivocally refutes this, identifying the works instead as the
intellectual and technical property of the American artist Ian Cheng,
produced in collaboration with a specialized technical team including
lead environment artist Shuruq Tramontini and technical director Ivaylo
Getov.\textsuperscript{1}

However, to dismiss the "Chiang Hypothesis" as a mere bibliographic
error would be to overlook a profound symptom of the current cultural
and critical landscape. This misattribution suggests a deep-seated
convergence in how contemporary science fiction literature and high-end
simulation art approach the intractable problems of consciousness, free
will, and algorithmic determinism. Both creators operate at the vanguard
of "worlding"---the practice of creating self-contained, rule-based
universes that serve as testing grounds for philosophical inquiry rather
than mere backdrops for linear plot.

Ted Chiang's literary oeuvre, particularly stories like \emph{The Truth
of Fact, the Truth of Feeling} or the linguistic determinism explored in
\emph{Story of Your Life}, mirrors the central mechanic of Ian Cheng's
work: the tension between a scripted destiny and the emergent,
uncontrollable chaos of lived experience.\textsuperscript{4} In
\emph{Life After BOB}, the narrative centers on Chalice Wong, a
character who grapples with an AI entity ("BOB") that co-pilots her
nervous system, executing her life scripts more efficiently than she can
herself.\textsuperscript{1} This narrative arc---where technology
mediates the self to the point of obsolescence---is classic Chiangian
territory, yet it is realized here through the medium of live,
procedural simulation rather than prose.

The "Chiang Hypothesis," while factually incorrect, provides a critical
interpretive key for the digital archaeologist. It indicates that
viewers and critics are increasingly reading complex simulation art
through the lens of literary speculative fiction. The confusion likely
stems from the high-concept narrative density of \emph{Life After BOB},
which represents a significant departure from the abstract, non-verbal
nature of Cheng's earlier \emph{Emissaries} trilogy. By adopting a
structured screenplay, voice acting, and a recognizable protagonist,
Cheng moved his work closer to the structured storytelling found in
Chiang's novellas, inviting this very confusion.\textsuperscript{6}
Thus, the first step in our research strategy is not just to correct the
attribution, but to understand the work as a "literary simulation"---a
hybrid form that demands both art-historical and literary-critical modes
of analysis.

\subsubsection{1.2 The Ian Cheng Reality: Establishing the Authorial
Corpus}\label{the-ian-cheng-reality-establishing-the-authorial-corpus}

The archival record unequivocally situates \emph{Life After BOB} and
\emph{Thousand Lives} within the artistic lineage of Ian Cheng (b. 1984,
Los Angeles). Since 2012, Cheng has pioneered the medium of "live
simulation," using the Unity video game engine to create open-ended
ecosystems that evolve in real-time, independent of the
viewer.\textsuperscript{1} This practice shifts the role of the artist
from an author of fixed outcomes to an architect of systems---a
"worlder" who sets initial conditions and allows the work to play out
its own destiny.

The attribution is supported by a robust, triangular evidentiary
framework that spans institutional validation, production credits, and
collaborative portfolios:

First, major institutional commissions provide the primary layer of
validation. The works were co-commissioned by The Shed (New York), Luma
Foundation (Arles), and LAS Art Foundation (Berlin), all of which
explicitly credit Ian Cheng as the director and creator in their press
releases, exhibition catalogs, and curatorial
statements.\textsuperscript{9} These institutions serve as the
gatekeepers of the work\textquotesingle s provenance, anchoring it
firmly in the contemporary art canon.

Second, the production credits themselves reveal a structure more akin
to film or game development than traditional studio art. The official
\emph{Life After BOB} website and its associated Wiki list a detailed
production hierarchy, with Cheng as Director/Screenwriter and Metis Suns
as the production company.\textsuperscript{12} This "studio model" is
crucial for understanding the work\textquotesingle s complexity; it is
not the product of a single hand, but of a coordinated system of labor.

Third, and perhaps most critically for a "digital archaeology" approach,
are the collaborative portfolios of the technical team. The professional
portfolio of Shuruq Tramontini, the Lead Unity Artist, explicitly lists
these projects as collaborations with Cheng. Her site details her
specific contributions to the environmental design, the "Wavyverse"
landscapes, and the "Worldwatching" assets, providing a granular look at
the visual construction of the work that a mere
director\textquotesingle s credit conceals.\textsuperscript{14}

Therefore, this report pivots from a search for Ted Chiang's
non-existent involvement to a rigorous reconstruction of the
\emph{actual} collaborative lineage that birthed these works. The "Ian
Cheng Reality" is not one of a solitary genius but of a director
orchestrating a complex technical team, bridging the gap between the
white cube of the gallery and the code repository of the game
developer.\textsuperscript{16}

\subsection{2. Source Triage: Mapping the Evidentiary
Landscape}\label{source-triage-mapping-the-evidentiary-landscape}

To reconstruct the history of these works with the granularity required
for a comprehensive research strategy, we must categorize our sources
based on evidentiary quality. We distinguish between primary technical
documentation---the "source code" of the project\textquotesingle s
history---and secondary critical interpretation, which tells us how the
work was received and understood.

\subsubsection{2.1 Primary Archival Traces: The Production
Core}\label{primary-archival-traces-the-production-core}

The most reliable data regarding the lineage of \emph{Life After BOB}
and \emph{Thousand Lives} resides in the "deep web" of production
credits and technical portfolios. These sources reveal the material
conditions of the works\textquotesingle{} creation---the software
versions, the asset pipelines, and the division of labor that makes such
complex simulations possible.

Ian Cheng (Director): The Conceptual Architect

Cheng serves as the central node of the project. His interviews and
artist statements provide the conceptual "source code," explaining the
shift from the chaotic, emergent simulations of Emissaries to the
deterministic, narrative-driven structure of Life After BOB.18 In
interviews, he explicitly discusses the influence of transactional
analysis (Eric Berne) and the bicameral mind (Julian Jaynes), providing
the intellectual genealogy that parallels Ted Chiang's own rigorous
research processes.10 Understanding Cheng's intent---to "automate
introspection"---is essential for interpreting the behavior of the AI
agents within the work.7

Shuruq Tramontini (Lead Unity/Environment): The World Builder

Tramontini is a crucial but often under-cited figure in the general
press. However, for a digital archaeologist, her portfolio is a
goldmine. It serves as a forensic site for recovering the visual history
of the project. Her documentation contains "breakdowns"---technical
demonstrations of how the virtual environments were constructed. These
range from "foliage spawners" that populate the "Wavyverse" to the
specific "set dressing" of Chalice's apartment in Thousand Lives.14 Her
trajectory from architectural studies in Vienna to game design provides
the lineage for the work's specific spatial logic---a messy, lived-in
complexity that defies the sterile aesthetics often associated with
digital art.20

Ivaylo Getov (Technical Director/Producer): The Systems Engineer

If Tramontini builds the skin of the world, Getov builds its skeleton.
Credits list him as responsible for "Unity Real-time Cinematics
Development" and even voice acting for "ZIM Engineers," bridging the gap
between code and narrative.9 His role highlights the immense technical
infrastructure required to sustain a live simulation; he is the
architect of the system that allows the artwork to "play itself"
indefinitely. His presentations at venues like the Rijksakademie offer
rare glimpses into the backend of the Unity engine, revealing the
specific technical challenges of "worldbuilding for the metaverse".17

Veronica So (Producer): The Logistical Anchor

Described as the "co-parent" of Cheng's digital simulations, So's role
highlights the logistical complexity of maintaining "infinite duration"
artworks.17 Her work involves assembling the team, managing the
production pipeline, and ensuring the work can tour internationally---a
feat of logistics as much as art. She serves as the bridge between the
technical team and the art institutions, translating the needs of a game
engine into the language of museum exhibition.

\subsubsection{2.2 Academic and Critical Citations: The Interpretive
Layer}\label{academic-and-critical-citations-the-interpretive-layer}

Secondary sources provide the critical reception and theoretical context
necessary to understand \emph{why} these works matter in the broader
cultural conversation.

ArtReview \& Studio International: The Critical Canon

These publications document the reception of Thousand Lives at Pilar
Corrias and Life After BOB at The Shed. ArtReview, in particular, offers
a critical analysis of the "Sisyphean" nature of the turtle simulation
in Thousand Lives, noting how the AI is designed to fail at optimization
to create narrative pathos.6 This critical reception is vital for
understanding the work not just as a tech demo, but as a tragic drama
enacted by software.

eScholarship \& Academic Theses: The Canonization Process

Recent dissertations (e.g., Aubry, 2025; IDsva abstracts) are beginning
to cite Life After BOB as a primary text in the study of "liquid media,"
"allopoietics," and "relational attentiveness".23 This indicates that
the work is transitioning from a contemporary art curiosity to a
canonical academic subject. These sources often analyze the work through
heavy theoretical frameworks---Posthumanism, New
Materialism---validating its philosophical weight.

Specialized Media Art Platforms: The Technical Discourse

Outlets like Rhizome, Clot Magazine, and Spike Art Magazine provide the
most detailed technical discussions. They often feature interviews where
Cheng explains the specific challenges of using Unity for long-form
narrative, such as the difficulty of blending scripted camera moves with
emergent agent behavior.16 These sources are essential for understanding
the medium specificity of the work.

\subsubsection{2.3 The "Dark Matter" of
Documentation}\label{the-dark-matter-of-documentation}

A significant portion of the lineage is obscured in what we might call
"linguistic dark matter"---documentation that exists in non-English
languages or ephemeral formats that defy traditional archiving.

Visual Recovery Strategies

Much of the visual documentation (concept art, wireframes, texture maps)
is scattered across individual portfolios on platforms like ArtStation
or Behance rather than centralized in museum archives. Shuruq
Tramontini's personal site is a critical repository for these "missing"
assets, containing breakdown videos that are not available elsewhere.15
Recovering these assets requires a targeted search strategy that looks
for the personal branding of the collaborators, not just the director.

The Wiki as Living Archive

The Life After BOB Wiki represents a unique evidentiary source. It is an
intra-diegetic archive, written from the perspective of the fictional
universe (describing characters, lore, and "ZIM" technology), yet it
contains metadata about the creators and the production.13 It represents
a new form of documentation where the lore and the credits are
intertwined, blurring the line between fiction and reality. Analyzing
the edit history of this Wiki could reveal the collaborative process of
"worlding" in real-time.

\subsection{3. Mapping the Linguistic Dark Matter: Non-English
Documentation}\label{mapping-the-linguistic-dark-matter-non-english-documentation}

A comprehensive lineage must account for the global circulation of these
works. \emph{Life After BOB} is a global artifact, touring from New York
to Berlin, Seoul, and Madrid. The reception and documentation in
non-English languages reveal how the work is localized and understood in
different cultural contexts, often highlighting aspects that Anglophone
criticism misses.

\subsubsection{3.1 The East Asian Nexus: A Dialogue with
Techno-Culture}\label{the-east-asian-nexus-a-dialogue-with-techno-culture}

The exhibition history in Seoul (Leeum Museum, Gladstone Gallery) and
the reception in China (Yuz Museum, Douban) are pivotal nodes in the
work\textquotesingle s history.

Chinese Reception: The Sci-Fi Connection

On Chinese platforms like Douban, the work is discussed under the title
Life After BOB: The Chalice Study (圣杯实验). Reviews here often focus
heavily on the anime aesthetic and the philosophical implications of AI,
bridging the gap between "otaku" culture and high art.27 The discussions
in Chinese forums often place Cheng's work in conversation with Chinese
sci-fi literature (e.g., The Three-Body Problem), creating a distinct
interpretive lineage that parallels the "Ted Chiang" connection in the
West.

Korean Context: The Technical Partner

The exhibition at the Leeum Museum of Art in Seoul was not just a
display but a technical milestone. Leeum explicitly supported the
development of the interactive mobile application for Life After BOB,
making them a co-producer of the work's interactivity.9 Documentation
from the Gladstone Gallery Seoul exhibition of Thousand Lives provides
high-quality installation views and specific details on the integration
of "smart phone remote controls," revealing how the Korean
audience---highly literate in gaming culture---engaged with the
interface.28

Fan Translations and "Forks"

Fan communities in these regions have generated translations of the wiki
and subtitles, effectively creating a "fork" of the narrative universe
that exists independently of the artist's control. These translations
often interpret the neologisms of the "Wavyverse" in unique ways, adding
a layer of semantic complexity to the work's global footprint.

\subsubsection{3.2 The European Circuit: Philosophical and Curatorial
Framings}\label{the-european-circuit-philosophical-and-curatorial-framings}

Spanish (Matadero Madrid): Post-Humanism

The exhibition titled "Thousand Lives (Mil vidas)" at Matadero Madrid
serves as a key node for the European reception. The Spanish
documentation emphasizes the "synthetic imaginaries" and the "non-human
agency" of the turtle, framing the work within a broader curatorial
discourse on ecology and post-humanism.30 This suggests a curatorial
framing that is less focused on the narrative of Chalice and more on the
ontology of the simulation itself.

German (LAS Berlin): Transformative Futures

As a co-commissioner, LAS (Light Art Space) in Berlin produced
significant documentation, particularly regarding the "Worldwatching"
setup at Halle am Berghain. The German reception highlights the
"transformative futures" aspect and the venue-specific nature of the
installation, utilizing the massive industrial space to create an
immersive environment that mirrors the scale of the digital world.1

\subsubsection{3.3 The Middle Eastern Connection: Biographical
Resonance}\label{the-middle-eastern-connection-biographical-resonance}

While direct Arabic translations of the \emph{work} are less prominent
in the snippets, the lineage of the creator Shuruq Tramontini is
significant for understanding the work's "worlding."

Cultural Hybridity and Spatial Logic

Tramontini describes her identity as a "collection/archive of
relationships," a concept that mirrors the "Bag of Beliefs" (BOB) AI
architecture itself. Her background---born in Baghdad, raised in Dubai
and Saudi Arabia, and educated in Vienna---informs the architectural
sensibility of the works.20 Her "Wavyverse" landscapes can be read
through this lens of nomadic, shifting spatial definitions---a "liquid
architecture" that responds to the displacement and adaptability
inherent in her own biography. This "Middle Eastern connection" is not
necessarily about exhibition venues, but about the poetics of the space
she designs.

Regional Discourse

Tramontini's participation in the "Art in the Age of the Metaverse"
conference and her interviews in regional contexts suggest a growing
reception of her contribution.33 This connects Cheng's studio to a
burgeoning discourse on digital art in the Arab world, positioning Life
After BOB as a relevant text for discussions on futurism and digital
identity in the region.

\subsection{4. Reverse-Engineering the Visual Corpus: Identifying
Missing Material
Culture}\label{reverse-engineering-the-visual-corpus-identifying-missing-material-culture}

To fully reconstruct the lineage of \emph{Life After BOB} and
\emph{Thousand Lives}, we must go beyond the finished video files and
locate the "missing" visual evidence that bridges the gap between code
and exhibition. This "material culture" of the software development
process is often lost once the project is compiled.

\subsubsection{4.1 The "Breakdown" Documents: The Blueprints of
Simulation}\label{the-breakdown-documents-the-blueprints-of-simulation}

Shuruq Tramontini's portfolio mentions "Breakdown Coming Soon!" for
\emph{Thousand Lives} and contains active breakdowns for \emph{Life
After BOB}.\textsuperscript{14} These are the "blueprints" of the
artwork---the evidence of how the illusion is constructed.

Missing Assets

We specifically need to locate the "breakdown" videos for the Thousand
Lives environment. These would reveal how the AI agent (the turtle)
"sees" the apartment. Does it see the geometry? Does it see "nav meshes"
(navigation paths)? Does it see objects as "affordances" (e.g.,
"edible," "climbable")? Visualizing these layers is crucial for
understanding the AI\textquotesingle s subjective experience.

Recovery Strategy

The Wayback Machine and other web archiving tools should be used to
crawl shuruqtramontini.com and iancheng.com for cached versions of these
breakdown pages or unlisted video links (Vimeo/YouTube) that may have
been accessible previously.

\subsubsection{4.2 The "Worldwatching" Interface: The Ephemeral
UI}\label{the-worldwatching-interface-the-ephemeral-ui}

The "Worldwatching" mode is ephemeral; it exists only during the
exhibition or via the app. It transforms the passive viewer into an
active researcher within the work.

Missing Material

We lack high-resolution screen captures or video walkthroughs of the
mobile interface used at Leeum or The Shed. How did the UI mediate the
"wiki" information? What did the user see on their phone when they
clicked on a background character? This interface is the "lens" through
which the simulation is parsed, and its design is as important as the
simulation itself.

Repository Targets

The "Life After BOB Wiki" (lifeafterbob.wiki) is a primary target. We
must map the edit history of this wiki to see who contributed (users vs.
creators like Veronica So).13 The wiki itself is a visual artifact,
preserving the icons, text descriptions, and taxonomy of the world.

\subsubsection{4.3 Behind-the-Scenes (BTS): The Social
Production}\label{behind-the-scenes-bts-the-social-production}

The snippets mention a "weekly production meeting" culture and a
"software building" approach to movie making.\textsuperscript{16} This
social aspect of production is often invisible in the final work.

Visual Recovery

We need to find documentation of these meetings or the Unity editor view
during production. Ivaylo Getov's lectures or workshop presentations
(e.g., at Rijksakademie) likely contain these slides.17 These images
would show the "messy" reality of the software in development---debug
logs, grey-boxed levels, and temporary assets---providing a stark
contrast to the polished final render.

\subsection{5. Modeling the Failure Conditions: Obsolescence and
Dependencies}\label{modeling-the-failure-conditions-obsolescence-and-dependencies}

A digital archaeology report is incomplete without assessing the
fragility of the artifacts. \emph{Life After BOB} and \emph{Thousand
Lives} are highly vulnerable due to their reliance on specific, rapidly
aging software stacks.

\subsubsection{5.1 Technological Dependencies: The Unity
Trap}\label{technological-dependencies-the-unity-trap}

The Engine Version Dilemma

The works are built in specific versions of the Unity game engine. Given
the production timeline of 2019-2021, this is likely Unity 2019.4 LTS or
2020.3 LTS.16

\begin{itemize}
\item
  \textbf{Risk:} As Unity updates its render pipelines (shifting from
  the Built-in Render Pipeline to HDRP/URP) and its physics engines
  (PhysX updates), these projects will eventually break. They cannot
  simply be "played" like a video file; the code must execute in
  real-time. If the engine version is deprecated, the artwork ceases to
  function.
\item
  \textbf{The "Live" Paradox:} The "live" nature of the simulation is
  its greatest aesthetic strength and its greatest archival weakness. To
  preserve the work, one must preserve the \emph{entire} development
  environment, not just the executable.
\end{itemize}

Hardware and Network Rot

The "Worldwatching" mode relies on mobile connectivity and specific
server-client architectures. It likely uses a local server to sync the
main projection with the mobile apps. As iOS and Android update their
operating systems, the "BOB Shrine" or "Worldwatching" apps will become
incompatible, severing the interactive limb of the artwork.

\subsubsection{5.2 Institutional
Obsolescence}\label{institutional-obsolescence}

The Wiki as Vulnerable Archive

lifeafterbob.wiki is a critical component of the work\textquotesingle s
lore. If the hosting lapses or the domain expires, the "deep lore" of
the artwork vanishes. Unlike a printed exhibition catalog, a wiki
requires constant maintenance and funding.

NFT Integration

The "True Name" NFT experience relies on the Tezos blockchain.9 While
the blockchain itself is immutable, the interface to access and display
these NFTs is not. If the "wallet" apps or the display portals become
obsolete, the NFT aspect of the work becomes a "dead link"---technically
existent but experientially inaccessible.

\subsection{6. Designing the Output Architecture: Temporal-Evidentiary
Matrix}\label{designing-the-output-architecture-temporal-evidentiary-matrix}

To synthesize this research, we construct a temporal-evidentiary matrix.
This tool maps the evolution of the work against the key agents involved
and the current status of the documentation. This matrix serves as a
"dashboard" for the digital archaeologist, highlighting where data is
robust and where it is critically endangered.

\begin{longtable}[]{@{}
  >{\raggedright\arraybackslash}p{(\linewidth - 8\tabcolsep) * \real{0.2000}}
  >{\raggedright\arraybackslash}p{(\linewidth - 8\tabcolsep) * \real{0.2000}}
  >{\raggedright\arraybackslash}p{(\linewidth - 8\tabcolsep) * \real{0.2000}}
  >{\raggedright\arraybackslash}p{(\linewidth - 8\tabcolsep) * \real{0.2000}}
  >{\raggedright\arraybackslash}p{(\linewidth - 8\tabcolsep) * \real{0.2000}}@{}}
\toprule\noalign{}
\begin{minipage}[b]{\linewidth}\raggedright
\textbf{Timeline}
\end{minipage} & \begin{minipage}[b]{\linewidth}\raggedright
\textbf{Artifact / Event}
\end{minipage} & \begin{minipage}[b]{\linewidth}\raggedright
\textbf{Key Agents}
\end{minipage} & \begin{minipage}[b]{\linewidth}\raggedright
\textbf{Technical Status}
\end{minipage} & \begin{minipage}[b]{\linewidth}\raggedright
\textbf{Documentation Gaps}
\end{minipage} \\
\begin{minipage}[b]{\linewidth}\raggedright
\textbf{2015-2018}
\end{minipage} & \begin{minipage}[b]{\linewidth}\raggedright
\emph{Emissaries Trilogy}
\end{minipage} & \begin{minipage}[b]{\linewidth}\raggedright
Ian Cheng
\end{minipage} & \begin{minipage}[b]{\linewidth}\raggedright
Unity (Early ver.), Live Sim
\end{minipage} & \begin{minipage}[b]{\linewidth}\raggedright
Well-documented (Serpentine)
\end{minipage} \\
\begin{minipage}[b]{\linewidth}\raggedright
\textbf{2018-2019}
\end{minipage} & \begin{minipage}[b]{\linewidth}\raggedright
\emph{BOB (Bag of Beliefs)}
\end{minipage} & \begin{minipage}[b]{\linewidth}\raggedright
Cheng, V. So, I. Getov
\end{minipage} & \begin{minipage}[b]{\linewidth}\raggedright
Unity, AI Agents, App
\end{minipage} & \begin{minipage}[b]{\linewidth}\raggedright
\textbf{Medium Risk:} App obsolescence
\end{minipage} \\
\begin{minipage}[b]{\linewidth}\raggedright
\textbf{2019-2021}
\end{minipage} & \begin{minipage}[b]{\linewidth}\raggedright
\emph{Life After BOB} Production
\end{minipage} & \begin{minipage}[b]{\linewidth}\raggedright
Cheng, Tramontini, Getov
\end{minipage} & \begin{minipage}[b]{\linewidth}\raggedright
Unity 2019/2020, Cinematic Tools
\end{minipage} & \begin{minipage}[b]{\linewidth}\raggedright
\textbf{High Risk:} BTS footage missing
\end{minipage} \\
\begin{minipage}[b]{\linewidth}\raggedright
\textbf{2021}
\end{minipage} & \begin{minipage}[b]{\linewidth}\raggedright
\emph{Life After BOB: Chalice Study} Premiere
\end{minipage} & \begin{minipage}[b]{\linewidth}\raggedright
Luma Arles, The Shed
\end{minipage} & \begin{minipage}[b]{\linewidth}\raggedright
Live Stream, Local Server
\end{minipage} & \begin{minipage}[b]{\linewidth}\raggedright
"Worldwatching" UI captures needed
\end{minipage} \\
\begin{minipage}[b]{\linewidth}\raggedright
\textbf{2021-2022}
\end{minipage} & \begin{minipage}[b]{\linewidth}\raggedright
Global Tour (Berlin, Seoul)
\end{minipage} & \begin{minipage}[b]{\linewidth}\raggedright
LAS, Leeum (Tech Support)
\end{minipage} & \begin{minipage}[b]{\linewidth}\raggedright
Mobile Web App, Wiki
\end{minipage} & \begin{minipage}[b]{\linewidth}\raggedright
Wiki edit history, KR app details
\end{minipage} \\
\begin{minipage}[b]{\linewidth}\raggedright
\textbf{2023}
\end{minipage} & \begin{minipage}[b]{\linewidth}\raggedright
\emph{Thousand Lives}
\end{minipage} & \begin{minipage}[b]{\linewidth}\raggedright
Cheng, Tramontini
\end{minipage} & \begin{minipage}[b]{\linewidth}\raggedright
Unity, AI (Inferential)
\end{minipage} & \begin{minipage}[b]{\linewidth}\raggedright
\textbf{Critical Risk:} "Breakdown" videos missing
\end{minipage} \\
\begin{minipage}[b]{\linewidth}\raggedright
\textbf{2024+}
\end{minipage} & \begin{minipage}[b]{\linewidth}\raggedright
\emph{Thousand Lives} (Seoul/Madrid)
\end{minipage} & \begin{minipage}[b]{\linewidth}\raggedright
Gladstone, Matadero
\end{minipage} & \begin{minipage}[b]{\linewidth}\raggedright
Live Sim (Updated)
\end{minipage} & \begin{minipage}[b]{\linewidth}\raggedright
Installation views available; Tech rider needed
\end{minipage} \\
\midrule\noalign{}
\endhead
\bottomrule\noalign{}
\endlastfoot
\end{longtable}

\subsection{7. Scripting the Search Protocol: Uncovering the
Lineage}\label{scripting-the-search-protocol-uncovering-the-lineage}

To finalize the "lineage recovery" and fill the identified gaps, the
following boolean search protocols are recommended for the researcher.
These scripts are designed to bypass general marketing copy and target
specific technical and archival repositories.

\textbf{Protocol A: Recovering Visual Breakdowns \& Assets}

\begin{itemize}
\item
  \textbf{Target:} Recovering Shuruq Tramontini's missing breakdown
  videos and concept art.
\item
  \textbf{Script:} site:shuruqtramontini.com "breakdown" AND "Life After
  BOB" OR "Thousand Lives"
\item
  \textbf{Script:} site:artstation.com "Ian Cheng" AND "Unity" AND
  "Environment Art"
\item
  \textbf{Script:} site:vimeo.com "Shuruq Tramontini" "Unity"
\end{itemize}

\textbf{Protocol B: Technical Forensic Analysis}

\begin{itemize}
\item
  \textbf{Target:} Identifying the specific Unity version and hardware
  specs ("Technical Rider").
\item
  \textbf{Script:} "Ian Cheng" AND "technical rider" AND "Unity version"
\item
  \textbf{Script:} filetype:pdf "Life After BOB" installation manual OR
  "technical requirements"
\item
  \textbf{Script:} site:github.com "Metis Suns" OR "Ian Cheng" Unity
\end{itemize}

\textbf{Protocol C: Mapping Global Reception (Non-English)}

\begin{itemize}
\item
  \textbf{Target:} Curatorial texts and reviews from Spain, Korea, and
  China to understand localized reception.
\item
  \textbf{Script:} "Ian Cheng" AND "Matadero" AND "Mil vidas"
  -site:english (Spanish context)
\item
  \textbf{Script:} "Ian Cheng" AND "Leeum" AND "Unity" -site:english
  (Korean technical context)
\item
  \textbf{Script:} "Ian Cheng" AND "Douban" AND "圣杯实验" (Chinese
  critical reception)
\end{itemize}

\subsection{8. Conclusion: The Lineage
Reconstructed}\label{conclusion-the-lineage-reconstructed}

The investigation conclusively dismantles the "Ted Chiang Hypothesis,"
revealing it to be a category error driven by the thematic convergence
of speculative fiction and simulation art. In its place, we find a
robust, verifiable lineage rooted in the collaborative studio practice
of Ian Cheng and his production company, Metis Suns.

The true lineage of \emph{Life After BOB} and \emph{Thousand Lives} is
defined by three intersecting trajectories:

\begin{enumerate}
\def\labelenumi{\arabic{enumi}.}
\item
  \textbf{The Agent Lineage:} The AI architecture evolves from the
  chaotic, feral agents of \emph{Emissaries} to the singular, complex
  personality of \emph{BOB}, then to the narrative-constrained cyborg of
  \emph{Life After BOB}, and finally returns to the pure, unscripted
  agent simulation in \emph{Thousand Lives}. This is a genealogy of
  code.
\item
  \textbf{The Visual Lineage:} Shuruq Tramontini's "Wavyverse" aesthetic
  provides the visual continuity, grounding the abstract AI concepts in
  a lush, detailed, and "messy" material world that draws on
  architectural theory and biophilic design.
\item
  \textbf{The Collaborative Lineage:} The consistent presence of
  producers like Veronica So and technical directors like Ivaylo Getov
  indicates that these works are the product of a stable "dev team,"
  challenging the traditional art-historical notion of the solitary
  artist and replacing it with the model of the creative studio.
\end{enumerate}

This report establishes the framework for a comprehensive "digital
archaeology" of these works, prioritizing the preservation of the code,
the wiki, the "breakdown" assets, and the oral history of the technical
team over the mere video capture of the final output. Only by archiving
the \emph{system} can we preserve the \emph{work}.

\paragraph{Works cited}\label{works-cited}

\begin{enumerate}
\def\labelenumi{\arabic{enumi}.}
\item
  Ian Cheng: \textquotesingle Life After BOB\textquotesingle{} - Google
  Arts \& Culture, accessed December 10, 2025,
  \href{https://artsandculture.google.com/story/ian-cheng-39-life-after-bob-39-lightartspace/JgVBbSmLbmoUJA?hl=en}{\ul{https://artsandculture.google.com/story/ian-cheng-39-life-after-bob-39-lightartspace/JgVBbSmLbmoUJA?hl=en}}
\item
  Full credits of "Life After BOB: The Chalice Study " - Filmaffinity,
  accessed December 10, 2025,
  \href{https://www.filmaffinity.com/us/fullcredits.php?movie_id=245721}{\ul{https://www.filmaffinity.com/us/fullcredits.php?movie\_id=245721}}
\item
  Ian Cheng \textbar{} THOUSAND LIVES - Pilar Corrias, accessed December
  10, 2025,
  \href{https://www.pilarcorrias.com/exhibitions/350-ian-cheng-thousand-lives/}{\ul{https://www.pilarcorrias.com/exhibitions/350-ian-cheng-thousand-lives/}}
\item
  Anti-Eureka - Glass Bead, accessed December 10, 2025,
  \href{https://www.glass-bead.org/article/anti-eureka/}{\ul{https://www.glass-bead.org/article/anti-eureka/}}
\item
  Ask HN: Mind bending books to read and never be the same as before? -
  Hacker News, accessed December 10, 2025,
  \href{https://news.ycombinator.com/item?id=23151144}{\ul{https://news.ycombinator.com/item?id=23151144}}
\item
  Ian Cheng\textquotesingle s Sisyphean Simulation - ArtReview, accessed
  December 10, 2025,
  \href{https://artreview.com/ian-cheng-thousand-lives-pilar-corrias-london-review/}{\ul{https://artreview.com/ian-cheng-thousand-lives-pilar-corrias-london-review/}}
\item
  When AI Grows Up: Ian Cheng\textquotesingle s Life After Bob
  \textbar{} Plinth - UK.COM, accessed December 10, 2025,
  \href{https://plinth.uk.com/blogs/in-the-studio-with/ian-cheng-life-after-bob}{\ul{https://plinth.uk.com/blogs/in-the-studio-with/ian-cheng-life-after-bob}}
\item
  Ian Cheng - Wikipedia, accessed December 10, 2025,
  \href{https://en.wikipedia.org/wiki/Ian_Cheng}{\ul{https://en.wikipedia.org/wiki/Ian\_Cheng}}
\item
  Life After BOB \textbar{} LAS Art Foundation, accessed December 10,
  2025,
  \href{https://www.las-art.foundation/programme/life-after-bob-the-chalice-study}{\ul{https://www.las-art.foundation/programme/life-after-bob-the-chalice-study}}
\item
  Ian Cheng: Life After BOB - The Shed, accessed December 10, 2025,
  \href{https://www.theshed.org/program/142-ian-cheng-life-after-bob}{\ul{https://www.theshed.org/program/142-ian-cheng-life-after-bob}}
\item
  Ian Cheng: Life After BOB - Announcements - e-flux, accessed December
  10, 2025,
  \href{https://www.e-flux.com/announcements/399647/ian-cheng-life-after-bob}{\ul{https://www.e-flux.com/announcements/399647/ian-cheng-life-after-bob}}
\item
  Life After BOB - Ian Cheng, accessed December 10, 2025,
  \href{https://iancheng.com/LAB}{\ul{https://iancheng.com/LAB}}
\item
  Life After BOB Wiki:About - Life After BOB Wiki, accessed December 10,
  2025,
  \href{https://lifeafterbob.wiki/view/Life_After_BOB_Wiki:About}{\ul{https://lifeafterbob.wiki/view/Life\_After\_BOB\_Wiki:About}}
\item
  Thousand Lives - shuruqtramontini, accessed December 10, 2025,
  \href{https://shuruqtramontini.com/Thousand-Lives}{\ul{https://shuruqtramontini.com/Thousand-Lives}}
\item
  Life After Bob - shuruqtramontini, accessed December 10, 2025,
  \href{https://shuruqtramontini.com/Life-After-Bob}{\ul{https://shuruqtramontini.com/Life-After-Bob}}
\item
  Insight: Ian Cheng\textquotesingle s \textquotesingle Life after BOB,
  The Chalice Study\textquotesingle{} - CLOT Magazine, accessed December
  10, 2025,
  \href{https://clotmag.com/news/insight-ian-chengs-life-after-bob-the-chalice-study-at-halle-am-berghain}{\ul{https://clotmag.com/news/insight-ian-chengs-life-after-bob-the-chalice-study-at-halle-am-berghain}}
\item
  Art in the Age of the Metaverse - Rijksakademie, accessed December 10,
  2025,
  \href{https://www.rijksakademie.nl/en/public-programme/2023-03-10-art-in-the-age-of-the-metaverse}{\ul{https://www.rijksakademie.nl/en/public-programme/2023-03-10-art-in-the-age-of-the-metaverse}}
\item
  Q\&A - Ian Cheng - The CCAM Maquette, accessed December 10, 2025,
  \href{https://yalemaquette.com/Q-A-Ian-Cheng}{\ul{https://yalemaquette.com/Q-A-Ian-Cheng}}
\item
  BOB (Bag of Beliefs) - Ian Cheng, accessed December 10, 2025,
  \href{https://iancheng.com/BOB}{\ul{https://iancheng.com/BOB}}
\item
  Meet Shuruq Tramontini - Voyage LA Magazine \textbar{} LA City Guide,
  accessed December 10, 2025,
  \href{https://voyagela.com/interview/meet-shuruq-tramontini-of-los-feliz-silver-lake-east-hollywood/}{\ul{https://voyagela.com/interview/meet-shuruq-tramontini-of-los-feliz-silver-lake-east-hollywood/}}
\item
  Fear and Wonder 3: Futures of AI Symposium - SCI-Arc, accessed
  December 10, 2025,
  \href{https://www.sciarc.edu/events/events/fear-and-wonder-3}{\ul{https://www.sciarc.edu/events/events/fear-and-wonder-3}}
\item
  Life After BOB, accessed December 10, 2025,
  \href{https://lifeafterbob.io/}{\ul{https://lifeafterbob.io/}}
\item
  Final Vinzenz Thesis Draft - v250513 08:52 - Unlinked - DSpace@MIT,
  accessed December 10, 2025,
  \href{https://dspace.mit.edu/bitstream/handle/1721.1/163535/aubry-vinzenz-smact-act-2025-thesis.pdf?sequence=1&isAllowed=y}{\ul{https://dspace.mit.edu/bitstream/handle/1721.1/163535/aubry-vinzenz-smact-act-2025-thesis.pdf?sequence=1\&isAllowed=y}}
\item
  Dissertation Abstracts \textbar{} Discover - IDSVA, accessed December
  10, 2025,
  \href{https://www.idsva.edu/discover/dissertation-abstracts}{\ul{https://www.idsva.edu/discover/dissertation-abstracts}}
\item
  Empathy Is an Open Circuit: Ian Cheng \textbar{} Spike Art Magazine,
  accessed December 10, 2025,
  \href{https://spikeartmagazine.com/articles/empathy-is-an-open-circuit-an-interview-with-ian-cheng}{\ul{https://spikeartmagazine.com/articles/empathy-is-an-open-circuit-an-interview-with-ian-cheng}}
\item
  Ian Cheng \textbar{} Life After BOB: The Chalice Study - Pilar
  Corrias, accessed December 10, 2025,
  \href{https://www.pilarcorrias.com/exhibitions/252-ian-cheng-life-after-bob-the-chalice-study/}{\ul{https://www.pilarcorrias.com/exhibitions/252-ian-cheng-life-after-bob-the-chalice-study/}}
\item
  圣杯实验- 电影- 豆瓣, accessed December 10, 2025,
  \href{https://m.douban.com/movie/subject/35617779/}{\ul{https://m.douban.com/movie/subject/35617779/}}
\item
  Ian Cheng\textquotesingle s thought-provoking AI-based art on show at
  Gladstone Gallery - The Korea Herald, accessed December 10, 2025,
  \href{https://www.koreaherald.com/article/3355571}{\ul{https://www.koreaherald.com/article/3355571}}
\item
  I feel like I\textquotesingle m in the game. Unlike its predecessor,
  however, it is not the vast number of character.. - MK, accessed
  December 10, 2025,
  \href{https://www.mk.co.kr/en/culture/10950432}{\ul{https://www.mk.co.kr/en/culture/10950432}}
\item
  Ian Cheng: Thousand Lives (Mil vidas) by mataderomadrid - Issuu,
  accessed December 10, 2025,
  \href{https://issuu.com/mataderomadrid/docs/ian_cheng-programa}{\ul{https://issuu.com/mataderomadrid/docs/ian\_cheng-programa}}
\item
  Ian Cheng: Thousand Lives (Mil vidas) en Medialab Matadero - Time Out,
  accessed December 10, 2025,
  \href{https://www.timeout.es/madrid/es/arte/ian-cheng-thousand-lives-mil-vidas}{\ul{https://www.timeout.es/madrid/es/arte/ian-cheng-thousand-lives-mil-vidas}}
\item
  Shuruq Tramontini: 5 Things I Wish Someone Told Me When I First Became
  A Content Creator \textbar{} by Authority Magazine Editorial Staff -
  Medium, accessed December 10, 2025,
  \href{https://medium.com/authority-magazine/shuruq-tramontini-5-things-i-wish-someone-told-me-when-i-first-became-a-content-creator-8fb9e948f658}{\ul{https://medium.com/authority-magazine/shuruq-tramontini-5-things-i-wish-someone-told-me-when-i-first-became-a-content-creator-8fb9e948f658}}
\item
  Art in the Age of Metaverse, accessed December 10, 2025,
  \href{https://www.ageofmetaverse.art/}{\ul{https://www.ageofmetaverse.art/}}
\item
  M.S. Fiction and Entertainment - SCI-Arc, accessed December 10, 2025,
  \href{https://www.sciarc.edu/academics/postgraduate/fiction-and-entertainment}{\ul{https://www.sciarc.edu/academics/postgraduate/fiction-and-entertainment}}
\end{enumerate}
