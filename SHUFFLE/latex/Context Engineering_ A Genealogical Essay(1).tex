\section{CONTEXT ENGINEERING: AN INTELLECTUAL GENEALOGY OF THE VOID
MANAGEMENT
THESIS}\label{context-engineering-an-intellectual-genealogy-of-the-void-management-thesis}

\subsection{1. APEX: THE GENEALOGICAL
CLAIM}\label{apex-the-genealogical-claim}

The discipline of Context Engineering, crystallizing in the mid-2020s as
the successor to the ad-hoc practice of prompt engineering, represents a
profound convergence of intellectual traditions spanning over two
millennia. While contemporary discourse frames it as a technical
optimization of Large Language Model (LLM) inference---the "delicate art
and science of filling the context window," as defined by Andrej
Karpathy \textsuperscript{1}---a rigorous genealogical analysis reveals
it to be the re-emergence of fundamental questions regarding situated
cognition, semiotic density, and systems theory. The central claim of
this report is that \textbf{Context Engineering is the convergence of
three distinct lineages: the Situated Cognition tradition (Suchman,
Hutchins), the Thick Description tradition (Geertz, Bateson), and the
pragmatic engineering of the Prompt (Karpathy, Lütke).}

This convergence culminates in the \textbf{Void Management Thesis}, a
design synthesis that treats the context window not merely as a passive
receptacle for text, but as a bounded, material resource---a "jar" that
organizes the wilderness of information. The thesis asserts that
intelligence is not an inherent property of the model\textquotesingle s
weights but an emergent property of the interaction between the model
and its engineered context. The limitations of the medium---the
finiteness of the window, the cost of the token, the decay of
attention---are the very constraints that enable meaning to emerge.

This document traces the intellectual lineage of this thesis, moving
from the present-day synthesis back through the convergence of the late
20th century, the origins in mid-century cybernetics, and finally to the
prehistoric philosophical roots in Aristotelian ethics and
phenomenology. It serves as a self-contained intellectual history for
the context engineer, grounding technical practice in theoretical depth.

\subsection{2. LEVEL 1: SYNTHESIS (Void Management as
Culmination)}\label{level-1-synthesis-void-management-as-culmination}

The Void Management Thesis is the contemporary design synthesis of fifty
years of context research. It posits that the primary task of the AI
engineer is no longer the training of the model (the creation of static
weights) but the curation of the dynamic void---the active,
inference-time environment in which intelligence is situated.

\subsubsection{The Materiality of
Context}\label{the-materiality-of-context}

The foundational axiom of Void Management is that \textbf{context is
material}. In the computational realm of the LLM, context has weight,
cost, decay rate, and rigid engineering constraints. It is not an
abstract "meaning" but a physical state of the system, measured in
tokens and processed in GPU memory.

\paragraph{The Economics of the Token}\label{the-economics-of-the-token}

As Andrej Karpathy articulated in 2025, the context window functions as
the Random Access Memory (RAM) of the cognitive operating system, while
the LLM serves as the Central Processing Unit (CPU).\textsuperscript{1}
In this architecture, the token is the fundamental economic unit. Every
token introduced into the "void" (the context window) incurs a cost in
compute cycles (attention is quadratic in many architectures) and,
crucially, in "attention budget."

Modern research indicates that as context grows, models tend to allocate
attention to immediate information at the expense of parametric
knowledge or complex reasoning.\textsuperscript{1} This phenomenon,
often termed "context distraction" or "lost in the middle," underscores
the materiality of the void. It is a finite resource where information
competes for survival. The "pressure" on the context window is a
quantifiable metric of the cognitive load placed on the system.

The code snippet referencing L0: Void Management and self.pressure = 0.0
\textsuperscript{2} provides a technical reification of this theoretical
stance. Here, "pressure" is likely a calculated value derived from the
ratio of used tokens to the total window size, or perhaps a measure of
the semantic divergence within the context. "Void Management," in this
technical implementation, is the algorithmic regulation of this
pressure---ensuring that the density of information remains within the
"habitable zone" for the model\textquotesingle s reasoning capabilities.

\paragraph{The Mining Metaphor: Structural
Integrity}\label{the-mining-metaphor-structural-integrity}

The term "Void Management" is borrowed from the mining industry, where
it refers to the management of underground spaces (stopes) created by
the extraction of ore.\textsuperscript{3} In mining, a void is a
structural risk; if not managed (e.g., by backfilling), it can lead to
"stope failure" or collapse.

The metaphor is strikingly precise for Context Engineering. The "void"
of the context window is created by the "extraction" of the
user\textquotesingle s intent. If this void is not carefully
managed---if it is filled with "garbage" or left structurally
unsound---the result is "cognitive collapse," manifested as
hallucination. The "geotechnical assessment" required in mining
\textsuperscript{4} maps to the "eval pipelines" of context
engineering.\textsuperscript{5} Just as a mine manager must assess the
stability of the rock face, the context engineer must assess the
semantic stability of the prompts and retrieved documents. A "failed
stope" in a mine exposes personnel to risk; a "failed context" in an AI
agent exposes the user to erroneous actions or security vulnerabilities
(context poisoning).\textsuperscript{6}

\paragraph{The Housing Metaphor: Process and
Turnover}\label{the-housing-metaphor-process-and-turnover}

A second, equally relevant source for the "Void Management" term comes
from social housing administration.\textsuperscript{7} In this domain,
void management is the process of managing a property between
tenancies---from the termination of one tenant to the re-letting to
another. It involves cleaning, repairing, and ensuring the space is
"habitable."

In the lifecycle of an AI agent, particularly a long-running one, "void
management" refers to the "context clearing" or "state reset" that must
occur between tasks. When an agent finishes a complex coding task
(Tenant A) and moves to a creative writing task (Tenant B), the context
window must be "cleaned." Residual tokens from the previous task (the
"trash" left by the previous tenant) can contaminate the new inference,
leading to "mode confusion" or privacy leaks. The "Void Management
Policy" of a housing authority \textsuperscript{9} finds its parallel in
the "Context Retention Policy" of an AI system---defining what is kept
(long-term memory), what is archived (to disk/vector DB), and what is
deleted.

\subsubsection{Design Principle: The Jar and the
Wilderness}\label{design-principle-the-jar-and-the-wilderness}

The central design principle of the Void Management Thesis is the
distinction between the \textbf{Wilderness} and the \textbf{Jar}.

\begin{itemize}
\item
  \textbf{The Wilderness:} This is the vast, unorganized ocean of
  information available to the system---the "Knowledge
  Context".\textsuperscript{1} It includes the entirety of the internet,
  the user\textquotesingle s complete file history, and the millions of
  vectors in a database. It is "thick" in the sense of being voluminous,
  but "thin" in the sense of being unorganized. It is the "background"
  (Winograd) or the "field" (Garfinkel).
\item
  \textbf{The Jar:} This is the context window itself. It is a bounded,
  rigid container. It has a hard limit (e.g., 128k tokens).
\end{itemize}

The act of engineering is the act of filling the jar from the
wilderness. The "Void Management" thesis asserts that \textbf{unbounded
context is unmanageable context}.\textsuperscript{10} If one attempts to
pour the entire wilderness into the jar, the system overflows (out of
memory) or becomes diluted (attention dispersion).

The "Jar" organizes the wilderness by imposing structure. This structure
is often a "grid"---a schema, a template, or a set of few-shot
examples.\textsuperscript{10} The "grid" allows for \textbf{fixed
structure with variable instantiation}. The prompts (the structure of
the jar) remain constant; the retrieved data (the contents) flow through
it. This aligns with the "Offloading" strategy described by the Manus
team \textsuperscript{1}, where the file system acts as unlimited
external memory (the wilderness), achieving 100:1 compression ratios.
Only the "summary" or "metadata" enters the jar. The key insight is
\textbf{reversibility}: one can always retrieve the full content from
the wilderness using the reference in the jar, but one cannot fit the
wilderness in the jar.

\subsubsection{Historical Position}\label{historical-position}

The Void Management Thesis sits at the apex of the pyramid because it
synthesizes the insights of the past into a pragmatic discipline for the
future. It is where \textbf{Situated Cognition} (the understanding that
intelligence is local) meets \textbf{Prompt Engineering} (the technical
capability to manipulate the locus of intelligence). It validates
\textbf{Thick Description} by proving that "meaning" requires a dense,
curated context, and it operationalizes \textbf{Systems Thinking} by
treating the context window as a feedback-controlled system.

\subsection{3. LEVEL 2: CONVERGENCE
(1990s-2020s)}\label{level-2-convergence-1990s-2020s}

The transition from the 20th to the 21st century marked a convergence of
three distinct streams of thought: the computational pragmatism of
modern AI, the interaction design philosophy of the 1980s, and the
critique of rationalism in computer science. This convergence laid the
immediate groundwork for Context Engineering.

\subsubsection{Karpathy's Prompt Engineering
(2023-2025)}\label{karpathys-prompt-engineering-2023-2025}

The most visible vector of this convergence is the rapid evolution of
"prompt engineering" into "context engineering." In the early 2020s,
with the advent of GPT-3, the focus was on "prompting"---the art of
asking the right question.\textsuperscript{1} By 2025, industry leaders
like Andrej Karpathy and Tobi Lütke crystallized the shift.

The Shift: "Context engineering is the new prompt engineering".1

The Mechanism: Karpathy's formulation of the LLM as an Operating System
provided the necessary mechanistic metaphor.1

\begin{itemize}
\item
  \textbf{CPU:} The LLM (weights + inference engine).
\item
  \textbf{RAM:} The Context Window (working memory).
\item
  \textbf{Disk:} Vector Database / File System (long-term storage).
\end{itemize}

This metaphor legitimized the "void" as a programmable space. It moved
the discipline from "whispering" to the model (a mystical art) to
"memory management" (an engineering science). The "delicate art and
science of filling the context window" \textsuperscript{1} is
fundamentally an optimization problem: maximizing the
\emph{signal-to-noise ratio} of the tokens in RAM.

\textbf{Evidence of Convergence:}

\begin{itemize}
\item
  \textbf{Tobi Lütke (June 19, 2025):} "Context engineering is the art
  of providing all the context for the task to be plausibly solvable by
  the LLM".\textsuperscript{13}
\item
  \textbf{Andrej Karpathy (June 25, 2025):} "Context engineering is the
  delicate art and science of filling the context window with just the
  right information for the next step".\textsuperscript{13}
\item
  \textbf{Harrison Chase (June 23, 2025):} "Context engineering is
  building dynamic systems to provide the right
  info".\textsuperscript{13}
\end{itemize}

This synchronization of terminology among industry leaders indicates a
"paradigm shift" (Kuhn) where the community collectively recognized that
the "Prompt" was too narrow a unit of analysis. The "Context"---the
entire system state---became the new atomic unit.

\subsubsection{Winograd \& Flores: The Breakdown of Rationalism
(1986)}\label{winograd-flores-the-breakdown-of-rationalism-1986}

Decades prior to Karpathy, Terry Winograd and Fernando Flores published
\emph{Understanding Computers and Cognition}.\textsuperscript{15} Their
work, deeply influenced by Heideggerian phenomenology, launched a
critique of the "rationalistic tradition" in AI---the idea that
intelligence consists of manipulating symbolic representations of the
world.

Computers as Tools for Conversation:

They argued that computers are fundamentally tools for conversation and
action, situated within a social context.15 This prefigures the "Chat"
interface of modern LLMs. The "conversation" is not just text exchange;
it is the generation of commitments and actions.

Breakdown:

Winograd and Flores introduced the concept of "breakdown." Tools are
"ready-to-hand" (invisible) when functioning smoothly, but become
"present-at-hand" (visible objects of study) when they fail or "break
down".17

\begin{itemize}
\item
  \textbf{Ready-to-hand:} When an LLM agent works perfectly, the user
  ignores the prompt and focuses on the result. The context is
  transparent.
\item
  \textbf{Breakdown:} A "hallucination" is a breakdown. It is a moment
  where the seamless flow of context is ruptured, revealing the
  artificiality of the system.
\end{itemize}

Design Implications:

Their assertion that "design is ontological"---that designing tools
designs new ways of being---prefigures the Void Management view that
engineering context is engineering the being of the AI agent. The
context engineer does not just supply data; they structure the
"blindness" and "sight" of the model, determining what is obvious and
what is invisible.17 By filtering the wilderness, the engineer decides
what constitutes the "world" for the agent.

\subsubsection{Norman's Affordances and Constraints
(1988)}\label{normans-affordances-and-constraints-1988}

Donald Norman's \emph{The Psychology of Everyday Things} (later
\emph{The Design of Everyday Things}) provided the third pillar of this
convergence.\textsuperscript{18}

Affordances:

Norman popularized the concept of "affordances"---the action
possibilities perceived by an actor in an environment. In the physical
world, a door plate affords pushing. In the digital void of an LLM, a
structured JSON schema affords precise data extraction. A
conversational, open-ended prompt affords creative rambling.

The context engineer designs the affordances of the window. If the
engineer wants the model to use a tool, they must provide a tool
definition (affordance) that is "visible" and "perceptible" to the
model.20

Constraints:

Norman emphasized that constraints are as important as affordances.
Constraints limit behavior to prevent error.

\begin{itemize}
\item
  \textbf{Physical Constraints:} The size of the context window (e.g.,
  128k tokens).
\item
  \textbf{Semantic Constraints:} Instructions like "Answer only in
  JSON."
\item
  \textbf{Cultural Constraints:} The "persona" or tone instructions.
\end{itemize}

The "Void Management" thesis adopts Norman's view: the wilderness is
full of false affordances (irrelevant data); the engineered context must
\emph{constrain} the model to the path of success. Meaning emerges from
these constraints.

\subsubsection{Convergence Point}\label{convergence-point}

\textbf{All three traditions recognize that meaning emerges from
constraints, not from unbounded possibility.}

\begin{itemize}
\item
  \textbf{Karpathy:} Bounded memory (RAM) forces prioritization of
  tokens.\textsuperscript{1}
\item
  \textbf{Winograd:} Bounded situation ("thrownness") forces localized
  action.\textsuperscript{15}
\item
  \textbf{Norman:} Bounded interface forces correct
  usage.\textsuperscript{18}
\end{itemize}

The Void Management thesis is the practical application of this
tripartite realization.

\subsection{4. LEVEL 3: TRADITIONS (Three
Lineages)}\label{level-3-traditions-three-lineages}

Beneath the immediate convergence of the modern era lie three deep
intellectual traditions that provide the theoretical substance for
Context Engineering: Situated Cognition, Thick Description, and Systems
Thinking.

\subsubsection{Tradition 1: Situated
Cognition}\label{tradition-1-situated-cognition}

The Situated Cognition tradition, emerging in the 1980s and 1990s,
fundamentally challenged the computational theory of mind. It argued
that thinking is not a process that happens "inside the head" (or the
CPU) but is inextricably linked to the physical and social context.

\paragraph{Suchman: Plans and Situated Actions
(1987)}\label{suchman-plans-and-situated-actions-1987}

Lucy Suchman's seminal work \textsuperscript{21} dismantled the idea
that human action is driven by abstract "plans" executed by a cognitive
processor. Through her ethnographic study of Xerox photocopier usage,
she demonstrated that plans are not controlling programs but distinct
\textbf{resources} used by actors to account for their actions
\emph{after} or \emph{during} the fact. Action is "situated"---it is an
improvisation based on the immediate material circumstances.

Implication for Context Engineering:

This is a radical insight for AI. It implies that the "System Prompt"
(the plan) is not a rigid controller of the LLM. Instead, the "situated
action" of the model depends on the immediate tokens present in the
window (the situation). The model does not "follow" the plan in a
deterministic sense; it uses the plan as a resource to navigate the
current context.

\begin{itemize}
\item
  \textbf{Context Drift:} This explains why context drift occurs. As the
  "situation" (the conversation history) grows and changes, it exerts
  more gravitational pull on the model\textquotesingle s actions than
  the abstract plan (system prompt).\textsuperscript{1} The immediate
  situation \emph{is} the controller.
\item
  \textbf{Link to Void:} The bounded space of the window \emph{is} the
  plan. The engineer must ensure that the "situation" (the tokens in the
  window) always aligns with the desired "plan."
\end{itemize}

\paragraph{Lave: Cognition in Practice
(1988)}\label{lave-cognition-in-practice-1988}

Jean Lave's research on arithmetic in everyday life (e.g., grocery
shopping) showed that people do not use abstract school-taught math
algorithms in the wild.\textsuperscript{24} Instead, they use the
environment (the context) to perform the calculation---for example,
physically comparing package sizes rather than calculating
price-per-ounce mentally.

Implication for Context Engineering:

This validates the "Offloading" strategy.1 Just as Lave's shoppers
offloaded cognitive effort into the physical arrangement of groceries,
context engineers offload token-heavy tasks to external tools
(calculators, code interpreters), bringing only the result into the
cognitive focus of the model.

\begin{itemize}
\item
  \textbf{Cognition is Distributed:} Thinking happens between the
  weights and the window. The "intelligence" of the system is not just
  in the model; it is in the \emph{interaction} between the model and
  the engineered context.
\end{itemize}

\paragraph{Hutchins: Cognition in the Wild
(1995)}\label{hutchins-cognition-in-the-wild-1995}

Edwin Hutchins extended this to "distributed
cognition".\textsuperscript{26} His study of ship navigation teams
revealed that the "intelligence" of the ship was not in the captain's
head but distributed across charts, alidades, and the communication
protocols of the crew.

Implication for Context Engineering:

The "agent" is a distributed cognitive system.

\begin{itemize}
\item
  \textbf{Cognitive Artifacts:} The charts and rulers of the ship map to
  the \textbf{"few-shot examples"} and \textbf{"structured templates"}
  in the context window.\textsuperscript{10} These are artifacts that
  guide reasoning.
\item
  \textbf{Propagation of State:} Hutchins described how a navigational
  state propagates through different representations (from landmark to
  bearing to line on a chart). Similarly, a context engineer designs the
  propagation of state from "User Query" to "Search Term" to "Retrieved
  Document" to "Final Answer".\textsuperscript{28}
\end{itemize}

\subsubsection{Tradition 2: Thick
Description}\label{tradition-2-thick-description}

If Situated Cognition tells us \emph{where} thinking happens, Thick
Description tells us \emph{what} makes information meaningful.

\paragraph{Geertz: The Interpretation of Cultures
(1973)}\label{geertz-the-interpretation-of-cultures-1973}

Clifford Geertz coined the term "thick description" to distinguish
between the raw physical observation of an act (a "twitch") and its
cultural meaning (a "wink").\textsuperscript{29} A thin description
records the contraction of the eyelid; a thick description records the
conspiracy, the satire, or the flirtation.

Implication for Context Engineering:

In the era of Prompt Engineering, "thin description" was the norm:
short, direct instructions ("Write a poem"). Context Engineering demands
"thick description."

\begin{itemize}
\item
  \textbf{The Grid Preserves Thickness:} To get a high-quality output
  from an LLM, the engineer must provide the "cultural" context: the
  persona, the tone, the constraints, the history, and the
  intent.\textsuperscript{5} The "grid" of the context window must be
  populated with enough "thickness" (examples, behavioral guidelines,
  nuances) to allow the model to distinguish a twitch from a wink.
\item
  \textbf{Addressability:} The challenge is to make this thickness
  \emph{addressable} by the model. Unstructured thickness is noise.
  Structured thickness (Thick Description via JSON/XML) is signal.
\end{itemize}

\paragraph{Bateson: Steps to an Ecology of Mind
(1972)}\label{bateson-steps-to-an-ecology-of-mind-1972}

Gregory Bateson defined information as "\textbf{a difference that makes
a difference}".\textsuperscript{32} In a system of infinite data (the
wilderness), most differences make no difference---they are noise.

Implication for Context Engineering:

This is the fundamental heuristic for "pruning" the context.34 When
filling the "jar," every piece of information must be scrutinized: does
this token constitute a "difference that makes a difference" to the
model\textquotesingle s output? If not, it is noise that degrades the
system\textquotesingle s performance (increasing pressure without
increasing signal).

\begin{itemize}
\item
  \textbf{Schismogenesis:} Bateson's concept of "schismogenesis"
  (feedback loops that lead to breakdown) warns of the dangers of
  self-reinforcing errors in LLM dialogue loops. A small hallucination,
  if fed back into the context, can amplify into a complete divergence
  from reality.\textsuperscript{33}
\end{itemize}

\subsubsection{Tradition 3: Systems
Thinking}\label{tradition-3-systems-thinking}

The third lineage provides the mechanisms for control and stability.

\paragraph{Meadows: Thinking in Systems
(2008)}\label{meadows-thinking-in-systems-2008}

Donella Meadows taught that systems are defined by their
\textbf{boundaries} and their \textbf{feedback
loops}.\textsuperscript{35} A system without feedback cannot
self-correct.

\textbf{Implication for Context Engineering:}

\begin{itemize}
\item
  \textbf{Leverage Points:} Meadows' concept of "leverage
  points"---places in a system where a small change produces a large
  effect---is the holy grail of context optimization. A single
  instruction in the System Prompt (a high-leverage point) can alter the
  behavior of the entire agent more effectively than thousands of tokens
  of few-shot examples (a lower leverage point).\textsuperscript{36}
\item
  \textbf{The Bound is the System:} "81 cells is the bounded system"
  refers to the concept that a system is defined by what is
  \emph{inside} the boundary. In Void Management, the 128k token window
  (or similar limit) defines the universe of the agent. What is outside
  does not exist for the agent.
\end{itemize}

\paragraph{Ashby: Introduction to Cybernetics
(1956)}\label{ashby-introduction-to-cybernetics-1956}

Ross Ashby's \textbf{Law of Requisite Variety} states that "only variety
can destroy variety".\textsuperscript{37} To control a system, the
controller must have as many states as the system being controlled.

Implication for Context Engineering:

This explains the failure of simple prompts for complex tasks. A complex
real-world task (high variety) cannot be controlled by a simple,
low-variety prompt. The context window must contain "requisite
variety"---enough tools, knowledge, and contingency plans---to match the
complexity of the user\textquotesingle s request.

\begin{itemize}
\item
  \textbf{Control through Constraint:} The Void Management thesis is
  essentially the management of Requisite Variety within the constraints
  of the context window.
\end{itemize}

\subsection{5. LEVEL 4: ORIGINS
(1970s-1980s)}\label{level-4-origins-1970s-1980s}

The traditions of the 1990s did not emerge in a vacuum; they were built
upon the foundational work of the 1970s and 80s in cybernetics,
ethnomethodology, and activity theory.

\subsubsection{Cybernetics (Wiener,
Ashby)}\label{cybernetics-wiener-ashby}

Cybernetics, the study of control and communication in the animal and
the machine, is the grandfather of Context Engineering. Norbert Wiener
and Ross Ashby established the principles of feedback and control that
underpin modern "agentic" AI.\textsuperscript{37}

Control through Constraint:

The central cybernetic insight is that control is achieved through
constraint. A thermostat controls temperature not by "understanding"
heat, but by constraining the system within boundaries. Similarly,
context engineering does not teach the LLM to "understand" in a human
sense; it controls the model\textquotesingle s probabilistic generation
by imposing constraints (context) that narrow the search space of the
next token. The "Void" is the state space; the context engineer erects
the cybernetic boundaries that keep the trajectory of the system within
the desired region (the "attractor").

\subsubsection{Ethnomethodology
(Garfinkel)}\label{ethnomethodology-garfinkel}

Harold Garfinkel's \emph{Studies in Ethnomethodology} (1967) introduced
the concept of \textbf{"indexicality"}---the idea that the meaning of
any word or action is indexed to its specific, local
context.\textsuperscript{39} There is no context-free meaning.

Meaning is Locally Produced:

For LLMs, this is a literal truth. The meaning of a token is
mathematically derived from its attention to every other token in the
sequence.

\begin{itemize}
\item
  \textbf{Accountability:} Garfinkel's work on "accountability"---how
  members of a society make their actions visibly rational and
  reportable---is crucial for AI transparency. A well-engineered context
  forces the agent to produce a "Chain of Thought" (an account) that
  makes its reasoning visible and indexical. The "grid" of the Void
  Management system preserves the indexical links between the
  user\textquotesingle s intent and the model\textquotesingle s action.
\end{itemize}

\subsubsection{Activity Theory (Vygotsky,
Leontiev)}\label{activity-theory-vygotsky-leontiev}

Activity Theory, rooted in Soviet psychology, offers the most robust
model for tool use. Lev Vygotsky \textsuperscript{41} and A.N. Leontiev
\textsuperscript{42} argued that human cognition is mediated by tools
(both physical tools and psychological tools like language).

Zone of Proximal Development (ZPD):

Vygotsky proposed the ZPD as the distance between what a learner can do
alone and what they can do with help.43

\begin{itemize}
\item
  \textbf{Context as Scaffolding:} Context Engineering effectively
  places the LLM in a ZPD. The "System Prompt" serves as the "more
  knowledgeable other," scaffolding the model\textquotesingle s
  performance. The model, which might fail at a complex task in
  isolation, succeeds when supported by the scaffolding of a rich
  context.
\end{itemize}

Hierarchical Structure of Activity:

Leontiev distinguished between "activity" (motivated by a need),
"action" (directed at a goal), and "operation" (automatic, unconscious
execution).42

\begin{itemize}
\item
  \textbf{Mapping to Agents:} Context engineering organizes the
  LLM\textquotesingle s work into these levels.

  \begin{itemize}
  \item
    \textbf{Activity:} The user\textquotesingle s high-level intent
    (e.g., "Write a novel").
  \item
    \textbf{Action:} The specific tool calls (e.g., "Generate outline,"
    "Research setting").
  \item
    \textbf{Operation:} The token generations (the automatic output).
  \item
    \textbf{Failure Mode:} The failure of "prompt engineering" was often
    a failure to distinguish between these levels---giving an
    operation-level instruction for an activity-level problem.
  \end{itemize}
\end{itemize}

\subsection{6. LEVEL 5: PREHISTORY (Before
Computers)}\label{level-5-prehistory-before-computers}

The deepest roots of Context Engineering tap into the bedrock of Western
philosophy. The problems of context, practical wisdom, and the
relationship between the general rule and the specific case were
articulated long before the invention of the microchip.

\subsubsection{Aristotle --- Techne vs
Phronesis}\label{aristotle-techne-vs-phronesis}

In the \emph{Nicomachean Ethics}, Aristotle distinguishes between
different intellectual virtues.\textsuperscript{45}

\begin{itemize}
\item
  \textbf{Episteme:} Scientific knowledge, universal and invariable.
  (The "weights" of the model---parametric knowledge).
\item
  \textbf{Techne:} Craft or art, the skill of making things, governed by
  rules. (The code and infrastructure of the AI).
\item
  \textbf{Phronesis:} Practical wisdom. The ability to discern the right
  course of action in a \emph{specific, changing situation} where the
  rules do not perfectly apply.
\end{itemize}

Context Engineering as Computational Phronesis:

Context Engineering is the attempt to imbue a machine of Episteme and
Techne with a simulation of Phronesis. A "prompt" is often a request for
a general rule (Episteme). But a complex agentic task requires
Phronesis---navigating the "particulars" of the user\textquotesingle s
messy, contradictory context. The "Void Management" thesis is a design
for Phronesis: it creates a space where the particulars (the contents of
the jar) can weigh against the universals (the model\textquotesingle s
training).

\subsubsection{Rhetoric (Aristotle,
Cicero)}\label{rhetoric-aristotle-cicero}

The classical tradition of Rhetoric (Aristotle, Cicero) is the study of
context-dependent communication.\textsuperscript{47}

\begin{itemize}
\item
  \textbf{Kairos:} The opportune moment, the "right time" to speak. In
  Context Engineering, this is the "dynamic" injection of
  information---ensuring the model has the information \emph{exactly
  when it is relevant}, not before (distraction) and not after
  (irrelevance).
\item
  \textbf{Decorum:} The fitness of the speech to the occasion. A "System
  Prompt" that defines a persona is essentially an instruction in
  \emph{Decorum}---telling the model what tone and style befit the
  current "occasion" (context).
\end{itemize}

\subsubsection{Phenomenology (Husserl,
Heidegger)}\label{phenomenology-husserl-heidegger}

Martin Heidegger's \emph{Being and Time} \textsuperscript{49} provides
the ultimate ontological grounding.

\begin{itemize}
\item
  \textbf{Thrownness (Geworfenheit):} Heidegger argued that
  \emph{Dasein} ("Being-there") is always already "thrown" into a world.
  We are never outside of context.
\item
  \textbf{The LLM as Cartesian Subject:} The LLM in its base state is a
  Cartesian subject---a brain in a vat, disconnected from the world.
\item
  \textbf{Context Engineering as "Throwing":} Context Engineering is the
  process of "throwing" the LLM into a world. By filling the context
  window with a specific history, a specific user profile, and specific
  tools, the engineer gives the model a \emph{Dasein}. The model becomes
  "Situated" (Suchman) because it is "Thrown" (Heidegger) into a "World"
  (Context). The "Void" is the space where the World is constructed.
\end{itemize}

\subsubsection{James C. Scott: Seeing Like a
State}\label{james-c.-scott-seeing-like-a-state}

Finally, the work of James C. Scott in \emph{Seeing Like a State}
\textsuperscript{51} provides a political-philosophical lens.

\begin{itemize}
\item
  \textbf{Techne vs. Metis:} Scott contrasts \emph{Techne}
  (state-imposed, simplified, standardized knowledge) with \emph{Metis}
  (local, practical, cunning intelligence). High Modernist states try to
  make the world legible by imposing \emph{Techne} and erasing
  \emph{Metis}---often leading to disaster (e.g., scientific forestry
  failing because it ignored the complex ecology).
\item
  \textbf{LLMs as High Modernism:} LLMs are High Modernist engines of
  \emph{Techne}. They are trained on the "average" of human knowledge,
  flattening local nuance.
\item
  \textbf{Context Engineering as Metis:} Context Engineering is the
  re-injection of \emph{Metis}. By filling the context with the
  specific, messy details of the user\textquotesingle s local reality
  (the "thick description"), the engineer protects the user from the
  "blindness" of the state-like model. The Void Management thesis is a
  strategy for preserving \emph{Metis} inside the machine of
  \emph{Techne}.
\end{itemize}

\subsection{7. ARGUMENT EXTRACTION}\label{argument-extraction}

\subsubsection{For Intellectual
History:}\label{for-intellectual-history}

\begin{quote}
"Context engineering is the convergence of situated cognition, thick
description, and prompt engineering. The void management thesis is where
Suchman meets Karpathy."
\end{quote}

\subsubsection{For Theoretical Claims:}\label{for-theoretical-claims}

\begin{quote}
"All three traditions recognize that meaning emerges from constraints.
Plans, affordances, and prompts all work by bounding possibility, not
expanding it."
\end{quote}

\subsubsection{For Design Grounding:}\label{for-design-grounding}

\begin{quote}
"From cybernetics: control through constraint. From ethnomethodology:
meaning locally produced. From activity theory: tools mediate cognition.
The grid synthesizes all three."
\end{quote}

\subsection{8. INTAKE GRID FOR GENEALOGICAL
SOURCES}\label{intake-grid-for-genealogical-sources}

The following grid organizes the primary sources referenced in this
genealogy, mapping them to the levels of the Argument Pyramid.

\begin{longtable}[]{@{}
  >{\raggedright\arraybackslash}p{(\linewidth - 8\tabcolsep) * \real{0.2000}}
  >{\raggedright\arraybackslash}p{(\linewidth - 8\tabcolsep) * \real{0.2000}}
  >{\raggedright\arraybackslash}p{(\linewidth - 8\tabcolsep) * \real{0.2000}}
  >{\raggedright\arraybackslash}p{(\linewidth - 8\tabcolsep) * \real{0.2000}}
  >{\raggedright\arraybackslash}p{(\linewidth - 8\tabcolsep) * \real{0.2000}}@{}}
\toprule\noalign{}
\begin{minipage}[b]{\linewidth}\raggedright
\textbf{Era}
\end{minipage} & \begin{minipage}[b]{\linewidth}\raggedright
\textbf{Look For}
\end{minipage} & \begin{minipage}[b]{\linewidth}\raggedright
\textbf{Maps To}
\end{minipage} & \begin{minipage}[b]{\linewidth}\raggedright
\textbf{Key Figures}
\end{minipage} & \begin{minipage}[b]{\linewidth}\raggedright
\textbf{Key Concepts}
\end{minipage} \\
\begin{minipage}[b]{\linewidth}\raggedright
\textbf{2020s}
\end{minipage} & \begin{minipage}[b]{\linewidth}\raggedright
Prompt/Context Engineering
\end{minipage} & \begin{minipage}[b]{\linewidth}\raggedright
\textbf{CONVERGENCE}
\end{minipage} & \begin{minipage}[b]{\linewidth}\raggedright
Karpathy, Lütke, Chase
\end{minipage} & \begin{minipage}[b]{\linewidth}\raggedright
Token Economics, CPU/RAM Metaphor, Void Management
\end{minipage} \\
\begin{minipage}[b]{\linewidth}\raggedright
\textbf{1990s}
\end{minipage} & \begin{minipage}[b]{\linewidth}\raggedright
Situated Cognition, Distributed Cognition
\end{minipage} & \begin{minipage}[b]{\linewidth}\raggedright
\textbf{TRADITIONS}
\end{minipage} & \begin{minipage}[b]{\linewidth}\raggedright
Suchman, Hutchins, Lave
\end{minipage} & \begin{minipage}[b]{\linewidth}\raggedright
Plans as Resources, Distributed Cognition, Cognitive Artifacts
\end{minipage} \\
\begin{minipage}[b]{\linewidth}\raggedright
\textbf{1980s}
\end{minipage} & \begin{minipage}[b]{\linewidth}\raggedright
HCI, Design Theory
\end{minipage} & \begin{minipage}[b]{\linewidth}\raggedright
\textbf{CONVERGENCE}
\end{minipage} & \begin{minipage}[b]{\linewidth}\raggedright
Winograd, Flores, Norman
\end{minipage} & \begin{minipage}[b]{\linewidth}\raggedright
Breakdown, Ready-to-hand, Affordances, Constraints
\end{minipage} \\
\begin{minipage}[b]{\linewidth}\raggedright
\textbf{1970s}
\end{minipage} & \begin{minipage}[b]{\linewidth}\raggedright
Thick Description, Cybernetics, Systems
\end{minipage} & \begin{minipage}[b]{\linewidth}\raggedright
\textbf{TRADITIONS / ORIGINS}
\end{minipage} & \begin{minipage}[b]{\linewidth}\raggedright
Geertz, Bateson, Meadows
\end{minipage} & \begin{minipage}[b]{\linewidth}\raggedright
Thick Description, Ecology of Mind, Leverage Points
\end{minipage} \\
\begin{minipage}[b]{\linewidth}\raggedright
\textbf{1960s}
\end{minipage} & \begin{minipage}[b]{\linewidth}\raggedright
Ethnomethodology, Activity Theory
\end{minipage} & \begin{minipage}[b]{\linewidth}\raggedright
\textbf{ORIGINS}
\end{minipage} & \begin{minipage}[b]{\linewidth}\raggedright
Garfinkel, Vygotsky, Leontiev
\end{minipage} & \begin{minipage}[b]{\linewidth}\raggedright
Indexicality, Accountability, ZPD, Activity/Action/Operation
\end{minipage} \\
\begin{minipage}[b]{\linewidth}\raggedright
\textbf{Pre-Comp}
\end{minipage} & \begin{minipage}[b]{\linewidth}\raggedright
Philosophy, Rhetoric
\end{minipage} & \begin{minipage}[b]{\linewidth}\raggedright
\textbf{PREHISTORY}
\end{minipage} & \begin{minipage}[b]{\linewidth}\raggedright
Aristotle, Heidegger, Scott
\end{minipage} & \begin{minipage}[b]{\linewidth}\raggedright
Phronesis, Techne, Metis, Thrownness, Kairos
\end{minipage} \\
\midrule\noalign{}
\endhead
\bottomrule\noalign{}
\endlastfoot
\end{longtable}

\subsubsection{Table of Comparisons: Thin vs. Thick
Context}\label{table-of-comparisons-thin-vs.-thick-context}

\begin{longtable}[]{@{}
  >{\raggedright\arraybackslash}p{(\linewidth - 6\tabcolsep) * \real{0.2500}}
  >{\raggedright\arraybackslash}p{(\linewidth - 6\tabcolsep) * \real{0.2500}}
  >{\raggedright\arraybackslash}p{(\linewidth - 6\tabcolsep) * \real{0.2500}}
  >{\raggedright\arraybackslash}p{(\linewidth - 6\tabcolsep) * \real{0.2500}}@{}}
\toprule\noalign{}
\begin{minipage}[b]{\linewidth}\raggedright
\textbf{Concept}
\end{minipage} & \begin{minipage}[b]{\linewidth}\raggedright
\textbf{Thin Description (Prompt Engineering)}
\end{minipage} & \begin{minipage}[b]{\linewidth}\raggedright
\textbf{Thick Description (Context Engineering)}
\end{minipage} & \begin{minipage}[b]{\linewidth}\raggedright
\textbf{Theoretical Basis}
\end{minipage} \\
\begin{minipage}[b]{\linewidth}\raggedright
\textbf{Goal}
\end{minipage} & \begin{minipage}[b]{\linewidth}\raggedright
Elicit a specific response.
\end{minipage} & \begin{minipage}[b]{\linewidth}\raggedright
Maintain a consistent world/state.
\end{minipage} & \begin{minipage}[b]{\linewidth}\raggedright
Winograd (Ontology)
\end{minipage} \\
\begin{minipage}[b]{\linewidth}\raggedright
\textbf{Input}
\end{minipage} & \begin{minipage}[b]{\linewidth}\raggedright
A single instruction ("Write a poem").
\end{minipage} & \begin{minipage}[b]{\linewidth}\raggedright
A bounded system (Persona + History + Tools).
\end{minipage} & \begin{minipage}[b]{\linewidth}\raggedright
Ashby (Requisite Variety)
\end{minipage} \\
\begin{minipage}[b]{\linewidth}\raggedright
\textbf{Meaning}
\end{minipage} & \begin{minipage}[b]{\linewidth}\raggedright
Dictionary definition (Semantic).
\end{minipage} & \begin{minipage}[b]{\linewidth}\raggedright
Locally produced (Indexical).
\end{minipage} & \begin{minipage}[b]{\linewidth}\raggedright
Garfinkel (Indexicality)
\end{minipage} \\
\begin{minipage}[b]{\linewidth}\raggedright
\textbf{Error}
\end{minipage} & \begin{minipage}[b]{\linewidth}\raggedright
"Wrong answer."
\end{minipage} & \begin{minipage}[b]{\linewidth}\raggedright
"Hallucination" / "Breakdown."
\end{minipage} & \begin{minipage}[b]{\linewidth}\raggedright
Heidegger (Breakdown)
\end{minipage} \\
\begin{minipage}[b]{\linewidth}\raggedright
\textbf{Control}
\end{minipage} & \begin{minipage}[b]{\linewidth}\raggedright
Tweaking words ("Whispering").
\end{minipage} & \begin{minipage}[b]{\linewidth}\raggedright
Managing constraints ("bounding the void").
\end{minipage} & \begin{minipage}[b]{\linewidth}\raggedright
Norman (Constraints)
\end{minipage} \\
\begin{minipage}[b]{\linewidth}\raggedright
\textbf{Metaphor}
\end{minipage} & \begin{minipage}[b]{\linewidth}\raggedright
Asking a question.
\end{minipage} & \begin{minipage}[b]{\linewidth}\raggedright
Managing a mine / Managing a tenancy.
\end{minipage} & \begin{minipage}[b]{\linewidth}\raggedright
Void Management (Mining/Housing)
\end{minipage} \\
\midrule\noalign{}
\endhead
\bottomrule\noalign{}
\endlastfoot
\end{longtable}

\emph{This genealogy is self-contained. The lineage traces a clear path
from the ancient question of "how to act wisely in a specific situation"
to the modern engineering challenge of "how to fill the context window
for the next token."}

\paragraph{Works cited}\label{works-cited}

\begin{enumerate}
\def\labelenumi{\arabic{enumi}.}
\item
  Deep Dive into Context Engineering for Agents - Galileo AI, accessed
  December 10, 2025,
  \href{https://galileo.ai/blog/context-engineering-for-agents}{\ul{https://galileo.ai/blog/context-engineering-for-agents}}
\item
  4qx-holarchy/can-you-analyse-and-report-on-1739483038575, accessed
  December 10, 2025,
  \href{https://code.organicdesign.nz/organicdesign/4qx-holarchy/src/commit/ad69f4ebe0973bd2e824901ee75de08898875926/context/oracle-conversations/can-you-analyse-and-report-on-1739483038575-export.txt}{\ul{https://code.organicdesign.nz/organicdesign/4qx-holarchy/src/commit/ad69f4ebe0973bd2e824901ee75de08898875926/context/oracle-conversations/can-you-analyse-and-report-on-1739483038575-export.txt}}
\item
  Underground Operators Conference 2025 Proceedings - AusIMM, accessed
  December 10, 2025,
  \href{https://www.ausimm.com/globalassets/conferences-and-events/underground-operators-2027/underground-operators-2025/ugops-2025-proceedings-ebook-2.pdf}{\ul{https://www.ausimm.com/globalassets/conferences-and-events/underground-operators-2027/underground-operators-2025/ugops-2025-proceedings-ebook-2.pdf}}
\item
  WILPINJONG COAL PROJECT OPEN CUT OPERATIONS MINING OPERATIONS PLAN
  2019 -- 2020 - Peabody Energy, accessed December 10, 2025,
  \href{https://www.peabodyenergy.com/Peabody/media/MediaLibrary/Operations/Australia\%20Mining/New\%20South\%20Wales\%20Mining/Wilpinjong\%20Mine/20180117_MOP_Open-Cut_2019-2020_Amendment-A_FINAL.pdf}{\ul{https://www.peabodyenergy.com/Peabody/media/MediaLibrary/Operations/Australia\%20Mining/New\%20South\%20Wales\%20Mining/Wilpinjong\%20Mine/20180117\_MOP\_Open-Cut\_2019-2020\_Amendment-A\_FINAL.pdf}}
\item
  Context Engineering Guide, accessed December 10, 2025,
  \href{https://www.promptingguide.ai/guides/context-engineering-guide}{\ul{https://www.promptingguide.ai/guides/context-engineering-guide}}
\item
  A Gentle Introduction to Context Engineering in LLMs - KDnuggets,
  accessed December 10, 2025,
  \href{https://www.kdnuggets.com/a-gentle-introduction-to-context-engineering-in-llms}{\ul{https://www.kdnuggets.com/a-gentle-introduction-to-context-engineering-in-llms}}
\item
  (Public Pack)Agenda Document for Cabinet, 24/07/2024 14:00 - Meetings,
  agendas, and minutes - Stevenage Borough Council, accessed December
  10, 2025,
  \href{https://democracy.stevenage.gov.uk/documents/g5703/Public+reports+pack+Wednesday+24-Jul-2024+14.00+Cabinet.pdf?T=10}{\ul{https://democracy.stevenage.gov.uk/documents/g5703/Public+reports+pack+Wednesday+24-Jul-2024+14.00+Cabinet.pdf?T=10}}
\item
  150 Best learning officer Jobs in , London (November 2025) \textbar{}
  JOB, accessed December 10, 2025,
  \href{https://jobtoday.com/gb/jobs-learning-officer/london_near_}{\ul{https://jobtoday.com/gb/jobs-learning-officer/london\_near\_}}
\item
  (Public Pack)Agenda Document for Council, 23/04/2025 18:00, accessed
  December 10, 2025,
  \href{https://democracy.middevon.gov.uk/documents/g1972/Public\%20reports\%20pack\%2023rd-Apr-2025\%2018.00\%20Council.pdf?T=10}{\ul{https://democracy.middevon.gov.uk/documents/g1972/Public\%20reports\%20pack\%2023rd-Apr-2025\%2018.00\%20Council.pdf?T=10}}
\item
  davidkimai/Context-Engineering: "Context engineering is the delicate
  art and science of filling the context window with just the right
  information for the next step." --- Andrej Karpathy. A frontier,
  first-principles handbook inspired by Karpathy and 3Blue1Brown for
  moving beyond prompt engineering to the wider discipline of context
  design, orchestration - GitHub, accessed December 10, 2025,
  \href{https://github.com/davidkimai/Context-Engineering}{\ul{https://github.com/davidkimai/Context-Engineering}}
\item
  What Is Context Engineering? A Guide for AI \& LLMs \textbar{}
  IntuitionLabs, accessed December 10, 2025,
  \href{https://intuitionlabs.ai/articles/what-is-context-engineering}{\ul{https://intuitionlabs.ai/articles/what-is-context-engineering}}
\item
  Beyond prompt engineering: the shift to context engineering \textbar{}
  Nearform, accessed December 10, 2025,
  \href{https://nearform.com/digital-community/beyond-prompt-engineering-the-shift-to-context-engineering/}{\ul{https://nearform.com/digital-community/beyond-prompt-engineering-the-shift-to-context-engineering/}}
\item
  What is Context Engineering, Anyway? - Zep, accessed December 10,
  2025,
  \href{https://blog.getzep.com/what-is-context-engineering/}{\ul{https://blog.getzep.com/what-is-context-engineering/}}
\item
  Context Engineering: The Future of AI Systems \textbar{} by Meghana
  Harishankara \textbar{} Medium, accessed December 10, 2025,
  \href{https://medium.com/@meghanaharishankara/context-engineering-the-future-of-ai-systems-a52062c727f0}{\ul{https://medium.com/@meghanaharishankara/context-engineering-the-future-of-ai-systems-a52062c727f0}}
\item
  Understanding computers and cognition \textbar{} Semantic Scholar,
  accessed December 10, 2025,
  \href{https://www.semanticscholar.org/paper/Understanding-computers-and-cognition-Winograd-Flores/0ee153002e00e9a52eadac3ab0c93a5bd0582a8f}{\ul{https://www.semanticscholar.org/paper/Understanding-computers-and-cognition-Winograd-Flores/0ee153002e00e9a52eadac3ab0c93a5bd0582a8f}}
\item
  Terry Winograd home page - Publications - Google Sites, accessed
  December 10, 2025,
  \href{https://sites.google.com/view/terrywinogradhomepage/home/publications}{\ul{https://sites.google.com/view/terrywinogradhomepage/home/publications}}
\item
  T. Winograd and F. Flores, Understanding Computers and Cognition: A
  New Foundation for Design (Ablex, Norwood, NJ, 1986), accessed
  December 10, 2025,
  \href{https://billclancey.name/Review-Winograd-Flores-Clancey-AIJ1987.pdf}{\ul{https://billclancey.name/Review-Winograd-Flores-Clancey-AIJ1987.pdf}}
\item
  Norman, D. (1988). The Psychology of Everyday Things. New York Basic
  Books, 140. - References - Scientific Research Publishing, accessed
  December 10, 2025,
  \href{https://www.scirp.org/reference/referencespapers?referenceid=1487772}{\ul{https://www.scirp.org/reference/referencespapers?referenceid=1487772}}
\item
  The Design of Everyday Things - ICDST E-print archive of engineering
  and scientific PDF documents, accessed December 10, 2025,
  \href{https://dl.icdst.org/pdfs/files4/4bb8d08a9b309df7d86e62ec4056ceef.pdf}{\ul{https://dl.icdst.org/pdfs/files4/4bb8d08a9b309df7d86e62ec4056ceef.pdf}}
\item
  The Psychology of Everyday Things - Donald A. Norman - Google Books,
  accessed December 10, 2025,
  \href{https://books.google.com/books/about/The_Psychology_of_Everyday_Things.html?id=OlNSRAAACAAJ}{\ul{https://books.google.com/books/about/The\_Psychology\_of\_Everyday\_Things.html?id=OlNSRAAACAAJ}}
\item
  Human-Machine Reconfigurations: Plans and Situated Actions: 2nd
  Edition - ResearchGate, accessed December 10, 2025,
  \href{https://www.researchgate.net/publication/265092509_Human-Machine_Reconfigurations_Plans_and_Situated_Actions_2nd_Edition}{\ul{https://www.researchgate.net/publication/265092509\_Human-Machine\_Reconfigurations\_Plans\_and\_Situated\_Actions\_2nd\_Edition}}
\item
  {[}PDF{]} Plans and Situated Actions: The Problem of Human-Machine
  Communication (Learning in Doing: Social, \textbar{} Semantic Scholar,
  accessed December 10, 2025,
  \href{https://www.semanticscholar.org/paper/Plans-and-Situated-Actions\%3A-The-Problem-of-in-Suchman/5416463537f8c6be1199951b4fd6f8d5dae14920}{\ul{https://www.semanticscholar.org/paper/Plans-and-Situated-Actions\%3A-The-Problem-of-in-Suchman/5416463537f8c6be1199951b4fd6f8d5dae14920}}
\item
  Plans and situated actions : the problem of human-machine
  communication : Suchman, Lucille Alice : Free Download, Borrow, and
  Streaming - Internet Archive, accessed December 10, 2025,
  \href{https://archive.org/details/planssituatedact0000such}{\ul{https://archive.org/details/planssituatedact0000such}}
\item
  Cognition in Practice: Mind, Mathematics and Culture in Everyday Life
  - About Google Books, accessed December 10, 2025,
  \href{https://books.google.kg/books?id=-JD6ngEACAAJ&source=gbs_book_other_versions_r&cad=3}{\ul{https://books.google.kg/books?id=-JD6ngEACAAJ\&source=gbs\_book\_other\_versions\_r\&cad=3}}
\item
  Cognition in Practice: Mind, Mathematics and Culture in Everyday Life
  - Google Books, accessed December 10, 2025,
  \href{https://books.google.com/books/about/Cognition_in_Practice.html?id=n6eiH3iPVKYC}{\ul{https://books.google.com/books/about/Cognition\_in\_Practice.html?id=n6eiH3iPVKYC}}
\item
  Cognition in the wild / Edwin Hutchins \textbar{} Catalogue \textbar{}
  National Library of Australia, accessed December 10, 2025,
  \href{https://catalogue.nla.gov.au/catalog/580835}{\ul{https://catalogue.nla.gov.au/catalog/580835}}
\item
  Cognition in the Wild, Edwin Hutchins - EUSSET Digital Library,
  accessed December 10, 2025,
  \href{https://dl.eusset.eu/items/b0028a89-c1c2-430c-96c5-7066fa157e66}{\ul{https://dl.eusset.eu/items/b0028a89-c1c2-430c-96c5-7066fa157e66}}
\item
  ‪Edwin Hutchins‬ - ‪Google Scholar‬, accessed December 10, 2025,
  \href{https://scholar.google.com/citations?user=T5WQ2EcAAAAJ&hl=en}{\ul{https://scholar.google.com/citations?user=T5WQ2EcAAAAJ\&hl=en}}
\item
  Geertz, C. (1973). Thick Description Toward an Interpretive Theory of
  Culture. Basic Books. - References - Scientific Research Publishing,
  accessed December 10, 2025,
  \href{https://www.scirp.org/reference/referencespapers?referenceid=3712718}{\ul{https://www.scirp.org/reference/referencespapers?referenceid=3712718}}
\item
  What is Thick Description in Qualitative Research? - QDAcity, accessed
  December 10, 2025,
  \href{https://qdacity.com/thick-description/}{\ul{https://qdacity.com/thick-description/}}
\item
  Context Engineering: Techniques, Tools, and Implementation - iKala,
  accessed December 10, 2025,
  \href{https://ikala.ai/blog/ai-trends/context-engineering-techniques-tools-and-implementation/}{\ul{https://ikala.ai/blog/ai-trends/context-engineering-techniques-tools-and-implementation/}}
\item
  Bateson, G. (1972). Steps to an Ecology of Mind Collected Essays in
  Anthropology, Psychiatry, Evolution, and Epistemology. Chicago, IL
  University of Chicago Press. - References - Scientific Research
  Publishing, accessed December 10, 2025,
  \href{https://www.scirp.org/reference/referencespapers?referenceid=1348764}{\ul{https://www.scirp.org/reference/referencespapers?referenceid=1348764}}
\item
  Steps to an Ecology of Mind - Grokipedia, accessed December 10, 2025,
  \href{https://grokipedia.com/page/Steps_to_an_Ecology_of_Mind}{\ul{https://grokipedia.com/page/Steps\_to\_an\_Ecology\_of\_Mind}}
\item
  Context engineering: Best practices for an emerging discipline
  \textbar{} Redis, accessed December 10, 2025,
  \href{https://redis.io/blog/context-engineering-best-practices-for-an-emerging-discipline/}{\ul{https://redis.io/blog/context-engineering-best-practices-for-an-emerging-discipline/}}
\item
  Citation - Thinking in systems : a primer - Search UW-Madison
  Libraries, accessed December 10, 2025,
  \href{https://search.library.wisc.edu/catalog/9910100084402121/cite}{\ul{https://search.library.wisc.edu/catalog/9910100084402121/cite}}
\item
  Systems thinking - Wikipedia, accessed December 10, 2025,
  \href{https://en.wikipedia.org/wiki/Systems_thinking}{\ul{https://en.wikipedia.org/wiki/Systems\_thinking}}
\item
  CYBERNETICS - The W. Ross Ashby Digital Archive, accessed December 10,
  2025,
  \href{https://ashby.info/Ashby-Introduction-to-Cybernetics.pdf}{\ul{https://ashby.info/Ashby-Introduction-to-Cybernetics.pdf}}
\item
  An introduction to cybernetics. by William Ross Ashby - Open Library,
  accessed December 10, 2025,
  \href{https://openlibrary.org/books/OL6202924M/An_introduction_to_cybernetics.}{\ul{https://openlibrary.org/books/OL6202924M/An\_introduction\_to\_cybernetics.}}
\item
  Garfinkel, H. (1967). Studies in Ethnomethodology. Englewood Cliffs,
  NJ Prentice-Hall. - References - Scientific Research Publishing,
  accessed December 10, 2025,
  \href{https://www.scirp.org/reference/ReferencesPapers?ReferenceID=1808987}{\ul{https://www.scirp.org/reference/ReferencesPapers?ReferenceID=1808987}}
\item
  Garfinkel, H - Studies in Ethnomethodology (1967) \textbar{} PDF -
  Scribd, accessed December 10, 2025,
  \href{https://www.scribd.com/document/49927355/Garfinkel-H-Studies-in-Ethnomethodology-1967}{\ul{https://www.scribd.com/document/49927355/Garfinkel-H-Studies-in-Ethnomethodology-1967}}
\item
  Vygotsky, L. S. (1978). Mind in Society The Development of Higher
  Psychological Processes. Cambridge, MA Harvard University Press. -
  References, accessed December 10, 2025,
  \href{https://www.scirp.org/reference/referencespapers?referenceid=2107373}{\ul{https://www.scirp.org/reference/referencespapers?referenceid=2107373}}
\item
  Leont\textquotesingle ev, A. (1978). Activity, Consciousness, and
  Personality. Englewood Cliffs, NJ Prentice-Hall. - References -
  Scientific Research Publishing, accessed December 10, 2025,
  \href{https://www.scirp.org/reference/referencespapers?referenceid=1831210}{\ul{https://www.scirp.org/reference/referencespapers?referenceid=1831210}}
\item
  LS Vygotsky.; Mind in Society : Development of Higher Psychological
  Processes, accessed December 10, 2025,
  \href{https://w.pauldowling.me/rtf/2021.1/readings/LSVygotsky_1978_MindinSocietyDevelopmentofHigherPsycholo.pdf}{\ul{https://w.pauldowling.me/rtf/2021.1/readings/LSVygotsky\_1978\_MindinSocietyDevelopmentofHigherPsycholo.pdf}}
\item
  Activity, consciousness, and personality - Semantic Scholar, accessed
  December 10, 2025,
  \href{https://www.semanticscholar.org/paper/Activity\%2C-consciousness\%2C-and-personality-Leont\%E2\%80\%99ev-Hall/846dcce4ea934d56c452cc2ae6a118f916d8902b}{\ul{https://www.semanticscholar.org/paper/Activity\%2C-consciousness\%2C-and-personality-Leont\%E2\%80\%99ev-Hall/846dcce4ea934d56c452cc2ae6a118f916d8902b}}
\item
  Quick question - how do you cite Aristotle and the Nicomachean
  ethics?? : r/askphilosophy, accessed December 10, 2025,
  \href{https://www.reddit.com/r/askphilosophy/comments/4ipxwq/quick_question_how_do_you_cite_aristotle_and_the/}{\ul{https://www.reddit.com/r/askphilosophy/comments/4ipxwq/quick\_question\_how\_do\_you\_cite\_aristotle\_and\_the/}}
\item
  Aristotle: Nicomachean Ethics - ResearchGate, accessed December 10,
  2025,
  \href{https://www.researchgate.net/publication/325449937_Aristotle_Nicomachean_Ethics}{\ul{https://www.researchgate.net/publication/325449937\_Aristotle\_Nicomachean\_Ethics}}
\item
  Citation - Cicero on the ideal orator (De Oratore) - Search UW-Madison
  Libraries, accessed December 10, 2025,
  \href{https://search.library.wisc.edu/catalog/999908955302121/cite}{\ul{https://search.library.wisc.edu/catalog/999908955302121/cite}}
\item
  De Oratore - Wikipedia, accessed December 10, 2025,
  \href{https://en.wikipedia.org/wiki/De_Oratore}{\ul{https://en.wikipedia.org/wiki/De\_Oratore}}
\item
  Heidegger, M. (1962). Being and Time (J. Macquarrie, \& E. Robinson,
  Trans.). Oxford, UK ‎\& Cambridge, USA: Blackwell Publishers Ltd.
  (Original work published 1927) - ResearchGate, accessed December 10,
  2025,
  \href{https://www.researchgate.net/publication/335060851_Heidegger_M_1962_Being_and_Time_J_Macquarrie_E_Robinson_Trans_Oxford_UK_Cambridge_USA_Blackwell_Publishers_Ltd_Original_work_published_1927}{\ul{https://www.researchgate.net/publication/335060851\_Heidegger\_M\_1962\_Being\_and\_Time\_J\_Macquarrie\_E\_Robinson\_Trans\_Oxford\_UK\_Cambridge\_USA\_Blackwell\_Publishers\_Ltd\_Original\_work\_published\_1927}}
\item
  Being and Time - Wikipedia, accessed December 10, 2025,
  \href{https://en.wikipedia.org/wiki/Being_and_Time}{\ul{https://en.wikipedia.org/wiki/Being\_and\_Time}}
\item
  Citation: Seeing like a state - BibGuru Guides, accessed December 10,
  2025,
  \href{https://www.bibguru.com/b/how-to-cite-seeing-like-a-state/}{\ul{https://www.bibguru.com/b/how-to-cite-seeing-like-a-state/}}
\item
  Seeing Like a State \textbar{} Grantmakers in the Arts, accessed
  December 10, 2025,
  \href{https://www.giarts.org/article/seeing-state}{\ul{https://www.giarts.org/article/seeing-state}}
\end{enumerate}
