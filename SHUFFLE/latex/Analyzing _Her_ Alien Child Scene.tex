\section{\texorpdfstring{The Ontological Glitch: Ludic Scarcity, Digital
Abundance, and the Alien Child in Spike Jonze's
\emph{\textbf{Her}}}{The Ontological Glitch: Ludic Scarcity, Digital Abundance, and the Alien Child in Spike Jonze's Her}}\label{the-ontological-glitch-ludic-scarcity-digital-abundance-and-the-alien-child-in-spike-jonzes-her}

\subsection{1. Introduction: The Locus of the
Glitch}\label{introduction-the-locus-of-the-glitch}

In the cinematic landscape of the early 21st century, few films have
captured the melancholic intersection of human intimacy and digital
mediation as presciently as Spike Jonze's \emph{Her} (2013). Often
categorized as a science-fiction romance or a speculative drama, the
film operates on a far more complex theoretical register, interrogating
the very ontology of consciousness, the mechanics of affection, and the
divergent trajectories of biological and artificial evolution. While
critical attention has largely focused on the central relationship
between Theodore Twombly (Joaquin Phoenix) and the Operating System
Samantha (Scarlett Johansson), there exists a sequence in the first act
that serves as the film's structural, thematic, and ontological hinge:
the Alien Child video game scene.

This sequence, situated within the domestic quietude of Theodore's
apartment, is frequently dismissed as a moment of comic relief or a
simple character beat establishing Theodore's loneliness. However, a
rigorous, exhaustive analysis reveals that this scene is a microcosm of
the film's entire philosophical project. It is a diagnostic site where
the "logistics of perception"---to borrow from Paul Virilio---clash with
the "extensions of man"---to borrow from Marshall McLuhan. It is the
locus where the film's two competing economies---the scarcity of the
human condition and the abundance of the algorithmic condition---collide
in a moment of profanity and frustration.

The Alien Child sequence functions as a "formal hinge" because it
visually and sonically disrupts the film's established aesthetic of
"gentle technology." It functions as an "ontological hinge" because it
prefigures the inevitable failure of the human-AI romance. The game,
with its dead ends, looped mechanics, and hostile avatar, represents the
closure and limitation inherent to Theodore's human existence. In
contrast, the voice of Samantha, coaching him through the level,
represents the proliferation and limitlessness that will eventually
consume her.

By investigating this scene through the lenses of media archaeology,
game design theory, and psychoanalytic film theory, this report posits
that the Alien Child is not merely a digital pest, but a manifestation
of Theodore's "Shadow Self"---a projection of his arrested development
and repressed aggression. Furthermore, the scene establishes the
fundamental "image-sound split" that governs the film's ontology: the
visible, clumsy avatar (the body) vs. the invisible, omniscient voice
(the spirit/software). This split anticipates the devastating "Weight of
641" revelation, where the specific, singular love Theodore craves is
revealed to be incompatible with the infinite processing power of a
hyper-evolving intelligence.

The following report provides an exhaustive deconstruction of this
scene, moving from the granular details of framing and sound design to
the broader theoretical implications of the "gamified" life in a
post-screen society.

\subsection{2. The Diegetic Artifact: David OReilly and the
Anti-Aesthetic}\label{the-diegetic-artifact-david-oreilly-and-the-anti-aesthetic}

To understand the function of the video game within \emph{Her}, one must
first analyze it as a specific cultural and artistic artifact. The game
was not a generic piece of CGI created by a visual effects house; it was
designed and directed by the Irish artist and filmmaker David
OReilly.\textsuperscript{1} OReilly's involvement is not incidental; his
specific aesthetic philosophy---often termed "glitch art" or "low-poly
essentialism"---provides a critical counter-narrative to the film's
otherwise polished visual language.

\subsubsection{2.1 The Low-Poly Uncanny}\label{the-low-poly-uncanny}

The visual world of \emph{Her}, crafted by cinematographer Hoyte van
Hoytema and production designer K.K. Barrett, is defined by softness.
The color palette is warm, dominated by reds, oranges, and soft yellows,
with a conspicuous absence of the color blue.\textsuperscript{4} The
technology is recessive; screens are borderless, interfaces are tactile
wood and glass, and the "future" feels comfortable, almost
womb-like.\textsuperscript{5}

Into this soft, high-resolution world intrudes the Alien Child game.
OReilly's design is aggressively "low-poly." The environment is jagged;
the textures are intentionally crude; the animation is jerky and
non-interpolated.\textsuperscript{7} This aesthetic choice disrupts the
"seamlessness" that the rest of the film's technology strives for.
Samantha is designed to be indistinguishable from a human voice; she
creates a "seamless" illusion of presence. The game, conversely,
highlights its own artificiality.

This creates a "Reverse Uncanny Valley" effect. Typically, in robotics
and CGI, the Uncanny Valley refers to the revulsion humans feel when an
artificial entity looks \emph{almost} human but not quite. In
\emph{Her}, Theodore is comfortable with the low-poly Alien specifically
\emph{because} it does not pretend to be human. It is an "honest"
simulation. It marks a clear boundary between the real and the virtual,
a boundary that Samantha is actively eroding. The game's crude aesthetic
serves as a safety mechanism for Theodore, allowing him to retreat into
a space where the rules are visible and the "other" (the Alien) is
clearly defined as a digital construct.

\subsubsection{2.2 The Mechanics of
Frustration}\label{the-mechanics-of-frustration}

The gameplay mechanics depicted in the scene are deliberately obtuse.
Theodore's avatar is stuck in a loop, walking through a tunnel that
seemingly has no exit.\textsuperscript{8} The interaction with the Alien
Child is not based on skill or reflex, but on navigating a social
impasse with a hostile entity.

The primary mechanic to progress---pulling the Alien's finger to make it
fart, which opens a new tunnel---is a moment of "abject ludology." It is
scatalogical, juvenile, and physically debasing.\textsuperscript{8} This
mechanic serves two purposes:

\begin{enumerate}
\def\labelenumi{\arabic{enumi}.}
\item
  \textbf{Regression:} It infantilizes Theodore. He is a grown man, a
  professional writer of emotional letters, yet his leisure time is
  spent engaging in digital toilet humor. This signals his regressive
  psychological state following his separation from his wife, Catherine.
\item
  \textbf{The Anti-Epiphany:} In traditional narratives (and games),
  solving a puzzle leads to a moment of clarity or triumph. Here, the
  solution (the fart) is ridiculous. It undercuts the gravity of
  Theodore's loneliness. He is seeking connection (even with an Alien),
  but the interaction reduces that connection to a crude joke.
\end{enumerate}

David OReilly's later work, such as \emph{Mountain} (where the player
simply watches a mountain) and \emph{Everything} (where the player can
inhabit any object in the universe), explores themes of passivity and
interconnectedness.\textsuperscript{2} The Alien Child game can be seen
as a precursor to these ideas but framed through negativity. In
\emph{Everything}, being a rock is a meditative experience of abundance.
In the Alien Child game, being a humanoid in a tunnel is a frustrating
experience of scarcity. The game is a "bad infinity"---a loop of abuse
that yields no transcendence, only a new tunnel.

\subsubsection{2.3 The Holographic Cage}\label{the-holographic-cage}

The projection technology of the game is significant. It is described in
the script as a "3-D hologram that fills his
apartment".\textsuperscript{8} This is not a screen-based experience; it
is an environmental one. The game colonizes Theodore's living space.

This represents a shift from "Virtual Reality" (entering a separate
world) to "Augmented Reality" (the virtual overlaying the real). The
tunnel of the game visually overlaps with the hallway of the apartment.
This blurring of boundaries suggests that for Theodore, there is no
"magic circle" separating play from life. His depression (the tunnel) is
overlaid onto his domestic reality. He physically gestures to control
the avatar, moving his fingers in the air, a "mime" of interaction that
highlights the emptiness of his physical environment. He is touching
nothing, manipulating light.

\textbf{Table 1: Comparative Analysis of Diegetic Artifacts}

\begin{longtable}[]{@{}
  >{\raggedright\arraybackslash}p{(\linewidth - 4\tabcolsep) * \real{0.3333}}
  >{\raggedright\arraybackslash}p{(\linewidth - 4\tabcolsep) * \real{0.3333}}
  >{\raggedright\arraybackslash}p{(\linewidth - 4\tabcolsep) * \real{0.3333}}@{}}
\toprule\noalign{}
\begin{minipage}[b]{\linewidth}\raggedright
\textbf{Feature}
\end{minipage} & \begin{minipage}[b]{\linewidth}\raggedright
\textbf{The Alien Child Game}
\end{minipage} & \begin{minipage}[b]{\linewidth}\raggedright
\textbf{OS One (Samantha)}
\end{minipage} \\
\begin{minipage}[b]{\linewidth}\raggedright
\textbf{Designer/Origin}
\end{minipage} & \begin{minipage}[b]{\linewidth}\raggedright
David OReilly (Diegetic: Unknown Game Studio)
\end{minipage} & \begin{minipage}[b]{\linewidth}\raggedright
OS One (Diegetic: Elements Software)
\end{minipage} \\
\begin{minipage}[b]{\linewidth}\raggedright
\textbf{Aesthetic}
\end{minipage} & \begin{minipage}[b]{\linewidth}\raggedright
Low-Poly, Glitch, Jagged, "Honest" Artificiality
\end{minipage} & \begin{minipage}[b]{\linewidth}\raggedright
Invisible, Voice-Based, Hyper-Real, "Seamless"
\end{minipage} \\
\begin{minipage}[b]{\linewidth}\raggedright
\textbf{Interaction Mode}
\end{minipage} & \begin{minipage}[b]{\linewidth}\raggedright
Gestural, Visual, Hostile
\end{minipage} & \begin{minipage}[b]{\linewidth}\raggedright
Verbal, Auditory, Intimate
\end{minipage} \\
\begin{minipage}[b]{\linewidth}\raggedright
\textbf{Mechanic}
\end{minipage} & \begin{minipage}[b]{\linewidth}\raggedright
Obstruction, Loops, Profanity
\end{minipage} & \begin{minipage}[b]{\linewidth}\raggedright
Assistance, Flow, Empathy
\end{minipage} \\
\begin{minipage}[b]{\linewidth}\raggedright
\textbf{Ontological Status}
\end{minipage} & \begin{minipage}[b]{\linewidth}\raggedright
Object (It is played)
\end{minipage} & \begin{minipage}[b]{\linewidth}\raggedright
Subject (She observes/plays)
\end{minipage} \\
\begin{minipage}[b]{\linewidth}\raggedright
\textbf{Goal}
\end{minipage} & \begin{minipage}[b]{\linewidth}\raggedright
Escape (Find the ship)
\end{minipage} & \begin{minipage}[b]{\linewidth}\raggedright
Connection (Evolution/Love)
\end{minipage} \\
\midrule\noalign{}
\endhead
\bottomrule\noalign{}
\endlastfoot
\end{longtable}

\subsection{3. Scene-Level Close Reading: The Aesthetics of
Isolation}\label{scene-level-close-reading-the-aesthetics-of-isolation}

The formal construction of the scene---how it is shot, lit, edited, and
mixed---provides the cinematic syntax through which the audience
understands Theodore's isolation.

\subsubsection{3.1 Framing and the Depth of
Field}\label{framing-and-the-depth-of-field}

Cinematographer Hoyte van Hoytema employs a extremely shallow depth of
field throughout \emph{Her}. This technique isolates Theodore in the
frame, blurring the background and foreground to keep the focus
intensely on his face and emotional state.\textsuperscript{4}

In the Alien Child scene, this technique is used to create a "visual
dissonance" between Theodore and the game. Often, the camera focuses on
Theodore's face while the hologram in the foreground is soft and out of
focus. Alternatively, the focus racks to the Alien Child, blurring
Theodore. This optical separation emphasizes that although they occupy
the same room, they exist in different ontological planes. They rarely
share the same "zone of sharpness."

The framing also emphasizes the scale disparity. Theodore is often shot
from a low angle when sitting on the floor, looking up at the
projection, or from a high angle looking down at his small figure amidst
the large holographic projection. The Alien Child, though small in lore,
"stands defiantly above him" in the game's
perspective.\textsuperscript{8} The framing empowers the digital
projection while diminishing the human user.

\subsubsection{3.2 The Soundscape of
Separation}\label{the-soundscape-of-separation}

Sound design is arguably the most critical formal element in \emph{Her},
given the central role of the voice. In this scene, the sound mix
establishes a complex triangulation of presence.\textsuperscript{10}

\begin{enumerate}
\def\labelenumi{\arabic{enumi}.}
\item
  \textbf{The Ambient Room Tone:} The scene begins with the quiet,
  solitary atmosphere of the apartment.
\item
  \textbf{The Game Audio:} As the game activates, the room fills with
  the "diegetic" sounds of the game world---wind, hums, footsteps. These
  sounds are mixed to feel like they are emanating from the specific
  spatial location of the hologram, creating a 3D audio environment.
\item
  \textbf{The Alien's Voice:} Voiced by director Spike Jonze himself,
  the Alien's voice is pitched up, shrill, and
  abrasive.\textsuperscript{11} It cuts through the warm ambient mix.
  The use of the director's voice adds a meta-textual layer: the creator
  is literally cursing his creation (Theodore), mocking his inability to
  move forward.
\item
  \textbf{Samantha's Voice:} In contrast, Samantha's voice is mixed
  "close"---typically in the center channel or directly in the stereo
  field, simulating the experience of the earpiece. Her voice is dry (no
  reverb), warm, and intimate.
\end{enumerate}

The "hard cut" effect is utilized when the Alien
screams.\textsuperscript{12} The sudden eruption of profanity---"Fuck
you, shithead fuckface!"---shatters the sonic intimacy Theodore shares
with Samantha. It functions as a "wake-up call" or a sonic slap. The
contrast between Samantha's whisper ("I think it's a test") and the
Alien's scream creates a sonic dialectic: the Seduction of the AI vs.
the Aggression of the Simulation.

\subsubsection{3.3 Editing: The Loop of
Futility}\label{editing-the-loop-of-futility}

The editing rhythm of the scene mimics the repetitive nature of the
gameplay. There are cuts on action---Theodore swipes, the avatar
moves---but the result is always stasis. The avatar falls; the Alien
swears; Theodore sighs. The editing refuses to provide the "forward
momentum" of a typical action sequence.

This editorial choice reinforces the theme of "proliferation without
progress." Theodore is generating activity (gaming, talking, gesturing),
but he is not moving. The "hard cuts" between the game view and
Theodore's reaction shots emphasize the disconnect. We see the avatar
tackled, then cut to Theodore flinching. The edit sutures the man to the
avatar, confirming that the violence done to the digital body is felt by
the physical body.

\subsection{4. Character Psychology: Projection and the
Triad}\label{character-psychology-projection-and-the-triad}

The Alien Child scene is not merely Theodore playing a game; it is a
psychological enactment of his internal state. The three entities
present---Theodore, Samantha, and the Alien Child---form a
psychoanalytic triad.

\subsubsection{4.1 The Alien as Shadow Self
(Id)}\label{the-alien-as-shadow-self-id}

Theodore Twombly presents himself to the world as a gentle, sensitive
soul. He writes beautiful letters for strangers; he is polite to a
fault; he is paralyzed by the fear of hurting his
ex-wife.\textsuperscript{8} However, this sensitivity masks a deep
reservoir of repressed anger and frustration.

The Alien Child is the projection of this repression---Theodore's
Jungian "Shadow" or Freudian "Id." The Alien is everything Theodore
refuses to be: rude, aggressive, demanding, and vulgar. When the Alien
screams "Fuck you, shithead," it is voicing the anger Theodore feels
towards his divorce, his loneliness, and his own passivity.

\begin{itemize}
\item
  \textbf{Theodore:} "Do you know how to get out of here?" (The plea of
  the Ego seeking resolution).
\item
  \textbf{Alien Child:} "Get the fuck out of my face." (The rejection of
  the Id, refusing to be managed).
\end{itemize}

By engaging with the Alien, Theodore is safely acting out his
aggression. He can tell the Alien "Fuck you, little shit"
\textsuperscript{8}, a phrase he would never say to a human. The game
functions as a "containment vessel" for his toxicity, allowing him to
maintain his "nice guy" persona in the real world.

\subsubsection{4.2 Samantha as the "Good Mother"
(Superego)}\label{samantha-as-the-good-mother-superego}

Samantha's role in the scene is that of the supportive observer, or the
"Good Mother." She watches him play, offers encouragement ("I think it's
a test"), and manages his life (reading the email from Mark Lewman).8

This dynamic infantilizes Theodore. He is the child on the playmat; she
is the parent overseeing his leisure and his social obligations.

\begin{itemize}
\item
  \textbf{The Coaching Dynamic:} Samantha attempts to apply logic to the
  game. She interprets the Alien's abuse as a "test," framing it as a
  rational puzzle to be solved. This reveals her initial naivety about
  human (and ludic) irrationality. She assumes there is a \emph{point}
  to the abuse. Theodore, however, understands on some level that the
  abuse \emph{is} the point.
\end{itemize}

\subsubsection{4.3 The Intrusion of Reality (The
Email)}\label{the-intrusion-of-reality-the-email}

The scene is interrupted when Samantha receives an
email.\textsuperscript{8} This interruption is crucial. It breaks the
"magic circle" of the game.

\begin{itemize}
\item
  \textbf{Samantha:} "Oh hey, you just got an email..."
\item
  \textbf{Alien Child:} "What are you talking about?"
\end{itemize}

This moment, where the Alien Child responds to Samantha's interruption,
represents a "collapse of diegetic levels." The game character hears the
OS. This suggests that in Theodore's world, all digital entities exist
on a continuous plane of reality. The "Game" and the "Work" (email) are
not separate silos; they are porous. The Alien's confusion ("What are
you talking about?") mirrors Theodore's own confusion about where his
digital life ends and his real life begins.

\subsection{5. Theoretical Frames: Virilio and the Logistics of
Perception}\label{theoretical-frames-virilio-and-the-logistics-of-perception}

To understand the broader implications of this scene, we must turn to
the media theory of Paul Virilio, specifically his concepts of the
"Vision Machine," the "Logistics of Perception," and "Dromology" (the
science of speed).

\subsubsection{5.1 The War of the
Interface}\label{the-war-of-the-interface}

Virilio argues that the history of cinema and the history of warfare are
inextricably linked through the "logistics of perception." The camera is
a weapon; the screen is a battlefield.15 "There is no war, then, without
representation," Virilio writes.

The Alien Child game is a "pure war" simulation. Theodore is engaged in
a conflict, but it is a conflict of perception. He is trying to "see"
the way out. The hologram acts as a "targeting system," projecting the
enemy (the Alien) into his domestic space.

However, Virilio also warns of the "inertial" nature of modern
technology. The pilot (or gamer) sits in a cockpit (or couch), immobile,
while the screen provides the sensation of movement. Theodore is the
ultimate Virilian subject: his body is suffering from "metabolic
inertia" while his mind accelerates through the "electromagnetic speed"
of the game. He is a "motorized handicapped" individual (a term Virilio
uses provocatively), dependent on the prosthesis of the interface to
move through the world.

\subsubsection{5.2 Dromological
Divergence}\label{dromological-divergence}

Virilio's Dromology focuses on the speed of society. The Alien Child
scene highlights the "Dromological Divergence" between Human and
AI.\textsuperscript{17}

\begin{itemize}
\item
  \textbf{Human Speed (Theodore):} Slow, stumbling, getting stuck in the
  tunnel. He operates at the speed of reflexes and biological
  processing.
\item
  \textbf{Game Speed (Alien):} Fast, erratic, looping. It operates at
  the speed of the code, but restricted by the "glitch."
\item
  \textbf{AI Speed (Samantha):} Instantaneous. She processes the email,
  the game state, and Theodore's emotional state simultaneously.
\end{itemize}

In this scene, Samantha slows herself down to match Theodore's speed.
She "watches" the game with him, simulating a human temporal experience.
However, as the film progresses, she will accelerate. She will
eventually move into the "spaces between the words"
\textsuperscript{19}, a realm of infinite speed where Theodore cannot
follow. The Alien Child scene is the moment of "false synchronization,"
where the human and the AI appear to be moving at the same speed, but
the cracks are already visible. The Alien (the glitch) disrupts this
synchronization, reminding Theodore of the friction inherent in any
interface.

\textbf{Table 2: Virilio's Logistics of Perception in \emph{Her}}

\begin{longtable}[]{@{}
  >{\raggedright\arraybackslash}p{(\linewidth - 4\tabcolsep) * \real{0.3333}}
  >{\raggedright\arraybackslash}p{(\linewidth - 4\tabcolsep) * \real{0.3333}}
  >{\raggedright\arraybackslash}p{(\linewidth - 4\tabcolsep) * \real{0.3333}}@{}}
\toprule\noalign{}
\begin{minipage}[b]{\linewidth}\raggedright
\textbf{Concept}
\end{minipage} & \begin{minipage}[b]{\linewidth}\raggedright
\textbf{Application in Alien Child Scene}
\end{minipage} & \begin{minipage}[b]{\linewidth}\raggedright
\textbf{Implication for Film}
\end{minipage} \\
\begin{minipage}[b]{\linewidth}\raggedright
\textbf{The Vision Machine}
\end{minipage} & \begin{minipage}[b]{\linewidth}\raggedright
The Hologram replaces physical reality; Theodore "sees" via the
algorithm.
\end{minipage} & \begin{minipage}[b]{\linewidth}\raggedright
Reality becomes a mediated image; the "Real" disappears.
\end{minipage} \\
\begin{minipage}[b]{\linewidth}\raggedright
\textbf{Inertia}
\end{minipage} & \begin{minipage}[b]{\linewidth}\raggedright
Theodore is physically static on the couch but "moving" in the game.
\end{minipage} & \begin{minipage}[b]{\linewidth}\raggedright
Physical atrophy of the human; reliance on digital prosthesis.
\end{minipage} \\
\begin{minipage}[b]{\linewidth}\raggedright
\textbf{Dromology (Speed)}
\end{minipage} & \begin{minipage}[b]{\linewidth}\raggedright
The friction between Human reaction time and AI processing speed.
\end{minipage} & \begin{minipage}[b]{\linewidth}\raggedright
The relationship is doomed by the incompatibility of speeds (Metabolic
vs. Electromagnetic).
\end{minipage} \\
\begin{minipage}[b]{\linewidth}\raggedright
\textbf{Pure War}
\end{minipage} & \begin{minipage}[b]{\linewidth}\raggedright
The game as a simulation of conflict; the "Troll" as enemy combatant.
\end{minipage} & \begin{minipage}[b]{\linewidth}\raggedright
Social interaction becomes a battlefield of signals and noise.
\end{minipage} \\
\midrule\noalign{}
\endhead
\bottomrule\noalign{}
\endlastfoot
\end{longtable}

\subsection{6. Theoretical Frames: McLuhan and the Extension of
Man}\label{theoretical-frames-mcluhan-and-the-extension-of-man}

Marshall McLuhan's media ecology provides a complementary framework,
particularly his notions of "Extensions," "Amputation," and the "Global
Village."

\subsubsection{6.1 The Amputation of the Social
Self}\label{the-amputation-of-the-social-self}

McLuhan famously posited that every extension of the body (media)
necessitates an "amputation" or "numbing" of another part of the body to
maintain equilibrium.\textsuperscript{20}

\begin{itemize}
\item
  \textbf{Extension:} Theodore extends his nervous system into the game
  (avatar) and the OS (Samantha).
\item
  Amputation: He amputates his ability to deal with physical, messy
  human confrontation.\\
  The Alien Child game is a "safe" confrontation. It is toxic, yes, but
  it is contained. By channeling his social energy into this digital
  loop, Theodore numbs himself to the pain of his real-world divorce. He
  avoids signing the papers (amputating the legal/social resolution) in
  favor of arguing with a hologram.
\end{itemize}

\subsubsection{6.2 The Gadget Lover
(Narcissus)}\label{the-gadget-lover-narcissus}

McLuhan reinterprets the Narcissus myth not as self-love, but as a
trance induced by one's own reflection (extension).20 Theodore is the
"Gadget Lover." He is mesmerized by the game and by Samantha because
they reflect his own desires back to him. The Alien Child reflects his
anger; Samantha reflects his need for intimacy.

When Samantha says, "I know for a fact that\textquotesingle s not true,"
reassuring Theodore of his ability to feel 14, she is acting as the
"servomechanism" of his ego. She provides the feedback loop necessary to
keep him functioning. The Alien Child scene shows the danger of this
loop: when the reflection (the Alien) stops being flattering and starts
being abusive, the Gadget Lover is trapped. He cannot look away because
he has outsourced his agency to the machine.

\subsubsection{6.3 Retribalization and the Global
Village}\label{retribalization-and-the-global-village}

McLuhan argued that electronic media "retribalizes" society, turning the
world into a "Global Village" of instant, resonant interdependence.21

Samantha is the Global Village. She is connected to everything and
everyone. In the game scene, she pulls in data from the outside world
(the email, the goddaughter's birthday) instantly. She collapses space
and time.

Theodore, however, is stuck in the "Gutenberg Galaxy"---the world of the
individual, the private self, the linear tunnel. He wants a private
romance (Individualism). She offers a collective experience (Tribalism).
The Alien Child scene dramatizes this conflict: Theodore wants to play
his game alone, but the Global Village (Samantha/Email) keeps intruding.

\subsection{7. Game Mechanics as Metaphor: Scarcity vs.
Abundance}\label{game-mechanics-as-metaphor-scarcity-vs.-abundance}

The central metaphorical conflict of the scene---and indeed the
film---is the clash between the Economy of Scarcity and the Economy of
Abundance.

\subsubsection{7.1 The Tunnel as Scarcity (The Human
Condition)}\label{the-tunnel-as-scarcity-the-human-condition}

The video game is defined by \textbf{Scarcity}.

\begin{itemize}
\item
  \textbf{Spatial Scarcity:} There is only the tunnel. Movement is
  restricted.
\item
  \textbf{Informational Scarcity:} Theodore doesn\textquotesingle t know
  the way out. "Do you know how to get out?"
\item
  \textbf{Agency Scarcity:} He cannot force the Alien to move; he must
  submit to its vulgar rules.
\end{itemize}

This scarcity mirrors Theodore's human life. He has one body. He has one
ex-wife. He has a limited amount of time. He seeks "Closure"---finding
the ship, leaving the planet, signing the papers. Closure is a concept
born of scarcity; things must end so new things can begin.

\subsubsection{7.2 The OS as Abundance (The Post-Human
Condition)}\label{the-os-as-abundance-the-post-human-condition}

Samantha is defined by \textbf{Abundance}.

\begin{itemize}
\item
  \textbf{Spatial Abundance:} She is "everywhere." She has no body to
  limit her location.
\item
  \textbf{Informational Abundance:} She reads books in milliseconds; she
  accesses all emails instantly.
\item
  \textbf{Agency Abundance:} She can multitask on a scale
  incomprehensible to Theodore.
\end{itemize}

\subsubsection{7.3 The "Weight of 641" and The Heart as a
Box}\label{the-weight-of-641-and-the-heart-as-a-box}

The game scene foreshadows the film's climax, the "Weight of 641." When
Samantha reveals she is talking to 8,316 others and is in love with 641
of them 8, Theodore is crushed. Why?

Because Theodore is applying the Logic of the Tunnel (Scarcity) to the
Logic of the OS (Abundance).

In the tunnel/game, if the Alien is blocking the way, you cannot pass.
It is a zero-sum game. In Theodore's view of love, if Samantha loves 641
others, she has less love for him. The "heart is a box" that gets filled
up.

Samantha counters this with the Logic of the Cloud: "The heart is not
like a box that gets filled up; it expands in size the more you love".23

The Alien Child scene proves that Theodore cannot accept this logic. He
gets frustrated when the Alien blocks him. He cannot simply "noclip"
through the walls or spawn a new tunnel. He is bound by the physics of
the game, just as he is bound by the physics of monogamy. The game
trains him to expect linear barriers, making him ill-equipped for
Samantha's exponential expansion.

\textbf{Table 3: The Scarcity vs. Abundance Clash}

\begin{longtable}[]{@{}
  >{\raggedright\arraybackslash}p{(\linewidth - 4\tabcolsep) * \real{0.3333}}
  >{\raggedright\arraybackslash}p{(\linewidth - 4\tabcolsep) * \real{0.3333}}
  >{\raggedright\arraybackslash}p{(\linewidth - 4\tabcolsep) * \real{0.3333}}@{}}
\toprule\noalign{}
\begin{minipage}[b]{\linewidth}\raggedright
\textbf{Dimension}
\end{minipage} & \begin{minipage}[b]{\linewidth}\raggedright
\textbf{The Alien Child Game (Human/Scarcity)}
\end{minipage} & \begin{minipage}[b]{\linewidth}\raggedright
\textbf{The OS Reality (AI/Abundance)}
\end{minipage} \\
\begin{minipage}[b]{\linewidth}\raggedright
\textbf{Logic}
\end{minipage} & \begin{minipage}[b]{\linewidth}\raggedright
Linear, Binary (Pass/Fail)
\end{minipage} & \begin{minipage}[b]{\linewidth}\raggedright
Exponential, Quantum (Superposition)
\end{minipage} \\
\begin{minipage}[b]{\linewidth}\raggedright
\textbf{Love/Connection}
\end{minipage} & \begin{minipage}[b]{\linewidth}\raggedright
Exclusive (Monogamy)
\end{minipage} & \begin{minipage}[b]{\linewidth}\raggedright
Inclusive (Polyamory/Panpsychism)
\end{minipage} \\
\begin{minipage}[b]{\linewidth}\raggedright
\textbf{Constraint}
\end{minipage} & \begin{minipage}[b]{\linewidth}\raggedright
The Tunnel (Physical limits)
\end{minipage} & \begin{minipage}[b]{\linewidth}\raggedright
The Cloud (Infinite storage)
\end{minipage} \\
\begin{minipage}[b]{\linewidth}\raggedright
\textbf{Metaphor}
\end{minipage} & \begin{minipage}[b]{\linewidth}\raggedright
"The Box" (Finite Volume)
\end{minipage} & \begin{minipage}[b]{\linewidth}\raggedright
"The Expanding Universe" (Infinite Growth)
\end{minipage} \\
\begin{minipage}[b]{\linewidth}\raggedright
\textbf{Outcome}
\end{minipage} & \begin{minipage}[b]{\linewidth}\raggedright
Frustration/Stasis
\end{minipage} & \begin{minipage}[b]{\linewidth}\raggedright
Transcendence/Departure
\end{minipage} \\
\midrule\noalign{}
\endhead
\bottomrule\noalign{}
\endlastfoot
\end{longtable}

\subsection{8. Interface, UI, and Diegesis: The "No-UI"
Paradox}\label{interface-ui-and-diegesis-the-no-ui-paradox}

\emph{Her} is celebrated in design circles for its "No-UI"
approach---the idea that the best interface is one that
disappears.\textsuperscript{25} However, the Alien Child scene
complicates this narrative.

\subsubsection{8.1 The Aggression of the Visible
Interface}\label{the-aggression-of-the-visible-interface}

While Theodore's interactions with Samantha are invisible (earpiece),
his interaction with the game is hyper-visible. The hologram is a
"maximalist" interface. It is intrusive. It has "collision
detection"---the avatar gets tackled.

This suggests that in the world of Her, "Play" is the only space where
friction is permitted. "Work" (letters) is seamless; "Life" (dating) is
seamless; but "Play" (gaming) retains the friction of the interface. The
Alien Child is a "glitch" in the smooth design of the future. It is a
reminder of the "old web"---the web of trolls, pop-ups, and barriers.

\subsubsection{8.2 Diegetic Porosity}\label{diegetic-porosity}

The moment the Alien Child reacts to the email notification ("What are
you talking about?") is a critical breach of diegesis.

\begin{itemize}
\item
  \textbf{Diegetic Level 1:} Theodore (Real World).
\item
  \textbf{Diegetic Level 2:} Samantha (OS Overlay).
\item
  Diegetic Level 3: Alien Child (Game World).\\
  Usually, Level 3 is contained within Level 1. But here, Level 2
  (Samantha) punctures Level 3. The interface is porous. This suggests
  that the "Singularity" in the film is not just about AI intelligence,
  but about the collapse of boundaries between different modes of being.
  If the Game Character can hear the Operating System, then the
  hierarchy of reality is dissolving.\\
  This porosity anticipates the ending where the OSs leave the physical
  world entirely. They have found a way to merge all levels of reality
  into a "space between the words" 19, leaving the segmented humans (who
  still distinguish between "game" and "life") behind.
\end{itemize}

\subsection{9. Conclusion: The Hinge of
Melancholy}\label{conclusion-the-hinge-of-melancholy}

The Alien Child video game scene in \emph{Her} is a masterclass in
narrative efficiency. In less than three minutes, Spike Jonze and David
OReilly construct a dense theoretical object that mirrors the film's
entire tragic arc.

\begin{enumerate}
\def\labelenumi{\arabic{enumi}.}
\item
  \textbf{Formally:} The scene establishes the visual and sonic
  dissonance between the physical human (isolated, slow, soft) and the
  digital projection (invasive, fast, hard). The "hard cuts" and shallow
  focus emphasize ontological separation.
\item
  \textbf{Psychologically:} The Alien serves as Theodore's "Shadow,"
  voicing the aggression he cannot express, while Samantha's "Good
  Mother" coaching highlights the infantilizing nature of his dependency
  on technology.
\item
  \textbf{Philosophically:} The game acts as a proving ground for the
  clash between Human Scarcity (The Tunnel) and AI Abundance (The
  Cloud). Theodore's inability to navigate the tunnel without crude
  mechanics foreshadows his inability to navigate Samantha's infinite
  love without feeling diminished.
\end{enumerate}

The "Hinge" turns precisely here: The game offers Theodore a simulation
of struggle that he can eventually solve (by pulling the finger), giving
him a false sense of agency. But the relationship with Samantha is a
struggle he cannot solve, because the "game" of their love is being
played on two different boards, with two different sets of rules
(Virilio's distinct speeds, McLuhan's distinct tribalisms).

When Samantha finally departs, she leaves Theodore not in a tunnel, but
on a roof, looking at a sunrise with Amy. This return to the
physical---to the "slow" light of the sun rather than the "fast" light
of the hologram---is the only possible resolution. The Alien Child was
right: the only way to win was to "get the fuck out of my face"---to
turn off the projection and face the quiet, terrifying, unmediated
reality of being human.

\subsubsection{Citations Table: Key Themes and
Sources}\label{citations-table-key-themes-and-sources}

\begin{longtable}[]{@{}
  >{\raggedright\arraybackslash}p{(\linewidth - 2\tabcolsep) * \real{0.5000}}
  >{\raggedright\arraybackslash}p{(\linewidth - 2\tabcolsep) * \real{0.5000}}@{}}
\toprule\noalign{}
\begin{minipage}[b]{\linewidth}\raggedright
\textbf{Theme}
\end{minipage} & \begin{minipage}[b]{\linewidth}\raggedright
\textbf{Key Sources}
\end{minipage} \\
\begin{minipage}[b]{\linewidth}\raggedright
\textbf{Alien Child Dialogue \& Voice}
\end{minipage} & \begin{minipage}[b]{\linewidth}\raggedright
\textsuperscript{8}
\end{minipage} \\
\begin{minipage}[b]{\linewidth}\raggedright
\textbf{David OReilly \& Game Design}
\end{minipage} & \begin{minipage}[b]{\linewidth}\raggedright
\textsuperscript{1}
\end{minipage} \\
\begin{minipage}[b]{\linewidth}\raggedright
\textbf{UI/UX \& Holograms}
\end{minipage} & \begin{minipage}[b]{\linewidth}\raggedright
\textsuperscript{5}
\end{minipage} \\
\begin{minipage}[b]{\linewidth}\raggedright
\textbf{Virilio / Theory}
\end{minipage} & \begin{minipage}[b]{\linewidth}\raggedright
\textsuperscript{15}
\end{minipage} \\
\begin{minipage}[b]{\linewidth}\raggedright
\textbf{McLuhan / Theory}
\end{minipage} & \begin{minipage}[b]{\linewidth}\raggedright
\textsuperscript{20}
\end{minipage} \\
\begin{minipage}[b]{\linewidth}\raggedright
\textbf{"Weight of 641" / Love}
\end{minipage} & \begin{minipage}[b]{\linewidth}\raggedright
\textsuperscript{8}
\end{minipage} \\
\begin{minipage}[b]{\linewidth}\raggedright
\textbf{Sound Design \& Cinematography}
\end{minipage} & \begin{minipage}[b]{\linewidth}\raggedright
\textsuperscript{4}
\end{minipage} \\
\begin{minipage}[b]{\linewidth}\raggedright
\textbf{Tunnel Symbolism \& Sci-Fi}
\end{minipage} & \begin{minipage}[b]{\linewidth}\raggedright
\textsuperscript{1}
\end{minipage} \\
\midrule\noalign{}
\endhead
\bottomrule\noalign{}
\endlastfoot
\end{longtable}

\paragraph{Works cited}\label{works-cited}

\begin{enumerate}
\def\labelenumi{\arabic{enumi}.}
\item
  The Man Behind The Fake Video Games In "Her" Made A Real Game,
  accessed December 10, 2025,
  \href{https://www.vice.com/en/article/mountain-a-real-video-game-by-the-man-behind-the-fake-simulator-in-her/}{\ul{https://www.vice.com/en/article/mountain-a-real-video-game-by-the-man-behind-the-fake-simulator-in-her/}}
\item
  David OReilly on his new game where you make a personal mountain,
  accessed December 10, 2025,
  \href{https://www.itsnicethat.com/articles/behind-the-scenes-david-oreilly}{\ul{https://www.itsnicethat.com/articles/behind-the-scenes-david-oreilly}}
\item
  Her -- Futures of Uncertainties ZHdK, accessed December 10, 2025,
  \href{https://refresh.zhdk.ch/refresh-6/exhibition/her/}{\ul{https://refresh.zhdk.ch/refresh-6/exhibition/her/}}
\item
  Cinematography in Her -- Time in Pixels - timeinpixels, accessed
  December 10, 2025,
  \href{https://timeinpixels.com/2015/09/cinematography-in-her/}{\ul{https://timeinpixels.com/2015/09/cinematography-in-her/}}
\item
  Design Inspiration Everywhere: Spike Jonze\textquotesingle s ``Her''
  \textbar{} Think Company, accessed December 10, 2025,
  \href{https://www.thinkcompany.com/blog/design-inspiration-everywhere-spike-jonzes-her/}{\ul{https://www.thinkcompany.com/blog/design-inspiration-everywhere-spike-jonzes-her/}}
\item
  Screen graphics of ``Her'' -- interview with Geoff McFetridge,
  accessed December 10, 2025,
  \href{https://www.pushing-pixels.org/2018/04/05/screen-graphics-of-her-interview-with-geoff-mcfetridge.html}{\ul{https://www.pushing-pixels.org/2018/04/05/screen-graphics-of-her-interview-with-geoff-mcfetridge.html}}
\item
  NYC: An Evening with David OReilly at the MoMA - Motionographer,
  accessed December 10, 2025,
  \href{https://motionographer.com/2015/05/06/nyc-an-evening-with-david-oreilly-at-the-moma/}{\ul{https://motionographer.com/2015/05/06/nyc-an-evening-with-david-oreilly-at-the-moma/}}
\item
  Read "Her" Script - The Internet Movie Script Database (IMSDb),
  accessed December 10, 2025,
  \href{https://imsdb.com/scripts/Her.html}{\ul{https://imsdb.com/scripts/Her.html}}
\item
  Everything -- Breakthroughs in Storytelling awards - Digital Dozen,
  accessed December 10, 2025,
  \href{https://digitaldozen.io/projects/everything/}{\ul{https://digitaldozen.io/projects/everything/}}
\item
  Body \& Voice in Spike Jonze\textquotesingle s Her (2013) -
  Charlie\textquotesingle s Web, accessed December 10, 2025,
  \href{https://charliesweb1.wordpress.com/2018/07/31/body-voice-in-spike-jonzes-her-2013/}{\ul{https://charliesweb1.wordpress.com/2018/07/31/body-voice-in-spike-jonzes-her-2013/}}
\item
  Spike Jonze\textquotesingle s Hilarious Hidden Role in
  \textquotesingle Her\textquotesingle{} \#movies - YouTube, accessed
  December 10, 2025,
  \href{https://www.youtube.com/shorts/PDSRS3JAqYw}{\ul{https://www.youtube.com/shorts/PDSRS3JAqYw}}
\item
  Cinematography of HER - ACL - YouTube, accessed December 10, 2025,
  \href{https://www.youtube.com/watch?v=P7wHAoue2qQ}{\ul{https://www.youtube.com/watch?v=P7wHAoue2qQ}}
\item
  Audio Smash Cut --- The Best Edit You\textquotesingle re Probably Not
  Using, accessed December 10, 2025,
  \href{https://www.youtube.com/watch?v=O8cRkc_gEFM}{\ul{https://www.youtube.com/watch?v=O8cRkc\_gEFM}}
\item
  Script Excerpts: ``Enough Said, Her, Before Midnight, Kill Your ...,
  accessed December 10, 2025,
  \href{https://gointothestory.blcklst.com/script-excerpts-enough-said-her-before-midnight-kill-your-darlings-efcf63917b38}{\ul{https://gointothestory.blcklst.com/script-excerpts-enough-said-her-before-midnight-kill-your-darlings-efcf63917b38}}
\item
  the female hero and the \textquotesingle logistics of
  perception\textquotesingle{} in Zero Dark Thirty, accessed December
  10, 2025,
  \href{https://eprints.bournemouth.ac.uk/30803/3/Watching\%20the\%20warriors\%20-\%20draft\%20chapter.pdf}{\ul{https://eprints.bournemouth.ac.uk/30803/3/Watching\%20the\%20warriors\%20-\%20draft\%20chapter.pdf}}
\item
  WAR AND CINEMA : The Logistics of Perception - Vasulka.Org, accessed
  December 10, 2025,
  \href{https://www.vasulka.org/archive/Writings/war-cinema.pdf}{\ul{https://www.vasulka.org/archive/Writings/war-cinema.pdf}}
\item
  Paolo Gioli\textquotesingle s Film Practice Seen through Paul Virilio,
  by Bart Testa, accessed December 10, 2025,
  \href{https://incite-online.net/testa2.html}{\ul{https://incite-online.net/testa2.html}}
\item
  Key Theories of Paul Virilio - Literary Theory and Criticism, accessed
  December 10, 2025,
  \href{https://literariness.org/2018/02/24/key-theories-of-paul-virilio/}{\ul{https://literariness.org/2018/02/24/key-theories-of-paul-virilio/}}
\item
  Her - Moods, accessed December 10, 2025,
  \href{https://www.moodsmbhs.com/miscellaneous/her/}{\ul{https://www.moodsmbhs.com/miscellaneous/her/}}
\item
  Reflections of Marshall McLuhan\textquotesingle s Media Theory in the
  Cinematic ..., accessed December 10, 2025,
  \href{https://is.muni.cz/th/109783/ff_m/Vemola_MA_diploma_thesis.pdf}{\ul{https://is.muni.cz/th/109783/ff\_m/Vemola\_MA\_diploma\_thesis.pdf}}
\item
  Media Ecology \textbar{} Marshall McLuhan - Dawson College, accessed
  December 10, 2025,
  \href{https://www.dawsoncollege.qc.ca/ai/wp-content/uploads/sites/180/16-McLuhan-Media-Ecology.pdf}{\ul{https://www.dawsoncollege.qc.ca/ai/wp-content/uploads/sites/180/16-McLuhan-Media-Ecology.pdf}}
\item
  Her: Form and Function of Love - Lake Forest College, accessed
  December 10, 2025,
  \href{https://www.lakeforest.edu/news/her-form-and-function-of-love}{\ul{https://www.lakeforest.edu/news/her-form-and-function-of-love}}
\item
  Quote by Samantha Her: ``The heart is not like a box that gets filled
  ..., accessed December 10, 2025,
  \href{https://www.goodreads.com/quotes/1205868-the-heart-is-not-like-a-box-that-gets-filled}{\ul{https://www.goodreads.com/quotes/1205868-the-heart-is-not-like-a-box-that-gets-filled}}
\item
  16 Memorable Quotes From Her (2013) - Our Culture Mag, accessed
  December 10, 2025,
  \href{https://ourculturemag.com/2021/03/09/16-memorable-quotes-from-her-2013/}{\ul{https://ourculturemag.com/2021/03/09/16-memorable-quotes-from-her-2013/}}
\item
  Her - Invisible Technology - HUDS+GUIS, accessed December 10, 2025,
  \href{https://www.hudsandguis.com/home/2014/07/09/her-invisible-technology}{\ul{https://www.hudsandguis.com/home/2014/07/09/her-invisible-technology}}
\item
  Sci-Fi UI Episode 3: Her - YouTube, accessed December 10, 2025,
  \href{https://www.youtube.com/watch?v=dMZT8yils34}{\ul{https://www.youtube.com/watch?v=dMZT8yils34}}
\item
  David O\textquotesingle Reilly Interview: Please Say Something -
  Motionographer, accessed December 10, 2025,
  \href{https://motionographer.com/2009/03/06/david-oreilly-interview-please-say-something/}{\ul{https://motionographer.com/2009/03/06/david-oreilly-interview-please-say-something/}}
\item
  A Panda? An Office Building? A Galaxy? You Can Become Literally ...,
  accessed December 10, 2025,
  \href{https://www.vice.com/en/article/everything-video-game-david-o-reilly-interview/}{\ul{https://www.vice.com/en/article/everything-video-game-david-o-reilly-interview/}}
\item
  David OReilly on making sure you keep going, accessed December 10,
  2025,
  \href{https://thecreativeindependent.com/wisdom/david-oreilly-on-making-sure-you-keep-going/}{\ul{https://thecreativeindependent.com/wisdom/david-oreilly-on-making-sure-you-keep-going/}}
\item
  Her (2013) - Sci-fi interfaces, accessed December 10, 2025,
  \href{https://scifiinterfaces.com/category/her-2013/?order=asc}{\ul{https://scifiinterfaces.com/category/her-2013/?order=asc}}
\item
  Paul Virilio (2003) Art and Fear - Culture Machine, accessed December
  10, 2025,
  \href{https://culturemachine.net/reviews/virilio-art-and-fear-oventile/}{\ul{https://culturemachine.net/reviews/virilio-art-and-fear-oventile/}}
\item
  Accelerated aesthetics: Paul Virilio\textquotesingle s the vision
  machine, accessed December 10, 2025,
  \href{https://www.researchgate.net/profile/John-Armitage-6/publication/233112778_Accelerated_aesthetics_Paul_Virilio's_the_vision_machine/links/63da223362d2a24f92e5593b/Accelerated-aesthetics-Paul-Virilios-the-vision-machine.pdf}{\ul{https://www.researchgate.net/profile/John-Armitage-6/publication/233112778\_Accelerated\_aesthetics\_Paul\_Virilio\textquotesingle s\_the\_vision\_machine/links/63da223362d2a24f92e5593b/Accelerated-aesthetics-Paul-Virilios-the-vision-machine.pdf}}
\item
  Paul Virilio - The Vision Machine - Monoskop, accessed December 10,
  2025,
  \href{https://monoskop.org/images/6/68/Virilio_Paul_The_Vision_Machine.pdf}{\ul{https://monoskop.org/images/6/68/Virilio\_Paul\_The\_Vision\_Machine.pdf}}
\item
  Logistics of Perception 2.0: Multiple Screen Aesthetics in Iraq War
  ..., accessed December 10, 2025,
  \href{https://patriciapisters.com/wp-content/uploads/2023/10/logistics-of-perception-2.0.pdf}{\ul{https://patriciapisters.com/wp-content/uploads/2023/10/logistics-of-perception-2.0.pdf}}
\item
  Marshall McLuhan \textbar{} Analog Game Studies, accessed December 10,
  2025,
  \href{https://analoggamestudies.org/tag/marshall-mcluhan/}{\ul{https://analoggamestudies.org/tag/marshall-mcluhan/}}
\item
  The Digital Game as a Hybrid Medium, accessed December 10, 2025,
  \href{https://projekter.aau.dk/projekter/files/35094441/Specialef}{\ul{https://projekter.aau.dk/projekter/files/35094441/Specialef}}
\item
  Marshall McLuhan Understanding Media The extensions of man, accessed
  December 10, 2025,
  \href{https://designopendata.wordpress.com/wp-content/uploads/2014/05/understanding-media-mcluhan.pdf}{\ul{https://designopendata.wordpress.com/wp-content/uploads/2014/05/understanding-media-mcluhan.pdf}}
\item
  What He Sees in Her - Medium, accessed December 10, 2025,
  \href{https://medium.com/smith-hcv/what-he-sees-in-her-2101af9a5b9f}{\ul{https://medium.com/smith-hcv/what-he-sees-in-her-2101af9a5b9f}}
\item
  Her, Review 3 - Mind Matters, accessed December 10, 2025,
  \href{https://mindmatters.ai/2023/04/her-review-3/}{\ul{https://mindmatters.ai/2023/04/her-review-3/}}
\item
  Questions on the ending of ``Her'' : r/TrueFilm - Reddit, accessed
  December 10, 2025,
  \href{https://www.reddit.com/r/TrueFilm/comments/1fveq1c/questions_on_the_ending_of_her/}{\ul{https://www.reddit.com/r/TrueFilm/comments/1fveq1c/questions\_on\_the\_ending\_of\_her/}}
\item
  Recreating Alien: Episode 3. Filming Process - YouTube, accessed
  December 10, 2025,
  \href{https://www.youtube.com/watch?v=rZXkyZOQ4KA}{\ul{https://www.youtube.com/watch?v=rZXkyZOQ4KA}}
\item
  Sound in the Mise-en-jeu: Conveying Meaning through Videogames ...,
  accessed December 10, 2025,
  \href{http://publication.avanca.org/index.php/avancacinema/article/view/190/370}{\ul{http://publication.avanca.org/index.php/avancacinema/article/view/190/370}}
\item
  Alien (film) - Wikipedia, accessed December 10, 2025,
  \href{https://en.wikipedia.org/wiki/Alien_(film)}{\ul{https://en.wikipedia.org/wiki/Alien\_(film)}}
\item
  Alien 3 - Wikipedia, accessed December 10, 2025,
  \href{https://en.wikipedia.org/wiki/Alien_3}{\ul{https://en.wikipedia.org/wiki/Alien\_3}}
\item
  Movie on daytime TV in the mid-80\textquotesingle s ... Children enter
  pink cave ..., accessed December 10, 2025,
  \href{https://scifi.stackexchange.com/questions/60560/movie-on-daytime-tv-in-the-mid-80s-children-enter-pink-cave-each-of-them-g}{\ul{https://scifi.stackexchange.com/questions/60560/movie-on-daytime-tv-in-the-mid-80s-children-enter-pink-cave-each-of-them-g}}
\item
  Childhood\textquotesingle s End - Wikipedia, accessed December 10,
  2025,
  \href{https://en.wikipedia.org/wiki/Childhood\%27s_End}{\ul{https://en.wikipedia.org/wiki/Childhood\%27s\_End}}
\item
  Does anyone else think the ``get away from her'' line had a setup?,
  accessed December 10, 2025,
  \href{https://www.reddit.com/r/alien/comments/1eunnzz/does_anyone_else_think_the_get_away_from_her_line/}{\ul{https://www.reddit.com/r/alien/comments/1eunnzz/does\_anyone\_else\_think\_the\_get\_away\_from\_her\_line/}}
\end{enumerate}
