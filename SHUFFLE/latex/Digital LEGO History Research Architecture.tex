\section{A Research Architecture for the Digital LEGO Ecosystem
(1990s--Present): Mapping Knowledge Gaps Through Source Triage,
Linguistic Analysis, and Visual Corpus
Reconstruction}\label{a-research-architecture-for-the-digital-lego-ecosystem-1990spresent-mapping-knowledge-gaps-through-source-triage-linguistic-analysis-and-visual-corpus-reconstruction}

\subsection{1. Introduction: The Material-Digital Dialectic of the
Brick}\label{introduction-the-material-digital-dialectic-of-the-brick}

The transformation of the LEGO brick from a tangible artifact of
injection-molded acrylonitrile butadiene styrene (ABS) into a purely
digital asset represents one of the most complex case studies in the
history of industrial design, ludology, and digital humanities. This
evolution was not a linear progression from physical to virtual; rather,
it was a fractured, multi-linear dialectic between corporate control and
community anarchy, between the precise engineering of Computer-Aided
Design (CAD) and the chaotic creativity of play. This report establishes
a comprehensive research architecture for the Digital LEGO Ecosystem,
spanning from the nascent, closed-source experiments of the LEGO Group
(TLG) in the mid-1980s to the contemporary era of AI-driven Gaussian
Splatting and generative design. By triangulating data from obsolete
software repositories, mailing list archives, patent filings, and recent
academic literature in computer graphics and human-computer interaction
(HCI), we reconstruct a timeline that is not merely chronological but
structural---revealing how file formats, rendering algorithms, and
community governance models reshaped the ontology of the "brick."

The central thesis of this architectural mapping is that the "digital
brick" is a contested object. To the LEGO Group, the digital brick began
as a marketing tool and an industrial necessity for generating
instruction manuals, eventually evolving into a controlled consumer
product through initiatives like \emph{LEGO Digital Designer}
(LDD).\textsuperscript{1} To the fan community, however, the digital
brick was a liberation from the scarcity of physical matter. Through the
\emph{LDraw} standard established by James Jessiman, the digital brick
became a linguistic construct---a line of text defining vector
geometry---that allowed for infinite reproduction and
preservation.\textsuperscript{3} The tension between these two
ontologies---the proprietary, physics-constrained brick of the
corporation and the open, infinite brick of the community---drove three
decades of software innovation.

Furthermore, this analysis exposes significant knowledge gaps in the
historical record. While the history of physical sets is
well-documented, the "Dark Age" of internal digital tools---specifically
the \emph{Panter} system (1986) and the \emph{L3D} database of SPU
Darwin (1996)---remains obscured by corporate secrecy and technological
obsolescence.\textsuperscript{4} Simultaneously, the sociological
dimensions of these tools reveal deeply entrenched biases. The
architecture of digital building software often codified gendered
marketing strategies and professional stigmas, creating a "digital
divide" where the "engineering" aesthetic of LDraw alienated casual
users while the "toy" aesthetic of LDD alienated professional
architects.\textsuperscript{6} As we move into an era of neural radiance
fields and generative AI, understanding this history is critical for
preserving the digital heritage of play.

\subsection{2. The Genesis of Digital Ontology: Primitive Accumulation
(1980s--1999)}\label{the-genesis-of-digital-ontology-primitive-accumulation-1980s1999}

The digitization of the LEGO system required a fundamental translation
of physical properties---clutch power, tolerance, and geometry---into
digital data. This process, described by media scholars as the
"primitive accumulation" of digital assets, occurred simultaneously
within the LEGO Group and among independent hobbyists, often with vastly
different philosophies regarding precision, utility, and visual
fidelity.

\subsubsection{2.1 The "Panter" Era and the Prehistory of Internal
CAD}\label{the-panter-era-and-the-prehistory-of-internal-cad}

While popular history locates the start of digital LEGO in the mid-1990s
with the advent of consumer CD-ROMs, evidence points to significantly
earlier internal efforts. The LEGO Group utilized a DOS-based program
known as \emph{Panter} as early as 1986 to generate building
instructions.\textsuperscript{4} This tool functioned as a primitive CAD
system, specifically designed to handle the isometric projection
required for the visual language of LEGO instructions. The existence of
\emph{Panter} challenges the narrative that LEGO was a "late adopter" of
digital technology; rather, it suggests that TLG recognized the
limitations of hand-drawn technical illustration well before the
consumer internet era.

However, these early internal tools were strictly functional. They
operated on an industrial logic: they were designed for optimization,
printing, and molding, not for creative exploration or consumer
interaction. \emph{Panter} and its successors were "closed" systems,
inaccessible to the public and likely reliant on hard-coded geometry
that did not account for the "play" element of the system. This
distinction is crucial because it left a vacuum for consumer-facing
digital building tools---a vacuum that would eventually be filled by the
community-driven \emph{LDraw} system. The lack of preserved code or
screenshots of \emph{Panter} represents a primary knowledge gap in the
history of LEGO\textquotesingle s digital infrastructure, a "lost
species" of software that defined the visual language of instructions
for a generation of builders.\textsuperscript{4}

\subsubsection{2.2 The Darwin Project: Ambition, Hubris, and
"L3D"}\label{the-darwin-project-ambition-hubris-and-l3d}

In the mid-1990s, the LEGO Group embarked on a secretive and highly
ambitious initiative known as Strategic Product Unit (SPU) Darwin. This
project represented the company\textquotesingle s first attempt to
create a unified digital theory of the brick. Initiated by the visionary
artist Dent-De-Lion Du Midi (Dandi), the project began with a
proof-of-concept video titled \emph{The LEGO Movie} (produced circa
1993/1994), which aimed to convince CEO Kjeld Kirk Kristiansen that
digital bricks could look and behave like their physical
counterparts.\textsuperscript{5} The "epiphany" for Du Midi occurred
when he realized that early 3D computer graphics naturally looked like
plastic---a limitation for realistic film, but a perfect affordance for
LEGO bricks.\textsuperscript{5}

The Darwin team, operating out of a reclaimed fishnet factory and
utilizing Silicon Graphics (SGI) supercomputers---then one of the most
powerful computing installations in Northern Europe---sought to create a
comprehensive database known as "L3D" (LEGO 3D).\textsuperscript{5} The
objective of L3D was not merely visual; it was to create a
"physics-aware" brick that understood connectivity, gravity, and
material properties. This was a radical departure from the static
geometry of \emph{Panter}.

However, the failure of SPU Darwin was multidimensional and instructive:

\begin{itemize}
\item
  \textbf{Technological Overreach:} The SGI infrastructure required to
  run the L3D simulations was prohibitively expensive and fundamentally
  inaccessible to the consumer market. The gap between the SGI
  workstations used by the developers and the Pentium-based PCs used by
  consumers was unbridgeable in 1996, making the technology viable only
  for high-end production (movies/commercials) rather than home
  play.\textsuperscript{5}
\item
  \textbf{Organizational Disconnect:} The SPU Darwin unit operated in
  isolation from the core product design teams in Billund. This created
  a "silo" effect where the digital innovations could not easily
  propagate back into the physical product development cycle. The
  traditional toy designers viewed the digital team with skepticism, and
  the digital team viewed the brick as merely data.\textsuperscript{10}
\item
  \textbf{The Uncanny Valley of Plastic:} As noted by early team
  members, while the SGI machines could render plastic, the
  computational cost to render this "plastic realism" in real-time was
  insurmountable. The result was often a disconnect between the vision
  of a "Living LEGO" world and the jagged, low-resolution reality of
  1990s consumer hardware.\textsuperscript{5}
\end{itemize}

The dissolution of the Darwin group dispersed talent and ideas that
would later re-emerge in \emph{LEGO Creator} and eventually \emph{LEGO
Digital Designer}, but it marked the end of the LEGO
Group\textquotesingle s first, failed attempt at a unified digital
ecosystem.

\subsubsection{2.3 The LDraw Revolution: James Jessiman and the Open
Standard}\label{the-ldraw-revolution-james-jessiman-and-the-open-standard}

In stark contrast to the closed, high-capital Darwin project, the
\emph{LDraw} standard emerged from the grassroots efforts of James
Jessiman in 1995. Jessiman\textquotesingle s contribution was not just
software but a \textbf{linguistic definition} of the brick. He
established a file format that used a plain-text coordinate system to
define the geometry of LEGO parts, relying on a library of primitives
(studs, tubes, boxes) to construct complex shapes.\textsuperscript{3}

The specific architecture of the LDraw format (.dat, .ldr, .mpd) was
revolutionary because of its modularity and readability. It treated the
brick not as a proprietary mesh, but as code.

\begin{itemize}
\item
  \textbf{Line Type 1 (The Reference):} This command referenced other
  files, allowing for a hierarchical library structure. A single "stud"
  file (stud.dat) could be referenced thousands of times within a model
  without duplicating the geometry data. This mimicked the physical
  reality of injection molding, where a single mold design is reused
  across millions of bricks.\textsuperscript{3}
\item
  \textbf{The LDraw Unit (LDU):} Jessiman established the LDraw Unit
  (LDU) as the de facto standard for digital LEGO measurement, where 1
  brick height equals 24 LDU and a stud diameter is 12 LDU. This
  coordinate system provided a mathematical perfection that allowed
  parts from different authors to interlock
  seamlessly.\textsuperscript{3}
\end{itemize}

Jessiman\textquotesingle s untimely death in 1997 created a "martyrdom
effect" that solidified the community\textquotesingle s resolve to
maintain the standard.\textsuperscript{12} The community organized
quickly, establishing the "LDraw.org Steering Committee" (SteerCo) and
the "LDraw Standards Board" (LSB) to govern the library. This turned
part authoring into a democratic, peer-reviewed process, where new parts
had to be "certified" by reviewers before entering the official
library.\textsuperscript{12} This governance structure allowed the LDraw
library to outpace LEGO\textquotesingle s own official digital libraries
in terms of part diversity, update speed, and accuracy for nearly two
decades.

\subsubsection{2.4 Comparative Analysis of Early Digital
Morphologies}\label{comparative-analysis-of-early-digital-morphologies}

\begin{longtable}[]{@{}
  >{\raggedright\arraybackslash}p{(\linewidth - 6\tabcolsep) * \real{0.2500}}
  >{\raggedright\arraybackslash}p{(\linewidth - 6\tabcolsep) * \real{0.2500}}
  >{\raggedright\arraybackslash}p{(\linewidth - 6\tabcolsep) * \real{0.2500}}
  >{\raggedright\arraybackslash}p{(\linewidth - 6\tabcolsep) * \real{0.2500}}@{}}
\toprule\noalign{}
\begin{minipage}[b]{\linewidth}\raggedright
\textbf{Feature}
\end{minipage} & \begin{minipage}[b]{\linewidth}\raggedright
\textbf{LDraw System}
\end{minipage} & \begin{minipage}[b]{\linewidth}\raggedright
\textbf{SPU Darwin (L3D)}
\end{minipage} & \begin{minipage}[b]{\linewidth}\raggedright
\textbf{LEGO Creator (Game)}
\end{minipage} \\
\begin{minipage}[b]{\linewidth}\raggedright
\textbf{Origin}
\end{minipage} & \begin{minipage}[b]{\linewidth}\raggedright
Community (James Jessiman)
\end{minipage} & \begin{minipage}[b]{\linewidth}\raggedright
Corporate (Internal R\&D)
\end{minipage} & \begin{minipage}[b]{\linewidth}\raggedright
Corporate (Superscape)
\end{minipage} \\
\begin{minipage}[b]{\linewidth}\raggedright
\textbf{Data Structure}
\end{minipage} & \begin{minipage}[b]{\linewidth}\raggedright
Text-based, procedural geometry, Primitives
\end{minipage} & \begin{minipage}[b]{\linewidth}\raggedright
High-fidelity SGI models, NURBS
\end{minipage} & \begin{minipage}[b]{\linewidth}\raggedright
Voxel / Simplified Mesh
\end{minipage} \\
\begin{minipage}[b]{\linewidth}\raggedright
\textbf{Accessibility}
\end{minipage} & \begin{minipage}[b]{\linewidth}\raggedright
Free, Open Source, DOS/Win
\end{minipage} & \begin{minipage}[b]{\linewidth}\raggedright
Internal / Proprietary
\end{minipage} & \begin{minipage}[b]{\linewidth}\raggedright
Consumer CD-ROM
\end{minipage} \\
\begin{minipage}[b]{\linewidth}\raggedright
\textbf{Philosophy}
\end{minipage} & \begin{minipage}[b]{\linewidth}\raggedright
"Digital CAD" (Precision)
\end{minipage} & \begin{minipage}[b]{\linewidth}\raggedright
"Digital Replica" (Physics)
\end{minipage} & \begin{minipage}[b]{\linewidth}\raggedright
"Digital Toy" (Play)
\end{minipage} \\
\begin{minipage}[b]{\linewidth}\raggedright
\textbf{Legacy}
\end{minipage} & \begin{minipage}[b]{\linewidth}\raggedright
Foundation of Studio/Mecabricks
\end{minipage} & \begin{minipage}[b]{\linewidth}\raggedright
Influenced LDD/Movies
\end{minipage} & \begin{minipage}[b]{\linewidth}\raggedright
Early gaming experiments
\end{minipage} \\
\midrule\noalign{}
\endhead
\bottomrule\noalign{}
\endlastfoot
\end{longtable}

\subsection{3. The Divergence of Tools: Engineering vs. Play
(2000--2015)}\label{the-divergence-of-tools-engineering-vs.-play-20002015}

Following the turn of the millennium, the digital LEGO ecosystem
bifurcated into two distinct cultures. The AFOL (Adult Fan of LEGO)
community pursued precision, documentation, and rendering quality
through LDraw-based tools, while the LEGO Group pursued accessibility
and "fluid play" through strictly controlled, user-friendly software
environments.

\subsubsection{3.1 The "MLCad" Era and the European
Influence}\label{the-mlcad-era-and-the-european-influence}

While LDraw provided the file format, editing LDraw files in a text
editor was impractical for complex models. \emph{MLCad}
(Mike\textquotesingle s Lego CAD), developed by Michael Lachman, became
the dominant editor for the sophisticated user base.\textsuperscript{16}
The development and adoption of MLCad revealed a strong European,
particularly Germanic, center of gravity in the digital LEGO community.
Extensive manuals, tutorials, and support forums were authored in German
and Italian, reflecting a culture that valued precision engineering and
complex instruction design.\textsuperscript{16}

MLCad introduced advanced features that professionalized digital
building, effectively turning the hobbyist into a digital draftsman:

\begin{itemize}
\item
  \textbf{Step Management:} MLCad allowed users to define building steps
  and sub-models (using the .mpd Multipart Document extension). This
  enabled users to author professional-grade building instructions, a
  capability that LEGO\textquotesingle s own consumer tools
  lacked.\textsuperscript{16}
\item
  \textbf{Buffer Exchange:} To handle large models on the limited
  hardware of the early 2000s, MLCad utilized "buffer exchange"
  techniques, optimizing memory usage---a direct response to the "bloat"
  of visual CAD systems.
\item
  \textbf{Flexible Parts Synthesis:} Integration with \emph{LSynth}
  allowed users to generate flexible elements like hoses, pneumatic
  tubes, and rubber bands. These elements are mathematically complex to
  model because their shape depends on their connection points and path.
  LSynth utilized a "synthesizing" approach, generating the geometry
  procedurally based on control points, a level of sophistication that
  mimicked professional engineering CAD tools.\textsuperscript{16}
\end{itemize}

The "MLCad" ecosystem represented a "high-technical" culture within the
LEGO community. Users were expected to understand file paths, library
structures, and coordinate geometry. This created a high barrier to
entry but resulted in models of professional CAD quality, often
exceeding the fidelity of LEGO\textquotesingle s own promotional
materials.

\subsubsection{3.2 L3P and the Pursuit of
Photorealism}\label{l3p-and-the-pursuit-of-photorealism}

The LDraw community\textquotesingle s obsession with fidelity extended
to visualization. The tool \emph{L3P}, developed by Lars C. Hassing in
1998, acted as a bridge between the mathematical precision of LDraw and
the ray-tracing capabilities of \emph{POV-Ray} (Persistence of Vision
Raytracer).\textsuperscript{19}

L3P was not a renderer itself; it was a \emph{compiler} that translated
the LDraw geometry into the POV-Ray scene description language.

\begin{itemize}
\item
  \textbf{The LGEO Library:} A significant limitation of LDraw was its
  mathematical perfection; LDraw definitions had infinitely sharp edges,
  which looked fake when rendered. To solve this, the community
  developed the \emph{LGEO} library. LGEO replaced standard LDraw
  primitives with POV-Ray definitions that included "fillets" (rounded
  edges) and realistic refraction indices for transparent ABS
  plastic.\textsuperscript{20}
\item
  \textbf{L3PAO:} To manage the complex command-line switches required
  by L3P, the community developed \emph{L3PAO} (L3P Add-On), a graphical
  interface that democratized high-end rendering. This allowed
  non-programmers to control lighting, camera angles, and background
  settings.\textsuperscript{20}
\end{itemize}

The visuals produced by this pipeline---characterized by reflective
studs, subsurface scattering on trans-neon parts, and soft shadows---set
a visual benchmark. For nearly a decade, fan-rendered images using
L3P/POV-Ray were often superior to the official renders produced by the
LEGO Group, which were constrained by the need for fast, real-time web
graphics.

\subsubsection{3.3 LEGO Digital Designer (LDD): The Corporate
Response}\label{lego-digital-designer-ldd-the-corporate-response}

In 2004, the LEGO Group released \emph{LEGO Digital Designer} (LDD),
built on the "Project Arena" code which itself was a descendant of the
\emph{LEGO Creator} game engine.\textsuperscript{23} LDD was designed
with a fundamentally different epistemology than LDraw: it was a toy
first, and a tool second.

\begin{itemize}
\item
  \textbf{The Interface:} LDD utilized a "click-and-snap" interface with
  a heuristic connectivity engine. The software "knew" where a brick
  could connect, preventing illegal connections. In contrast, LDraw
  editors allowed parts to intersect or float, prioritizing user freedom
  over physical constraints.\textsuperscript{1}
\item
  \textbf{Design byME:} LDD was initially inextricably linked to a
  physical fulfillment service, \emph{LEGO Design byME} (formerly LEGO
  Factory). This service allowed users to order their custom digital
  creations as boxed sets. This ambitious integration of digital design
  and physical logistics failed due to the high cost of manual picking,
  quality control issues, and the sheer complexity of managing millions
  of individual elements. The service was cancelled in 2012, severing
  the link between the digital and physical products.\textsuperscript{1}
\item
  \textbf{The LXF Format:} LDD used a proprietary .lxf (LEGO Exchange
  Format) based on XML. Unlike LDraw, the part geometry was encrypted or
  compressed within the application assets (db.lif), making it difficult
  for the community to modify or expand the library. This "black box"
  approach limited the tool\textquotesingle s utility for advanced users
  who needed parts that LEGO had not yet officially
  digitized.\textsuperscript{24}
\end{itemize}

LDD suffered from "technical debt" and a lack of sustained investment.
As internal teams at LEGO moved to more advanced tools (like \emph{LDD
Pro} and eventual integrations with Maya/Unity), the public version of
LDD stagnated. It was officially "defunded" in 2016, leaving the
community in a state of limbo until the acquisition of
BrickLink.\textsuperscript{28}

\subsection{4. The Sociological Architecture: Communities, Gender, and
Knowledge}\label{the-sociological-architecture-communities-gender-and-knowledge}

The digital ecosystem was not merely a repository of parts; it was a
repository of culture. The architecture of these platforms shaped who
participated, how knowledge was transferred, and how the value of the
"digital brick" was perceived by different demographics.

\subsubsection{4.1 The "Japanese Anomaly": CUUSOO and the Origins of
LEGO
Ideas}\label{the-japanese-anomaly-cuusoo-and-the-origins-of-lego-ideas}

While the Western narrative focuses on LDraw, LUGNET, and Eurobricks, a
parallel digital innovation occurred in Japan through the \emph{CUUSOO}
platform. Launched in 2008 as a partnership with a Japanese open
innovation company, \emph{LEGO CUUSOO} (meaning "daydream" or "fantasy"
in Japanese) established the mechanism for crowdsourcing product
development.\textsuperscript{25}

The architecture of CUUSOO was distinct:

\begin{itemize}
\item
  \textbf{Beta Phase:} Initially restricted to Japan with a threshold of
  1,000 votes, the platform tapped into a unique Japanese fan culture
  that revered "mecha" and detailed miniaturization. The first success,
  the \emph{Shinkai 6500} submersible, was a hyper-realistic scientific
  model, setting a tone of serious "adult" building.\textsuperscript{25}
\item
  \textbf{Global Expansion:} The platform went global in 2011, raising
  the threshold to 10,000 votes and rebranding as \emph{LEGO Ideas} in
  2014. This shift represented a critical architectural change: the
  digital model (often built in LDD or LDraw) became a \emph{currency}
  of social capital. The ability to render a persuasive, photorealistic
  image (using tools like POV-Ray or later Blender) became a
  prerequisite for commercial success. The digital file was no longer
  just a blueprint; it was a marketing asset.\textsuperscript{31}
\end{itemize}

\subsubsection{4.2 Gendered Technicity and "Brooms and
Bonnets"}\label{gendered-technicity-and-brooms-and-bonnets}

Research into the sociology of the digital LEGO ecosystem reveals deeply
entrenched gender biases that were codified into the software tools
themselves. The "LEGOfied" media studies analysis highlights how the
technical language of LDraw---with its emphasis on coordinates,
"primitive substitution," and CAD-like interfaces---historically
codified the digital ecosystem as a male-dominated engineering
domain.\textsuperscript{32}

\begin{itemize}
\item
  \textbf{Marketing Bias:} Analysis of marketing imagery shows a
  persistent bias where "City" and "Technic" themes heavily feature male
  representation, while "Friends" (and its specific color palette) is
  segregated. This segregation seeped into the digital
  tools.\textsuperscript{34}
\item
  \textbf{The "Pink Brick" Problem:} In digital tools like LDD and
  LDraw, the "Friends" color palette (pastels, vibrant pinks/purples)
  and specific mini-doll parts were often added later or segregated in
  the user interface. This reinforced the "othering" of feminine-coded
  play. The default palettes and "starter sets" in LDD often prioritized
  standard "boy" colors (red, blue, yellow, grey), implicitly guiding
  the user toward specific types of construction.\textsuperscript{32}
\item
  \textbf{Professional Stigma:} Within professional architecture, the
  use of LEGO (physical or digital) is often stigmatized as "childish."
  Despite the modular utility of the brick, Reddit threads from
  architecture students and professionals reveal a strong bias against
  using LEGO for serious modeling in university settings. The "toy"
  connotation overrides the "tool" utility, preventing LDD or Studio
  from being adopted as a sketch tool alongside SketchUp or
  Rhino.\textsuperscript{6}
\end{itemize}

\subsubsection{4.3 The Mindstorms Discontinuation and the "Black Box" of
Education}\label{the-mindstorms-discontinuation-and-the-black-box-of-education}

The trajectory of \emph{LEGO Mindstorms} (1998--2022) traces a shift
from "hacker-friendly" openness to "classroom-safe" closure.

\begin{itemize}
\item
  \textbf{The Hacker Era:} The original RCX and NXT bricks were cracked
  by the community almost immediately. Tools like \emph{BricxCC}
  (developed by John Hansen) and languages like \emph{NQC} (Not Quite C)
  allowed users to bypass the limitations of the official graphical
  software. This era fostered a generation of engineers who learned
  low-level programming through reverse-engineering the
  brick.\textsuperscript{36}
\item
  \textbf{The Closure:} The discontinuation of Mindstorms in 2022 in
  favor of \emph{SPIKE Prime} signals a strategic pivot. \emph{SPIKE} is
  strictly educational, Python-based (official), and lacks the rugged
  "hacker" ethos of the EV3 era.\textsuperscript{39} The "Robot
  Inventor" app will eventually cease to function, rendering the
  hardware potentially obsolete---a form of digital obsolescence that
  creates a massive knowledge gap for future robotics preservation. The
  "Black Box" has closed; the user is now a student following a
  curriculum, not a hacker exploring a system.\textsuperscript{41}
\end{itemize}

\subsection{5. Convergence and the Modern Synthesis
(2016--Present)}\label{convergence-and-the-modern-synthesis-2016present}

The fragmentation of the 2000s has resolved into a consolidated,
corporate-owned ecosystem, primarily through the LEGO
Group\textquotesingle s strategic acquisition of \emph{BrickLink} in
2019. This acquisition brought the community\textquotesingle s premier
marketplace and its emerging software standard under the corporate
umbrella.

\subsubsection{5.1 BrickLink Studio: The Unification
Tool}\label{bricklink-studio-the-unification-tool}

\emph{BrickLink Studio} (often referred to simply as Studio) emerged as
a third-party competitor to LDD but was integrated into the official
LEGO ecosystem following the acquisition. It successfully synthesized
the two historical tracks that had been divergent for twenty years:

\begin{enumerate}
\def\labelenumi{\arabic{enumi}.}
\item
  \textbf{LDraw DNA:} Studio utilizes a part library architecture that
  is fundamentally compatible with LDraw. It allows for the import of
  .ldr and .mpd files, satisfying the AFOL demand for legacy
  compatibility and precision.\textsuperscript{42}
\item
  \textbf{LDD Usability:} It incorporates the "snap" connectivity and
  stability checking of LDD, satisfying the casual user who requires
  heuristic guidance.
\item
  \textbf{Commercial Integration:} Crucially, Studio is directly linked
  to the BrickLink marketplace. It allows users to "price out" their
  digital designs and generate "Wanted Lists" for purchase. This finally
  realizes the failed promise of \emph{Design byME}---integration of
  design and logistics---but it achieves it by offloading the logistics
  to the decentralized network of BrickLink sellers rather than a
  centralized LEGO factory.\textsuperscript{43}
\end{enumerate}

The official retirement of LDD in 2022 marked the formal end of
LEGO\textquotesingle s internal consumer CAD development. The LEGO Group
effectively outsourced the maintenance of the digital building
experience to the (now owned) community platform, acknowledging that the
community tool had surpassed the corporate tool.\textsuperscript{2}

\subsubsection{5.2 Technical Debt and the "Missing Parts"
Crisis}\label{technical-debt-and-the-missing-parts-crisis}

Despite this unification, significant architectural gaps remain. The
digital library is never complete. Snippet analysis reveals a constant
struggle with "missing parts," particularly complex new molds,
dual-molded elements, or licensed parts with unique
prints.\textsuperscript{27}

\begin{itemize}
\item
  \textbf{The Lag:} The LDraw library (which feeds Studio) often lags
  behind new part releases. The \emph{LDraw Parts Tracker} operates on a
  volunteer basis, and the backlog of "uncertified" parts creates a
  versioning crisis. A model built today using an uncertified part might
  not open correctly in five years if the part number changes or the
  reference is lost.
\item
  \textbf{Licensing Restructuring:} The transition to \textbf{CC-BY 4.0}
  licensing for LDraw parts in 2023 represents a major legal
  restructuring. This move attempts to secure the
  library\textquotesingle s future usage in open-source projects while
  navigating the complex trademark waters of the LEGO Group, ensuring
  that the "digital brick" remains a commons rather than a purely
  corporate asset.\textsuperscript{44}
\end{itemize}

\subsection{6. Future Horizons: AI, Generative Design, and Gaussian
Splatting}\label{future-horizons-ai-generative-design-and-gaussian-splatting}

The frontier of the digital LEGO ecosystem is moving beyond manual CAD
into the realm of Artificial Intelligence and Machine Learning (ML). The
static mesh is giving way to the dynamic field.

\subsubsection{6.1 Generative LEGO Design}\label{generative-lego-design}

Recent research from the University of Guelph \textsuperscript{46}
explores the use of Deep Generative Models to automate the construction
of LEGO structures. By representing bricks as nodes in a graph and
connections as edges, ML models can "learn" to build. This fundamentally
shifts the user\textquotesingle s role from "builder" to "curator,"
where the software suggests structural completions or optimizations.
This parallels the "Auto-Complete" functionality in text, suggesting a
future where digital building tools assist in the creative process by
predicting the next brick.

\subsubsection{6.2 Gaussian Splatting and the End of
Geometry}\label{gaussian-splatting-and-the-end-of-geometry}

The paper \emph{GaussianUpdate: Continual 3D Gaussian Splatting Update
for Changing Environments} \textsuperscript{47} introduces a paradigm
shift in visualization that renders the debate between LDraw and LDD
moot. Unlike traditional rendering (POV-Ray/L3P) which relies on defined
polygon geometry, Gaussian Splatting represents scenes as a cloud of 3D
Gaussians (volumetric blobs) that can be updated in real-time.

\begin{itemize}
\item
  \textbf{Implication:} This technology allows for the capture and
  rendering of \emph{dynamic} LEGO scenes (including lighting changes
  and object removal) without the computationally expensive ray-tracing
  of the past. It does not require a mesh; it requires a neural
  representation of the light field.
\item
  \textbf{Application:} This could enable "Mixed Reality" play where
  physical LEGO builds are scanned and augmented with digital effects in
  real-time, fulfilling the "Fluid Play" promise that LEGO has chased
  since the \emph{Darwin} era. It suggests a future where the "digital
  brick" is not a CAD file, but a learned neural
  feature.\textsuperscript{49}
\end{itemize}

\subsubsection{6.3 Table: The Evolution of LEGO Rendering
Technologies}\label{table-the-evolution-of-lego-rendering-technologies}

\begin{longtable}[]{@{}
  >{\raggedright\arraybackslash}p{(\linewidth - 8\tabcolsep) * \real{0.2000}}
  >{\raggedright\arraybackslash}p{(\linewidth - 8\tabcolsep) * \real{0.2000}}
  >{\raggedright\arraybackslash}p{(\linewidth - 8\tabcolsep) * \real{0.2000}}
  >{\raggedright\arraybackslash}p{(\linewidth - 8\tabcolsep) * \real{0.2000}}
  >{\raggedright\arraybackslash}p{(\linewidth - 8\tabcolsep) * \real{0.2000}}@{}}
\toprule\noalign{}
\begin{minipage}[b]{\linewidth}\raggedright
\textbf{Era}
\end{minipage} & \begin{minipage}[b]{\linewidth}\raggedright
\textbf{Technology}
\end{minipage} & \begin{minipage}[b]{\linewidth}\raggedright
\textbf{Mechanism}
\end{minipage} & \begin{minipage}[b]{\linewidth}\raggedright
\textbf{Key Advantage}
\end{minipage} & \begin{minipage}[b]{\linewidth}\raggedright
\textbf{Key Limitation}
\end{minipage} \\
\begin{minipage}[b]{\linewidth}\raggedright
\textbf{1995}
\end{minipage} & \begin{minipage}[b]{\linewidth}\raggedright
\textbf{Wireframe/Flat}
\end{minipage} & \begin{minipage}[b]{\linewidth}\raggedright
Vector lines (LEdit)
\end{minipage} & \begin{minipage}[b]{\linewidth}\raggedright
Fast on 486 PCs
\end{minipage} & \begin{minipage}[b]{\linewidth}\raggedright
No depth, abstract
\end{minipage} \\
\begin{minipage}[b]{\linewidth}\raggedright
\textbf{1998}
\end{minipage} & \begin{minipage}[b]{\linewidth}\raggedright
\textbf{L3P + POV-Ray}
\end{minipage} & \begin{minipage}[b]{\linewidth}\raggedright
Ray-tracing scripts
\end{minipage} & \begin{minipage}[b]{\linewidth}\raggedright
Photorealism, reflections
\end{minipage} & \begin{minipage}[b]{\linewidth}\raggedright
Slow render times, static
\end{minipage} \\
\begin{minipage}[b]{\linewidth}\raggedright
\textbf{2004}
\end{minipage} & \begin{minipage}[b]{\linewidth}\raggedright
\textbf{LDD (Real-time)}
\end{minipage} & \begin{minipage}[b]{\linewidth}\raggedright
OpenGL/DirectX
\end{minipage} & \begin{minipage}[b]{\linewidth}\raggedright
Instant feedback
\end{minipage} & \begin{minipage}[b]{\linewidth}\raggedright
"Video game" look, low fidelity
\end{minipage} \\
\begin{minipage}[b]{\linewidth}\raggedright
\textbf{2014}
\end{minipage} & \begin{minipage}[b]{\linewidth}\raggedright
\textbf{Eyesight (Studio)}
\end{minipage} & \begin{minipage}[b]{\linewidth}\raggedright
Photorealistic rendering
\end{minipage} & \begin{minipage}[b]{\linewidth}\raggedright
Integrated, easy to use
\end{minipage} & \begin{minipage}[b]{\linewidth}\raggedright
High GPU cost
\end{minipage} \\
\begin{minipage}[b]{\linewidth}\raggedright
\textbf{2025+}
\end{minipage} & \begin{minipage}[b]{\linewidth}\raggedright
\textbf{Gaussian Splatting}
\end{minipage} & \begin{minipage}[b]{\linewidth}\raggedright
Radiance Fields
\end{minipage} & \begin{minipage}[b]{\linewidth}\raggedright
Real-time dynamic updates
\end{minipage} & \begin{minipage}[b]{\linewidth}\raggedright
High memory usage, new tech
\end{minipage} \\
\midrule\noalign{}
\endhead
\bottomrule\noalign{}
\endlastfoot
\end{longtable}

\subsection{7. Conclusion: The Knowledge Gaps and Preservation
Risks}\label{conclusion-the-knowledge-gaps-and-preservation-risks}

The architecture of the Digital LEGO Ecosystem is characterized by a
"Dual-Core" processor history: the Community Core (LDraw) and the
Corporate Core (Darwin/LDD). While these cores have recently merged via
BrickLink Studio, significant risks and gaps remain in the historical
and technical record.

\textbf{Identified Knowledge Gaps:}

\begin{enumerate}
\def\labelenumi{\arabic{enumi}.}
\item
  \textbf{The "Dark Age" of Internal Tools:} There is almost no public
  documentation, code preservation, or visual record of the
  \emph{Panter} (1986) or \emph{L3D} (1996) database structures. These
  represent "lost species" in the digital evolution of the brick.
  Without oral histories or archival recovery, the origins of
  LEGO\textquotesingle s digital syntax remain obscure.
\item
  \textbf{Robotics Firmware Preservation:} With the shutdown of the
  \emph{Mindstorms} servers and apps, the specific "handshake" protocols
  and firmware versions required to run NXT/EV3 robots are at risk of
  becoming abandonware. The ecosystem relies entirely on aging
  third-party tools like \emph{BricxCC}, which may not function on
  future operating systems.
\item
  \textbf{Algorithmic Bias:} The "Pink Brick" phenomenon suggests that
  the very tools used to design LEGO sets may encode gender biases
  (e.g., categorizing certain colors or parts as "decorative" vs.
  "structural"). Further research is needed to deconstruct the "default"
  settings of these CAD tools.
\item
  \textbf{Legal Fragility:} The reliance on the LDraw library---which
  exists in a legal grey area of trademark tolerance---poses a long-term
  risk. If the LEGO Group were to enforce strict IP control on the
  digital representation of its patented stud-and-tube coupling, the
  entire open ecosystem could collapse.
\end{enumerate}

Final Assessment:

The Digital LEGO Ecosystem is a triumph of participatory culture over
corporate secrecy. It proved that a community of volunteers could
maintain a more accurate, diverse, and functional digital library than
the manufacturer itself. However, as the ecosystem moves toward
AI-driven generation and cloud-based platforms, the
user\textquotesingle s ability to "own" the digital brick---to see the
text file behind the render---is diminishing. The future challenge lies
not in rendering the brick more realistically, but in preserving the
open standards that allow it to be virtually interlocked by anyone,
anywhere. The digital brick must remain a language, not just an image.

\subsubsection{Source Data Integration}\label{source-data-integration}

\begin{itemize}
\item
  \textbf{LDraw Policies \& Standards:} \textsuperscript{12}
\item
  \textbf{Internal History (Darwin/Panter):} \textsuperscript{4}
\item
  \textbf{Rendering Tech (L3P/Gaussian):} \textsuperscript{19}
\item
  \textbf{Discontinuation \& Corporate Strategy:} \textsuperscript{28}
\item
  \textbf{Sociological \& Gender Analysis:} \textsuperscript{6}
\item
  \textbf{Japanese Ecosystem (CUUSOO):} \textsuperscript{25}
\end{itemize}

\paragraph{Works cited}\label{works-cited}

\begin{enumerate}
\def\labelenumi{\arabic{enumi}.}
\item
  Lego Digital Designer - Wikipedia, accessed December 10, 2025,
  \href{https://en.wikipedia.org/wiki/Lego_Digital_Designer}{\ul{https://en.wikipedia.org/wiki/Lego\_Digital\_Designer}}
\item
  RIP LEGO Digital Designer - True North Bricks, accessed December 10,
  2025,
  \href{https://truenorthbricks.com/2022/01/13/rip-lego-digital-designer/}{\ul{https://truenorthbricks.com/2022/01/13/rip-lego-digital-designer/}}
\item
  LDraw - Wikipedia, accessed December 10, 2025,
  \href{https://en.wikipedia.org/wiki/LDraw}{\ul{https://en.wikipedia.org/wiki/LDraw}}
\item
  The meaning or story behind the ``reversed'' Octan logo - lego -
  Reddit, accessed December 10, 2025,
  \href{https://www.reddit.com/r/lego/comments/1o23c6g/the_meaning_or_story_behind_the_reversed_octan/}{\ul{https://www.reddit.com/r/lego/comments/1o23c6g/the\_meaning\_or\_story\_behind\_the\_reversed\_octan/}}
\item
  Inside the LEGO Group\textquotesingle s Secretive Strategic Product
  Unit Darwin, accessed December 10, 2025,
  \href{https://www.lego.com/cdn/cs/set/assets/blt4212e2be20008c99/bits_n_bricks_s01e16_darwin_feature_and_transcript.pdf}{\ul{https://www.lego.com/cdn/cs/set/assets/blt4212e2be20008c99/bits\_n\_bricks\_s01e16\_darwin\_feature\_and\_transcript.pdf}}
\item
  Is it acceptable to use Legos to make models? : r/architecture -
  Reddit, accessed December 10, 2025,
  \href{https://www.reddit.com/r/architecture/comments/vozoip/is_it_acceptable_to_use_legos_to_make_models/}{\ul{https://www.reddit.com/r/architecture/comments/vozoip/is\_it\_acceptable\_to\_use\_legos\_to\_make\_models/}}
\item
  Can i make my uni architect models with lego? : r/architecture -
  Reddit, accessed December 10, 2025,
  \href{https://www.reddit.com/r/architecture/comments/uq8bim/can_i_make_my_uni_architect_models_with_lego/}{\ul{https://www.reddit.com/r/architecture/comments/uq8bim/can\_i\_make\_my\_uni\_architect\_models\_with\_lego/}}
\item
  software angle tool: Topics by Science.gov, accessed December 10,
  2025,
  \href{https://www.science.gov/topicpages/s/software+angle+tool.html}{\ul{https://www.science.gov/topicpages/s/software+angle+tool.html}}
\item
  Bits N\textquotesingle{} Bricks Season 5 Episode 47: The Rise of
  BrickLink Feature and Transcript - LEGO, accessed December 10, 2025,
  \href{https://www.lego.com/cdn/cs/set/assets/bltf643219fa5bd3d27/bits_n_bricks_s05e47_feature_and_transcript.pdf}{\ul{https://www.lego.com/cdn/cs/set/assets/bltf643219fa5bd3d27/bits\_n\_bricks\_s05e47\_feature\_and\_transcript.pdf}}
\item
  How Lego Became The Apple Of Toys \textbar{} Flip Consulting, accessed
  December 10, 2025,
  \href{https://flipconsulting.studio/wp-content/uploads/2020/02/How-Lego-Became-The-Apple-Of-Toys.pdf}{\ul{https://flipconsulting.studio/wp-content/uploads/2020/02/How-Lego-Became-The-Apple-Of-Toys.pdf}}
\item
  LEGO in a Digital World - Technology and Operations Management,
  accessed December 10, 2025,
  \href{https://d3.harvard.edu/platform-rctom/submission/lego-in-a-digital-world/}{\ul{https://d3.harvard.edu/platform-rctom/submission/lego-in-a-digital-world/}}
\item
  About Us - LDraw.org, accessed December 10, 2025,
  \href{https://www.ldraw.org/help/about-us}{\ul{https://www.ldraw.org/help/about-us}}
\item
  Community: James Jessiman Memorial - LDraw, accessed December 10,
  2025,
  \href{https://www.ldraw.org/community/james-jessiman-memorial.html}{\ul{https://www.ldraw.org/community/james-jessiman-memorial.html}}
\item
  Just James - LDraw.org, accessed December 10, 2025,
  \href{https://www.ldraw.org/community/just-james.html}{\ul{https://www.ldraw.org/community/just-james.html}}
\item
  Community: LDraw.org Standards Board Charter, accessed December 10,
  2025,
  \href{https://www.ldraw.org/article/382.html}{\ul{https://www.ldraw.org/article/382.html}}
\item
  Holly-Wood.it \textgreater{} MLCad, accessed December 10, 2025,
  \href{https://www.holly-wood.it/mlcad-en.html}{\ul{https://www.holly-wood.it/mlcad-en.html}}
\item
  MLCAD: A Survey of Research in Machine Learning for CAD Keynote Paper,
  accessed December 10, 2025,
  \href{https://www.researchgate.net/publication/355870373_MLCAD_A_Survey_of_Research_in_Machine_Learning_for_CAD_Keynote_Paper}{\ul{https://www.researchgate.net/publication/355870373\_MLCAD\_A\_Survey\_of\_Research\_in\_Machine\_Learning\_for\_CAD\_Keynote\_Paper}}
\item
  TobyLobster/ImportLDraw: A Blender plug-in for importing LDraw file
  format Lego models and parts. - GitHub, accessed December 10, 2025,
  \href{https://github.com/TobyLobster/ImportLDraw}{\ul{https://github.com/TobyLobster/ImportLDraw}}
\item
  L3P - render any LDRAW model in POV-Ray - Google Groups, accessed
  December 10, 2025,
  \href{https://groups.google.com/g/rec.toys.lego/c/mSZ-SgwL9-g}{\ul{https://groups.google.com/g/rec.toys.lego/c/mSZ-SgwL9-g}}
\item
  Conversion 101 - LDraw.org Wiki, accessed December 10, 2025,
  \href{https://wiki.ldraw.org/wiki/Conversion_101}{\ul{https://wiki.ldraw.org/wiki/Conversion\_101}}
\item
  POV-Ray - Studio Help Center, accessed December 10, 2025,
  \href{https://studiohelp.bricklink.com/hc/en-us/articles/6506022103447-POV-Ray}{\ul{https://studiohelp.bricklink.com/hc/en-us/articles/6506022103447-POV-Ray}}
\item
  The NXT STEP in LEGO® Robotics - RSSing.com, accessed December 10,
  2025,
  \href{https://mindstorms82.rssing.com/chan-37719032/all_p4.html}{\ul{https://mindstorms82.rssing.com/chan-37719032/all\_p4.html}}
\item
  Inside one of the most important LEGO® games ever made, accessed
  December 10, 2025,
  \href{https://www.lego.com/cdn/cs/set/assets/blte53dbf634332aa73/bits_n_bricks_s04e43_feature_and_transcript.pdf}{\ul{https://www.lego.com/cdn/cs/set/assets/blte53dbf634332aa73/bits\_n\_bricks\_s04e43\_feature\_and\_transcript.pdf}}
\item
  lego digital designer - dkor, accessed December 10, 2025,
  \href{https://dkor.wordpress.com/tag/lego-digital-designer/}{\ul{https://dkor.wordpress.com/tag/lego-digital-designer/}}
\item
  LEGO® Ideas \textbar{} LEGO® History \textbar{} LEGO.com US, accessed
  December 10, 2025,
  \href{https://www.lego.com/en-us/history/articles/j-lego-ideas}{\ul{https://www.lego.com/en-us/history/articles/j-lego-ideas}}
\item
  Category:LEGO Digital Designer - LDraw.org Wiki, accessed December 10,
  2025,
  \href{https://wiki.ldraw.org/wiki/Category:LEGO_Digital_Designer}{\ul{https://wiki.ldraw.org/wiki/Category:LEGO\_Digital\_Designer}}
\item
  LDraw to LDD conversion, accessed December 10, 2025,
  \href{https://wiki.ldraw.org/wiki/LDraw_to_LDD_conversion}{\ul{https://wiki.ldraw.org/wiki/LDraw\_to\_LDD\_conversion}}
\item
  LEGO Digital Designer officially defunded and unsupported {[}News{]} -
  The Brothers Brick, accessed December 10, 2025,
  \href{https://www.brothers-brick.com/2016/01/21/lego-digital-designer-officially-defunded-and-unsupported-news/}{\ul{https://www.brothers-brick.com/2016/01/21/lego-digital-designer-officially-defunded-and-unsupported-news/}}
\item
  The LEGO Group will focus on BrickLink Studio and pull back support
  for LEGO® Digital Designer. - LEGO Ambassador Network, accessed
  December 10, 2025,
  \href{https://lan.lego.com/news/overview/the-lego-group-will-focus-on-bricklink-studio-and-pull-back-support-for-lego\%C2\%AE-digital-designer-r301/}{\ul{https://lan.lego.com/news/overview/the-lego-group-will-focus-on-bricklink-studio-and-pull-back-support-for-lego\%C2\%AE-digital-designer-r301/}}
\item
  From Tokyo Dream to Global Playground: The Story of LEGO® Cuusoo, the
  Fan-Powered Dream Machine - Brick Legions, accessed December 10, 2025,
  \href{https://www.bricklegions.com/2024/01/02/from-tokyo-dream-to-global-playground-the-story-of-lego-cuusoo-the-fan-powered-dream-machine/}{\ul{https://www.bricklegions.com/2024/01/02/from-tokyo-dream-to-global-playground-the-story-of-lego-cuusoo-the-fan-powered-dream-machine/}}
\item
  LEGO Ideas Sets: From Fan Creation to Valuable Collection - BlockApps
  Inc., accessed December 10, 2025,
  \href{https://blockapps.net/blog/lego-ideas-sets-from-fan-creation-to-valuable-collection/}{\ul{https://blockapps.net/blog/lego-ideas-sets-from-fan-creation-to-valuable-collection/}}
\item
  LEGOfied: Building Blocks as Media: Nicholas Taylor - Bloomsbury
  Publishing, accessed December 10, 2025,
  \href{https://www.bloomsbury.com/us/legofied-9781501354052/}{\ul{https://www.bloomsbury.com/us/legofied-9781501354052/}}
\item
  LEGOfied: building blocks as media, edited by Nicholas Taylor \& Chris
  Ingraham New York: Bloomsbury Academic 2020 FOREWORD - Microethology,
  accessed December 10, 2025,
  \href{https://www.microethology.net/wp-content/uploads/2020/03/LEGOfied-foreword.pdf}{\ul{https://www.microethology.net/wp-content/uploads/2020/03/LEGOfied-foreword.pdf}}
\item
  The Faces of LEGO 2021: Setting A Baseline For Gender Representation
  In Marketing \textquotesingle Hero Shots\textquotesingle{} \textbar{}
  The Rambling Brick, accessed December 10, 2025,
  \href{https://ramblingbrick.com/2021/10/15/the-faces-of-lego-2021-setting-a-baseline-for-gender-representation-in-marketing-hero-shots/}{\ul{https://ramblingbrick.com/2021/10/15/the-faces-of-lego-2021-setting-a-baseline-for-gender-representation-in-marketing-hero-shots/}}
\item
  LEGOfied: Building Blocks as Media 9781501354045, 9781501354076,
  9781501354069 - DOKUMEN.PUB, accessed December 10, 2025,
  \href{https://dokumen.pub/legofied-building-blocks-as-media-9781501354045-9781501354076-9781501354069.html}{\ul{https://dokumen.pub/legofied-building-blocks-as-media-9781501354045-9781501354076-9781501354069.html}}
\item
  Bricx Command Center - Wikipedia, accessed December 10, 2025,
  \href{https://en.wikipedia.org/wiki/Bricx_Command_Center}{\ul{https://en.wikipedia.org/wiki/Bricx\_Command\_Center}}
\item
  Bricx Command Center, accessed December 10, 2025,
  \href{https://bricxcc.sourceforge.net/}{\ul{https://bricxcc.sourceforge.net/}}
\item
  Bricx Command Center 3.3 History, accessed December 10, 2025,
  \href{https://bricxcc.sourceforge.net/bricxcc_history.htm}{\ul{https://bricxcc.sourceforge.net/bricxcc\_history.htm}}
\item
  LEGO discontinuing Mindstorms brand, retiring 51515 Robot Inventor at
  end of 2022 {[}News{]}, accessed December 10, 2025,
  \href{https://robot-academy.com/lego-discontinuing-mindstorms-brand-retiring-51515-robot-inventor-at-end-of-2022-news/}{\ul{https://robot-academy.com/lego-discontinuing-mindstorms-brand-retiring-51515-robot-inventor-at-end-of-2022-news/}}
\item
  Lego Mindstorms Robotics Kits Are Being Discontinued - PCMag, accessed
  December 10, 2025,
  \href{https://www.pcmag.com/news/lego-mindstorms-robotics-kits-are-being-discontinued}{\ul{https://www.pcmag.com/news/lego-mindstorms-robotics-kits-are-being-discontinued}}
\item
  The future of LEGO Education amidst the greatest turnaround in history
  TEACHING NOTE, accessed December 10, 2025,
  \href{https://lup.lub.lu.se/student-papers/record/9150944/file/9150946.pdf}{\ul{https://lup.lub.lu.se/student-papers/record/9150944/file/9150946.pdf}}
\item
  BrickLink Studio Replaces LEGO Digital Designer - theBrickBlogger.com,
  accessed December 10, 2025,
  \href{https://thebrickblogger.com/2022/01/bricklink-studio-replaces-lego-digital-designer/}{\ul{https://thebrickblogger.com/2022/01/bricklink-studio-replaces-lego-digital-designer/}}
\item
  What software do you use to plan/design your LEGO builds? - Reddit,
  accessed December 10, 2025,
  \href{https://www.reddit.com/r/lego/comments/1n7l10z/what_software_do_you_use_to_plandesign_your_lego/}{\ul{https://www.reddit.com/r/lego/comments/1n7l10z/what\_software\_do\_you\_use\_to\_plandesign\_your\_lego/}}
\item
  LDraw.org 2023-03 Parts Update Now Available, accessed December 10,
  2025,
  \href{https://forums.ldraw.org/thread-27515.html}{\ul{https://forums.ldraw.org/thread-27515.html}}
\item
  How to import an unofficial parts archive into LDraw? - Bricks
  Stackexchange, accessed December 10, 2025,
  \href{https://bricks.stackexchange.com/questions/6220/how-to-import-an-unofficial-parts-archive-into-ldraw}{\ul{https://bricks.stackexchange.com/questions/6220/how-to-import-an-unofficial-parts-archive-into-ldraw}}
\item
  LEGO Advances Automated Physical Design - University of Guelph,
  accessed December 10, 2025,
  \href{https://www.uoguelph.ca/ccmps/news/2020/12/lego-advances-automated-physical-design}{\ul{https://www.uoguelph.ca/ccmps/news/2020/12/lego-advances-automated-physical-design}}
\item
  GaussianUpdate: Continual 3D Gaussian Splatting Update for Changing
  Environments - arXiv, accessed December 10, 2025,
  \href{https://arxiv.org/html/2508.08867v1}{\ul{https://arxiv.org/html/2508.08867v1}}
\item
  (PDF) GaussianUpdate: Continual 3D Gaussian Splatting Update for
  Changing Environments - ResearchGate, accessed December 10, 2025,
  \href{https://www.researchgate.net/publication/394458431_GaussianUpdate_Continual_3D_Gaussian_Splatting_Update_for_Changing_Environments}{\ul{https://www.researchgate.net/publication/394458431\_GaussianUpdate\_Continual\_3D\_Gaussian\_Splatting\_Update\_for\_Changing\_Environments}}
\item
  How the LEGO Group Blends the Physical and Digital to Create New Forms
  of Play, accessed December 10, 2025,
  \href{https://www.lego.com/cdn/cs/set/assets/bltfb18e128b11cda97/bits_n_bricks_s01e02_fluid_play_feature_and_transcript.pdf}{\ul{https://www.lego.com/cdn/cs/set/assets/bltfb18e128b11cda97/bits\_n\_bricks\_s01e02\_fluid\_play\_feature\_and\_transcript.pdf}}
\item
  LDraw.org Library Policies and FAQ, accessed December 10, 2025,
  \href{https://www.ldraw.org/pt-policies.html}{\ul{https://www.ldraw.org/pt-policies.html}}
\item
  GoldieBlox™ in the 2nd and 3rd Grade: Final Report, accessed December
  10, 2025,
  \href{http://girlsschools.org/wp-content/uploads/2017/11/GoldieBlox-in-the-2nd-and-3rd-Grade-Final-Report.pdf}{\ul{http://girlsschools.org/wp-content/uploads/2017/11/GoldieBlox-in-the-2nd-and-3rd-Grade-Final-Report.pdf}}
\item
  Japanese LEGO.... - General LEGO Discussion - Eurobricks Forums,
  accessed December 10, 2025,
  \href{https://www.eurobricks.com/forum/forums/topic/81168-japanese-lego/}{\ul{https://www.eurobricks.com/forum/forums/topic/81168-japanese-lego/}}
\item
  Brickipedia:LEGO Wikis in Other Languages - Fandom, accessed December
  10, 2025,
  \href{https://brickipedia.fandom.com/wiki/Brickipedia:LEGO_Wikis_in_Other_Languages}{\ul{https://brickipedia.fandom.com/wiki/Brickipedia:LEGO\_Wikis\_in\_Other\_Languages}}
\end{enumerate}
