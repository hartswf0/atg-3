\section{The Dialectics of the Digital Commons: Tracing Critical
Pedagogy from Freirean Praxis to Liberatory
Simulation}\label{the-dialectics-of-the-digital-commons-tracing-critical-pedagogy-from-freirean-praxis-to-liberatory-simulation}

\subsection{Introduction: The Ontological Vocation in a Rendered
World}\label{introduction-the-ontological-vocation-in-a-rendered-world}

The trajectory of critical pedagogy, a philosophy of education rooted in
the struggle for humanization and social justice, has undergone a
profound metamorphosis in the transition from the analog literacy
campaigns of the mid-20th century to the algorithmic environments of the
21st. At the heart of this evolution remains the seminal work of Paulo
Freire, whose insistence on \emph{conscientização}---the development of
critical consciousness---provides the theoretical bedrock for
contemporary investigations into digital literacy, creative coding, and
participatory urbanism. This report offers an exhaustive genealogical
analysis of these developments, tracing the remediation of Freirean
principles through the lineage of Constructionism, critical digital
pedagogy, and the emerging genre of "serious toys" in architectural and
urban design.

The inquiry posits that the tools of digital simulation---specifically
game engines like Unity and social platforms like Roblox---are not
merely neutral instruments of representation but are active sites of
pedagogical and political contestation. As these platforms are adopted
by educators, architects, and community planners, they generate new
"limit-situations" where the tension between liberatory agency and
"platformized" surveillance plays out. By synthesizing primary sources,
archival materials from major design conferences (ACADIA, CAADRIA,
ISEA), and technical reports on "civic tech," this analysis illuminates
how the "ontological vocation" of becoming more human, as defined by
Freire, is being renegotiated in the virtual commons. The report further
investigates the rise of "participatory digital twins" and
"counter-twinning" as methodologies that challenge the technocratic
hegemony of the "Smart City," proposing instead a "Playable City"
framework that recovers the right to the city through dialogic play.

\subsection{Part I: The Genealogical Roots of Critical Pedagogy and the
Politics of
Literacy}\label{part-i-the-genealogical-roots-of-critical-pedagogy-and-the-politics-of-literacy}

\subsubsection{\texorpdfstring{1.1 The Brazilian Context and the
Philosophy of
\emph{\textbf{Conscientização}}}{1.1 The Brazilian Context and the Philosophy of Conscientização}}\label{the-brazilian-context-and-the-philosophy-of-conscientizauxe7uxe3o}

To understand the migration of critical pedagogy into digital spaces,
one must first rigorously examine its origins in the material conditions
of Northeastern Brazil during the mid-20th century. Paulo Freire's
philosophy was not an abstraction but a direct response to the "culture
of silence" imposed by centuries of colonial rule and economic
oppression. Brazil, a Portuguese colony from 1500 to 1822, developed a
socioeconomic structure heavily reliant on slavery and the latifundia
system, leaving a legacy of profound inequality and illiteracy that
persisted well into the 1960s.\textsuperscript{1} Freire's work in the
1960s, particularly his adult literacy campaigns in Angicos, was
predicated on the understanding that illiteracy was not a personal
failing but a political condition designed to maintain the subjugation
of the rural poor.

Freire's central theoretical contribution, \emph{conscientização},
describes the dialectical process by which learners move from a "naive
consciousness"---accepting reality as a fixed, unchangeable given---to a
"critical consciousness" that perceives the historical and mutable
nature of social structures. This awakening is the precursor to
\emph{praxis}, defined as "reflection and action upon the world in order
to transform it".\textsuperscript{2} Freire drew heavily from Hegelian
dialectics, particularly the master-slave dialectic, and Marxist class
analysis, arguing that oppression dehumanizes both the oppressor and the
oppressed. The goal of education, therefore, is the "ontological
vocation" of all people to become fully human, a state achievable only
through liberation from oppressive dynamics.\textsuperscript{1}

\subsubsection{1.2 The Banking Model vs. Problem-Posing
Education}\label{the-banking-model-vs.-problem-posing-education}

The mechanism of oppression in education was identified by Freire as the
"banking model." In this paradigm, students are conceptualized as empty
vessels or depositories, and the teacher as the depositor of static
knowledge. This narration of reality---where the teacher speaks and the
students listen---mechanically domesticates the learner, inhibiting
creativity and reinforcing a passive acceptance of the status
quo.\textsuperscript{2} The banking concept mirrors the oppressive
society as a whole, projecting an ideology of adaptation rather than
transformation.

In opposition, Freire proposed "problem-posing education," a dialogic
model where the vertical hierarchy of teacher-student is dismantled in
favor of a horizontal relationship of "teacher-student" and
"students-teachers." In this model, the object of knowledge is not the
private property of the teacher but a medium for critical reflection by
both parties. Learning takes place through the investigation of
"generative themes"---words and concepts drawn directly from the
existential reality of the learners (e.g., \emph{tijolo} for brick,
\emph{favela} for slum, \emph{salário} for wages). By decoding these
themes, learners begin to "read the word and the world" simultaneously,
understanding language not just as a set of abstract symbols but as a
code for power relations.\textsuperscript{5}

\subsubsection{1.3 Dialogic Learning and the Conditions of True
Communication}\label{dialogic-learning-and-the-conditions-of-true-communication}

Central to this liberatory project is the concept of dialogue. For
Freire, dialogue is not merely conversation; it is an existential
necessity and a political act. True dialogue cannot exist in the absence
of a profound love for the world and for people, nor can it exist
without humility, faith in humanity, and hope. It requires critical
thinking that perceives reality as a process rather than a static
entity.\textsuperscript{6} This dialogic framework is essential for
understanding later digital pedagogies, where the "interface" becomes
the medium of dialogue (or anti-dialogue).

The suppression of dialogue is a characteristic of "anti-dialogical
action," which includes conquest, divide and rule, manipulation, and
cultural invasion---tactics used by the oppressor to prevent critical
consciousness. Conversely, dialogic action involves cooperation, unity
for liberation, organization, and cultural synthesis.\textsuperscript{6}
These categories provide a robust heuristic for analyzing contemporary
digital platforms: does a Learning Management System (LMS) or a "Smart
City" dashboard facilitate dialogic cooperation, or does it impose a
form of cultural invasion and technocratic manipulation?

\subsubsection{1.4 The Spatial Turn: Critical Pedagogy of
Place}\label{the-spatial-turn-critical-pedagogy-of-place}

While Freire focused on the social and political dimensions of
situationality, later scholars synthesized his work with place-based
education to form a "critical pedagogy of place." This framework argues
that "being in a situation" has an inherent spatial dimension. To
critically reflect on one\textquotesingle s existence is to reflect on
the spaces one inhabits; to act is to transform those
spaces.\textsuperscript{8} This "spatial turn" is pivotal for connecting
Freirean pedagogy to architecture and urban planning. It suggests that
the built environment---like language---is a text to be read and
rewritten. This synthesis laid the groundwork for "participatory design"
and "radical planning" movements, which seek to decolonize spatial
practices and empower inhabitants to shape their own
communities.\textsuperscript{8}

\subsection{Part II: The Digital Remediation -- From Constructionism to
Critical Code
Studies}\label{part-ii-the-digital-remediation-from-constructionism-to-critical-code-studies}

As education entered the digital age, Freire's ideas found resonance and
remediation in the theories of Seymour Papert and the burgeoning field
of critical digital pedagogy. This section traces the intellectual
lineage from the "generative themes" of literacy circles to the
"microworlds" of creative coding.

\subsubsection{2.1 Constructionism: The Maker Mindset as
Praxis}\label{constructionism-the-maker-mindset-as-praxis}

Seymour Papert, a mathematician and student of Jean Piaget, developed
the theory of Constructionism, which shares deep epistemological roots
with Freirean pedagogy. While Piaget emphasized that children construct
knowledge inside their heads (constructivism), Papert argued that this
happens "felicitously" when the learner is engaged in constructing a
public entity, whether it be a sandcastle on the beach or a theory of
the universe.\textsuperscript{10}

Papert's work with the LOGO programming language introduced the concept
of the "microworld"---a constrained, safe, yet open-ended digital
environment where learners could explore mathematical concepts by
"teaching" a turtle to move. This reversal of agency---the child
programming the computer rather than the computer programming the
child---is a direct digital analogue to Freire's rejection of the
banking model.\textsuperscript{10} Scholars have noted that Freire laid
the groundwork for the "maker mindset" before the term existed, viewing
the act of creation as a tool to question oppression. The convergence of
Freire and Papert suggests that "making"---coding, designing,
building---is a form of literacy that enables the learner to externalize
internal models and subject them to public critique and
dialogue.\textsuperscript{10}

\subsubsection{2.2 Critical Digital Pedagogy: Interrogating the
Tool}\label{critical-digital-pedagogy-interrogating-the-tool}

In the late 20th and early 21st centuries, the field of Critical Digital
Pedagogy (CDP) emerged to apply Freirean critique to educational
technology. CDP asserts that digital tools are never neutral; they are
encoded with political, economic, and cultural values. A critical
digital pedagogy asks not just "how" to use a tool, but "why" and
"whether" it should be used at all. It investigates the "hidden
curriculum" of software---the assumptions about learning, authority, and
knowledge that are hard-coded into interfaces.\textsuperscript{12}

Key practitioners in this field argue that Learning Management Systems
(LMS) often function as digital banking models, or "walled gardens,"
that stifle agency and enforce surveillance. In contrast, CDP advocates
for "domain of one\textquotesingle s own" initiatives and the use of
open-web tools that allow students to narrate their own identities and
construct their own digital architectures.\textsuperscript{12} The goal
is to foster "critical digital literacy," which involves understanding
the politics of the interface, the economics of data, and the power
dynamics of algorithmic culture.

\subsubsection{2.3 Creative Coding and the Politics of Open
Source}\label{creative-coding-and-the-politics-of-open-source}

The "creative coding" movement represents a practical application of
these liberatory principles. By positioning programming as an expressive
medium for artists and humanists, rather than solely a technical skill
for engineers, this movement seeks to democratize access to
computational power.

\paragraph{2.3.1 The Processing Foundation and
p5.js}\label{the-processing-foundation-and-p5.js}

A central institution in this landscape is the Processing Foundation,
founded in 2012 to promote software literacy within the visual arts. The
development of \emph{Processing} (by Casey Reas and Ben Fry) and its
web-based successor \emph{p5.js} (led by Lauren McCarthy) was driven by
an explicit philosophy of inclusion. Reas views software as culture,
arguing that code creates "conditional systems" that reflect the biases
and values of their creators.\textsuperscript{14}

Lauren McCarthy's leadership of the \emph{p5.js} project represents a
significant evolution in critical coding pedagogy. Unlike many
open-source projects where diversity is an afterthought, McCarthy
centered community outreach, accessibility, and inclusivity as the
\emph{foundation} of the platform. \emph{p5.js} was designed to be
accessible through a web browser, lowering the technical barrier to
entry and allowing for immediate visual feedback.\textsuperscript{16}
This design choice is deeply political; it asserts that coding belongs
to everyone, including those historically marginalized in tech (women,
people of color, non-binary individuals). The platform's documentation
and tutorials prioritize "accessible" language and diverse examples,
embodying a "pedagogy of care".\textsuperscript{17}

\paragraph{2.3.2 Critical Code Studies and Aesthetic
Programming}\label{critical-code-studies-and-aesthetic-programming}

The academic discourse surrounding creative coding has given rise to
"Critical Code Studies" (CCS) and "Aesthetic Programming." CCS applies
the hermeneutic strategies of the humanities to the interpretation of
computer code, reading it as a text that contains rhetoric, culture, and
ideology.\textsuperscript{19} Scholars argue that programming education
often perpetuates a "myth of axiological neutrality," presenting code as
objective logic while ignoring its entanglement with colonial and
capitalist power structures.

"Aesthetic Programming," as articulated by Winnie Soon and Geoff Cox,
uses \emph{p5.js} to teach coding as a critical practice. It encourages
learners to "program or be programmed," fostering a politicized literacy
of how software structures the world. By examining the material
conditions of coding---time, energy, labor---students learn to see code
not just as a tool but as a system of relations.\textsuperscript{17}
This aligns with Freire's goal of making the invisible visible; in the
digital age, the invisible forces are the algorithms that govern our
lives.

\subsection{Part III: "Serious Toys" -- The Architectural Turn in
Simulation}\label{part-iii-serious-toys-the-architectural-turn-in-simulation}

The principles of critical pedagogy have found a particularly fertile
ground in the intersection of architecture, urban planning, and
simulation. The concept of "serious toys" provides a theoretical
framework for understanding how game engines and virtual environments
are reshaping the design disciplines.

\subsubsection{3.1 Defining "Serious Toys": From Eames to
Cybernetics}\label{defining-serious-toys-from-eames-to-cybernetics}

The term "serious toys" was popularized by Charles and Ray Eames, the
mid-century modern designers who famously declared, "Toys are not as
innocent as they look... Toys and games are the preludes to serious
ideas." In a 1958 interview and subsequent writings, Charles Eames
argued that many transformative technologies, such as electricity and
flight, began as curiosities or playthings. He viewed toys as
instruments for uninhibited experimentation, allowing the mind to model
complex concepts without the fear of failure associated with "serious"
work.\textsuperscript{21}

This concept has deep resonances with cybernetics and systems theory. A
"serious toy," in this context, is a system that allows for the modeling
of variables and feedback loops. It is a "transitional object"
(Winnicott) that bridges the gap between the internal world of
imagination and the external world of physical constraints. In
contemporary architectural education, the "serious toy" has evolved from
physical blocks and erector sets to digital game engines like Unity and
Unreal.

\subsubsection{3.2 Unity and the Simulation of
Possibility}\label{unity-and-the-simulation-of-possibility}

The adoption of the Unity game engine by architecture schools and firms
marks a paradigm shift from \emph{representation} to \emph{simulation}.
Traditional CAD tools (like AutoCAD or Revit) are designed to produce
static representations of buildings for construction. Unity, however, is
designed to simulate environments for experience. It allows for the
introduction of physics, time, lighting, and
interactivity.\textsuperscript{23}

\paragraph{3.2.1 Experiential Learning and Scenario
Building}\label{experiential-learning-and-scenario-building}

In architectural pedagogy, Unity serves as a "microworld" where students
can test design hypotheses in real-time. This aligns with Experiential
Learning Theory (Kolb), which emphasizes learning through concrete
experience and active experimentation.\textsuperscript{25} Projects like
the "Spaces, Places and Possibilities" initiative in Squamish, Canada,
utilized Unity to create interactive "model explorers." These tools
allowed community members to navigate 3D visualizations of different
urban development scenarios, transforming abstract planning data into
visceral, first-person experiences.\textsuperscript{26} By "playing"
through these scenarios, stakeholders could understand the trade-offs of
density, zoning, and environmental impact in a way that static maps
could never convey.

\subsubsection{3.3 The Plethora Project: Gamifying the
Commons}\label{the-plethora-project-gamifying-the-commons}

A pivotal figure in the theorization of "serious toys" for architecture
is Jose Sanchez, director of the Plethora Project. Sanchez explicitly
merges video game design with critical architectural theory to challenge
the "platform capitalism" of the design industry. His work argues for
"Architecture for the Commons"---systems that allow for distributed
authorship and participatory design.\textsuperscript{28}

\begin{itemize}
\item
  \textbf{\emph{Block\textquotesingle hood}:} Sanchez's game
  \emph{Block\textquotesingle hood} is a city-building simulator that
  critiques the growth-at-all-costs logic of traditional games like
  \emph{SimCity}. In \emph{Block\textquotesingle hood}, the core
  mechanic is ecological balance and entropy. Players must construct
  neighborhoods from a catalog of blocks (apartments, trees, solar
  panels), but each block has inputs and outputs. If a block's needs are
  not met, it decays and collapses. This "procedural rhetoric" teaches
  systems thinking and sustainability, embodying Freirean praxis by
  forcing the player to reflect on the consequences of their design
  decisions.\textsuperscript{30}
\item
  \textbf{\emph{Common\textquotesingle hood}:} Building on this,
  \emph{Common\textquotesingle hood} focuses on the social dimensions of
  space. It is a "squatter settlement simulation" where players manage a
  community of characters with unique needs and backstories, scavenging
  materials to build shelter and infrastructure. The game explicitly
  deals with themes of labor, inequality, and mutual aid, serving as a
  "serious toy" for understanding the political economy of the built
  environment.\textsuperscript{33}
\end{itemize}

\subsubsection{3.4 Zaha Hadid Architects: Parametricism and Gamified
Participation}\label{zaha-hadid-architects-parametricism-and-gamified-participation}

The application of these principles extends to high-end architectural
practice. The Bhooshan Studio at Zaha Hadid Architects (ZHA), part of
the Architectural Association's Design Research Lab (DRL), investigates
"participatory agency" through gamification. They employ "cyber-physical
platforms" that allow users to configure housing layouts or urban
densities using game-like interfaces. These inputs are then rationalized
by parametric algorithms to generate fabrication-ready
designs.\textsuperscript{35}

While this approach claims to democratize design, it has drawn critique
for potentially retaining a top-down power structure. Critics argue that
while the user is given a "sandbox," the "rules of the game" (the
parametric constraints) are still dictated by the architect, limiting
true emancipatory potential. This tension highlights the difference
between "consultation" (users selecting options) and true "co-creation"
(users defining the problem), a distinction central to Freirean
pedagogy.\textsuperscript{36}

\subsection{Part IV: The City as Classroom -- Participatory Planning,
Digital Twins, and the Playable
City}\label{part-iv-the-city-as-classroom-participatory-planning-digital-twins-and-the-playable-city}

The principles of critical pedagogy---dialogue, participation, and
critical consciousness---are increasingly being applied at the urban
scale. Digital tools are transforming urban planning from a technocratic
exercise into a participatory pedagogical process, often referred to as
"civic tech" or "participatory design."

\subsubsection{\texorpdfstring{4.1 UN-Habitat and the
\emph{\textbf{Block by Block}}
Initiative}{4.1 UN-Habitat and the Block by Block Initiative}}\label{un-habitat-and-the-block-by-block-initiative}

One of the most successful applications of "serious toys" in the Global
South is the \emph{Block by Block} initiative, a partnership between
UN-Habitat and Mojang (creators of Minecraft). This program addresses
the exclusion of marginalized communities from urban planning processes
due to the technical barriers of traditional architectural drawings.

\begin{itemize}
\item
  \textbf{Methodology:} The program uses Minecraft to create rough 3D
  models of public spaces in developing countries (e.g., Kibera in
  Nairobi, spaces in Nepal and Haiti). Community members---including
  women, youth, and slum dwellers---are invited to workshops where they
  redesign these spaces within the game.
\item
  \textbf{Freirean Impact:} Minecraft serves as a "lingua franca" or a
  "digital pidgin" that transcends literacy barriers. It allows
  participants to visualize their "generative themes"---safety,
  lighting, sanitation---and translate them into a spatial language that
  planners can understand. In Kibera, youth used the game to identify
  dark spots prone to crime and proposed lighting solutions that were
  subsequently implemented. This process validates the "local knowledge"
  of the residents, shifting them from objects of planning to subjects
  of design.\textsuperscript{37}
\end{itemize}

\subsubsection{4.2 euPOLIS and the Gamification of Nature-Based
Solutions}\label{eupolis-and-the-gamification-of-nature-based-solutions}

The \emph{euPOLIS} project represents a European approach to gamified
participation, focusing on Nature-Based Solutions (NBS) for urban
health. The project developed the \emph{euPOLIS} game, a simulation tool
that allows citizens to propose and test urban interventions. By
shifting technical calculations (engineering, cost) to the backend
simulation, the game empowers citizens to focus on value-based
propositions. It creates a dialogic interface where the expert knowledge
of planners and the lived experience of residents can
meet.\textsuperscript{39}

\subsubsection{4.3 Digital Twins: Between Technocratic Control and
"Counter-Twinning"}\label{digital-twins-between-technocratic-control-and-counter-twinning}

The concept of the "Digital Twin"---a virtual replica of a physical
city---is currently a site of intense ideological contestation. In the
dominant "Smart City" discourse, digital twins are tools for
optimization and control, fed by real-time sensor data to manage
traffic, energy, and policing. This model often aligns with
"surveillance capitalism" and the banking model of information, where
data is extracted from citizens to manage them more
efficiently.\textsuperscript{41}

However, a "critical digital twin" movement is emerging. Researchers
advocate for "participatory digital twins" or "counter-twinning," which
incorporate qualitative data---stories, memories, and
aspirations---alongside quantitative metrics. A notable case study in
Gothenburg, Sweden, explored "counter-twinning" as a form of resistance.
By allowing citizens to annotate the digital twin with their own
narratives and "invisible" data, the project sought to foreground
marginalized perspectives and challenge the commodification of urban
life. This approach transforms the digital twin from a tool of
surveillance into a "material-dialogic space" for collective memory and
future-making.\textsuperscript{44}

\subsubsection{4.4 The "Playable City" vs. The "Smart
City"}\label{the-playable-city-vs.-the-smart-city}

The "Playable City" framework has emerged as a direct critique of the
"Smart City." While the Smart City prioritizes efficiency and
seamlessness, the Playable City emphasizes friction, spontaneity, and
human connection. It uses the same infrastructure (sensors, screens,
networks) to create moments of play and "carnivalesque" disruption.

\begin{itemize}
\item
  \textbf{Critical Pedagogy of Play:} Playable City interventions often
  function as "urban hacks" that reveal the hidden rules of the city. By
  turning a staircase into a piano or a streetlight into a shadow-puppet
  theater, these interventions invite citizens to re-engage with their
  environment and with each other. This aligns with Freire's notion of
  humanization; it reclaims the city as a space for life and interaction
  rather than just transit and commerce. It suggests that the "right to
  the city" includes the right to play and to reshape the urban
  fabric.\textsuperscript{46}
\end{itemize}

\subsection{Part V: Critique, Contradiction, and the Political Economy
of
Platforms}\label{part-v-critique-contradiction-and-the-political-economy-of-platforms}

While digital tools offer immense potential for liberatory education,
they are deeply embedded in capitalist structures that often contradict
Freirean ideals. A critical analysis must confront these tensions,
particularly the "platformization" of education and the neocolonial
implications of certain design practices.

\subsubsection{5.1 The Paradox of Corporate Platforms: Roblox and
Labor}\label{the-paradox-of-corporate-platforms-roblox-and-labor}

Roblox represents a complex paradox for critical pedagogy. On one hand,
it is a powerful constructionist tool that democratizes game design for
the "Alpha Generation," fostering peer-to-peer learning and
creativity.\textsuperscript{49} On the other hand, it is a closed,
proprietary platform driven by profit.

\begin{itemize}
\item
  \textbf{Exploitation of Digital Labor:} Critics point out that
  Roblox's business model relies on the unpaid or underpaid labor of
  millions of child developers. The platform extracts value from their
  creativity while exposing them to "surveillance capitalism"
  mechanisms, such as behavioral profiling and data harvesting. This
  relationship mimics the "feudal" dynamics of the gig economy rather
  than the liberatory dynamics of the commons.\textsuperscript{50}
\item
  \textbf{Surveillance in the Classroom:} The integration of platforms
  like Roblox into school curriculums introduces commercial surveillance
  into the learning environment. This violates Freire's insistence on
  the classroom as a safe space for critical consciousness, transforming
  students into data subjects whose behavior is mined for prediction and
  modification.\textsuperscript{12}
\end{itemize}

\subsubsection{5.2 Neocolonialism in Design: The Case of
Próspera}\label{neocolonialism-in-design-the-case-of-pruxf3spera}

The tension between liberatory rhetoric and neocolonial reality is
starkly illustrated in the \emph{Próspera} project in Roatán, Honduras.
This "charter city" or ZEDE (Zone for Employment and Economic
Development) utilizes advanced digital governance and architectural
design (partnering with Zaha Hadid Architects) to create a
semi-autonomous jurisdiction.

\begin{itemize}
\item
  \textbf{Critique:} While proponents frame it as a laboratory for
  innovation and economic freedom, critics and local communities view it
  as a neocolonial enclave. The project uses digital tools to bypass
  national laws and establish a private governance structure,
  effectively commodifying sovereignty. This "start-up city" model,
  often justified through the language of "leaping" development stages,
  can be seen as a digital reincarnation of colonial extraction, where
  local populations are displaced or disenfranchised by a techno-elite.
  This stands in direct opposition to the "critical pedagogy of place,"
  which demands accountability to the local historical and social
  context.\textsuperscript{53}
\end{itemize}

\subsubsection{5.3 The Struggle for the Digital
Commons}\label{the-struggle-for-the-digital-commons}

The antidote to these exploitative dynamics, as suggested by scholars
like Jose Sanchez and the open-education movement, is the construction
of a "Digital Commons." This involves:

\begin{itemize}
\item
  \textbf{Open Source as Praxis:} Using and creating open-source
  software (like Godot, Blender, or p5.js) is a political act that
  reclaims the means of digital production from corporate monopolies.
\item
  \textbf{Data Justice and Sovereignty:} Educational initiatives must
  prioritize "data justice," teaching students not just how to code, but
  how to interrogate the ownership and ethics of data.
\item
  \textbf{Cooperative Governance:} Platforms for "commoning" should be
  governed by their users, not by shareholders. This aligns with
  Freire's vision of a society where "men and women are not objects of
  history, but Subjects".\textsuperscript{28}
\end{itemize}

\subsection{Conclusion: Toward a Critical AI
Literacy}\label{conclusion-toward-a-critical-ai-literacy}

The trajectory of critical pedagogy from the sugar cane fields of
Pernambuco to the voxelated landscapes of the Metaverse is not a linear
progression but a complex remediation of emancipatory ideals. Paulo
Freire's fundamental insight---that education is a political act of
humanization---remains as relevant in the age of Artificial Intelligence
as it was in the age of the typewriter.

The integration of "serious toys" and game engines into education and
urban planning offers profound opportunities for \textbf{dialogic
learning} and \textbf{participatory design}. Tools like Unity and
Minecraft allow communities to "write the world" in three dimensions,
visualizing alternatives to the status quo and challenging the hegemony
of expert planners. However, the "platformization" of these tools
threatens to enclose these liberatory practices within "walled gardens"
of surveillance and extraction.

The future of liberatory digital design lies in "Critical AI
Literacy"---a pedagogy that empowers learners to understand not just the
mechanics of generative AI and simulation, but the power structures they
reinforce. It demands that we move beyond the "banking model" of digital
consumption to a "problem-posing" model of digital creation, where the
"Digital Twin" becomes a "Counter-Twin," and the "Smart City" becomes a
"Playable City" of the commons.

\subsubsection{Table 1: Evolution of Pedagogical Models from Analog to
Digital}\label{table-1-evolution-of-pedagogical-models-from-analog-to-digital}

\begin{longtable}[]{@{}
  >{\raggedright\arraybackslash}p{(\linewidth - 6\tabcolsep) * \real{0.2500}}
  >{\raggedright\arraybackslash}p{(\linewidth - 6\tabcolsep) * \real{0.2500}}
  >{\raggedright\arraybackslash}p{(\linewidth - 6\tabcolsep) * \real{0.2500}}
  >{\raggedright\arraybackslash}p{(\linewidth - 6\tabcolsep) * \real{0.2500}}@{}}
\toprule\noalign{}
\begin{minipage}[b]{\linewidth}\raggedright
\textbf{Component}
\end{minipage} & \begin{minipage}[b]{\linewidth}\raggedright
\textbf{Freirean Pedagogy (Analog)}
\end{minipage} & \begin{minipage}[b]{\linewidth}\raggedright
\textbf{Critical Digital Pedagogy (Digital)}
\end{minipage} & \begin{minipage}[b]{\linewidth}\raggedright
\textbf{Key Tools/Platforms}
\end{minipage} \\
\begin{minipage}[b]{\linewidth}\raggedright
\textbf{Role of Learner}
\end{minipage} & \begin{minipage}[b]{\linewidth}\raggedright
Co-investigator; Subject (not Object)
\end{minipage} & \begin{minipage}[b]{\linewidth}\raggedright
Maker; Coder; User-Designer
\end{minipage} & \begin{minipage}[b]{\linewidth}\raggedright
p5.js, Scratch, Zines
\end{minipage} \\
\begin{minipage}[b]{\linewidth}\raggedright
\textbf{Role of Teacher}
\end{minipage} & \begin{minipage}[b]{\linewidth}\raggedright
Facilitator; Problem-poser
\end{minipage} & \begin{minipage}[b]{\linewidth}\raggedright
Mentor; Co-learner; Guide on the side
\end{minipage} & \begin{minipage}[b]{\linewidth}\raggedright
GitHub, Discord, Wikis
\end{minipage} \\
\begin{minipage}[b]{\linewidth}\raggedright
\textbf{Methodology}
\end{minipage} & \begin{minipage}[b]{\linewidth}\raggedright
Culture Circles; Generative Themes
\end{minipage} & \begin{minipage}[b]{\linewidth}\raggedright
Critical Making; Creative Coding; Glitch Art
\end{minipage} & \begin{minipage}[b]{\linewidth}\raggedright
Processing, Digital Storytelling
\end{minipage} \\
\begin{minipage}[b]{\linewidth}\raggedright
\textbf{Literacy}
\end{minipage} & \begin{minipage}[b]{\linewidth}\raggedright
Reading the Word \& World
\end{minipage} & \begin{minipage}[b]{\linewidth}\raggedright
Critical Code Literacy; Algorithmic Awareness
\end{minipage} & \begin{minipage}[b]{\linewidth}\raggedright
Python, Data Visualization
\end{minipage} \\
\begin{minipage}[b]{\linewidth}\raggedright
\textbf{Objective}
\end{minipage} & \begin{minipage}[b]{\linewidth}\raggedright
\emph{Conscientização} (Critical Consciousness)
\end{minipage} & \begin{minipage}[b]{\linewidth}\raggedright
Digital Agency; Data Justice; "Hackability"
\end{minipage} & \begin{minipage}[b]{\linewidth}\raggedright
Open Source Movements
\end{minipage} \\
\midrule\noalign{}
\endhead
\bottomrule\noalign{}
\endlastfoot
\end{longtable}

\subsubsection{Table 2: Comparative Analysis of "Serious Toys" in Urban
Planning}\label{table-2-comparative-analysis-of-serious-toys-in-urban-planning}

\begin{longtable}[]{@{}
  >{\raggedright\arraybackslash}p{(\linewidth - 8\tabcolsep) * \real{0.2000}}
  >{\raggedright\arraybackslash}p{(\linewidth - 8\tabcolsep) * \real{0.2000}}
  >{\raggedright\arraybackslash}p{(\linewidth - 8\tabcolsep) * \real{0.2000}}
  >{\raggedright\arraybackslash}p{(\linewidth - 8\tabcolsep) * \real{0.2000}}
  >{\raggedright\arraybackslash}p{(\linewidth - 8\tabcolsep) * \real{0.2000}}@{}}
\toprule\noalign{}
\begin{minipage}[b]{\linewidth}\raggedright
\textbf{Platform / Project}
\end{minipage} & \begin{minipage}[b]{\linewidth}\raggedright
\textbf{Core Mechanic}
\end{minipage} & \begin{minipage}[b]{\linewidth}\raggedright
\textbf{Pedagogical Goal}
\end{minipage} & \begin{minipage}[b]{\linewidth}\raggedright
\textbf{Freirean Alignment}
\end{minipage} & \begin{minipage}[b]{\linewidth}\raggedright
\textbf{Critique / Limitation}
\end{minipage} \\
\begin{minipage}[b]{\linewidth}\raggedright
\textbf{Block by Block (Minecraft)}
\end{minipage} & \begin{minipage}[b]{\linewidth}\raggedright
Voxel-based construction; simplified modeling
\end{minipage} & \begin{minipage}[b]{\linewidth}\raggedright
Participatory design; Community engagement
\end{minipage} & \begin{minipage}[b]{\linewidth}\raggedright
High: Empowers marginalized voices to "name" their space.
\end{minipage} & \begin{minipage}[b]{\linewidth}\raggedright
Dependency on proprietary software (Microsoft); translation to reality.
\end{minipage} \\
\begin{minipage}[b]{\linewidth}\raggedright
\textbf{Block\textquotesingle hood / Common\textquotesingle hood}
\end{minipage} & \begin{minipage}[b]{\linewidth}\raggedright
Ecological balance; resource management
\end{minipage} & \begin{minipage}[b]{\linewidth}\raggedright
Systems thinking; Entropy; Mutual aid
\end{minipage} & \begin{minipage}[b]{\linewidth}\raggedright
High: Focuses on interdependence and "commons" rather than profit.
\end{minipage} & \begin{minipage}[b]{\linewidth}\raggedright
Niche audience; complexity barrier; high system requirements.
\end{minipage} \\
\begin{minipage}[b]{\linewidth}\raggedright
\textbf{euPOLIS Game}
\end{minipage} & \begin{minipage}[b]{\linewidth}\raggedright
NBS (Nature-Based Solutions) simulation
\end{minipage} & \begin{minipage}[b]{\linewidth}\raggedright
Co-creation; Health \& Well-being metrics
\end{minipage} & \begin{minipage}[b]{\linewidth}\raggedright
Medium: Dialogic interface between experts and citizens.
\end{minipage} & \begin{minipage}[b]{\linewidth}\raggedright
Potential for technocratic "gamification" rather than true agency.
\end{minipage} \\
\begin{minipage}[b]{\linewidth}\raggedright
\textbf{Roblox Urban Planning}
\end{minipage} & \begin{minipage}[b]{\linewidth}\raggedright
Open sandbox; Social gameplay
\end{minipage} & \begin{minipage}[b]{\linewidth}\raggedright
Youth engagement; Future city visioning
\end{minipage} & \begin{minipage}[b]{\linewidth}\raggedright
Medium: Agentic play; peer-to-peer learning.
\end{minipage} & \begin{minipage}[b]{\linewidth}\raggedright
\textbf{High Risk:} Surveillance capitalism; labor exploitation; walled
garden.
\end{minipage} \\
\begin{minipage}[b]{\linewidth}\raggedright
\textbf{Digital Twins (Smart Cities)}
\end{minipage} & \begin{minipage}[b]{\linewidth}\raggedright
Real-time sensor data; predictive modeling
\end{minipage} & \begin{minipage}[b]{\linewidth}\raggedright
Optimization; Efficiency; "What-if" scenarios
\end{minipage} & \begin{minipage}[b]{\linewidth}\raggedright
Low (usually): Often top-down and technocratic.
\end{minipage} & \begin{minipage}[b]{\linewidth}\raggedright
"Counter-twinning" is needed to add social/qualitative layers.
\end{minipage} \\
\midrule\noalign{}
\endhead
\bottomrule\noalign{}
\endlastfoot
\end{longtable}

\subsubsection{Table 3: Key Figures and Their Contributions to
Liberatory
Design}\label{table-3-key-figures-and-their-contributions-to-liberatory-design}

\begin{longtable}[]{@{}
  >{\raggedright\arraybackslash}p{(\linewidth - 6\tabcolsep) * \real{0.2500}}
  >{\raggedright\arraybackslash}p{(\linewidth - 6\tabcolsep) * \real{0.2500}}
  >{\raggedright\arraybackslash}p{(\linewidth - 6\tabcolsep) * \real{0.2500}}
  >{\raggedright\arraybackslash}p{(\linewidth - 6\tabcolsep) * \real{0.2500}}@{}}
\toprule\noalign{}
\begin{minipage}[b]{\linewidth}\raggedright
\textbf{Figure}
\end{minipage} & \begin{minipage}[b]{\linewidth}\raggedright
\textbf{Domain}
\end{minipage} & \begin{minipage}[b]{\linewidth}\raggedright
\textbf{Key Contribution / Concept}
\end{minipage} & \begin{minipage}[b]{\linewidth}\raggedright
\textbf{Connection to Critical Pedagogy}
\end{minipage} \\
\begin{minipage}[b]{\linewidth}\raggedright
\textbf{Paulo Freire}
\end{minipage} & \begin{minipage}[b]{\linewidth}\raggedright
Education / Philosophy
\end{minipage} & \begin{minipage}[b]{\linewidth}\raggedright
\emph{Conscientização}; Banking Model; Dialogic Learning
\end{minipage} & \begin{minipage}[b]{\linewidth}\raggedright
The foundational theorist of liberatory education.
\end{minipage} \\
\begin{minipage}[b]{\linewidth}\raggedright
\textbf{Seymour Papert}
\end{minipage} & \begin{minipage}[b]{\linewidth}\raggedright
Comp Sci / Education
\end{minipage} & \begin{minipage}[b]{\linewidth}\raggedright
Constructionism; Microworlds
\end{minipage} & \begin{minipage}[b]{\linewidth}\raggedright
Linked "making" with learning; bridged Piaget and Freire.
\end{minipage} \\
\begin{minipage}[b]{\linewidth}\raggedright
\textbf{Charles \& Ray Eames}
\end{minipage} & \begin{minipage}[b]{\linewidth}\raggedright
Design / Architecture
\end{minipage} & \begin{minipage}[b]{\linewidth}\raggedright
"Serious Toys"
\end{minipage} & \begin{minipage}[b]{\linewidth}\raggedright
Reframed toys as preludes to serious ideas and scientific discovery.
\end{minipage} \\
\begin{minipage}[b]{\linewidth}\raggedright
\textbf{bell hooks}
\end{minipage} & \begin{minipage}[b]{\linewidth}\raggedright
Cultural Criticism
\end{minipage} & \begin{minipage}[b]{\linewidth}\raggedright
Engaged Pedagogy; Transgressing boundaries
\end{minipage} & \begin{minipage}[b]{\linewidth}\raggedright
Emphasized the "soul" of students and the classroom as a site of
resistance.
\end{minipage} \\
\begin{minipage}[b]{\linewidth}\raggedright
\textbf{Casey Reas}
\end{minipage} & \begin{minipage}[b]{\linewidth}\raggedright
Digital Art / Code
\end{minipage} & \begin{minipage}[b]{\linewidth}\raggedright
Processing; Software as culture
\end{minipage} & \begin{minipage}[b]{\linewidth}\raggedright
Democratized coding; framed code as an expressive, non-neutral medium.
\end{minipage} \\
\begin{minipage}[b]{\linewidth}\raggedright
\textbf{Lauren McCarthy}
\end{minipage} & \begin{minipage}[b]{\linewidth}\raggedright
Art / Tech
\end{minipage} & \begin{minipage}[b]{\linewidth}\raggedright
p5.js; Diversity in tech
\end{minipage} & \begin{minipage}[b]{\linewidth}\raggedright
Explicitly centered inclusion and community in the design of coding
tools.
\end{minipage} \\
\begin{minipage}[b]{\linewidth}\raggedright
\textbf{Jose Sanchez}
\end{minipage} & \begin{minipage}[b]{\linewidth}\raggedright
Architecture / Games
\end{minipage} & \begin{minipage}[b]{\linewidth}\raggedright
\emph{Block\textquotesingle hood}; Architecture for the Commons
\end{minipage} & \begin{minipage}[b]{\linewidth}\raggedright
Applied game engines to architectural theory; anti-platform capitalism.
\end{minipage} \\
\begin{minipage}[b]{\linewidth}\raggedright
\textbf{Shajay Bhooshan}
\end{minipage} & \begin{minipage}[b]{\linewidth}\raggedright
Architecture (ZHA)
\end{minipage} & \begin{minipage}[b]{\linewidth}\raggedright
Gamified Design; Participatory Agency
\end{minipage} & \begin{minipage}[b]{\linewidth}\raggedright
Uses "serious games" for high-density urban design participation.
\end{minipage} \\
\midrule\noalign{}
\endhead
\bottomrule\noalign{}
\endlastfoot
\end{longtable}

\paragraph{Works cited}\label{works-cited}

\begin{enumerate}
\def\labelenumi{\arabic{enumi}.}
\item
  Paulo Freire \textbar{} Internet Encyclopedia of Philosophy, accessed
  December 10, 2025,
  \href{https://iep.utm.edu/freire/}{\ul{https://iep.utm.edu/freire/}}
\item
  Critical Pedagogy - Rollins School of Public Health - Emory
  University, accessed December 10, 2025,
  \href{https://sph.emory.edu/info/faculty-staff/rollins-teaching-learning-core/teaching-learning-principles/critical-pedagogy}{\ul{https://sph.emory.edu/info/faculty-staff/rollins-teaching-learning-core/teaching-learning-principles/critical-pedagogy}}
\item
  Critical pedagogy - Wikipedia, accessed December 10, 2025,
  \href{https://en.wikipedia.org/wiki/Critical_pedagogy}{\ul{https://en.wikipedia.org/wiki/Critical\_pedagogy}}
\item
  Teacher and Student Roles in Freire\textquotesingle s Critical
  Pedagogy: A Qualitative Case Study - ScholarWorks@BGSU, accessed
  December 10, 2025,
  \href{https://scholarworks.bgsu.edu/cgi/viewcontent.cgi?article=1777&context=mwer}{\ul{https://scholarworks.bgsu.edu/cgi/viewcontent.cgi?article=1777\&context=mwer}}
\item
  TRAVELS IN TROY WITH FREIRE - Beyond Bits \& Atoms, accessed December
  10, 2025,
  \href{http://beyondbitsandatoms.org/readings/blikstein_travels_2008.pdf}{\ul{http://beyondbitsandatoms.org/readings/blikstein\_travels\_2008.pdf}}
\item
  Pedagogy of the Oppressed - Wikipedia, accessed December 10, 2025,
  \href{https://en.wikipedia.org/wiki/Pedagogy_of_the_Oppressed}{\ul{https://en.wikipedia.org/wiki/Pedagogy\_of\_the\_Oppressed}}
\item
  Paulo Freire\textquotesingle s Educational Thought of Dialogue and Its
  Implications for Teachers and Teaching, accessed December 10, 2025,
  \href{https://pesa.org.au/images/papers/2011-papers/syi2011.pdf}{\ul{https://pesa.org.au/images/papers/2011-papers/syi2011.pdf}}
\item
  The best of both worlds: a critical pedagogy of place* - City Tech
  OpenLab, accessed December 10, 2025,
  \href{https://openlab.citytech.cuny.edu/genedseminar-winter-2025/files/2020/01/Critical-Pedagogy-and-Place-based-Learning.pdf}{\ul{https://openlab.citytech.cuny.edu/genedseminar-winter-2025/files/2020/01/Critical-Pedagogy-and-Place-based-Learning.pdf}}
\item
  İSTANBUL TEKNİK ÜNİVERSİTESİ FEN BİLİMLERİ ENSTİTÜSÜ DOKTORA TEZİ
  KASIM 2019 TÜRKİYE\textquotesingle DEKİ KENTSEL DÖNÜŞ, accessed
  December 10, 2025,
  \href{https://polen.itu.edu.tr/bitstream/11527/18589/1/606169.pdf}{\ul{https://polen.itu.edu.tr/bitstream/11527/18589/1/606169.pdf}}
\item
  Constructionism, a Learning Theory and a Model for Maker Education -
  FabLearn Fellows, accessed December 10, 2025,
  \href{https://fellows.fablearn.org/constructionism-a-learning-theory-and-a-model-for-maker-education/}{\ul{https://fellows.fablearn.org/constructionism-a-learning-theory-and-a-model-for-maker-education/}}
\item
  From In-the-Head to In-the-World: Frameworks for Understanding and
  Applying Computational Thinking - UNL Digital Commons, accessed
  December 10, 2025,
  \href{https://digitalcommons.unl.edu/cgi/viewcontent.cgi?article=1577&context=teachlearnfacpub}{\ul{https://digitalcommons.unl.edu/cgi/viewcontent.cgi?article=1577\&context=teachlearnfacpub}}
\item
  Critical Instructional Design -- An Urgency of Teachers, accessed
  December 10, 2025,
  \href{https://pressbooks.pub/criticaldigitalpedagogy/chapter/critical-pedagogy-and-learning-online/}{\ul{https://pressbooks.pub/criticaldigitalpedagogy/chapter/critical-pedagogy-and-learning-online/}}
\item
  Critical Digital Pedagogy in the Platform Society \textbar{} Oxford
  Research Encyclopedia of Education, accessed December 10, 2025,
  \href{https://oxfordre.com/education/display/10.1093/acrefore/9780190264093.001.0001/acrefore-9780190264093-e-1888?d=/10.1093/acrefore/9780190264093.001.0001/acrefore-9780190264093-e-1888&p=emailAEFH9ooinKMlY}{\ul{https://oxfordre.com/education/display/10.1093/acrefore/9780190264093.001.0001/acrefore-9780190264093-e-1888?d=\%2F10.1093\%2Facrefore\%2F9780190264093.001.0001\%2Facrefore-9780190264093-e-1888\&p=emailAEFH9ooinKMlY}}
\item
  Online Courses from University of California, Los Angeles, Department
  of Arts \textbar{} Kadenze, accessed December 10, 2025,
  \href{https://www.kadenze.com/partners/university-of-california-los-angeles-department-of-arts}{\ul{https://www.kadenze.com/partners/university-of-california-los-angeles-department-of-arts}}
\item
  Casey Reas -- EGS -- Division of Philosophy, Art, and Critical
  Thought, accessed December 10, 2025,
  \href{https://pact.egs.edu/biography/casey-reas/}{\ul{https://pact.egs.edu/biography/casey-reas/}}
\item
  Chapter 08: p5.js in the Media Studies Classroom by J.J. Sylvia IV
  \textbar{} Coding Pedagogy, accessed December 10, 2025,
  \href{http://codingpedagogy.net/chapter08/index.html}{\ul{http://codingpedagogy.net/chapter08/index.html}}
\item
  1. Getting started - Aesthetic Programming, accessed December 10,
  2025,
  \href{https://aesthetic-programming.net/pages/1-getting-started.html}{\ul{https://aesthetic-programming.net/pages/1-getting-started.html}}
\item
  Digital Pedagogy and Critical Pedagogy: Understanding Critical
  Pedagogy \textbar{} Academic Technology Solutions, accessed December
  10, 2025,
  \href{https://academictech.uchicago.edu/2023/10/25/digital-pedagogy-and-critical-pedagogy-understanding-critical-pedagogy/}{\ul{https://academictech.uchicago.edu/2023/10/25/digital-pedagogy-and-critical-pedagogy-understanding-critical-pedagogy/}}
\item
  (PDF) Decoloniality, Digital-coloniality and Computer Programming
  Education, accessed December 10, 2025,
  \href{https://www.researchgate.net/publication/385489112_Decoloniality_Digital-coloniality_and_Computer_Programming_Education}{\ul{https://www.researchgate.net/publication/385489112\_Decoloniality\_Digital-coloniality\_and\_Computer\_Programming\_Education}}
\item
  Theorising while() Practising: A Review of Aesthetic Programming -
  Computational Culture, accessed December 10, 2025,
  \href{http://computationalculture.net/theorising-while-practising-a-review-of-aesthetic-programming/}{\ul{http://computationalculture.net/theorising-while-practising-a-review-of-aesthetic-programming/}}
\item
  Urban co-creation - UPCommons, accessed December 10, 2025,
  \href{https://upcommons.upc.edu/bitstreams/f7f0500b-3da2-44eb-a0dc-f62430862ad4/download}{\ul{https://upcommons.upc.edu/bitstreams/f7f0500b-3da2-44eb-a0dc-f62430862ad4/download}}
\item
  An Eames anthology: articles, film scripts, interviews, letters,
  notes, speeches 9780300212839, 0300212836 - DOKUMEN.PUB, accessed
  December 10, 2025,
  \href{https://dokumen.pub/an-eames-anthology-articles-film-scripts-interviews-letters-notes-speeches-9780300212839-0300212836.html}{\ul{https://dokumen.pub/an-eames-anthology-articles-film-scripts-interviews-letters-notes-speeches-9780300212839-0300212836.html}}
\item
  Opening the Gate to Urban Repair: A Tool for Citizen-Led Design -
  ResearchGate, accessed December 10, 2025,
  \href{https://www.researchgate.net/publication/359810720_Opening_the_Gate_to_Urban_Repair_A_Tool_for_Citizen-Led_Design}{\ul{https://www.researchgate.net/publication/359810720\_Opening\_the\_Gate\_to\_Urban\_Repair\_A\_Tool\_for\_Citizen-Led\_Design}}
\item
  Community Assistance to Improve the Ability to Use Basic Unity and
  Virtual Reality - Semantic Scholar, accessed December 10, 2025,
  \href{https://pdfs.semanticscholar.org/fae2/d5a48ea8e7b69ea0b5f27b366958dc6b3960.pdf}{\ul{https://pdfs.semanticscholar.org/fae2/d5a48ea8e7b69ea0b5f27b366958dc6b3960.pdf}}
\item
  Where Critical Inquiry, Empirical Making, and Experiential Learning
  Shape Architectural Pedagogy - MDPI, accessed December 10, 2025,
  \href{https://www.mdpi.com/2673-8392/5/3/129}{\ul{https://www.mdpi.com/2673-8392/5/3/129}}
\item
  Communicating complexity: interactive model explorers and immersive
  visualizations as tools for local planning and community engagement -
  Facets Journal, accessed December 10, 2025,
  \href{https://www.facetsjournal.com/doi/10.1139/facets-2020-0045}{\ul{https://www.facetsjournal.com/doi/10.1139/facets-2020-0045}}
\item
  (PDF) Communicating complexity: interactive model explorers and
  immersive visualizations as tools for local planning and community
  engagement - ResearchGate, accessed December 10, 2025,
  \href{https://www.researchgate.net/publication/349822998_Communicating_complexity_interactive_model_explorers_and_immersive_visualizations_as_tools_for_local_planning_and_community_engagement}{\ul{https://www.researchgate.net/publication/349822998\_Communicating\_complexity\_interactive\_model\_explorers\_and\_immersive\_visualizations\_as\_tools\_for\_local\_planning\_and\_community\_engagement}}
\item
  Jose Sanchez - Architecture For The Commons - Participatory Systems in
  The Age of Platforms - Scribd, accessed December 10, 2025,
  \href{https://www.scribd.com/document/617049073/Jose-Sanchez-Architecture-for-the-Commons-Participatory-Systems-in-the-Age-of-Platforms}{\ul{https://www.scribd.com/document/617049073/Jose-Sanchez-Architecture-for-the-Commons-Participatory-Systems-in-the-Age-of-Platforms}}
\item
  {[}PDF{]} Architecture for the Commons by Jose Sanchez \textbar{}
  9781138362352, 9780429778018 - Perlego, accessed December 10, 2025,
  \href{https://www.perlego.com/book/1628963/architecture-for-the-commons-participatory-systems-in-the-age-of-platforms-pdf}{\ul{https://www.perlego.com/book/1628963/architecture-for-the-commons-participatory-systems-in-the-age-of-platforms-pdf}}
\item
  Block\textquotesingle hood --- Jose Sanchez - Plethora Project,
  accessed December 10, 2025,
  \href{https://www.plethora-project.com/blockhood}{\ul{https://www.plethora-project.com/blockhood}}
\item
  Block\textquotesingle hood - ZKM Karlsruhe, accessed December 10,
  2025,
  \href{https://zkm.de/en/blockhood}{\ul{https://zkm.de/en/blockhood}}
\item
  Block\textquotesingle Hood: Can a Video Game Change How We Design a
  City?, accessed December 10, 2025,
  \href{https://sociologyofvideogames.com/2016/03/09/blockhood-can-a-video-game-change-how-we-design-a-city/}{\ul{https://sociologyofvideogames.com/2016/03/09/blockhood-can-a-video-game-change-how-we-design-a-city/}}
\item
  Data Justice, AI, and Design - MIDAS - University of Michigan,
  accessed December 10, 2025,
  \href{https://midas.umich.edu/events/data-justice-and-design/}{\ul{https://midas.umich.edu/events/data-justice-and-design/}}
\item
  Next Week on Xbox: New Games for July 10 to 14, accessed December 10,
  2025,
  \href{https://news.xbox.com/en-us/2023/07/07/next-week-on-xbox-new-games-for-july-10-to-14/}{\ul{https://news.xbox.com/en-us/2023/07/07/next-week-on-xbox-new-games-for-july-10-to-14/}}
\item
  Architecture and Urbanism (DRL), accessed December 10, 2025,
  \href{https://www.aaschool.ac.uk/academicprogrammes/postgraduate/architecture-and-urbanism}{\ul{https://www.aaschool.ac.uk/academicprogrammes/postgraduate/architecture-and-urbanism}}
\item
  Architecture and Gaming -- What\textquotesingle s with all the hype? -
  Shelidon, accessed December 10, 2025,
  \href{https://www.shelidon.it/architecture-and-gaming-whats-with-all-the-hype/}{\ul{https://www.shelidon.it/architecture-and-gaming-whats-with-all-the-hype/}}
\item
  The best Minecraft mod is the one we live in - GamesBeat, accessed
  December 10, 2025,
  \href{https://gamesbeat.com/mojang-block-by-block-interview/?utm_source=altnate&utm_medium=tumblr}{\ul{https://gamesbeat.com/mojang-block-by-block-interview/?utm\_source=altnate\&utm\_medium=tumblr}}
\item
  Minecraft - Wikipedia, accessed December 10, 2025,
  \href{https://en.wikipedia.org/wiki/Minecraft}{\ul{https://en.wikipedia.org/wiki/Minecraft}}
\item
  Empowering Communities Through Gamified Urban Design Solutions - MDPI,
  accessed December 10, 2025,
  \href{https://www.mdpi.com/2624-6511/8/2/44}{\ul{https://www.mdpi.com/2624-6511/8/2/44}}
\item
  euPOLIS planning methodology \textbar{} Download Scientific Diagram -
  ResearchGate, accessed December 10, 2025,
  \href{https://www.researchgate.net/figure/euPOLIS-planning-methodology_fig1_354834016}{\ul{https://www.researchgate.net/figure/euPOLIS-planning-methodology\_fig1\_354834016}}
\item
  The risk of algorithmic injustice for interactive learning
  environments - WestminsterResearch, accessed December 10, 2025,
  \href{https://westminsterresearch.westminster.ac.uk/item/w7601/the-risk-of-algorithmic-injustice-for-interactive-learning-environments}{\ul{https://westminsterresearch.westminster.ac.uk/item/w7601/the-risk-of-algorithmic-injustice-for-interactive-learning-environments}}
\item
  The Role of Digital Twins in Urban Mobility Planning → Scenario -
  Prism → Sustainability Directory, accessed December 10, 2025,
  \href{https://prism.sustainability-directory.com/scenario/the-role-of-digital-twins-in-urban-mobility-planning/}{\ul{https://prism.sustainability-directory.com/scenario/the-role-of-digital-twins-in-urban-mobility-planning/}}
\item
  Digital Twins and Urban Planning: Designing Smarter, More Inclusive
  Cities - EA Journals, accessed December 10, 2025,
  \href{https://eajournals.org/wp-content/uploads/sites/21/2025/06/Digital-Twins-and-Urban-Planning.pdf}{\ul{https://eajournals.org/wp-content/uploads/sites/21/2025/06/Digital-Twins-and-Urban-Planning.pdf}}
\item
  Oracles: Specu ativ Enactments for urban digital twins \&
  participatio, accessed December 10, 2025,
  \href{https://www.scriptiewerkplaats-dhzw.nl/uploads/cfswdhzw/attachments/David\%20Ti.pdf}{\ul{https://www.scriptiewerkplaats-dhzw.nl/uploads/cfswdhzw/attachments/David\%20Ti.pdf}}
\item
  COUNTER-TWINNING IN GOTHENBURG - GUPEA, accessed December 10, 2025,
  \href{https://gupea.ub.gu.se/bitstream/handle/2077/89928/HP_Thesis_GUPEA.pdf?sequence=1&isAllowed=y}{\ul{https://gupea.ub.gu.se/bitstream/handle/2077/89928/HP\_Thesis\_GUPEA.pdf?sequence=1\&isAllowed=y}}
\item
  Together We Can Make It Work! Toward a Design Framework for Inclusive
  and Participatory City-Making of Playable Cities - Frontiers, accessed
  December 10, 2025,
  \href{https://www.frontiersin.org/journals/computer-science/articles/10.3389/fcomp.2020.600654/full}{\ul{https://www.frontiersin.org/journals/computer-science/articles/10.3389/fcomp.2020.600654/full}}
\item
  (PDF) Critical Playable Cities - ResearchGate, accessed December 10,
  2025,
  \href{https://www.researchgate.net/publication/334638061_Critical_Playable_Cities}{\ul{https://www.researchgate.net/publication/334638061\_Critical\_Playable\_Cities}}
\item
  Citizens of Play: Revisiting the Relationship Between Playable and
  Smart Cities, accessed December 10, 2025,
  \href{https://www.researchgate.net/publication/334656025_Citizens_of_Play_Revisiting_the_Relationship_Between_Playable_and_Smart_Cities}{\ul{https://www.researchgate.net/publication/334656025\_Citizens\_of\_Play\_Revisiting\_the\_Relationship\_Between\_Playable\_and\_Smart\_Cities}}
\item
  Media Technologies and Epistemologies: The Platforming of ...,
  accessed December 10, 2025,
  \href{https://ijoc.org/index.php/ijoc/article/view/21902}{\ul{https://ijoc.org/index.php/ijoc/article/view/21902}}
\item
  Teaching and Learning About Audio-Visual Media: - Norwegian Research
  Information Repository, accessed December 10, 2025,
  \href{https://munin.uit.no/bitstream/handle/10037/26853/article.pdf?sequence=2}{\ul{https://munin.uit.no/bitstream/handle/10037/26853/article.pdf?sequence=2}}
\item
  THREE MORAL CHALLENGES OF SURVEILLANCE CAPITALISM IN THE METAVERSE,
  accessed December 10, 2025,
  \href{https://ijlet.org/wp-content/uploads/2025/01/3.3.4.pdf}{\ul{https://ijlet.org/wp-content/uploads/2025/01/3.3.4.pdf}}
\item
  Lawsuit Accuses Roblox of Covertly Harvesting Kids\textquotesingle{}
  Data - Security Buzz, accessed December 10, 2025,
  \href{https://securitybuzz.com/cybersecurity-news/lawsuit-accuses-roblox-of-covertly-harvesting-kids-data/}{\ul{https://securitybuzz.com/cybersecurity-news/lawsuit-accuses-roblox-of-covertly-harvesting-kids-data/}}
\item
  Próspera - Grokipedia, accessed December 10, 2025,
  \href{https://grokipedia.com/page/Pr\%C3\%B3spera}{\ul{https://grokipedia.com/page/Pr\%C3\%B3spera}}
\item
  EMBEDDED SPATIAL PLANNING INSTRUMENTS OF COLONIAL HERITAGE IN HONDURAS
  - Utrecht University Student Theses Repository Home, accessed December
  10, 2025,
  \href{https://studenttheses.uu.nl/bitstream/handle/20.500.12932/47997/FINAL\%20THESIS_Embedded\%20SP\%20instruments\%20of\%20colonial\%20heritage\%20in\%20Honduras_Klaussova.pdf?sequence=1}{\ul{https://studenttheses.uu.nl/bitstream/handle/20.500.12932/47997/FINAL\%20THESIS\_Embedded\%20SP\%20instruments\%20of\%20colonial\%20heritage\%20in\%20Honduras\_Klaussova.pdf?sequence=1}}
\item
  Charter Cities and de- democratization in neoliberalism: the case of
  Honduras, accessed December 10, 2025,
  \href{https://revistas.uece.br/index.php/tensoesmundiais/article/download/12405/11708/57609}{\ul{https://revistas.uece.br/index.php/tensoesmundiais/article/download/12405/11708/57609}}
\end{enumerate}
