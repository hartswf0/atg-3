\section{The Generative Void: Intersections of Modular Granularity,
Critical Pedagogy, and Algorithmic Order in Physical and Digital
Architectures}\label{the-generative-void-intersections-of-modular-granularity-critical-pedagogy-and-algorithmic-order-in-physical-and-digital-architectures}

\subsection{Abstract}\label{abstract}

This paper presents an exhaustive synthesis of contemporary research
regarding the convergence of architectural theory, modular systems, and
critical pedagogy within the framework of digital and physical
environments. By integrating diverse datasets---ranging from the
behavior of "voids" in sensor networks and domestic architecture to the
historical trajectory of Lego-based modularity and the emergence of
Prompt Orchestration Markup Language (POML)---this analysis infers a
latent narrative: the transition from static, top-down structuralism to
dynamic, generative systems. The "void" is identified not merely as an
absence of matter but as a productive operational space that
necessitates adaptive strategies, whether through biological analogues
like "slime" architecture, constructionist learning environments, or
algorithmic pattern orchestration. Furthermore, the investigation
highlights the critical role of context---material, cultural, and
digital---in shaping these environments, positing that the future of
design lies in the "middle ground" between rigid modularity and fluid
adaptability. The paper explores the mathematical entropy of modular
systems, the routing protocols of underwater networks, and the
socio-political implications of "Afrofuturist" constructionism,
ultimately arguing for a unified theory of "Void Operations" that
transcends the physical-digital divide.

\subsection{I. Introduction: The Presence of
Absence}\label{i.-introduction-the-presence-of-absence}

The discipline of architecture has historically been defined by the
manipulation of mass---the arrangement of stones, bricks, and beams to
enclose space. However, a latent narrative emerging across disparate
fields---from computational design to network topology and educational
theory---suggests a paradigmatic shift. We are moving from an
architecture of \emph{objects} to an architecture of \emph{voids}. In
this contemporary framework, the "empty" space is no longer a passive
recipient of form but an active, generative agent that dictates the
behavior of the system, whether that system is a single-family dwelling,
a wireless sensor network, or a large language model.

This paper synthesizes a wide array of research materials to construct a
comprehensive theory of this transition. It examines the tension between
the "discrete unit" (the brick, the voxel, the prompt) and the
"continuum" (the void, the flow, the narrative). The analysis begins by
exploring the \textbf{Ontology of the Void} in physical space, detailing
how domestic architecture utilizes vertical emptiness for environmental
regulation and psychological balance.\textsuperscript{1} It then
transitions to the \textbf{Digital Void}, specifically the phenomenon of
"energy holes" in Underwater Wireless Sensor Networks (UWSNs), where the
void represents a critical failure state that must be managed through
complex routing algorithms.\textsuperscript{3}

The investigation then pivots to \textbf{Modular Granularity}, the
mechanism by which we attempt to fill or define these voids. Through a
detailed case study of the Lego and Modulex systems, we analyze the
entropy of construction and the tension between professional
standardization and creative play.\textsuperscript{4} This leads into a
discussion of \textbf{Pedagogical Infrastructures}, where the principles
of "Critical Pedagogy" and "Constructionism" repurpose these modular
tools to challenge institutional hierarchies and empower learners
through "Afrofuturist" design.\textsuperscript{6}

Finally, the paper addresses the emerging \textbf{Algorithmic Syntax} of
the information age, specifically the Prompt Orchestration Markup
Language (POML). We argue that POML represents the "voxelization" of
logic, an attempt to impose the same modular order on Artificial
Intelligence that Modulex attempted to impose on architectural
drafting.\textsuperscript{8} By weaving these threads together, the
paper demonstrates that the management of the void---through modularity,
pedagogy, and code---is the central design challenge of our era.

\subsection{II. The Architecture of Nothingness: Void Operations and
Spatial
Logic}\label{ii.-the-architecture-of-nothingness-void-operations-and-spatial-logic}

\subsubsection{A. The Domestic Void as Environmental and Psychological
Regulator}\label{a.-the-domestic-void-as-environmental-and-psychological-regulator}

In the context of high-density urban living, the void is frequently
misunderstood as wasted space---an economic inefficiency to be
minimized. However, recent theoretical and practical applications
redefine the void as a critical environmental regulator and a
psychological necessity. The principle of "designing void" posits that
empty spaces function similarly to skylights but with enhanced vertical
connectivity, facilitating air circulation and light penetration across
multiple levels.\textsuperscript{1}

For residences with significant depth (specifically those exceeding 20
meters), the insertion of voids becomes a structural and environmental
necessity rather than a mere aesthetic choice. These spaces break the
monotony of the floor plate, acting as "partial partitions" that
separate functional zones (e.g., living room vs. kitchen) without the
visual occlusion of solid walls.\textsuperscript{1} The strategic
placement of these voids is governed by a rigorous logic:

\begin{itemize}
\item
  \textbf{Central Voids:} When placed in the middle of the house, the
  void acts as a partition for the living room and kitchen, creating a
  feeling of spaciousness and illuminating the elevator
  chamber.\textsuperscript{1}
\item
  \textbf{Terminal Voids:} When placed at the end of the house, the void
  creates ventilation for the kitchen and dining room, ensuring that
  even the deepest parts of the plan receive natural
  light.\textsuperscript{1}
\end{itemize}

This functionalist approach is overlaid with a sophisticated
understanding of "Feng Shui" and elemental balance. The research
highlights that for small houses, voids should lean towards the element
of "Water" to generate "Wood." This is achieved through the use of pale
colors and winding lines, which create a softness that reduces the
feeling of narrowness.\textsuperscript{1} Conversely, a void in a living
room featuring a chandelier interacts with the element of "Fire,"
creating a mutualistic relationship between the space and its
decor.\textsuperscript{1} Thus, the void is not static; it is an active
elemental force that manipulates human perception of volume,
temperature, and comfort.

Furthermore, the design of the void must account for acoustic privacy. A
straight, transparent void acts like a skylight, conducting sound
clearly and resonantly between floors, which can disrupt family life. To
mitigate this, the walls of the void should be "rough, rough, and lumpy"
to absorb sound, utilizing materials like paint spikes or ceiling
bricks.\textsuperscript{1} This demonstrates a nuanced understanding of
materiality where the surface texture is dictated by the acoustic
properties of the void it encloses.

\subsubsection{B. Speculating the Architecture of
Nothingness}\label{b.-speculating-the-architecture-of-nothingness}

Moving beyond the domestic scale, speculative architectural research
reframes "nothingness" as a generative condition. Grounded in nihilist
and ontological perspectives, particularly those discussed by Brett
(2016), this approach treats the void not as a byproduct of construction
but as the "very site where spatial consciousness and meaning can
emerge".\textsuperscript{10}

This theoretical framework challenges the traditional architectural
prioritization of material presence. Instead, it proposes "void
operations"---typological categorizations of emptiness that allow for
new spatial possibilities. The void makes "being perceptible"; it is the
silence that allows the architectural "sound" to be heard. This
interdependence suggests that meaning in architecture is derived from
the rhythm of solid and void, much like the pauses in a musical
composition.\textsuperscript{10}

The study of "nothingness" classifies voids based on spatial categories
and formulates the potential these voids have in shaping perception. It
captures spaces that "project nothingness" and are lacking definition,
arguing that these undefined aspects are not failures of design but
opportunities for "spatial reinterpretation".\textsuperscript{10} In
urban spaces, these voids serve as potential areas for various
activities, interpreted differently by each individual, thus becoming a
productive condition rather than an absence.\textsuperscript{10}

\subsubsection{C. The "Slime" Analog: Adaptive Occupation of the
Void}\label{c.-the-slime-analog-adaptive-occupation-of-the-void}

To navigate and utilize these voids, theoretical models have turned to
biological analogues, specifically the behavior of slime molds. The
"Slime" concept represents a "liquid, adaptive, and responsive entity
capable of occupying, deforming, and reorganising
voids".\textsuperscript{10} Unlike rigid structural elements, "Slime"
operations---oozing, stretching, flowing, splattering,
contracting---allow for a perfect conformity to the irregular boundaries
of a void.\textsuperscript{10}

This biological metaphor extends to the "connectivity architecture"
observed in the works of Toyo Ito and SANAA, where the focus shifts from
pre-set structural elements to the inter-relationships among
components.\textsuperscript{11} The slime mold's ability to optimize
routes between food sources mirrors the architectural desire to optimize
flow between functional zones, suggesting that the future of "void
filling" is not modular stacking, but organic growth. This contrasts
sharply with the "formlessness of nature" by introducing an artificial,
synthetic growth that owes nothing to chaos but everything to
algorithmic adaptability.\textsuperscript{12}

In the "Slime" model, the architectural solution for human survival in a
crowded world is to design spaces based on available voids, optimizing
their potential to create habitable environments. The "Slime" is the
tool capable of recomposing objects and spaces into adaptive forms,
seamlessly adjusting to diverse characteristics. Its movement---oozing
and pulsating---serves as a response to different void characteristics,
mapping the positions of emptiness to create a new architecture of
survival.\textsuperscript{10}

\subsection{III. The Topology of Absence: Voids in Wireless Sensor
Networks}\label{iii.-the-topology-of-absence-voids-in-wireless-sensor-networks}

\subsubsection{A. The Energy Hole
Problem}\label{a.-the-energy-hole-problem}

In the domain of Underwater Wireless Sensor Networks (UWSNs), the
concept of the "void" shifts from a generative aesthetic asset to a
malignant functional failure. Here, the network is composed of sensor
nodes deployed at various depths (D0 to D3) to monitor oceanographic
data.\textsuperscript{3} The data must be transmitted from the deep
nodes (D3) to the surface sink (D0) via multi-hop routing.

A "void node" or "energy hole" occurs when a node depletes its battery
and dies. Because nodes closer to the surface sink must relay data for
all the nodes below them, they deplete their energy much faster than the
leaf nodes. This creates an "energy hole" around the sink, severing the
connection between the deep sensors and the surface.\textsuperscript{3}
This is known as the "hotspot problem."

The presence of a void forces the remaining nodes to route data around
the hole. This often requires increasing transmission power to reach
more distant neighbors, which in turn accelerates the depletion of those
nodes, causing the void to expand rapidly.\textsuperscript{3} The void
in this context is a contagion; its presence stresses the system,
leading to cascading failure.

\subsubsection{B. Routing Protocols as Void
Management}\label{b.-routing-protocols-as-void-management}

To mitigate the impact of these voids, complex routing protocols are
employed. The "Generalized Energy-Efficient Distance-Based Routing"
(GEDAR) protocol and others utilize "depth adjustment" to physically
move nodes to void areas, effectively healing the network.3

The network\textquotesingle s depth is split into layers:

\begin{itemize}
\item
  \textbf{D0:} Sink Node (Surface)
\item
  \textbf{D1:} Near Surface Nodes
\item
  \textbf{D2:} Mid-depth Nodes
\item
  \textbf{D3:} Seabed Nodes.\textsuperscript{3}
\end{itemize}

Nodes at the intersection of layers act as "cluster relays," forwarding
data from the deeper layer to the sink. For example, node S13 on the
seabed transmits through the chain S12--S9--S5--S3--S1--sink.3 The cost
of a link (\$C\_\{L, u \textbackslash to v\}\$) is calculated based on
the transmission power (\$P\_\{t, u\}\$) and reception power (\$P\_\{r,
v\}\$):

\$\$C\_\{L, u \textbackslash to v\} = \textbackslash frac\{P\_\{t,
u\}\}\{P\_\{r, v\}\}\$\$

This equation 3 dictates the "cost" of crossing the void. The protocol
must dynamically calculate this cost to avoid "void nodes." When a void
is discovered, a "detour path" is identified, guiding knowledge up to
lower depths. This highlights a fundamental difference between physical
and digital architecture: in the home, we build around the void to
preserve it; in the network, we build across the void to eliminate it.

\subsubsection{C. Density and the Packing
Problem}\label{c.-density-and-the-packing-problem}

The management of voids in both architecture and networks is
fundamentally a "packing problem".13 In UWSNs, the density of nodes
(\$N\$) must meet a threshold (\$N\_\{th\}\$) to ensure connectivity:

\$\$N\_\{th\} = \textbackslash begin\{cases\} 1 \&
\textbackslash text\{if \} N \textbackslash ge N\_\{th\}
\textbackslash\textbackslash{} 0 \& \textbackslash text\{if \} N
\textless{} N\_\{th\} \textbackslash end\{cases\}\$\$

If the area is sparse (\$N \textless{} N\_\{th\}\$), it is a potential
void.3 This mathematical definition of density parallels the granular
packing of material in physical construction. The efficiency of "void
structures" in materials science is a variation of this packing problem,
determining how particles (or sensor nodes) can be arranged to maximize
coverage while minimizing resource use.13

\subsection{IV. Modular Granularity: From Bricks to
Voxels}\label{iv.-modular-granularity-from-bricks-to-voxels}

The management of the void---whether filling it with sensor nodes or
carving it out of a building---is typically achieved through modular
systems. These discrete units of matter or information allow for the
standardization of construction and the measurement of complexity. The
history of the Lego brick and its professional counterpart, Modulex,
offers a profound case study in the tension between standardization and
adaptability.

\subsubsection{A. The Modulex Experiment: Professionalizing the
Brick}\label{a.-the-modulex-experiment-professionalizing-the-brick}

In 1963, the Lego Group launched "Modulex," a line of bricks
specifically designed for architects and planners.\textsuperscript{4}
Unlike the standard Lego brick, which has a height-to-width ratio of
6:5, Modulex was based on a perfect 1:1 cube (5mm x 5mm x
5mm).\textsuperscript{4} This M20 system was calibrated to a 1:20 scale,
where a single stud represented 100mm (4 inches), the standard module
for architectural wall widths.\textsuperscript{4}

The development of Modulex was driven by a desire for professional
precision. Architects frequently approximate widths to 100mm to simplify
scale drawings; Modulex provided a physical instantiation of this
simplification. The system was marketed as a tool for "statistical
purposes," where the base unit could symbolize a definite quantity,
size, or period.\textsuperscript{4}

However, Modulex failed commercially. One primary reason was the "clutch
power" issue. While standard Lego bricks rely on the interference fit of
studs and tubes to hold together ("clutch power"), Modulex bricks often
required glue for permanent models, or had different tolerances that
made them less satisfying to use as a temporary modeling
tool.\textsuperscript{4} Furthermore, the color palette was restricted
to "neutral tones" (terracotta, grey, white) to reflect the "post-war
minimalism" and "brutalism" of the era, rejecting the bright primaries
of the toy line.\textsuperscript{4} This attempt to strip the "play" out
of the brick ultimately rendered it a dry, utilitarian object that could
not compete with traditional drafting or the emerging digital tools.

The incompatibility was also a major friction point. A standard Lego
brick could not connect to a Modulex brick without illegal connections
or friction-based hacks.\textsuperscript{14} This bifurcation of the
ecosystem meant that the professional tool could not leverage the
massive availability of the toy system, isolating it in a niche market
that eventually evaporated.

\subsubsection{B. Entropy and Information Content in Modular
Systems}\label{b.-entropy-and-information-content-in-modular-systems}

The arrangement of these modular units can be analyzed through the lens
of information theory. Research into the "Entropy of Lego" suggests that
a building modeled as a set of repeating parts has a measurable
entropy.\textsuperscript{5} The "unusualness" of the pieces used
contributes to the information density of the model.

Lego models of famous buildings, such as the Empire State Building, can
be read as a grid of cells (voxels). For instance, an Empire State model
might be an 8 x 9 x 50 block of cells, of which 563 are occupied.5 The
probability that a unit is empty (void) or occupied (solid) allows for
the calculation of entropy.

\$\$P(r) = k(r + v)\^{}\{-a\}\$\$

This Zipf-Mandelbrot distribution describes the frequency of piece
usage.5 Standard Lego Architecture kits are found to have a relatively
low entropy (around 7 bits per piece), utilizing common bricks to create
iconic forms. In contrast, "MovieMaker" sets or complex MOCs (My Own
Creations) display higher entropy (9.6 bits), indicating a greater
diversity of parts and a more complex internal language.5

This quantification of design complexity implies that "creativity" in
modular systems is linked to the expansion of the vocabulary (the number
of distinct piece types) and the unpredictability of their combination.
Cities "amplify the entropy of buildings" because they present a chaotic
juxtaposition of forms, whereas the controlled environment of a single
Lego kit reduces entropy to a manageable, instruction-based
sequence.\textsuperscript{5}

\subsubsection{C. The Voxel and the Atom
Brick}\label{c.-the-voxel-and-the-atom-brick}

The legacy of Modulex persists in the concept of the "Voxel" (volumetric
pixel) and modern iterations like "The Atom Brick".\textsuperscript{15}
Atom Bricks are 3/4 the size of standard Lego bricks (12x24mm vs
16x32mm), allowing for higher resolution models that capture
architectural details impossible with the coarser grain of standard
Lego.\textsuperscript{15}

For example, the Atom Brick model of Frank Lloyd Wright's \emph{Darwin
D. Martin House} uses 1,961 pieces to achieve a footprint of 26x37 cm.
If built with standard Lego, the model would be 35x50 cm, losing the
delicate scale required for a desktop display.\textsuperscript{15} This
"voxelization" allows for a finer granularity of expression, moving
closer to the Modulex ideal of the "perfect scale model" while retaining
the interlocking playability of the System brick.

This modular logic extends to industrial logistics. P\&O Maritime
Logistics (POML) utilizes a "high resolution voxel" system (250mm x
250mm x 50mm) for modular cargo storage.\textsuperscript{16} This
demonstrates that the logic of the toy brick---standardized,
interlocking units---has permeated heavy industry, allowing for
efficient packing and reconfiguration of physical goods in a manner
identical to the manipulation of data packets or plastic bricks.

\subsection{V. Digital Materiality: Virtual Reality and
Simulation}\label{v.-digital-materiality-virtual-reality-and-simulation}

The physical limitations of the brick---gravity, friction, material
cost---are transcended in virtual environments. "Dreamscape Bricks VR"
represents a "Digital Reality Environment As a Medium for Studio
Collaboration".\textsuperscript{17} This system allows architects to
design using virtual Lego bricks via direct manipulation, maintaining
the "connection rules" of the physical toy while enabling features like
"anti-gravity" and infinite supply.\textsuperscript{17}

\subsubsection{A. The Dreamscape
Architecture}\label{a.-the-dreamscape-architecture}

"Dreamscape Bricks VR" is built on the Unreal Engine 4 and utilizes a
framework called DREAMSCAPE (Digital Reality Environment As a Medium for
Studio Collaboration in Architectural Production \&
Education).\textsuperscript{17} The system architecture leverages the
user\textquotesingle s pre-existing familiarity with Lego. Because most
users have played with these bricks as children, the "germane cognitive
load" required to learn the VR tool is significantly reduced. Users
already possess the "brick building schemas" in their long-term
memory.\textsuperscript{17}

The system allows for "Direct Manipulation," where users pick up,
rotate, and snap bricks together using hand controllers. A key feature
is the visualization of "ghost" pieces---blue translucent bricks that
appear at valid connection points when a user brings a brick close to
the model.\textsuperscript{18} This visual feedback replaces the tactile
"click" of the physical brick, guiding the user to valid structural
configurations.

\subsubsection{B. Scales of Interaction}\label{b.-scales-of-interaction}

The virtual environment allows for dynamic scaling, creating a
"mid-ground between imagination and perception".\textsuperscript{18} The
system supports two distinct scales:

\begin{enumerate}
\def\labelenumi{\arabic{enumi}.}
\item
  \textbf{Precision Building Scale (1:10):} The virtual bricks are ten
  times larger than real life. This allows for precise manipulation and
  detailed assembly, treating the bricks as large construction
  blocks.\textsuperscript{18}
\item
  \textbf{Figure-Sized User Scale (1:42.5):} The user is shrunk to the
  size of a Lego Minifigure (1:42.5 scale). A 170cm tall human becomes
  the same height as a 4cm Minifigure. In this mode, the virtual bricks
  are effectively massive masonry units.\textsuperscript{18}
\end{enumerate}

This scaling capability allows the designer to embody the inhabitant.
They can build a wall at 1:10 scale, then shrink down to 1:42.5 scale to
walk through the "void" they just created, checking sightlines and
spatial feeling from the perspective of the occupant. This recursive
loop of design and experience is impossible in physical modeling, where
the architect is always a giant looming over the model.

\subsection{VI. Pedagogical Infrastructures: Critical Pedagogy and
Constructionism}\label{vi.-pedagogical-infrastructures-critical-pedagogy-and-constructionism}

The architecture of physical space has a direct corollary in the
architecture of learning. The synthesized research identifies a strong
link between "Constructionism"---learning by building---and the modular
systems described above.

\subsubsection{A. Critical Pedagogy: Disrupting the
Flow}\label{a.-critical-pedagogy-disrupting-the-flow}

Critical Pedagogy, as discussed in the context of architectural
education, posits that educational environments often reinforce dominant
power structures by transmitting "canonical
discourses".\textsuperscript{7} Traditional "Vorkurs" (preliminary
courses) reinforce hierarchies and stifle innovation. To counter this,
architecture schools must cultivate "contradiction and conflict,"
producing "moments of crisis" that challenge professional
incorporation.\textsuperscript{7}

This pedagogy rejects the "banking model" (filling the student-void with
facts) in favor of "problem-posing" education. It requires spaces that
are "modifiable" and "convertible"---physical environments that invite
manipulation and appropriation.\textsuperscript{19} A "multi-purpose
room" is often generic and oppressive; a truly flexible space possesses
"modularity and open-endedness at a structural level," allowing users to
redesign the space itself.\textsuperscript{19}

True flexibility in educational architecture involves:

\begin{itemize}
\item
  \textbf{Modifiability:} Spaces comprised of mobile components (walls,
  partitions) that invite active manipulation.\textsuperscript{19}
\item
  \textbf{Convertibility:} Spaces designed with a "core and shell" model
  (like office buildings) that allow for total programmatic
  reassignment.\textsuperscript{19}
\item
  \textbf{Scaleability:} Spaces that can expand or contract based on
  enrollment flows.\textsuperscript{19}
\end{itemize}

\subsubsection{B. Constructionism and the Lego/Logo
Environment}\label{b.-constructionism-and-the-legologo-environment}

Seymour Papert's theory of Constructionism argues that knowledge is most
effectively constructed when the learner is actively engaged in building
a public entity.\textsuperscript{20} This is where the Lego brick
(physical or digital) becomes a powerful pedagogical tool.

"Digital Lego-Based Learning Environments" have been developed for
teaching abstract concepts like fraction ordering.\textsuperscript{21}
These environments use the familiar logic of the brick to scaffold
complex mathematical thinking. For example, a "fraction cube" game
requires students to break cubes in ascending order, linking the
physical act of destruction/construction with the abstract concept of
numeracy.\textsuperscript{21} The "game environment" provides the
motivation, while the modular nature of the task allows for
"tinkering"---an iterative process of design and reflection essential to
deep learning.\textsuperscript{6}

Research confirms that these Lego representations support conceptual
understanding for both high-ability and low-ability
students.\textsuperscript{21} The modular system allows for
"differentiation" naturally---advanced students can build complex,
high-entropy structures, while struggling students can focus on basic,
low-entropy connections, all within the same material framework.

\subsubsection{C. Afrofuturism and Critical Constructionist
Design}\label{c.-afrofuturism-and-critical-constructionist-design}

The synthesis extends Constructionism into the political realm through
\textbf{Afrofuturism} as "Critical Constructionist
Design".\textsuperscript{6} This framework invites learners to use their
cultural histories to "design futuristic artifacts that critique
existing social inequities."

In this pedagogical model, the "void" is the erased history or the
unimagined future of marginalized communities. The "module" is the
cultural artifact or story. The act of construction is a political act
of reclaiming space (physical and narrative). Participants use personal
experiences and family histories to design artifacts that challenge
anti-Blackness and environmental instability.\textsuperscript{6}

This aligns with "Critical Compassionate Pedagogy" and "Ecopedagogy,"
which seek to detach education from anthropocentric and colonial
paradigms.\textsuperscript{23} By engaging in "speculative design,"
students move beyond the "rhetoric of mastery and control" often found
in maker education, instead focusing on "multispecies and multimattered
creativity".\textsuperscript{6} The goal is not just to teach STEM
skills, but to use those skills to "build futures from the past and
present," creating a "safe disciplinary space" where wellbeing and
identity are central.\textsuperscript{6}

\subsubsection{D. The Politics of Educational
Infrastructure}\label{d.-the-politics-of-educational-infrastructure}

Just as architectural infrastructure shapes movement, digital
infrastructure shapes pedagogy. The adoption of specific Learning
Management Systems (LMS) or platforms like Microsoft Teams is not a
neutral technical choice but a pedagogical one.\textsuperscript{24}
These platforms encode assumptions about "efficiency, scalability, and
control" that may conflict with the values of "trust, autonomy, and
learner agency".\textsuperscript{24}

Infrastructures function as "knowledge-producing machines" that
stabilize certain worldviews while rendering others
invisible.\textsuperscript{24} A university that standardizes on a rigid
LMS is like a housing block that refuses to design voids---it creates a
stifling environment where "messy" critical dialogue is impossible. The
"Digital University" must therefore be reimagined not as a consumer of
corporate tools but as a builder of "critical infrastructure," designing
digital voids that foster dialogue and experimentation rather than
surveillance and compliance.\textsuperscript{24}

\subsection{VII. Algorithmic Syntax: POML and the Architecture of
Logic}\label{vii.-algorithmic-syntax-poml-and-the-architecture-of-logic}

The final thread of this narrative concerns the ordering systems that
govern these modular and void-based interactions in the realm of
Artificial Intelligence. The acronym \textbf{POML} appears in the
research with multiple, yet thematically resonant, definitions,
symbolizing the attempt to structure the "void" of latent space.

\subsubsection{A. Prompt Orchestration Markup Language
(POML)}\label{a.-prompt-orchestration-markup-language-poml}

In the realm of Large Language Models (LLMs), POML stands for
\textbf{Prompt Orchestration Markup Language}.\textsuperscript{8} As
LLMs become integral to healthcare (e.g., triage, diagnosis) and other
high-stakes fields, the inherent instability of natural language
prompts---the "hallucination" problem---becomes unacceptable.

POML introduces a structured framework to address this. It employs a
component-based markup (similar to HTML or XML) to define "roles, tasks,
and examples" within a prompt.\textsuperscript{9} It utilizes tags to
separate logic from content, much like CSS separates style from HTML.

\begin{itemize}
\item
  \textbf{Structure:} It uses specialized tags for seamless data
  integration.
\item
  \textbf{Styling:} It uses a CSS-like system to decouple content from
  presentation.\textsuperscript{9}
\end{itemize}

POML is to AI what Modulex was to architecture. It attempts to impose a
rigid, professional grid (markup) onto a fluid, organic medium (natural
language). Just as Modulex failed because it lacked the "clutch power"
of the chaotic real world and was too rigid for play, POML faces the
challenge of taming the unpredictability of AI. However, unlike Modulex,
POML is succeeding because the stakes in AI (medical diagnosis) demand
the rigidity that the "toy" model of basic prompting cannot provide.

\subsubsection{B. Pattern-Oriented Modeling
Language}\label{b.-pattern-oriented-modeling-language}

A secondary definition, \textbf{Pattern-Oriented Modeling Language}
(also POML), appears in software architecture.\textsuperscript{25} Here,
it is used to represent design patterns---reusable solutions to common
problems. It utilizes tags like \textless poml:scope\textgreater,
\textless poml:pattern\textgreater,
\textless poml:methodname\textgreater, and
\textless poml:calls\textgreater{} to define the relationships between
software components.\textsuperscript{25}

For example, a C++ scoped name systemGUI.myWindow.currentPosition would
be represented in POML as:

\begin{quote}
XML
\end{quote}

\textless poml:scope\textgreater systemGUI\\
\textless poml:scope\textgreater myWindow\\
currentPosition\\
\textless/poml:scope\textgreater{}\\
\textless/poml:scope\textgreater{}

.\textsuperscript{25} This hierarchical nesting mirrors the "cluster
head" logic of the sensor networks and the nested "voids" of the
domestic house. It is a universal syntax of containment.

\subsubsection{C. Physical Order and Moral
Liberty}\label{c.-physical-order-and-moral-liberty}

A tertiary, philosophical definition found in the snippets is
\textbf{Physical Order and Moral Liberty (POML)}.\textsuperscript{26}
While distinct from the technical markup, it provides a poetic caption
for the entire report. The "Physical Order" (the brick, the wall, the
markup tag, the routing protocol) exists to create the space for "Moral
Liberty" (the void, the critical agency of the student, the creative
interpretation, the flow of air).

This philosophical POML argues that "Spirituality is not permanent the
way a physical object is".\textsuperscript{26} The physical object (the
architecture) provides the "foundation" for the ephemeral "moral
liberty" (the life lived within). This aligns perfectly with the
"Architecture of Nothingness" \textsuperscript{10}, where the material
form exists only to frame the void where meaning emerges.

\subsection{VIII. Conclusion: The Unified Field of Void
Operations}\label{viii.-conclusion-the-unified-field-of-void-operations}

This exhaustive analysis of the provided research snippets uncovers a
latent narrative centered on the \textbf{structuring of potential}.
Whether through the physical interlocking of Modulex bricks, the digital
markup of AI prompts (POML), or the subtractive carving of architectural
voids, human agency is defined by how we organize space and information.

The evidence leads to several critical conclusions:

\begin{enumerate}
\def\labelenumi{\arabic{enumi}.}
\item
  \textbf{The Void is the Primary Actor:} In both physical architecture
  and digital networks, the void is not a passive background. In the
  home, it is a "chimney" for life.\textsuperscript{1} In the network,
  it is a "black hole" for data.\textsuperscript{3} Design must center
  on the management of these voids---enhancing the productive ones and
  bridging the destructive ones.
\item
  \textbf{Modularity is a Double-Edged Sword:} The history of Modulex
  \textsuperscript{4} proves that hyper-rationality and rigid
  standardization can kill utility. The "perfect" 1:1 cube failed
  because it lacked the messy "clutch power" of the toy. Digital systems
  like POML \textsuperscript{9} must be careful not to over-constrain
  the "latent space" of AI, or they risk stifling the very creativity
  they seek to harness.
\item
  \textbf{Pedagogy Must Be Architectural:} Education cannot occur in
  generic "multi-purpose" voids. It requires "Critical Constructionist"
  spaces---environments that can be hacked, rebuilt, and owned by the
  learners.\textsuperscript{6} The tools of learning (Lego, VR) must be
  scalable and culturally responsive (Afrofuturism) to be effective.
\item
  \textbf{Context Overrides Code:} Whether it is the choice of
  "Laterite" stone in India \textsuperscript{28} or the "depth
  adjustment" of a sensor in the ocean \textsuperscript{3}, the context
  dictates the solution. There is no universal brick, and there is no
  universal algorithm.
\end{enumerate}

In sum, the "Physical Order" of our technologies must be designed to
maximize the "Moral Liberty" of their users. We must build with "Slime"
\textsuperscript{10} as much as with "Bricks," ensuring that our systems
can flow, adapt, and breathe in the complex voids of the real world. The
future of design is not in the placing of the solid, but in the
orchestration of the empty.

\subsection{IX. Appendix: Data Tables}\label{ix.-appendix-data-tables}

\subsubsection{Table 1: Comparative Analysis of Void
Operations}\label{table-1-comparative-analysis-of-void-operations}

\begin{longtable}[]{@{}
  >{\raggedright\arraybackslash}p{(\linewidth - 8\tabcolsep) * \real{0.2000}}
  >{\raggedright\arraybackslash}p{(\linewidth - 8\tabcolsep) * \real{0.2000}}
  >{\raggedright\arraybackslash}p{(\linewidth - 8\tabcolsep) * \real{0.2000}}
  >{\raggedright\arraybackslash}p{(\linewidth - 8\tabcolsep) * \real{0.2000}}
  >{\raggedright\arraybackslash}p{(\linewidth - 8\tabcolsep) * \real{0.2000}}@{}}
\toprule\noalign{}
\begin{minipage}[b]{\linewidth}\raggedright
\textbf{Domain}
\end{minipage} & \begin{minipage}[b]{\linewidth}\raggedright
\textbf{The Void}
\end{minipage} & \begin{minipage}[b]{\linewidth}\raggedright
\textbf{Function/Threat}
\end{minipage} & \begin{minipage}[b]{\linewidth}\raggedright
\textbf{Management Strategy}
\end{minipage} & \begin{minipage}[b]{\linewidth}\raggedright
\textbf{Key Metric}
\end{minipage} \\
\begin{minipage}[b]{\linewidth}\raggedright
\textbf{Domestic Architecture}
\end{minipage} & \begin{minipage}[b]{\linewidth}\raggedright
Light Well / Stairwell
\end{minipage} & \begin{minipage}[b]{\linewidth}\raggedright
Air Circulation / Light
\end{minipage} & \begin{minipage}[b]{\linewidth}\raggedright
"Designing Void" \textsuperscript{1}
\end{minipage} & \begin{minipage}[b]{\linewidth}\raggedright
Verticality / Depth (\textgreater10m)
\end{minipage} \\
\begin{minipage}[b]{\linewidth}\raggedright
\textbf{UWSN (Sensor Networks)}
\end{minipage} & \begin{minipage}[b]{\linewidth}\raggedright
Energy Hole
\end{minipage} & \begin{minipage}[b]{\linewidth}\raggedright
Data Loss / Network Partition
\end{minipage} & \begin{minipage}[b]{\linewidth}\raggedright
Depth Adjustment / Routing \textsuperscript{3}
\end{minipage} & \begin{minipage}[b]{\linewidth}\raggedright
Link Cost (\$C\_\{L, u \textbackslash to v\}\$)
\end{minipage} \\
\begin{minipage}[b]{\linewidth}\raggedright
\textbf{Speculative Design}
\end{minipage} & \begin{minipage}[b]{\linewidth}\raggedright
"Nothingness"
\end{minipage} & \begin{minipage}[b]{\linewidth}\raggedright
Generative Space
\end{minipage} & \begin{minipage}[b]{\linewidth}\raggedright
"Slime" Adaptation \textsuperscript{10}
\end{minipage} & \begin{minipage}[b]{\linewidth}\raggedright
Fluidity / Conformity
\end{minipage} \\
\begin{minipage}[b]{\linewidth}\raggedright
\textbf{Pedagogy}
\end{minipage} & \begin{minipage}[b]{\linewidth}\raggedright
The Classroom
\end{minipage} & \begin{minipage}[b]{\linewidth}\raggedright
Epistemic Space
\end{minipage} & \begin{minipage}[b]{\linewidth}\raggedright
Critical Constructionism \textsuperscript{6}
\end{minipage} & \begin{minipage}[b]{\linewidth}\raggedright
Student Agency / Modifiability
\end{minipage} \\
\midrule\noalign{}
\endhead
\bottomrule\noalign{}
\endlastfoot
\end{longtable}

\subsubsection{Table 2: Modular System
Specifications}\label{table-2-modular-system-specifications}

\begin{longtable}[]{@{}
  >{\raggedright\arraybackslash}p{(\linewidth - 8\tabcolsep) * \real{0.2000}}
  >{\raggedright\arraybackslash}p{(\linewidth - 8\tabcolsep) * \real{0.2000}}
  >{\raggedright\arraybackslash}p{(\linewidth - 8\tabcolsep) * \real{0.2000}}
  >{\raggedright\arraybackslash}p{(\linewidth - 8\tabcolsep) * \real{0.2000}}
  >{\raggedright\arraybackslash}p{(\linewidth - 8\tabcolsep) * \real{0.2000}}@{}}
\toprule\noalign{}
\begin{minipage}[b]{\linewidth}\raggedright
\textbf{System}
\end{minipage} & \begin{minipage}[b]{\linewidth}\raggedright
\textbf{Dimensions}
\end{minipage} & \begin{minipage}[b]{\linewidth}\raggedright
\textbf{Scale}
\end{minipage} & \begin{minipage}[b]{\linewidth}\raggedright
\textbf{Philosophy}
\end{minipage} & \begin{minipage}[b]{\linewidth}\raggedright
\textbf{Fate}
\end{minipage} \\
\begin{minipage}[b]{\linewidth}\raggedright
\textbf{Lego System}
\end{minipage} & \begin{minipage}[b]{\linewidth}\raggedright
8mm pitch (approx)
\end{minipage} & \begin{minipage}[b]{\linewidth}\raggedright
Variable
\end{minipage} & \begin{minipage}[b]{\linewidth}\raggedright
Play / Remixability
\end{minipage} & \begin{minipage}[b]{\linewidth}\raggedright
Global Standard
\end{minipage} \\
\begin{minipage}[b]{\linewidth}\raggedright
\textbf{Modulex (M20)}
\end{minipage} & \begin{minipage}[b]{\linewidth}\raggedright
5mm x 5mm x 5mm
\end{minipage} & \begin{minipage}[b]{\linewidth}\raggedright
1:20
\end{minipage} & \begin{minipage}[b]{\linewidth}\raggedright
Professional Precision
\end{minipage} & \begin{minipage}[b]{\linewidth}\raggedright
Discontinued (Commercial Failure)
\end{minipage} \\
\begin{minipage}[b]{\linewidth}\raggedright
\textbf{Atom Brick}
\end{minipage} & \begin{minipage}[b]{\linewidth}\raggedright
12mm x 24mm
\end{minipage} & \begin{minipage}[b]{\linewidth}\raggedright
3/4 Lego
\end{minipage} & \begin{minipage}[b]{\linewidth}\raggedright
High Resolution Model
\end{minipage} & \begin{minipage}[b]{\linewidth}\raggedright
Niche Collector Market
\end{minipage} \\
\begin{minipage}[b]{\linewidth}\raggedright
\textbf{Dreamscape VR}
\end{minipage} & \begin{minipage}[b]{\linewidth}\raggedright
Variable (Digital)
\end{minipage} & \begin{minipage}[b]{\linewidth}\raggedright
1:10 \& 1:42.5
\end{minipage} & \begin{minipage}[b]{\linewidth}\raggedright
Immersive Simulation
\end{minipage} & \begin{minipage}[b]{\linewidth}\raggedright
Academic Prototype
\end{minipage} \\
\midrule\noalign{}
\endhead
\bottomrule\noalign{}
\endlastfoot
\end{longtable}

\subsubsection{Table 3: POML Definitions and
Contexts}\label{table-3-poml-definitions-and-contexts}

\begin{longtable}[]{@{}
  >{\raggedright\arraybackslash}p{(\linewidth - 8\tabcolsep) * \real{0.2000}}
  >{\raggedright\arraybackslash}p{(\linewidth - 8\tabcolsep) * \real{0.2000}}
  >{\raggedright\arraybackslash}p{(\linewidth - 8\tabcolsep) * \real{0.2000}}
  >{\raggedright\arraybackslash}p{(\linewidth - 8\tabcolsep) * \real{0.2000}}
  >{\raggedright\arraybackslash}p{(\linewidth - 8\tabcolsep) * \real{0.2000}}@{}}
\toprule\noalign{}
\begin{minipage}[b]{\linewidth}\raggedright
\textbf{Acronym}
\end{minipage} & \begin{minipage}[b]{\linewidth}\raggedright
\textbf{Full Name}
\end{minipage} & \begin{minipage}[b]{\linewidth}\raggedright
\textbf{Domain}
\end{minipage} & \begin{minipage}[b]{\linewidth}\raggedright
\textbf{Function}
\end{minipage} & \begin{minipage}[b]{\linewidth}\raggedright
\textbf{Key Mechanism}
\end{minipage} \\
\begin{minipage}[b]{\linewidth}\raggedright
\textbf{POML}
\end{minipage} & \begin{minipage}[b]{\linewidth}\raggedright
Prompt Orchestration Markup Language
\end{minipage} & \begin{minipage}[b]{\linewidth}\raggedright
AI / Healthcare
\end{minipage} & \begin{minipage}[b]{\linewidth}\raggedright
Standardize LLM Inputs
\end{minipage} & \begin{minipage}[b]{\linewidth}\raggedright
Tags, Roles, CSS-like styling \textsuperscript{9}
\end{minipage} \\
\begin{minipage}[b]{\linewidth}\raggedright
\textbf{POML}
\end{minipage} & \begin{minipage}[b]{\linewidth}\raggedright
Pattern-Oriented Modeling Language
\end{minipage} & \begin{minipage}[b]{\linewidth}\raggedright
Software Engineering
\end{minipage} & \begin{minipage}[b]{\linewidth}\raggedright
Define Design Patterns
\end{minipage} & \begin{minipage}[b]{\linewidth}\raggedright
Scoped Tags (poml:scope) \textsuperscript{25}
\end{minipage} \\
\begin{minipage}[b]{\linewidth}\raggedright
\textbf{POML}
\end{minipage} & \begin{minipage}[b]{\linewidth}\raggedright
Physical Order and Moral Liberty
\end{minipage} & \begin{minipage}[b]{\linewidth}\raggedright
Philosophy
\end{minipage} & \begin{minipage}[b]{\linewidth}\raggedright
Metaphysical Framework
\end{minipage} & \begin{minipage}[b]{\linewidth}\raggedright
Interaction of Matter and Spirit \textsuperscript{26}
\end{minipage} \\
\begin{minipage}[b]{\linewidth}\raggedright
\textbf{POML}
\end{minipage} & \begin{minipage}[b]{\linewidth}\raggedright
P\&O Maritime Logistics
\end{minipage} & \begin{minipage}[b]{\linewidth}\raggedright
Industrial Logistics
\end{minipage} & \begin{minipage}[b]{\linewidth}\raggedright
Modular Cargo
\end{minipage} & \begin{minipage}[b]{\linewidth}\raggedright
Voxelized Storage (250mm unit) \textsuperscript{16}
\end{minipage} \\
\midrule\noalign{}
\endhead
\bottomrule\noalign{}
\endlastfoot
\end{longtable}

\paragraph{Works cited}\label{works-cited}

\begin{enumerate}
\def\labelenumi{\arabic{enumi}.}
\item
  The principle of designing void in a house must be known, accessed
  December 10, 2025,
  \href{https://longsoncement.com.vn/en/the-principle-of-designing-void-in-a-house-must-be-known/}{\ul{https://longsoncement.com.vn/en/the-principle-of-designing-void-in-a-house-must-be-known/}}
\item
  Evaluation of Quality of Daylight in a Contemporary Residential ...,
  accessed December 10, 2025,
  \href{https://www.ijee.net/article_183272_4563966eebd3cd2385cd40c12eafb98d.pdf}{\ul{https://www.ijee.net/article\_183272\_4563966eebd3cd2385cd40c12eafb98d.pdf}}
\item
  A Novel Power-Based Sparsity-Aware and Energy-Efficient ... - MDPI,
  accessed December 10, 2025,
  \href{https://www.mdpi.com/2079-9292/10/7/854}{\ul{https://www.mdpi.com/2079-9292/10/7/854}}
\item
  Saving Modulex - MiniBricks Madness, accessed December 10, 2025,
  \href{https://minibricksmadness.com/wp-content/uploads/2011/05/Saving-Modulex2.pdf}{\ul{https://minibricksmadness.com/wp-content/uploads/2011/05/Saving-Modulex2.pdf}}
\item
  The entropy of LEGOÕ - Andrew Crompton, accessed December 10, 2025,
  \href{http://www.cromp.com/download/pdfdocs/Entropy\%20of\%20Lego\%20E&PB.pdf}{\ul{http://www.cromp.com/download/pdfdocs/Entropy\%20of\%20Lego\%20E\&PB.pdf}}
\item
  Afrofuturism as critical constructionist design: building futures from
  ..., accessed December 10, 2025,
  \href{https://www.researchgate.net/publication/340791336_Afrofuturism_as_critical_constructionist_design_building_futures_from_the_past_and_present}{\ul{https://www.researchgate.net/publication/340791336\_Afrofuturism\_as\_critical\_constructionist\_design\_building\_futures\_from\_the\_past\_and\_present}}
\item
  Digital Architectural Design in Foundation Courses - CumInCAD,
  accessed December 10, 2025,
  \href{https://papers.cumincad.org/data/works/att/ijac20119407.pdf}{\ul{https://papers.cumincad.org/data/works/att/ijac20119407.pdf}}
\item
  13186 PDFs \textbar{} Review articles in HL7 - ResearchGate, accessed
  December 10, 2025,
  \href{https://www.researchgate.net/topic/HL7/publications}{\ul{https://www.researchgate.net/topic/HL7/publications}}
\item
  人工智能2025\_8\_20 - arXiv每日学术速递, accessed December 10, 2025,
  \href{http://www.arxivdaily.com/thread/70752}{\ul{http://www.arxivdaily.com/thread/70752}}
\item
  Speculating the architecture of nothingness through void operations,
  accessed December 10, 2025,
  \href{https://www.researchgate.net/publication/397239774_Speculating_the_architecture_of_nothingness_through_void_operations}{\ul{https://www.researchgate.net/publication/397239774\_Speculating\_the\_architecture\_of\_nothingness\_through\_void\_operations}}
\item
  Bridging the Philosophy of Science and Architecture - MDPI, accessed
  December 10, 2025,
  \href{https://www.mdpi.com/2075-5309/15/10/1646}{\ul{https://www.mdpi.com/2075-5309/15/10/1646}}
\item
  Possibility of critical practice in computational design, accessed
  December 10, 2025,
  \href{https://journals.openedition.org/craup/361?lang=en}{\ul{https://journals.openedition.org/craup/361?lang=en}}
\item
  Universidad Autónoma de Nuevo León, accessed December 10, 2025,
  \href{http://eprints.uanl.mx/25704/1/1080328927.pdf}{\ul{http://eprints.uanl.mx/25704/1/1080328927.pdf}}
\item
  A Lego brick from 1949 can still connect with a brand new ... -
  Reddit, accessed December 10, 2025,
  \href{https://www.reddit.com/r/Damnthatsinteresting/comments/1hs2eyr/a_lego_brick_from_1949_can_still_connect_with_a/}{\ul{https://www.reddit.com/r/Damnthatsinteresting/comments/1hs2eyr/a\_lego\_brick\_from\_1949\_can\_still\_connect\_with\_a/}}
\item
  Three new models by LEGO Architecture founder Adam Reed Tucker,
  accessed December 10, 2025,
  \href{https://brickarchitect.com/2020/adam-reed-tucker-the-atom-brick/}{\ul{https://brickarchitect.com/2020/adam-reed-tucker-the-atom-brick/}}
\item
  FEATURES The world\textquotesingle s leading and only monthly magazine
  for the ..., accessed December 10, 2025,
  \href{https://www.drycargomag.com/ThreeDmags/Magazine-Editions/November-2025-Issue/offline/download.pdf}{\ul{https://www.drycargomag.com/ThreeDmags/Magazine-Editions/November-2025-Issue/offline/download.pdf}}
\item
  (PDF) An investigation of architectural design process in physical
  ..., accessed December 10, 2025,
  \href{https://www.researchgate.net/publication/365144041_An_investigation_of_architectural_design_process_in_physical_medium_and_VR/download}{\ul{https://www.researchgate.net/publication/365144041\_An\_investigation\_of\_architectural\_design\_process\_in\_physical\_medium\_and\_VR/download}}
\item
  An Experimental Virtual Reality Tool for Architectural Design -
  IxD\&A, accessed December 10, 2025,
  \href{https://ixdea.org/wp-content/uploads/IxDEA_art/52/52_13.pdf}{\ul{https://ixdea.org/wp-content/uploads/IxDEA\_art/52/52\_13.pdf}}
\item
  Flexible Space \& Built Pedagogy: emerging IT embodiments, accessed
  December 10, 2025,
  \href{https://www.researchgate.net/publication/228444411_Flexible_Space_Built_Pedagogy_emerging_IT_embodiments}{\ul{https://www.researchgate.net/publication/228444411\_Flexible\_Space\_Built\_Pedagogy\_emerging\_IT\_embodiments}}
\item
  2 Philosophy, theory, and pedagogy of Technologies education, accessed
  December 10, 2025,
  \href{https://api.taylorfrancis.com/content/chapters/oa-edit/download?identifierName=doi&identifierValue=10.4324/9781003490715-2&type=chapterpdf}{\ul{https://api.taylorfrancis.com/content/chapters/oa-edit/download?identifierName=doi\&identifierValue=10.4324/9781003490715-2\&type=chapterpdf}}
\item
  Architecture of Digital Lego-Based Learning Environment, accessed
  December 10, 2025,
  \href{https://www.researchgate.net/figure/Architecture-of-Digital-Lego-Based-Learning-Environment_fig1_297680369}{\ul{https://www.researchgate.net/figure/Architecture-of-Digital-Lego-Based-Learning-Environment\_fig1\_297680369}}
\item
  A Digital Lego-Based Learning Environment for Fraction Ordering,
  accessed December 10, 2025,
  \href{https://www.researchgate.net/publication/297680369_A_Digital_Lego-Based_Learning_Environment_for_Fraction_Ordering}{\ul{https://www.researchgate.net/publication/297680369\_A\_Digital\_Lego-Based\_Learning\_Environment\_for\_Fraction\_Ordering}}
\item
  Towards an Ecosystem-of-Learning for Architectural Education, accessed
  December 10, 2025,
  \href{https://www.ingentaconnect.com/contentone/arched/char/2021/00000007/00000001/art00002?crawler=true&mimetype=application/pdf}{\ul{https://www.ingentaconnect.com/contentone/arched/char/2021/00000007/00000001/art00002?crawler=true\&mimetype=application/pdf}}
\item
  Critical Infrastructure: Reimagining the Digital University -
  eLearning, accessed December 10, 2025,
  \href{https://elearning.home.nomagic.uk/0032_critical_infrastructure_reimagining_the_digital_university.html}{\ul{https://elearning.home.nomagic.uk/0032\_critical\_infrastructure\_reimagining\_the\_digital\_university.html}}
\item
  SPQR: Formal Foundations and Practical Support for the Automated ...,
  accessed December 10, 2025,
  \href{http://www.cs.unc.edu/xcms/wpfiles/dissertations/smith_jason.pdf}{\ul{http://www.cs.unc.edu/xcms/wpfiles/dissertations/smith\_jason.pdf}}
\item
  Introductory Seville Pages - Santayana Edition - Indiana University,
  accessed December 10, 2025,
  \href{https://santedit.reclaimhosting.iu.edu/wp-content/uploads/2014/10/Overheard-in-Seville.26.2008.pdf}{\ul{https://santedit.reclaimhosting.iu.edu/wp-content/uploads/2014/10/Overheard-in-Seville.26.2008.pdf}}
\item
  A Meaningful Life amidst a Pluralism of Cultures and Values - Brill,
  accessed December 10, 2025,
  \href{https://brill.com/downloadpdf/display/title/62305.pdf}{\ul{https://brill.com/downloadpdf/display/title/62305.pdf}}
\item
  Materials and techniques in vernacular architecture - RTF, accessed
  December 10, 2025,
  \href{https://www.re-thinkingthefuture.com/architectural-styles/a10350-materials-and-techniques-in-vernacular-architecture/}{\ul{https://www.re-thinkingthefuture.com/architectural-styles/a10350-materials-and-techniques-in-vernacular-architecture/}}
\end{enumerate}
