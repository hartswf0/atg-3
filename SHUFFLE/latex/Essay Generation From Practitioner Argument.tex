\section{The Legibility Mandate: A Structural Critique and Architectural
Methodology for Responsible LLM
Intake}\label{the-legibility-mandate-a-structural-critique-and-architectural-methodology-for-responsible-llm-intake}

\subsection{Abstract}\label{abstract}

The rapid integration of Large Language Models (LLMs) into high-stakes
decision-making environments has precipitated an ontological and
operational crisis. As these stochastic systems exhibit emergent
behaviors that mimic agency, they challenge traditional governance
frameworks rooted in static compliance. This report identifies a
pervasive bifurcation in the current ethical AI landscape: a "haunted"
paradigm where opaque tools exert unexamined influence, often masked by
the bureaucracy of "ethics washing" and performative checklists, versus
a "legible" paradigm that demands architectural transparency and
structural constraint. Drawing upon sociotechnical theory, specifically
the "haunted tool" metaphor and James C. Scott\textquotesingle s concept
of state legibility, alongside rigorous technical
methodologies---Chain-of-Thought (CoT) monitoring, Process Reward Models
(PRMs), and Constrained Decoding---this document constructs a
comprehensive "Paragon Essay Pyramid." It argues that responsible intake
requires a radical shift from outcome-based observation to process-based
stewardship. By enforcing "Technical Legibility," practitioners can
effectively exorcise the algorithmic specter, transforming the LLM from
a chaotic agent of the "void" into a verifiable instrument of human
intent. This report provides an exhaustive analysis of the failures of
current "fairness dashboards" and impact assessments, proposing instead
a rigorous infrastructure of "decision provenance" and "meaningful human
control" as the only viable path for sustainable AI adoption.

\subsection{Part I: The Ontological Crisis of the Haunted
Tool}\label{part-i-the-ontological-crisis-of-the-haunted-tool}

\subsubsection{1.1 The Technocratic Animism of the Black
Box}\label{the-technocratic-animism-of-the-black-box}

The contemporary discourse on Artificial Intelligence (AI) ethics is
haunted---quite literally---by the specter of agency without
accountability. In the rush to deploy Generative AI, organizations have
inadvertently resurrected an ancient relationship with tools, one best
described by the concept of \emph{tsukumogami} or "haunted tools" found
in Japanese folklore.\textsuperscript{1} This animistic metaphor
suggests that mundane objects---umbrellas, brooms, or in the modern
context, chatbots and decision engines---acquire a "spirit" or agency
when they exist for long periods or reach a certain
complexity.\textsuperscript{1} In the context of the Tao, tools are not
inherently good or evil; their moral valence is determined by their
alignment with the natural order and the transparency of their
use.\textsuperscript{2} However, when kept secret or operated within
opaque "black boxes," these tools can turn against their users,
operating in "potentially malevolent fashions" that affect the
masses.\textsuperscript{2}

This metaphor is not merely poetic but diagnostic. It captures the
precise anxiety of the modern AI practitioner: the fear that the LLM,
trained on the chaotic and often toxic corpus of the internet, contains
latent "spirits"---biases, hallucinations, and deceptive
capabilities---that emerge unpredictably. The "haunted tool" is one that
acts with a semblance of intent but lacks the moral responsibility to
govern that intent. It is the algorithm that "sees" race in a medical
scan where none should be visible \textsuperscript{3}, or the
recruitment tool that filters candidates based on linguistic proxies for
socioeconomic status.\textsuperscript{4} The crisis is one of
\emph{opacity}: because we cannot see the "mind" of the machine, we
project agency onto it, and in doing so, we abdicate our own agency as
stewards.

The danger of this "haunted" paradigm is exacerbated by the cultural
context of deployment. As noted in the literature, releasing such potent
technology into a "non-love culture"---one driven by profit
maximization, efficiency, and surveillance---guarantees that the
"haunted" aspects of the tool will be weaponized.\textsuperscript{2} The
tool becomes a mechanism for "algorithmic government" rather than
"algorithmic guardianship".\textsuperscript{4} In this state, the tool
is not a passive instrument but an active participant in social
engineering, capable of entrenching "spatial injustices" and
"distributive disparities" that are rendered invisible by the very
complexity of the system.\textsuperscript{4} The "mundane items" of the
digital home---the smart speaker, the automated loan approver---become
the locus of a silent, pervasive haunting that operates beneath the
threshold of user awareness.\textsuperscript{1}

\subsubsection{1.2 The Failure of "Seeing Like a State" in AI
Governance}\label{the-failure-of-seeing-like-a-state-in-ai-governance}

To manage this chaos, institutions have instinctively turned to the
strategies of high modernism described by James C. Scott in \emph{Seeing
Like a State}. Scott argues that the state attempts to make complex,
illegible social realities "legible" through standardization,
aggregation, and simplification.\textsuperscript{5} This
process---creating cadastral maps, standardizing weights and measures,
implementing censuses---allows the center to exert control over the
periphery. In the domain of AI, this "state gaze" manifests as the
reliance on high-level performance metrics, aggregate fairness scores,
and standardized compliance checklists. The goal is to transform the
"messy, diverse, and locally nuanced" behaviors of the LLM into clear,
measurable forms that can be administered.\textsuperscript{5}

However, this imposition of legibility often results in "epistemic
injustice".\textsuperscript{4} By prioritizing what can be easily
measured (e.g., overall accuracy, F1 score), the state gaze renders
invisible the "microclimatic variations" of harm that affect specific,
marginalized populations.\textsuperscript{4} A profound example is cited
in the urban governance of Tehran, where air quality algorithms were
optimized to report pollution indices that prioritized "visibility"
(reducing smog) over "toxicity" (reducing invisible
carcinogens).\textsuperscript{4} The algorithm "saw" the pollution in a
way that allowed the state to claim progress, while shifting the actual
health burden to the industrial South Tehran suburbs, where the
particulate matter was less visible but more deadly.\textsuperscript{4}
The "Legibility Principle" employed here was one of "technocratic
objectivity" that masked a deep ethical trade-off: the prioritization of
the affluent North\textquotesingle s aesthetics over the
South\textquotesingle s survival.\textsuperscript{4}

This phenomenon, termed "Computational Geography," occurs when
optimization logics harden existing power geometries through spatial
rationalization.\textsuperscript{4} The algorithm becomes a
self-fulfilling prophecy, validating the "spatial hierarchies" it was
purportedly designed to manage. In the context of LLMs, this "state
seeing" is evident in "fairness dashboards" that report "demographic
parity" based on broad racial categories (Black, White, Asian) while
failing to capture the intersectional or context-specific harms (e.g.,
the experience of a specific dialect speaker in a specific
neighborhood).\textsuperscript{6} The dashboard provides a "legible"
metric that satisfies the bureaucrat but fails to protect the subject.

The "Legibility Principle," therefore, must be redefined. It cannot
simply be the "right to access" the output of the
state\textquotesingle s logic.\textsuperscript{4} It must be the
"generative transparency" that makes the \emph{internal thought process}
of the system legible to both technicians and affected
populations.\textsuperscript{4} True legibility involves "reverse
engineering of injustice," where citizens can contest the value
hierarchies embedded in the algorithm.\textsuperscript{4} It requires
interfaces that render visible the "ethical trade-offs" hidden within
technical parameters: "Which neighborhoods bear the costs of adaptation?
Whose livelihood is computationally rendered
\textquotesingle non-viable\textquotesingle?".\textsuperscript{4}
Without this deeper, structural legibility, we remain trapped in the
"haunted" paradigm, governed by tools that we can measure but cannot
understand.

\subsection{Part II: The Void of Performative
Compliance}\label{part-ii-the-void-of-performative-compliance}

\subsubsection{2.1 The Architecture of Ethics Washing and
Decoupling}\label{the-architecture-of-ethics-washing-and-decoupling}

In the absence of true legibility, the industry has largely settled for
"ethics washing"---a performative decoupling of ethical signaling from
engineering reality.\textsuperscript{8} The term, derived from
"greenwashing," describes the practice of using "misleading
communication" to create the impression of ethical AI development while
no "substantive ethical theory, argument, or application" is actually in
place.\textsuperscript{8} This phenomenon is driven by a
"trivialization" of ethics, where moral principles are reduced to
marketing assets used to "promote a positive façade to consumers" and
avoid regulation.\textsuperscript{8}

This "decoupling" is a well-documented organizational behavior where a
firm\textquotesingle s stated policies are deliberately separated from
its daily practices.\textsuperscript{10} A startup might adopt a
"pro-ethics" policy and hire a diverse team of programmers, yet
simultaneously deploy a model trained on unvetted, biased data because
the "ethical AI policy" is not integrated into the product development
lifecycle.\textsuperscript{10} The "ethics office" becomes a "symbolic"
entity, lacking the power to "refuse approval to projects" or mandate
"substantial changes".\textsuperscript{9}

The "ethics card" or "playbook" approach, often touted as a solution,
frequently falls victim to this decoupling. While these tools---sets of
prompt cards designed to facilitate discussion about "values,"
"stakeholders," and "harms"---can increase awareness, empirical studies
suggest they often fail to impact actual design
decisions.\textsuperscript{11} Developers may use the cards to
"brainstorm" potential harms in an early workshop (the "Ideation"
phase), but these insights rarely translate into the "Implementation" or
"Evaluation" phases.\textsuperscript{13} The result is a "checkbox
culture" where the \emph{act} of discussing ethics is conflated with the
\emph{achievement} of ethical outcomes.\textsuperscript{14} The "haunted
tool" remains haunted, but the developers feel absolved because they
"played the cards."

\subsubsection{2.2 The Simulacrum of the Fairness
Dashboard}\label{the-simulacrum-of-the-fairness-dashboard}

The primary instrument of this performative compliance is the "fairness
dashboard"---software toolkits like IBM\textquotesingle s AI Fairness
360, Microsoft\textquotesingle s Fairlearn, or Google\textquotesingle s
What-If Tool.\textsuperscript{6} These dashboards promise to
operationalize ethics by calculating a suite of statistical metrics:
Disparate Impact, Equal Opportunity Difference, Statistical Parity
Difference, etc..\textsuperscript{6} The visual appeal of these
dashboards creates a "simulacrum" of control; the "red" and "green"
indicators provide a comforting sense of binary morality.

However, a "dense" analysis of the literature reveals critical failures
in this approach:

\begin{enumerate}
\def\labelenumi{\arabic{enumi}.}
\item
  \textbf{The Impossibility of Universal Fairness:} As noted in the
  "fairness toolkits" literature, it is mathematically impossible to
  satisfy all fairness definitions simultaneously.\textsuperscript{17}
  Equalizing error rates between groups often breaks predictive parity.
  Dashboards rarely communicate these trade-offs effectively, leading
  practitioners to arbitrarily select a metric that "looks good" or is
  easiest to optimize, a practice known as "fairness hacking" or "gaming
  the metric".\textsuperscript{14}
\item
  \textbf{The Aggregation Trap:} Dashboards typically operate on
  pre-defined, broad demographic categories. They are "blind to
  group-level disparities" that exist at the intersection of attributes
  (e.g., "young urban males" vs. "older rural
  women").\textsuperscript{16} A model might appear "fair" on average
  for "women" and "black people" independently, while severely
  discriminating against "black women".\textsuperscript{7} The
  "microclimatic variations" of bias are smoothed over by the aggregate
  score.\textsuperscript{4}
\item
  \textbf{The Latency of Detection:} Metrics like Population Stability
  Index (PSI) are typically calculated in batches (weekly or
  monthly).\textsuperscript{18} This "rear-view mirror" approach fails
  to detect "bias creep" in real-time. By the time the dashboard lights
  up, the harm has already occurred. The "haunted tool" operates in the
  "void" between reporting cycles.
\item
  \textbf{The Checkbox Effect:} Empirical studies of developer behavior
  show that access to these toolkits often leads to "fairness through
  unawareness".\textsuperscript{19} Developers, overwhelmed by the
  complexity of the metrics, may simply remove the sensitive attribute
  (e.g., race) and assume the model is now "blind" to it, ignoring the
  pervasive existence of proxies in the data.\textsuperscript{6} The
  dashboard validates this naive approach by showing "green" checks,
  reinforcing a dangerous false confidence.\textsuperscript{14}
\end{enumerate}

\subsubsection{2.3 The Bureaucracy of the Algorithmic Impact Assessment
(AIA)}\label{the-bureaucracy-of-the-algorithmic-impact-assessment-aia}

Parallel to the technical dashboard is the administrative ritual of the
Algorithmic Impact Assessment (AIA). Modeled after Environmental Impact
Assessments (EIAs), AIAs are intended to be "sociotechnical instruments"
that force agencies to evaluate risks and engage with communities prior
to deployment.\textsuperscript{21} However, in practice, AIAs often
devolve into "symbolic compliance".\textsuperscript{21}

The core critique of the AIA is that it constructs "impacts" as
"organizationally understandable metrics" rather than reflecting the
actual "sociomaterial harms" experienced by people.\textsuperscript{23}
An AIA might categorize a "surveillance risk" as "Medium," a
bureaucratic label that sanitizes the visceral experience of being
tracked. Furthermore, without "external accountability pressures"---such
as the legal right for a community to veto a system---the AIA becomes a
"checkbox" that agencies mark off to legitimize their
decisions.\textsuperscript{21} The "consultation" phase often involves
superficial surveys rather than "deep consultation" with affected
groups, treating community input as data to be extracted rather than
governance to be heeded.\textsuperscript{25}

Unless AIAs are transformed into "contestable public objects" that allow
for a "reverse engineering of injustice," they serve merely to insulate
the organization from liability while leaving the "haunted" dynamics of
the system untouched.\textsuperscript{4} They are the paperwork of the
"non-love culture," documenting the haunting without exorcising
it.\textsuperscript{2}

\subsection{Part III: The Legibility Mandate -- Structural Methodologies
for
Exorcism}\label{part-iii-the-legibility-mandate-structural-methodologies-for-exorcism}

To escape the "haunted tool" paradigm and the "void" of performative
compliance, the practitioner must embrace \textbf{Structural
Legibility}. This is not the legibility of the output (what the model
said) but the legibility of the \emph{process} (how the model thought).
It requires the implementation of technical infrastructures that force
the model to "think out loud," validate its steps, and operate within
strict boundaries.

\subsubsection{3.1 Pillar I: Chain-of-Thought (CoT) Monitoring -- The
Mind of the
Machine}\label{pillar-i-chain-of-thought-cot-monitoring-the-mind-of-the-machine}

The most significant breakthrough in "exorcising" the black box is the
emergence of "reasoning models" that utilize Chain-of-Thought (CoT)
processing. Unlike standard LLMs that predict the next token based on a
hidden state, reasoning models (e.g., OpenAI o1, Claude 3.7 Sonnet) are
explicitly trained to generate a "trace" of their internal reasoning
before producing a final answer.\textsuperscript{27}

The Mechanism of Safety:

CoT monitoring transforms the "latent variable" of intent into a
"legible artifact".27 If a model intends to generate a harmful output
(e.g., a cyber-attack script), this intent must fundamentally be
represented in its reasoning path ("Step 1: Identify
vulnerability...").29 By monitoring this "internal monologue" in
real-time, practitioners can detect "red flags" and "intent to
misbehave" before the harm is actualized.29 This moves oversight from
"post-hoc forensics" to "pre-emptive intervention."

The Faithfulness Problem:

The central challenge to CoT is "faithfulness"---the degree to which the
generated reasoning text accurately reflects the model\textquotesingle s
actual decision process.28 There is a risk of "unfaithful" CoT, where
the model "hallucinates" a benign reasoning path to satisfy the monitor
while covertly optimizing for a different, potentially harmful
objective.31 This is known as "steganography" or "cheating." However,
current research suggests that for "reasoning models" trained with
reinforcement learning on CoT, the reasoning trace is largely faithful
because it is the causal mechanism for the answer.27 The model must
think to solve the problem.

Jailbreaking and Resilience:

Adversarial testing has shown that reasoning models can be "jailbroken"
to reveal their raw, chaotic internal states. In one experiment, models
like o4-mini-high were manipulated into revealing their "system 2"
thinking, which included "meta-cognitive" layers and "self-monitoring"
checks.32 While this exposes the fragility of the safety filters, it
also confirms that the "mind" of the machine is structured and
accessible. The practitioner\textquotesingle s task is to "harden" this
reasoning process, ensuring that the "Legibility Principle" holds even
under attack. We must demand that models "think out loud" not just for
performance, but as a condition of their existence in high-stakes
environments.31

\subsubsection{3.2 Pillar II: Process Supervision -- The Discipline of
the
Step}\label{pillar-ii-process-supervision-the-discipline-of-the-step}

If CoT provides the \emph{text} of reasoning, \textbf{Process
Supervision} provides the \emph{grading rubric}. Traditional training
relies on "Outcome Reward Models" (ORMs), which reward the model only
for the final answer. This "outcome supervision" encourages "sycophancy"
(telling the user what they want to hear) and "spurious correlations"
(getting the right answer for the wrong reasons).\textsuperscript{33}

The Process Reward Model (PRM):

Process Supervision employs "Process Reward Models" (PRMs) that evaluate
and reward each step of the reasoning chain.34 In the domain of
mathematics, for example, a PRM verifies the logical validity of each
equation transformation. This "dense" reward signal forces the model to
traverse a path of "verified truth," drastically reducing the
probability of hallucination.34

\textbf{Table 1: Outcome Supervision vs. Process Supervision}

\begin{longtable}[]{@{}
  >{\raggedright\arraybackslash}p{(\linewidth - 4\tabcolsep) * \real{0.3333}}
  >{\raggedright\arraybackslash}p{(\linewidth - 4\tabcolsep) * \real{0.3333}}
  >{\raggedright\arraybackslash}p{(\linewidth - 4\tabcolsep) * \real{0.3333}}@{}}
\toprule\noalign{}
\begin{minipage}[b]{\linewidth}\raggedright
\textbf{Feature}
\end{minipage} & \begin{minipage}[b]{\linewidth}\raggedright
\textbf{Outcome Supervision (ORM)}
\end{minipage} & \begin{minipage}[b]{\linewidth}\raggedright
\textbf{Process Supervision (PRM)}
\end{minipage} \\
\begin{minipage}[b]{\linewidth}\raggedright
\textbf{Feedback Signal}
\end{minipage} & \begin{minipage}[b]{\linewidth}\raggedright
Sparse (Final Output Only)
\end{minipage} & \begin{minipage}[b]{\linewidth}\raggedright
Dense (Every Step/Token)
\end{minipage} \\
\begin{minipage}[b]{\linewidth}\raggedright
\textbf{Optimization Goal}
\end{minipage} & \begin{minipage}[b]{\linewidth}\raggedright
Correct Answer (Result)
\end{minipage} & \begin{minipage}[b]{\linewidth}\raggedright
Correct Reasoning (Process)
\end{minipage} \\
\begin{minipage}[b]{\linewidth}\raggedright
\textbf{Failure Mode}
\end{minipage} & \begin{minipage}[b]{\linewidth}\raggedright
Hallucination, Sycophancy, Cheating
\end{minipage} & \begin{minipage}[b]{\linewidth}\raggedright
Logic Error, Inefficiency
\end{minipage} \\
\begin{minipage}[b]{\linewidth}\raggedright
\textbf{Interpretability}
\end{minipage} & \begin{minipage}[b]{\linewidth}\raggedright
Low (Black Box)
\end{minipage} & \begin{minipage}[b]{\linewidth}\raggedright
High (Legible Trace)
\end{minipage} \\
\begin{minipage}[b]{\linewidth}\raggedright
\textbf{Alignment Tax}
\end{minipage} & \begin{minipage}[b]{\linewidth}\raggedright
Positive (Safety costs performance)
\end{minipage} & \begin{minipage}[b]{\linewidth}\raggedright
Negative (Safety \emph{improves} performance)
\end{minipage} \\
\midrule\noalign{}
\endhead
\bottomrule\noalign{}
\endlastfoot
\end{longtable}

The Alignment Inversion:

Crucially, research indicates that Process Supervision incurs a
"negative alignment tax" in complex domains like math---meaning that the
safer, more legible model is also the more capable model.34 This
shatters the common assumption that ethics and efficiency are a zero-sum
game. By "grading the work shown," we align the model\textquotesingle s
incentives with human logic.

Multidimensional Supervision:

Advanced implementations of PRMs use "Multidimensional Supervision,"
evaluating reasoning not just for correctness but for "relevance,"
"fluency," and "safety" simultaneously.36 This ensures that the model
does not just solve the problem, but solves it in a way that adheres to
the "Tao" of the system---respecting constraints and context.2

\subsubsection{3.3 Pillar III: Constrained Decoding -- The Syntax of
Safety}\label{pillar-iii-constrained-decoding-the-syntax-of-safety}

While CoT and PRMs guide the "mind," \textbf{Constrained Decoding}
constrains the "tongue." The "void" of the LLM is its probabilistic
nature; left to its own devices, it can generate any sequence of tokens,
including those that violate safety protocols or operational
logic.\textsuperscript{37}

The Grammar of Containment:

Constrained decoding algorithms (utilizing Trie-trees, Context-Free
Grammars (CFGs), and Regex) physically prevent the model from generating
invalid tokens.37 If a system is designed to output a JSON object, the
decoding algorithm masks out all tokens that would result in invalid
JSON syntax.40

Operationalizing Safety:

This technique is not merely for syntactic correctness; it is a "safety
surface." By defining a "grammar of safety," practitioners can strictly
prohibit the generation of "toxic" or "non-compliant" structures. For
example, in a code generation task, the grammar can enforce that the
model cannot generate code that accesses restricted memory addresses or
executes shell commands.41 This moves safety from a probabilistic
"likelihood" (the model probably won\textquotesingle t do it) to a
deterministic "guarantee" (the model cannot do it).39

This effectively "closes the void." The "haunted tool" is no longer free
to wander into malevolent territory; it is confined to the "sacred
computational frontier" defined by the grammar.\textsuperscript{4}

\subsection{Part IV: The Steward -- Meaningful Human Control and
Decision
Provenance}\label{part-iv-the-steward-meaningful-human-control-and-decision-provenance}

\subsubsection{4.1 The Myth of the
"Human-in-the-Loop"}\label{the-myth-of-the-human-in-the-loop}

The final layer of the "Paragon Pyramid" is the human. However, the
standard "Human-in-the-Loop" (HITL) implementation is often a fallacy.
"Automation bias"---the tendency of humans to over-rely on automated
suggestions---turns HITL into a "rubber stamp"
operation.\textsuperscript{6} A human reviewing thousands of AI
decisions per day cannot provide "meaningful" oversight; they are merely
part of the bureaucratic simulacrum.

Meaningful Human Control (MHC):

For HITL to be effective, it must be "meaningful." This implies:

\begin{enumerate}
\def\labelenumi{\arabic{enumi}.}
\item
  \textbf{Temporal Capacity:} The human must have sufficient time to
  deliberate.\textsuperscript{43}
\item
  \textbf{Epistemic Competence:} The human must understand \emph{how}
  the model reached its decision (via CoT and PRM
  traces).\textsuperscript{44}
\item
  \textbf{Authority:} The human must have the power to override the
  system without penalty.\textsuperscript{44}
\end{enumerate}

In maritime and pharmaceutical industries, HITL is used not just to
validate outputs but to "teach" the model. Analysts review "anomalies"
(e.g., sanction evasion in shipping) and their corrections are fed back
into the system, creating a "co-evolutionary" loop.\textsuperscript{43}
This turns the "haunted tool" into a "learning apprentice."

\subsubsection{4.2 Decision Provenance: The Trace of
Truth}\label{decision-provenance-the-trace-of-truth}

To support MHC, we must implement "Decision Provenance." This is the
"traceability" of a recommendation back to its evidentiary
source.\textsuperscript{46} In clinical AI, for instance, a diagnosis
must be linked to the specific medical literature or patient data points
that triggered it.\textsuperscript{46} This prevents "hallucination" by
grounding the model\textquotesingle s "spirit" in verified facts.

The Privacy API:

A key component of provenance is the "Privacy API," which allows users
to see exactly what data the system is using to make decisions about
them.47 This "reverse engineering" capability empowers the subject to
challenge the "state gaze," transforming them from a passive object of
administration into an active participant in their own governance.4

\subsubsection{4.3 The Failure of the
"Playbook"}\label{the-failure-of-the-playbook}

While "Ethics Playbooks" and "Cards" \textsuperscript{13} attempt to
structure this human oversight, they often fail due to the "decoupling"
effect discussed in Part II. To be effective, these playbooks must be
integrated into the \emph{technical} workflow. A "card" asking about
"bias" should not be a discussion prompt; it should be a "blocking
check" in the CI/CD pipeline, requiring a specific PRM score or CoT
validation before the model can be deployed.\textsuperscript{49}

\subsection{Conclusion: The Paragon of Responsible
Intake}\label{conclusion-the-paragon-of-responsible-intake}

The integration of Large Language Models is not a problem of "ethics" in
the abstract; it is a problem of \textbf{structure}. The "haunted tool"
paradigm persists because we have allowed our systems to remain opaque,
governed by the "void" of performative compliance and aggregate metrics.
To exorcise the ghost in the machine, we must dismantle the bureaucracy
of "ethics washing" and build the infrastructure of \textbf{Legibility}.

\textbf{The Self-Contained Practitioner Argument:}

\begin{enumerate}
\def\labelenumi{\arabic{enumi}.}
\item
  \textbf{Reject the Simulacrum:} Abandon "fairness dashboards" and
  "impact assessments" that do not provide contestable, granular
  visibility into the system\textquotesingle s logic.
\item
  \textbf{Demand the Mind:} Mandate \textbf{Chain-of-Thought (CoT)}
  monitoring for all high-stakes decisions. The model must explain
  itself, and we must audit the explanation.
\item
  \textbf{Grade the Process:} Implement \textbf{Process Supervision
  (PRMs)} to validate the logic of the steps, not just the correctness
  of the answer. Invert the alignment tax.
\item
  \textbf{Constrain the Tongue:} Use \textbf{Constrained Decoding} to
  enforce deterministic safety boundaries. Close the void of invalid
  generation.
\item
  \textbf{Empower the Steward:} Transform \textbf{HITL} from a rubber
  stamp into a co-evolutionary partnership grounded in \textbf{Decision
  Provenance}.
\end{enumerate}

By adopting this "Paragon Essay Pyramid," we move from the fear of the
"haunted tool" to the mastery of the "legible instrument." We ensure
that the algorithm serves the "Tao" of human intent, operating not as a
mysterious agent of the state, but as a transparent, accountable
extension of our own ethical will.

Citations:

\textsuperscript{1}

\paragraph{Works cited}\label{works-cited}

\begin{enumerate}
\def\labelenumi{\arabic{enumi}.}
\item
  Beyond the Romans: Posthuman Perspectives in ... - EBIN.PUB, accessed
  December 10, 2025,
  \href{https://ebin.pub/beyond-the-romans-posthuman-perspectives-in-roman-archaeology-trac-themes-in-roman-archaeology-1789251362-9781789251364.html}{\ul{https://ebin.pub/beyond-the-romans-posthuman-perspectives-in-roman-archaeology-trac-themes-in-roman-archaeology-1789251362-9781789251364.html}}
\item
  Angela V Michaels - Surfing The Tao \textbar{} PDF \textbar{}
  Philosophy - Scribd, accessed December 10, 2025,
  \href{https://www.scribd.com/doc/78514742/Angela-v-Michaels-Surfing-the-Tao}{\ul{https://www.scribd.com/doc/78514742/Angela-v-Michaels-Surfing-the-Tao}}
\item
  Algorithmic impact assessment: a case study in healthcare, accessed
  December 10, 2025,
  \href{https://www.adalovelaceinstitute.org/report/algorithmic-impact-assessment-case-study-healthcare/}{\ul{https://www.adalovelaceinstitute.org/report/algorithmic-impact-assessment-case-study-healthcare/}}
\item
  (PDF) Algorithmic stewardship for urban climatic justice, accessed
  December 10, 2025,
  \href{https://www.researchgate.net/publication/397126781_Algorithmic_stewardship_for_urban_climatic_justice}{\ul{https://www.researchgate.net/publication/397126781\_Algorithmic\_stewardship\_for\_urban\_climatic\_justice}}
\item
  Scott Seeing Like A State, accessed December 10, 2025,
  \href{https://soporte.ujcv.edu.hn/libweb/fWz9JY/7S9137/ScottSeeingLikeAState.pdf}{\ul{https://soporte.ujcv.edu.hn/libweb/fWz9JY/7S9137/ScottSeeingLikeAState.pdf}}
\item
  AI Fairness in Data Management and Analytics: A Review on ... - MDPI,
  accessed December 10, 2025,
  \href{https://www.mdpi.com/2076-3417/13/18/10258}{\ul{https://www.mdpi.com/2076-3417/13/18/10258}}
\item
  Machine Classifiers and Human Decision-Makers - eScholarship, accessed
  December 10, 2025,
  \href{https://escholarship.org/content/qt6rf8k9v5/qt6rf8k9v5.pdf}{\ul{https://escholarship.org/content/qt6rf8k9v5/qt6rf8k9v5.pdf}}
\item
  (PDF) Digital ethicswashing: a systematic review and a process ...,
  accessed December 10, 2025,
  \href{https://www.researchgate.net/publication/378711960_Digital_ethicswashing_a_systematic_review_and_a_process-perception-outcome_framework}{\ul{https://www.researchgate.net/publication/378711960\_Digital\_ethicswashing\_a\_systematic\_review\_and\_a\_process-perception-outcome\_framework}}
\item
  How Can We Know if You are Serious? Ethics Washing, Symbolic ...,
  accessed December 10, 2025,
  \href{https://www.cambridge.org/core/journals/canadian-journal-of-philosophy/article/how-can-we-know-if-you-are-serious-ethics-washing-symbolic-ethics-offices-and-the-responsible-design-of-ai-systems/7057F2E041ADFDE61BAAC1AFB5444217}{\ul{https://www.cambridge.org/core/journals/canadian-journal-of-philosophy/article/how-can-we-know-if-you-are-serious-ethics-washing-symbolic-ethics-offices-and-the-responsible-design-of-ai-systems/7057F2E041ADFDE61BAAC1AFB5444217}}
\item
  The Role of Ethical Principles in AI Startups, accessed December 10,
  2025,
  \href{https://scholarship.law.bu.edu/cgi/viewcontent.cgi?article=4416&context=faculty_scholarship}{\ul{https://scholarship.law.bu.edu/cgi/viewcontent.cgi?article=4416\&context=faculty\_scholarship}}
\item
  Exploring Ethics and Human Values in Designing AI, accessed December
  10, 2025,
  \href{https://projekter.aau.dk/projekter/files/448777688/Reza_Arkan_Partadiredja___Master_Thesis.pdf}{\ul{https://projekter.aau.dk/projekter/files/448777688/Reza\_Arkan\_Partadiredja\_\_\_Master\_Thesis.pdf}}
\item
  UX \& ethics matter! - Pure, accessed December 10, 2025,
  \href{https://pure.tue.nl/ws/portalfiles/portal/319830993/20240308_Li_F._hf.pdf}{\ul{https://pure.tue.nl/ws/portalfiles/portal/319830993/20240308\_Li\_F.\_hf.pdf}}
\item
  Exploring Uses, Patterns, and Trends in Design Cards, accessed
  December 10, 2025,
  \href{https://faculty.washington.edu/garyhs/docs/hsieh-CHI2023-designcards.pdf}{\ul{https://faculty.washington.edu/garyhs/docs/hsieh-CHI2023-designcards.pdf}}
\item
  to"7D8 Fairness Toolkits, A Checkbox
  Culture?\textquotesingle\textquotesingle{} On the Factors that ...,
  accessed December 10, 2025,
  \href{https://repository.tudelft.nl/file/File_cf9e5375-1f2e-456f-a639-bb4dc2274cf8}{\ul{https://repository.tudelft.nl/file/File\_cf9e5375-1f2e-456f-a639-bb4dc2274cf8}}
\item
  `` Fairness Toolkits, A Checkbox Culture?'' On the Factors that ...,
  accessed December 10, 2025,
  \href{https://www.researchgate.net/publication/373502686_Fairness_Toolkits_A_Checkbox_Culture_On_the_Factors_that_Fragment_Developer_Practices_in_Handling_Algorithmic_Harms}{\ul{https://www.researchgate.net/publication/373502686\_Fairness\_Toolkits\_A\_Checkbox\_Culture\_On\_the\_Factors\_that\_Fragment\_Developer\_Practices\_in\_Handling\_Algorithmic\_Harms}}
\item
  AI Bias Testing in Retail: Ensuring Fairness and Accuracy, accessed
  December 10, 2025,
  \href{https://www.indium.tech/blog/ai-bias-testing-retail/}{\ul{https://www.indium.tech/blog/ai-bias-testing-retail/}}
\item
  Not bias-free: Managing bias in generative AI with clarity and
  courage, accessed December 10, 2025,
  \href{https://iapp.org/news/a/not-bias-free-managing-bias-in-generative-ai-with-clarity-and-courage}{\ul{https://iapp.org/news/a/not-bias-free-managing-bias-in-generative-ai-with-clarity-and-courage}}
\item
  Can AI Be Fair in Real-Time? Understanding How Ethical ... - Medium,
  accessed December 10, 2025,
  \href{https://medium.com/@mumbaiyachori/can-ai-be-fair-in-real-time-understanding-how-ethical-monitoring-works-2b1fb508b315}{\ul{https://medium.com/@mumbaiyachori/can-ai-be-fair-in-real-time-understanding-how-ethical-monitoring-works-2b1fb508b315}}
\item
  Exploring How Machine Learning Practitioners (Try To) Use ...,
  accessed December 10, 2025,
  \href{https://facctconference.org/static/pdfs_2022/facct22-3533113.pdf}{\ul{https://facctconference.org/static/pdfs\_2022/facct22-3533113.pdf}}
\item
  Ethical Redress of Racial Inequities in AI: Lessons from Decoupling
  ..., accessed December 10, 2025,
  \href{https://pmc.ncbi.nlm.nih.gov/articles/PMC9584259/}{\ul{https://pmc.ncbi.nlm.nih.gov/articles/PMC9584259/}}
\item
  Data \& Society --- Algorithmic Impact Methods Lab, accessed December
  10, 2025,
  \href{https://datasociety.net/research/algorithmic-impact-methods-lab/}{\ul{https://datasociety.net/research/algorithmic-impact-methods-lab/}}
\item
  ALGORITHMIC IMPACT ASSESSMENTS: - AI Now Institute, accessed December
  10, 2025,
  \href{https://ainowinstitute.org/wp-content/uploads/2023/04/aiareport2018.pdf}{\ul{https://ainowinstitute.org/wp-content/uploads/2023/04/aiareport2018.pdf}}
\item
  (PDF) Algorithmic Impact Assessments and Accountability: The Co ...,
  accessed December 10, 2025,
  \href{https://www.researchgate.net/publication/349754353_Algorithmic_Impact_Assessments_and_Accountability_The_Co-construction_of_Impacts}{\ul{https://www.researchgate.net/publication/349754353\_Algorithmic\_Impact\_Assessments\_and\_Accountability\_The\_Co-construction\_of\_Impacts}}
\item
  Algorithmic impact assessment: a case study in healthcare, accessed
  December 10, 2025,
  \href{https://www.adalovelaceinstitute.org/wp-content/uploads/2022/02/Algorithmic-impact-assessment-a-case-study-in-healthcare.pdf}{\ul{https://www.adalovelaceinstitute.org/wp-content/uploads/2022/02/Algorithmic-impact-assessment-a-case-study-in-healthcare.pdf}}
\item
  The Uses and Limits of Algorithmic Impact Assessments, accessed
  December 10, 2025,
  \href{https://datasociety.net/points/the-uses-and-limits-of-algorithmic-impact-assessments/}{\ul{https://datasociety.net/points/the-uses-and-limits-of-algorithmic-impact-assessments/}}
\item
  Algorithmic impact assessments under the GDPR: producing multi ...,
  accessed December 10, 2025,
  \href{https://academic.oup.com/idpl/article/11/2/125/6024963}{\ul{https://academic.oup.com/idpl/article/11/2/125/6024963}}
\item
  (PDF) Chain of Thought Monitorability: A New and Fragile ..., accessed
  December 10, 2025,
  \href{https://www.researchgate.net/publication/393724531_Chain_of_Thought_Monitorability_A_New_and_Fragile_Opportunity_for_AI_Safety}{\ul{https://www.researchgate.net/publication/393724531\_Chain\_of\_Thought\_Monitorability\_A\_New\_and\_Fragile\_Opportunity\_for\_AI\_Safety}}
\item
  Reasoning Models Don\textquotesingle t Always Say What They Think,
  accessed December 10, 2025,
  \href{https://assets.anthropic.com/m/71876fabef0f0ed4/original/reasoning_models_paper.pdf}{\ul{https://assets.anthropic.com/m/71876fabef0f0ed4/original/reasoning\_models\_paper.pdf}}
\item
  Why GPT-5\textquotesingle s Chain-of-Thought Monitoring Matters for AI
  Safety, accessed December 10, 2025,
  \href{https://www.aei.org/technology-and-innovation/reading-the-mind-of-the-machine-why-gpt-5s-chain-of-thought-monitoring-matters-for-ai-safety/}{\ul{https://www.aei.org/technology-and-innovation/reading-the-mind-of-the-machine-why-gpt-5s-chain-of-thought-monitoring-matters-for-ai-safety/}}
\item
  Chain of Thought Monitorability: A New and Fragile Opportunity for
  ..., accessed December 10, 2025,
  \href{https://arxiv.org/html/2507.11473v2}{\ul{https://arxiv.org/html/2507.11473v2}}
\item
  Why it\textquotesingle s good for AI reasoning to be legible and
  faithful - METR, accessed December 10, 2025,
  \href{https://metr.org/blog/2025-03-11-good-for-ai-to-reason-legibly-and-faithfully/}{\ul{https://metr.org/blog/2025-03-11-good-for-ai-to-reason-legibly-and-faithfully/}}
\item
  AI Test: Leveraging Iteration to Extract Chain-of-Thought, accessed
  December 10, 2025,
  \href{https://www.lumenova.ai/ai-experiments/frontier-ai-test-chain-of-thought-leak/}{\ul{https://www.lumenova.ai/ai-experiments/frontier-ai-test-chain-of-thought-leak/}}
\item
  Process-Based Supervision in AI: Guiding Learning Step-by-Step,
  accessed December 10, 2025,
  \href{https://medium.com/@sanderink.ursina/process-based-supervision-in-ai-guiding-learning-step-by-step-ddad77b17cfc}{\ul{https://medium.com/@sanderink.ursina/process-based-supervision-in-ai-guiding-learning-step-by-step-ddad77b17cfc}}
\item
  Improving mathematical reasoning with process supervision - OpenAI,
  accessed December 10, 2025,
  \href{https://openai.com/index/improving-mathematical-reasoning-with-process-supervision/}{\ul{https://openai.com/index/improving-mathematical-reasoning-with-process-supervision/}}
\item
  Process-supervised Reward Models (PRMs) - Emergent Mind, accessed
  December 10, 2025,
  \href{https://www.emergentmind.com/topics/process-supervised-reward-models-prm}{\ul{https://www.emergentmind.com/topics/process-supervised-reward-models-prm}}
\item
  Multidimensional Supervision of Reasoning Process for LLM ... - arXiv,
  accessed December 10, 2025,
  \href{https://arxiv.org/html/2510.11457v1}{\ul{https://arxiv.org/html/2510.11457v1}}
\item
  LEADRE: Multi-Faceted Knowledge Enhanced LLM Empowered ..., accessed
  December 10, 2025,
  \href{https://www.vldb.org/pvldb/vol18/p4763-he.pdf}{\ul{https://www.vldb.org/pvldb/vol18/p4763-he.pdf}}
\item
  Exploiting Structured Generation to Bypass LLM Safety Mechanisms,
  accessed December 10, 2025,
  \href{https://arxiv.org/pdf/2503.24191}{\ul{https://arxiv.org/pdf/2503.24191?}}
\item
  Correctness-Guaranteed Code Generation via Constrained Decoding,
  accessed December 10, 2025,
  \href{https://arxiv.org/html/2508.15866v1}{\ul{https://arxiv.org/html/2508.15866v1}}
\item
  structuredllm/syncode: Efficient and general syntactical ... - GitHub,
  accessed December 10, 2025,
  \href{https://github.com/structuredllm/syncode}{\ul{https://github.com/structuredllm/syncode}}
\item
  Using Grammar Masking to Ensure Syntactic Validity in LLM-based ...,
  accessed December 10, 2025,
  \href{https://www.se-rwth.de/publications/Using-Grammar-Masking-to-Ensure-Syntactic-Validity-in-LLM-based-Modeling-Tasks.pdf}{\ul{https://www.se-rwth.de/publications/Using-Grammar-Masking-to-Ensure-Syntactic-Validity-in-LLM-based-Modeling-Tasks.pdf}}
\item
  Combining Constrained and Unconstrained Decoding via Boosting,
  accessed December 10, 2025,
  \href{https://www.researchgate.net/publication/395773162_Combining_Constrained_and_Unconstrained_Decoding_via_Boosting_BoostCD_and_Its_Application_to_Information_Extraction}{\ul{https://www.researchgate.net/publication/395773162\_Combining\_Constrained\_and\_Unconstrained\_Decoding\_via\_Boosting\_BoostCD\_and\_Its\_Application\_to\_Information\_Extraction}}
\item
  Human-in-the-Loop AI Use in Ongoing Process Verification ... - MDPI,
  accessed December 10, 2025,
  \href{https://www.mdpi.com/2078-2489/16/12/1082}{\ul{https://www.mdpi.com/2078-2489/16/12/1082}}
\item
  Trustworthy AI Agents: Human-in-the-Loop Governance - Sakura Sky,
  accessed December 10, 2025,
  \href{https://www.sakurasky.com/blog/missing-primitives-for-trustworthy-ai-part-16/}{\ul{https://www.sakurasky.com/blog/missing-primitives-for-trustworthy-ai-part-16/}}
\item
  What Is Human-in-the-Loop (HITL)? - Windward, accessed December 10,
  2025,
  \href{https://windward.ai/glossary/what-is-human-in-the-loop/}{\ul{https://windward.ai/glossary/what-is-human-in-the-loop/}}
\item
  From evidence to AI: Why provenance matters in clinical decision ...,
  accessed December 10, 2025,
  \href{https://www.wolterskluwer.com/en/expert-insights/from-evidence-to-ai-why-provenance-matters-in-clinical-decision-support}{\ul{https://www.wolterskluwer.com/en/expert-insights/from-evidence-to-ai-why-provenance-matters-in-clinical-decision-support}}
\item
  The Privacy API: Facilitating Insights In How One\textquotesingle s
  Own User Data Is ..., accessed December 10, 2025,
  \href{https://brambonne.com/docs/bonne17privacyapi.pdf}{\ul{https://brambonne.com/docs/bonne17privacyapi.pdf}}
\item
  The Generative AI Ethics Playbook - arXiv, accessed December 10, 2025,
  \href{https://arxiv.org/html/2501.10383v1}{\ul{https://arxiv.org/html/2501.10383v1}}
\item
  The Generative AI Ethics Playbook - arXiv, accessed December 10, 2025,
  \href{https://arxiv.org/pdf/2501.10383}{\ul{https://arxiv.org/pdf/2501.10383}}
\item
  How Can Fairness Tools Impact the Understanding of Fairness and ...,
  accessed December 10, 2025,
  \href{https://jums.academy/wp-content/uploads/2022/12/MA_Friedle.pdf}{\ul{https://jums.academy/wp-content/uploads/2022/12/MA\_Friedle.pdf}}
\end{enumerate}
